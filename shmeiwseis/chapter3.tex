\documentclass{book}

\usepackage{etex}
\usepackage{etoolbox}

%%%%%%%%%%%%%%%%%%%%%%%%%%%%%%%%%%%%%%
% Babel language package
%\usepackage[english,greek]{babel}
% Inputenc font encoding
%\usepackage[utf8]{inputenc}


% \usepackage{xltxtra} 
% \usepackage{xgreek} 
% \setmainfont[Mapping=tex-text]{GFS Didot} 

%\usepackage{kmath,kerkis} % The order of the packages matters; kmath changes the default text font
%\usepackage[T1]{fontenc}
\usepackage{ifxetex}
\ifxetex
    % IF XELATEX

% MINION new font
\usepackage{fontspec}
%\usepackage{mathspec}

\setmainfont[Extension=.ttf,UprightFont=*-Regular,BoldFont=*-Bold,ItalicFont=*-Italic,BoldItalicFont=*-Bold-Italic]{Minion-Pro}
\setsansfont[Extension=.ttf,UprightFont=*H,BoldFont=*HB,ItalicFont=*HI,BoldItalicFont=*HBI]{Vera}
\usepackage{unicode-math}
\setmathfont{latinmodern-math.otf}
%\setmathfont[range=\mathup]{Minion-Pro-Regular.ttf}
%\setmathfont[range=\mathit]{Minion-Pro-Italic.ttf}
%\setmathfont[range=\mathbf]{Minion-Pro-Bold.ttf}
%\setmathfont[range={0222B}]{Minion-Pro-Italic.ttf}
%\setmathfontface\mathfoo{Minion-Pro-Regular.ttf}
%\setoperatorfont\mathfoo
    \usepackage{polyglossia}
    \setdefaultlanguage{greek}
    \setotherlanguage{english}
    %\RequirePackage{unicode-math}
    %\setmathfont{Latin Modern Math}
    %\newcommand{\smblkcircle}{•}
\else
    % IF PDFLATEX
    %\usepackage{tgheros}
    %\renewcommand*\familydefault{\sfdefault}
    %\usepackage[eulergreek]{sansmath}
    %\sansmath
    \usepackage[T1]{fontenc}
    \usepackage[utf8]{inputenc}
    \usepackage[english,greek]{babel}
\fi


\usepackage{anyfontsize}
\newlength{\FONTmain}\setlength{\FONTmain}{9pt}
\newlength{\FONTmainbl}\setlength{\FONTmainbl}{1.2\FONTmain}
\renewcommand{\tiny}        {\fontsize{0.6\FONTmain}{0.6\FONTmainbl}\selectfont}
\renewcommand{\scriptsize}  {\fontsize{0.7\FONTmain}{0.7\FONTmainbl}\selectfont}
\renewcommand{\footnotesize}{\fontsize{0.8\FONTmain}{0.8\FONTmainbl}\selectfont}
\renewcommand{\small}       {\fontsize{0.9\FONTmain}{0.9\FONTmainbl}\selectfont}
\renewcommand{\normalsize}  {\fontsize{1.0\FONTmain}{1.0\FONTmainbl}\selectfont}
\renewcommand{\large}       {\fontsize{1.2\FONTmain}{1.2\FONTmainbl}\selectfont}
\renewcommand{\Large}       {\fontsize{1.4\FONTmain}{1.4\FONTmainbl}\selectfont}
\renewcommand{\LARGE}       {\fontsize{1.6\FONTmain}{1.6\FONTmainbl}\selectfont}
\renewcommand{\huge}        {\fontsize{1.8\FONTmain}{1.8\FONTmainbl}\selectfont}

%%%%%%%%%%%%%%%%%%%%%%%%%%%%%%%%%%%%%%
\usepackage[table,RGB]{xcolor}

\usepackage{geometry}
\geometry{a5paper,top=15mm,bottom=15mm,left=15mm,right=15mm}
\setlength{\parindent}{0pt}


%\usepackage{extsizes}
\usepackage{multicol}

%%%%% math packages %%%%%%%%%%%%%%%%%%
\usepackage[intlimits]{amsmath}
\usepackage{amssymb}
\usepackage{amsfonts}
\usepackage{amsthm}
\usepackage{proof}
\usepackage{mathtools}
\usepackage{extarrows}

\usepackage[italicdiff]{physics}
\usepackage{siunitx}
\usepackage{xfrac}

%%%%%%% symbols packages %%%%%%%%%%%%%%
\usepackage{bm} %for use \bm instead \boldsymbol in math mode
\usepackage{dsfont}
%\usepackage{stmaryrd}
%%%%%%%%%%%%%%%%%%%%%%%%%%%%%%%%%%%%%%%


%%%%%% graphics %%%%%%%%%%%%%%%%%%%%%%%
\usepackage{graphicx}
%\usepackage{color}
%\usepackage{xypic}
%\usepackage[all]{xy}
%\usepackage{calc}

%%%%%% tables %%%%%%%%%%%%%%%%%%%%%%%%%
\usepackage{array}
\usepackage{booktabs}
\usepackage{multirow}
\usepackage{makecell}
\usepackage{minibox}
\usepackage{systeme}
%%%%%%%%%%%%%%%%%%%%%%%%%%%%%%%%%%%%%%%

\usepackage{enumitem}
\usepackage{tikz}
\usetikzlibrary{shapes,angles,calc,arrows,arrows.meta,quotes,intersections}
\usetikzlibrary{decorations.pathmorphing}
\usetikzlibrary{decorations.pathreplacing} 
\usetikzlibrary{decorations.markings,patterns} 
\usepackage{pgfplots}
\pgfplotsset{compat=1.15}

\tikzset{dot/.style={ draw, fill, circle, inner sep=1pt, minimum size=3pt }}
\usepackage{fancyhdr}
%%%%% header and footer rule %%%%%%%%%
%\setlength{\headheight}{14pt}
\renewcommand{\headrulewidth}{0pt}
\renewcommand{\footrulewidth}{0pt}
\fancypagestyle{plain}{\fancyhf{}\rfoot{\thepage}}
\fancypagestyle{vangelis}{\fancyhf{}
    \fancyfootoffset[LE,RO]{10mm}
    \rfoot[]{\thepage}
    \lfoot[\thepage]{}
    \rhead[]{\tikz[remember picture,overlay]{\node[rotate=90,anchor=east] (text) at ([shift={(-5mm,-8mm)}]current page.north east) {\textcolor{Col\thechapter}{\small\strut\leftmark}};
    \fill[Col\thechapter] ([xshift={-2.5mm}]text.east) rectangle++(5mm,5mm);
    }}
    %\lhead[\textcolor{Col\thechapter}{\leftmark}]{}
}
%%%%%%%%%%%%%%%%%%%%%%%%%%%%%%%%%%%%%%%

\usepackage[space]{grffile}


% \definecolor{Col1}{HTML}{eb3b79}
% \definecolor{Col2}{HTML}{9a529f}
% \definecolor{Col3}{HTML}{775ba6}
% \definecolor{Col4}{HTML}{5a68b0}
% \definecolor{Col5}{HTML}{55a0d8}
% \definecolor{Col6}{HTML}{34b0e5}
% \definecolor{Col7}{HTML}{34c1d7}
% \definecolor{Col8}{HTML}{65bc6a}
% \definecolor{Col9}{HTML}{9acb62}
% \definecolor{Col10}{HTML}{d1dd5b}
% \definecolor{Col11}{HTML}{f9ec5d}
% \definecolor{Col12}{HTML}{fbc82a}
% \definecolor{Col13}{HTML}{faa725}
% \definecolor{Col14}{HTML}{f26f47}
% \definecolor{Col15}{HTML}{8e6d65}
% \definecolor{Col16}{HTML}{bdbcbc}
% \definecolor{Col17}{HTML}{79919d}

\definecolor{Col1}{rgp}{0.74, 0.2, 0.64}
\definecolor{Col2}{rgp}{0.0, 0.55, 0.55}
\definecolor{Col3}{rgp}{0.74, 0.2, 0.64}
\definecolor{Col4}{rgp}{0.0, 0.55, 0.55}
\definecolor{Col5}{rgp}{0.74, 0.2, 0.64}
\definecolor{Col6}{rgp}{0.0, 0.55, 0.55}
\definecolor{Col7}{rgp}{0.74, 0.2, 0.64}
\definecolor{Col8}{rgp}{0.0, 0.55, 0.55}
\definecolor{Col9}{rgp}{0.74, 0.2, 0.64}
\definecolor{Col10}{rgp}{0.0, 0.55, 0.55}

\everymath{\displaystyle}

\usepackage[most]{tcolorbox}

\usepackage[explicit]{titlesec}
%%%%%% titlesec settings %%%%%%%%%%%%%
% \titleformat{ command }[ shape ]{ format }{ label }{ sep }{ before-code }[ after-code 
% \titlespacing*{ command }{ left }{ before-sep }{ after-sep }[ right-sep ]
% Chapter


% \titleformat{\chapter}[block]{\huge\bfseries}{\begin{tcolorbox}[colback=Col\thechapter,left=3pt,right=3pt,top=18pt,bottom=18pt,sharp
% corners,boxrule=0pt]\centering\huge\bfseries\textcolor{white}{#1}\end{tcolorbox}}{0pt}{\markboth{#1}}[\clearpage]
% \titlespacing*{\chapter}{0cm}{6\baselineskip}{0\baselineskip}[0ex]
% % Section
% \titleformat{\section}[hang]{\pagestyle{plain}\Large\bfseries\centering}{\begin{tcolorbox}[colback=Col\thechapter!75!white,left=1pt,right=1pt,top=2pt,bottom=2pt,sharp
% corners,boxrule=0pt]\centering\strut\textcolor{white}{#1}\end{tcolorbox}}{0ex}{}
% \titlespacing*{\section}{0cm}{2\baselineskip}{\baselineskip}[0ex]
% % subsection
% \titleformat{\subsection}[hang]{\pagestyle{plain}\large\bfseries\centering}{\begin{tcolorbox}[colback=Col\thechapter!55!white,left=1pt,right=1pt,top=2pt,bottom=2pt,sharp
% corners,boxrule=0pt]\centering\strut\textcolor{white}{#1}\end{tcolorbox}}{0ex}{}
% \titlespacing*{\section}{0cm}{2\baselineskip}{\baselineskip}[0ex]
% % Subsubsection
% \titleformat{\subsubsection}[hang]{\normalsize\bfseries\centering}{}{0ex}{\color{Col\thechapter!45}{#1}}{}
% \titlespacing*{\subsubsection}{0cm}{\baselineskip}{\baselineskip}[0ex]

%% Subsection
%\titleformat{\subsection}[hang]{\large\bfseries\centering}{\tcbox[colback=Col\thechapter!50!white,left=1pt,right=1pt,top=1pt,bottom=1pt,sharp corners]{#1}}{0ex}{}
%\titlespacing*{\subsection}{0cm}{2\baselineskip}{\baselineskip}[0ex]
%% Subsubsection
%\titleformat{\subsubsection}[hang]{\normalsize\bfseries\centering}{}{0ex}{\color{Col\thechapter}{#1}}{}
%\titlespacing*{\subsubsection}{0cm}{\baselineskip}{\baselineskip}[0ex]
%%%%%%%%%%%%%%%%%%%%%%%%%%%%%%%%%%%%%%%




\AtBeginDocument{\pagestyle{vangelis}\normalsize\raggedright}


\newcommand{\twocolumnside}[2]{\begin{minipage}[t]{0.45\linewidth}\raggedright
#1
\end{minipage}\hfill{\color{Col\thechapter}{\vrule width 1pt}}\hfill\begin{minipage}[t]{0.45\linewidth}\raggedright
#2
\end{minipage}
}

\newcommand{\twocolumnsides}[2]{\begin{minipage}[t]{0.45\linewidth}\raggedright
#1
\end{minipage}\hfill\begin{minipage}[t]{0.45\linewidth}\raggedright
#2
\end{minipage}
}

\newcommand{\twocolumnsidesc}[2]{\begin{minipage}{0.45\linewidth}\raggedright
#1
\end{minipage}\hfill\begin{minipage}[c]{0.45\linewidth}\raggedright
#2
\end{minipage}
}

\newcommand{\twocolumnsidesp}[2]{\begin{minipage}[t]{0.35\linewidth}\raggedright
#1
\end{minipage}\hfill\begin{minipage}[t]{0.55\linewidth}\raggedright
#2
\end{minipage}
}

\newcommand{\twocolumnsidesl}[2]{\begin{minipage}[t]{0.55\linewidth}\raggedright
#1
\end{minipage}\hfill\begin{minipage}[t]{0.35\linewidth}\raggedright
#2
\end{minipage}
}


\usepackage{calc}
\usepackage{array}
\definecolor{TabLine}{RGB}{254,254,254}
\newcommand{\TabRowHead}{\rowcolor{TabHeadRow}}
\newcommand{\TabRowHeadCor}{\cellcolor{white}}
\newcommand{\TabRowHCol}{\color{white}\bfseries\boldmath}
\newcommand{\TabCellHead}{\cellcolor{TabHeadRow}\TabRowHCol}
\newenvironment{Mytable}%
    {\begingroup\setlength{\arrayrulewidth}{2pt}\arrayrulecolor{TabLine}
    \colorlet{TabHeadRow}{Col\thechapter}
    \colorlet{TabRowOdd}{Col\thechapter!50!white}
    \colorlet{TabRowEven}{Col\thechapter!25!white}
    \rowcolors{1}{TabRowOdd}{TabRowEven}
    }%
    {\endgroup

}

\usepackage{fancyhdr}
%%%%% header and footer rule %%%%%%%%%
\setlength{\headheight}{14pt}
\renewcommand{\headrulewidth}{0pt}
\renewcommand{\footrulewidth}{0pt}
\fancypagestyle{plain}{\fancyhf{}
\fancyhead{}
\lfoot{\small \hrule \vspace{5pt}\color{Col1} Βαγγέλης Σαπουνάκης}
\cfoot{\small \hrule \vspace{5pt}\color{Col2!75} Φοιτητικό Πρόσημο}
\rfoot{\small \hrule \vspace{5pt} \thepage}}
\fancypagestyle{vangelis}{\fancyhf{}
\lfoot{\small \hrule \vspace{5pt}\color{Col1} Βαγγέλης Σαπουνάκης}
\cfoot{\small \hrule \vspace{5pt}\color{Col2!75} Φοιτητικό Πρόσημο}
\rfoot{\small \hrule \vspace{5pt} \thepage}}

%%%%%%%%%%%%Watermark%%%%%%%%%%%%%%%%%%
 \usepackage[printwatermark]{xwatermark} 
 \newwatermark[allpages,color=blue!8,angle=45,scale=3,xpos=0,ypos=0]{ΠΡΟΣΗΜΟ}
%%%%%%%%%%%%%%%%%%%%%%%%%%%%%%%%%%%%%%

%%%%%%%%% mdframed theorem boxes, breakable and with ref support %%%%%%%%

\usepackage[framemethod=TikZ]{mdframed}

\mdfdefinestyle{mythm}{innertopmargin=0pt,linecolor=Col2!75,linewidth=2pt,
  backgroundcolor=Col2!15, %background color of the box
  shadow=false,shadowcolor=Col2,shadowsize=5pt,% shadows
  frametitleaboveskip=\dimexpr-1.3\ht\strutbox\relax, 
  frametitlealignment={\hspace*{0.03\linewidth}},%
}

\mdfdefinestyle{mydfn}{innertopmargin=0pt,linecolor=Col1!75,linewidth=2pt,
  backgroundcolor=Col1!15, %background color of the box
  shadow=false,shadowcolor=Col1,shadowsize=5pt,% shadows
  frametitleaboveskip=\dimexpr-1.3\ht\strutbox\relax, 
  frametitlealignment={\hspace*{0.03\linewidth}},%
}

\mdfdefinestyle{myprop}{innertopmargin=0pt,linecolor=blue!75,linewidth=2pt,
  backgroundcolor=blue!10, %background color of the box
  shadow=false,shadowcolor=blue,shadowsize=5pt,% shadows
  frametitleaboveskip=\dimexpr-1.3\ht\strutbox\relax, 
  frametitlealignment={\hspace*{0.03\linewidth}},%
}

\mdfdefinestyle{myboxs}{innertopmargin=0pt,linecolor=blue!75,linewidth=0pt,
  backgroundcolor=blue!15, %background color of the box
  shadow=false,shadowcolor=blue,shadowsize=5pt,% shadows
}

% \newcounter{theo}[section]
% \setcounter{theo}{0}
% \renewcommand{\thetheo}{\arabic{section}.\arabic{theo}}


\newenvironment{mythm}[2][]{%
  \refstepcounter{thm}
  % Code for box design goes here.
  \ifstrempty{#1}%
    % if condition (without title)
    {\mdfsetup{
  frametitle={%
    \tikz[baseline=(current bounding box.east),outer sep=0pt]
    \node[anchor=east,rectangle,fill=Col2!75,text=white]
  {\strut Θεώρημα~\thethm};},%
      }%
      % else condition (with title)
      }{\mdfsetup{
  frametitle={%
    \tikz[baseline=(current bounding box.east),outer sep=0pt]
    \node[anchor=east,rectangle,fill=Col2!75,text=white]
  {\strut Θεώρημα~\thethm~~({#1})};},%
      }%
    }%
    % Both conditions
    \mdfsetup{
      style=mythm
    }
    \begin{mdframed}[]\relax\label{#2}}{%
  \end{mdframed}}
  %%%%%%%%%%%%%%%%%%%%%%%%%%%%%%%%%%%%%%%%%%%%%%%%%%%%%%%%%%

\newenvironment{mydfn}[2][]{%
  \refstepcounter{thm}
  % Code for box design goes here.
  \ifstrempty{#1}%
    % if condition (without title)
    {\mdfsetup{
  frametitle={%
    \tikz[baseline=(current bounding box.east),outer sep=0pt]
    \node[anchor=east,rectangle,fill=Col1!75,text=white,draw=Col1!75]
  {\strut Ορισμός~\thethm};},%
      }%
      % else condition (with title)
      }{\mdfsetup{
  frametitle={%
    \tikz[baseline=(current bounding box.east),outer sep=0pt]
    \node[anchor=east,rectangle,fill=Col1!75,text=white,draw=Col1!75]
  {\strut Ορισμός~\thethm~~({#1})};},%
      }%
    }%
    % Both conditions
    \mdfsetup{
      style=mydfn
    }
    \begin{mdframed}[]\relax\label{#2}}{%
  \end{mdframed}}
  %%%%%%%%%%%%%%%%%%%%%%%%%%%%%%%%%%%%%%%%%%%%%%%%%%%%%%%%%%

\newenvironment{myprop}[2][]{%
  \refstepcounter{thm}
  % Code for box design goes here.
  \ifstrempty{#1}%
    % if condition (without title)
    {\mdfsetup{
  frametitle={%
    \tikz[baseline=(current bounding box.east),outer sep=0pt]
    \node[anchor=east,rectangle,fill=blue!50,text=white]
  {\strut Πρόταση~\thethm};},%
      }%
      % else condition (with title)
      }{\mdfsetup{
  frametitle={%
    \tikz[baseline=(current bounding box.east),outer sep=0pt]
    \node[anchor=east,rectangle,fill=blue!50,text=white]
  {\strut Πρόταση~\thethm~~({#1})};},%
      }%
    }%
    % both conditions
    \mdfsetup{
      style=myprop
    }
    \begin{mdframed}[]\relax\label{#2}}{%
  \end{mdframed}}
  %%%%%%%%%%%%%%%%%%%%%%%%%%%%%%%%%%%%%%%%%%%%%%%%%%%%%%%%%%
\newenvironment{myboxs}{%
  % Code for box design goes here.
    \mdfsetup{
      style=myboxs
    }
    \begin{mdframed}[]\relax}{%
\end{mdframed}}
  %%%%%%%%%%%%%%%%%%%%%%%%%%%%%%%%%%%%%%%%%%%%%%%%%%%%%%%%%%
% \renewcommand{\qedsymbol}{$\blacksquare$}

\newcommand{\comb}[2]{\lambda_{1}\vec{#1}_{1} + \cdots + \lambda_{#2}\vec{#1}_{#2}}
\newcommand{\combc}[3]{#2_{1}\vec{#1}_{1} + \cdots + #2_{#3}\vec{#1}_{#3}}
\newcommand{\combb}[2]{\lambda_{1}\vec{#1}_{1} + \lambda_{2}\vec{#1}_{2} + \cdots + 
\lambda_{#2}\vec{#1}_{#2}}

\newcommand{\me}{\mathrm{e}}


\newlist{myitemize}{itemize}{3}
\setlist[myitemize]{label=\textcolor{Col1}{\tiny$\blacksquare$},leftmargin=*}

\newlist{myitemize*}{itemize*}{3}
\setlist[myitemize*]{itemjoin=\hspace{2\baselineskip},label=\textcolor{Col1}{\tiny$\blacksquare$}}

\newlist{myenumerate}{enumerate}{3}
\setlist[enumerate,1]{label=\textcolor{Col1}{\theenumi.},leftmargin=*}
\setlist[enumerate,2]{label=\textcolor{Col1}{\roman*)},leftmargin=*}

\setlist[description]{labelindent=1em,widest=Ιανουα0000,labelsep*=1em,itemindent=0pt,leftmargin=*}

% %%%%%%%%%%%%%%%%%% fancy headings %%%%%%%%%%%%%%%%%%

% %%%%%%%%%%%%%%%%%%%%%%% my boxes %%%%%%%%%%%%%%%%%%%%%%%%%%%%
% \newcommand{\mythm}[1]{
%       \refstepcounter{thm}
%     \begin{tikzpicture}
%         \node[myboxthm] (box1) 
%         {
%             \begin{minipage}{0.9\textwidth}
%                 #1
%             \end{minipage}
%         } ;

%         \node[myboxtitlethm] at (box1.north west) {\strut Θεώρημα~\thethm} ;
%     \end{tikzpicture}
% }

% \newcommand{\mythmm}[2]{
%       \refstepcounter{thm}
%     \begin{tikzpicture}
%         \node[myboxthm] (box1) 
%         {
%             \begin{minipage}{0.9\textwidth}
%                 #2
%             \end{minipage}
%         } ;

%         \node[myboxtitlethm] at (box1.north west) {\strut Θεώρημα~\thethm \; (#1)} ;
%     \end{tikzpicture}
% }

% \newcommand{\mydfn}[1]{
%       \refstepcounter{dfn}
%     \begin{tikzpicture}
%         \node[myboxdfn] (box1) 
%         {
%             \begin{minipage}{0.9\textwidth}
%                 #1
%             \end{minipage}
%         } ;

%         \node[myboxtitledfn] at (box1.north west) {\strut Ορισμός~\thedfn} ;
%     \end{tikzpicture}
% }


% \newcommand{\myprop}[1]{
%       \refstepcounter{thm}
%     \begin{tikzpicture}
%         \node[myboxprop] (box1) 
%         {
%             \begin{minipage}{0.9\textwidth}
%                 #1
%             \end{minipage}
%         } ;

%         \node[myboxtitleprop] at (box1.north west) {\strut Πρόταση~\theprop} ;
%     \end{tikzpicture}
% }

% \newcommand{\mypropp}[2]{
%       \refstepcounter{thm}
%     \begin{tikzpicture}
%         \node[myboxprop] (box1) 
%         {
%             \begin{minipage}{0.9\textwidth}
%                 #2
%             \end{minipage}
%         } ;

%         \node[myboxtitleprop] at (box1.north west) {\strut Πρόταση~\theprop (#1)} ;
%     \end{tikzpicture}
% }

%%%\mybrace{<first>}{<second>}[<Optional text>]
\newcommand{\tikzmark}[1]{\tikz[baseline={(#1.base)},overlay,remember picture] \node[outer
sep=0pt, inner sep=0pt] (#1) {\phantom{A}};}
%% syntax
\NewDocumentCommand\mybrace{mmo}{%
  \IfValueTF {#3}{%
    \begin{tikzpicture}[overlay, remember picture,decoration={brace,amplitude=1ex}]
      \draw[decorate,thick] (#1.north east) -- (#2.south east) 
        node (b) [midway,xshift=13pt,label={right=of b}:{#3}] {};
    \end{tikzpicture}%
  }%
  {%
    \begin{tikzpicture}[overlay, remember picture,decoration={brace,amplitude=1ex}]
      \draw[decorate,thick] (#1.north east) -- (#2.south east);
    \end{tikzpicture}%
  }%
}%

%%%%%%%How to use this %%%%%%%%%%%%%%%%%%%%%%%%%
%use \tikzmark{a} and \tikzmark{b} at first and last \item where the brace is
%wanted
%use the following command after \end{enumerate}
%\mybrace{a}{b}[Text comes here to describe these to items and justify for your
%case]]



%%%%% label inline equations and don't allow reference
\newcommand\inlineeqno{\stepcounter{equation} (\theequation)}

%%%%%defines \inlineequation[<label name>]{<equation>}
%%%%%%%%format use \inlineequation[<label name>]{<equation>}%%%%%%%
\makeatletter
\newcommand*{\inlineequation}[2][]{%
    \begingroup
    % Put \refstepcounter at the beginning, because
    % package `hyperref' sets the anchor here.
    \refstepcounter{equation}%
    \ifx\\#1\\%
\else
    \label{#1}%
\fi
% prevent line breaks inside equation
\relpenalty=10000 %
\binoppenalty=10000 %
\ensuremath{%
    % \displaystyle % larger fractions, ...
    #2%
}%
\quad ~\@eqnnum
\endgroup
}
\makeatother


%%%%%%%%%%%%%%%%%% fancy enumitem cicled label %%%%%%%%%%%%%%%%%%
\newcommand*\circled[1]{\tikz[baseline=(char.base)]{
\node[shape=circle,draw,inner sep=0.3pt] (char) {#1};}}
% use it like \begin{enumerate}[label=\protect\circled{\Alph{enumi}}]
%%%\mybrace{<first>}{<second>}[<Optional text>]
%%% wrap with braces list environments




%%%%%%%%%%%%% puts brace under matrix
\newcommand\undermat[2]{%
  \makebox[0pt][l]{$\smash{\underbrace{\phantom{%
\begin{matrix}#2\end{matrix}}}_{\text{$#1$}}}$}#2}


%circle item inside array or matrix
\newcommand\Circle[1]{%
\tikz[baseline=(char.base)]\node[circle,draw,inner sep=2pt] (char) {#1};}


  %redeftine \eqref so that parenthesis () have the color the link
\makeatletter
\renewcommand*{\eqref}[1]{%
  \hyperref[{#1}]{\textup{\tagform@{\ref*{#1}}}}%
}
\makeatother

%removes qedsymbol and additional vertical space at the end 
\makeatletter
\renewenvironment{proof}[1][\proofname]{\par
  % \pushQED{\hfill\qedhere}% <--- remove the QED business
  \normalfont \topsep6\p@\@plus6\p@\relax
  \trivlist
  \item[\hskip\labelsep
        \itshape
        #1\@addpunct{.}]\ignorespaces
}{%
 % \popQED% <--- remove the QED business
  \endtrivlist\@endpefalse
}
\renewcommand\qedhere{$\blacksquare$} % to ensure code portability
\makeatother


\input{tikz.tex}
\input{myboxes.tex}

\pagestyle{askhseis}

\setcounter{chapter}{2}
\begin{document}

\chapter{Σειρές}



\begin{mybox1}
\begin{dfn}
Έστω $ {(a_{n})}_{n \in \mathbb{N}}$ ακολουθία. Τότε ορίζουμε την ακολουϑία 
\begin{align*}
    S_{1} &= a_{1} \\
    S_{2} &= a_{1}+ a_{2} \\
    \vdots \\
    S_{N} &= a_{1}+ a_{2}+ \cdots + a_{N} \\
    \vdots 
\end{align*}

Η ακολουθία $ {(S_{N})}_{N \in \mathbb{N}} $ της οποίας οι όροι δίνονται από την 
σχέση \[ S_{N} = \sum_{n=1}^{N} a_{n}, \; \forall N \in \mathbb{N} \] ονομάζεται 
\textcolor{Col2}{σειρά} και συμβολίζεται
$
    \sum_{n=1}^{\infty} a_{n} 
$
\end{dfn}
\end{mybox1}

\begin{rem}
Οι όροι της ακολουθίας $ {(S_{N})}_{N \in \mathbb{N}} $ ονομάζονται 
\textcolor{Col2}{μερικά αθροίσματα} της σειράς.
\end{rem}

\section{Παραδείγματα}

\begin{enumerate}
    \item $ \sum_{n=1}^{\infty} \frac{1}{n} $, είναι η ακολουθία 
        $ {(S_{n})}_{n \in \mathbb{N}} $, όπου $ S_{1}=1, \; S_{2}=1+ \frac{1}{2}, \; 
        S_{3}= 1 + \frac{1}{2} + \frac{1}{3}, \ldots  $

    \item $ \sum_{n=1}^{\infty} n  $, είναι η ακολουθία ${(S_{n})}_{n \in \mathbb{N}}$,
        όπου $ S_{1}=1, \; S_{2}=1+2, \; S_{3}=1+2+3, \ldots $ η οποία προφανώς 
        απειρίζεται θετικά

    \item $ \sum_{n=1}^{\infty} 1  $, είναι η ακολουθία ${(S_{n})}_{n \in \mathbb{N}}$,
        όπου $ S_{1}=1, \; S_{2}=1+1, \; S_{3}=1+1+1, \ldots $ η οποία προφανώς 
        απειρίζεται θετικά.

    \item $ \sum_{n=1}^{\infty} {(-1)}^{n}  $, είναι η ακολουθία 
        $ {(S_{n})}_{n \in \mathbb{N}} $, όπου $ S_{1}=-1, \; S_{2}=0, \; S_{3}=-1, 
        \ldots $, η οποία προφανώς δεν συγκλίνει, γιατί έχει δύο διαφορετικές, σταθερές 
        συγκλίνουσες υπακολουθίες.
\end{enumerate}

\begin{rems}
\item {}
    \begin{myitemize}
    \item Η σειρά $ \sum_{n=1}^{\infty} \frac{1}{n}  $, λέγεται 
        \textcolor{Col2}{αρμονική σειρά} και αποκλίνει. (απειρίζεται θετικά)
    \item Η σειρά $ \sum_{n=1}^{\infty} \frac{1}{n^{\rho}}  $, λέγεται 
        \textcolor{Col2}{γενικευμένη αρμονική σειρά} και συγκλίνει αν και μόνον αν 
        $ \rho > 1 $.
    \end{myitemize}
\end{rems}

\section{Σύγκλιση Σειρών}

\begin{mybox1}
\begin{dfn}
Έστω $ \sum_{n=1}^{\infty} a_{n}  $ σειρά και έστω $ {(S_{n})}_{n \in \mathbb{N}} $ 
η ακολουθία των μερικών αθροισμάτων της. 

Λέμε ότι η σειρά \textcolor{Col2}{συγκλίνει} στον πραγματικό αριθμό 
$ S \in \mathbb{R} $, ο οποίος 
τότε καλείται \textcolor{Col2}{άθροισμα} της σειράς, αν η ακολουθία των 
μερικών αθροισμάτων της συγκλίνει στον $ S $. Δηλαδή
\[
    \sum_{n=1}^{\infty} a_{n} = S \Leftrightarrow \lim_{n \to \infty} S_{n} = S  
\] 
\end{dfn}
\end{mybox1}

\begin{rem}
\item {}
  \begin{myitemize}
    \item Αν $ \lim_{n \to \infty} S_{n} = + \infty $, τότε λέμε ότι η σειρά
      \textcolor{Col1}{απειρίζεται
      θετικά} και γράφουμε $ \sum_{n=1}^{\infty} a_{n} = + \infty $.
    \item Αν $ \lim_{n \to \infty} S_{n} = - \infty $, τότε λέμε ότι η σειρά
      \textcolor{Col1}{απειρίζεται
      αρνητικά} και γράφουμε $ \sum_{n=1}^{\infty} a_{n} = - \infty $.
    \item Αν δεν υπάρχει το $ \lim_{n \to \infty} S_{n} $, ή $ \lim_{n \to \infty} S_{n}=
      \pm \infty$, τότε λέμε ότι η σειρά \textcolor{Col1}{αποκλίνει}.

  \end{myitemize}
\end{rem}

\begin{example}
  Να αποδείξετε ότι η σειρά $ \sum_{n=1}^{\infty} (-1)^{n} $ αποκλίνει.
\end{example}
  \begin{proof}
    Θεωρούμε την ακολουθία $ {(S_{n})}_{n=2}^{\infty} $, όπου $ S_{n}= a_{2}+ a_{3} +
    \cdots a_{n}, \; \forall n \in \mathbb{N} $ και έχουμε
  \begin{align*}
    S_{2} &= a_{2} = (-1)^{2}=1 \\
    S_{3} &= a_{2}+ a_{3} = 1 + (-1)^{3} = 0 \\
    S_{4} &= a_{2}+ a_{3} + a_{4} = 0 + (-1)^{4} = 1 \\
    &\vdots \\
    S_{n} &= 
    \begin{cases}
      1,  &n \; \text{άρτιος} \\
      0,  &n \; \text{περιττός} 
    \end{cases}
   \end{align*} 
 Οπότε, έχουμε δύο διαφορετικές ακολουθίες, της $ {(S_{n})}_{n=2}^{\infty} $ 
 που συγκλίνουν σε διαφορετικό αριθμό, επομένως, δεν υπάρχει το 
 $ \lim_{n \to \infty} S_{n} $, άρα η σειρά $ \sum_{n=1}^{\infty} (-1)^{n} $ αποκλίνει.
  \end{proof}

\begin{example}
  Να αποδείξετε ότι η σειρά $ \sum_{n=1}^{\infty} \frac{1}{3^{n}} $ συγκλίνει.
\end{example}
\begin{proof}
  Θεωρούμε την ακολουθία $ {(S_{n})}_{n \in \mathbb{N}} $, όπου $ S_{n}= a_{1}+ a_{2} +
  \cdots a_{n}, \; \forall n \in \mathbb{N} $. 
  Παρατηρούμε ότι ο $ S_{n} $ είναι το άθροισμα των $n$ πρώτων όρων της γεωμετρικής 
  προόδου
  $ (\frac{1}{3^{n}})_{n \in \mathbb{N}} $, άρα όπως είναι γνωστό, έχουμε
  \[
    S_{n} = a_{1} \cdot \frac{\lambda ^{n}-1}{\lambda -1} = 
    \frac{1}{3} \cdot \frac{(\frac{1}{3} )^{n}-1}{\frac{1}{3} - 1} 
    = \frac{1}{\cancel{3}} \cdot \frac{\frac{1}{3^{n}} - 1}{- \frac{2}{\cancel{3}}} 
    = \frac{1}{2} \cdot \left(1 - \frac{1}{3^{n}}\right)
  \] 
  Άρα
  \[ 
    \lim_{n \to \infty} S_{n} = \lim_{n \to \infty} \frac{1}{2} \cdot 
    \left(1 - \frac{1}{3^{n}}\right) = \frac{1}{2} \cdot \lim_{n \to \infty} 
    \left(1 - \frac{1}{3^{n}}\right) = \frac{1}{2} \cdot (1-0) = \frac{1}{2}
  \]
  Επομένως, η σειρά συγκλίνει και μάλιστα 
  $ \sum_{n=1}^{\infty} \frac{1}{3^{n}} = \frac{1}{2} $
\end{proof}

\begin{mybox3}
\begin{prop}
  \begin{enumerate}[i)]
    \mbox{}
    \item $ \sum_{n=1}^{\infty} a_{n}  $ συγκλίνει 
      $ \Leftrightarrow \sum_{n= n_{0}}^{\infty} a_{n}, \; \; \forall n_{0} 
      \in \mathbb{N}  $ συγκλίνει.
    \item\label{prop:diff2} $ \sum_{n=1}^{\infty} a_{n}  $ συγκλίνει 
      $ \Rightarrow \sum_{n=1}^{\infty} a_{n} - \sum_{n= n_{0}}^{\infty} a_{n}  
      = \sum_{n=1}^{n_{0}-1} a_{n} $
  \end{enumerate}
  \label{prop:diff}
\end{prop}
\end{mybox3}
\begin{proof}
\item {}
    \begin{enumerate}
        \item Θέτουμε $ S_{N} = \sum_{n=1}^{N} a_{n}, \; 
            \forall N \in \mathbb{N} $ και $ T_{N} = \sum_{n= n_{0}}^{N}, \; N= n_{0}, 
            n_{0}+1, \ldots $. Τότε για $ n \geq n_{0} $ έχουμε:
            \[
                S_{N}-T_{N} = a_{1}+ a_{2}+ a_{3}+ \cdots + a_{n_0-1} 
            \] 
            Δηλαδή οι ακολουθίες $ (S_{N})_{N \in \mathbb{N}} $ και 
            $ (T_{N})_{n \in \mathbb{N}} $ διαφέρουν κατά σταθερά. Σύμφωνα 
            με τις ιδιότητες των ορίων, αν συγκλίνει η μία τότε συγκλίνει και η άλλη.

        \item Ας υποθέσουμε ότι η σειρά $ \sum_{n=1}^{\infty} $ συγκλίνει, 
            δηλαδή ότι η ακολουθία $ (S_{N})_{N \in \mathbb{N}} $ συγκλίνει. 

            Τότε σύμφωνα με το πρώτο ερώτημα και η $ (T_{N})_{N \in \mathbb{N}} $
            συγκλίνει και μάλιστα $ \lim_{N \to \infty} S_{N} = \lim_{N \to \infty} 
            T_{N} + a_{1}+ a_{2} + \cdots + a_{n_{0}-1} 
            \Rightarrow \sum_{n=1}^{\infty} a_{n}- \sum_{n= n_{0}}^{\infty} a_{n} = 
            \sum_{n=1}^{n_{0}-1} a_{n}    $
    \end{enumerate}
\end{proof}

\begin{rem}
  Σύμφωνα με την  πρόταση~\ref{prop:diff}, η πρόσθεση ή η αφαίρεση πεπερασμένου 
  πλήθους όρων, σε μία συγκλίνουσα σειρά, δεν επηρεάζει τη σύγκλισή της. Προσοχή, όμως,
  γιατί επηρεάζει το άθροισμά της.
\end{rem}

\begin{mybox3}
\begin{prop}
  Έστω $ \sum_{n=1}^{\infty} a_{n} $ σειρά πραγματικών αριθμών και έστω $ a_{n} \geq 0,
  \; \forall n \in \mathbb{N} $. Τότε $ \sum_{n=1}^{\infty} a_{n} = S \in \mathbb{R} $ 
  ή $ \sum_{n=1}^{\infty} = + \infty $.
\end{prop}
\end{mybox3}
\begin{proof}
  Θεωρούμε την ακολουθία των μερικών αθροισμάτων $ (S_{n})_{n \in \mathbb{N}} $, 
  όπου $ S_{n} = a_{1}+ \cdots + a_{n} $. Παρατηρούμε ότι
  \[
    S_{n} = a_{1}+ \cdots + a_{n} \leq a_{1}+ \cdots + a_{n}+ a_{n+1} = S_{n+1} , 
    \; \forall n \in \mathbb{N}
   \] 
   Άρα η ακολουθία $ (S_{n})_{n \in \mathbb{N}} $ είναι αύξουσα. Υπάρχουν δύο
   περιπτώσεις: 
   \begin{description}
     \item [1η Περίπτωση:] $ {(S_{n})}_{n \in \mathbb{N}} $ μη φραγμένη, οπότε 
       $ \sum_{n=1}^{\infty} a_{n} = \lim_{n \to \infty} S_{n} = + \infty $ 
     \item [2η Περίπτωση:] $ {(S_{n})}_{n \in \mathbb{N}} $ φραγμένη, οπότε
       $ \sum_{n=1}^{\infty} a_{n} = \lim_{n \to \infty} S_{n} = S \in \mathbb{R} $,
       όπου $ S = \sup \{ S_{n} \} \in \mathbb{R} $.
   \end{description}
\end{proof}

\begin{rem}
    Μια σειρά στην οποία δεν γνωρίζουμε τα πρόσημα των όρων της, μπορεί:
    \begin{enumerate}[i)]
        \item να συγκλίνει
        \item να αποκλίνει στο $ + \infty $ ή στο $ - \infty $
        \item τίποτε από τα παραπάνω.
    \end{enumerate}
\end{rem}


\section{Τηλεσκοπικές Σειρές}

\begin{mybox1}
\begin{dfn}
    Οι σειρές $ \sum_{n=1}^{\infty} a_{n} $ των οποίων 
    οι όροι γράφονται στη μορφή 
    $ a_{n} = b_{n} - b_{n+1}, \; \forall n \in \mathbb{N} $ όπου για την ακολουθία
    $ {(b_{n})}_{n \in \mathbb{N}}$, μπορεί να υπολογιστεί το όριο της, λέγονται 
    τηλεσκοπικές.
\end{dfn}
\end{mybox1}

\begin{examples}
\item {}
    \begin{enumerate}
        \item Να υπολογιστεί το άθροισμα της σειράς $ \sum_{n=1}^{\infty} 
            \frac{1}{n(n+1)} $.
            \begin{proof}
            \item 
                Έχουμε
                \[
                    a_{n} = \frac{1}{n(n+1)} = \frac{A}{n} + \frac{B}{n+1} = 
                    \frac{1}{n} - \frac{1}{n+1} 
                \]
                επομένως η σειρά γράφεται
                \[
                    \sum_{n=1}^{\infty} \frac{1}{n(n+1)} = 
                    \sum_{n=1}^{\infty} \left(\frac{1}{n} - \frac{1}{n+1}\right) 
                \] 
                Υπολογίζουμε την ακολουθία των μερικών αθροισμάτων και έχουμε
                \begin{align*}
                    S_{1} &= 1- \frac{1}{2} \\
                    S_{2} &= 1- \frac{1}{2} + \frac{1}{2} - \frac{1}{3} = 
                    1 - \frac{1}{3}  \\
                    S_{3} &=  1- \frac{1}{2} + \frac{1}{2} - \frac{1}{3} + 
                    \frac{1}{3} - \frac{1}{4} = 1 - \frac{1}{4}  \\
                    \vdots \\
                    S_{n} &= 1 - \frac{1}{n+1} \\
                    \vdots
                \end{align*}
            Επομένως $ \lim_{n \to \infty} S_{n} = 
            \lim_{n \to \infty} \left(1 - \frac{1}{n+1}\right) = 1 - 0 = 1 $
        \end{proof}

        \item Να υπολογιστεί το άθροισμα της σειράς $ \sum_{n=1}^{\infty} 
            \ln{\left(\frac{n}{n+1}\right)} $
            \begin{proof}
            \item {}
                Έχουμε
                \[
                    a_{n} = \ln{\left(\frac{n}{n+1}\right)} = \ln{n} - \ln{(n+1)}  
                \]
                επομένως η σειρά γράφεται
                \[
                    \sum_{n=1}^{\infty} \ln{\left(\frac{n}{n+1} \right)} = 
                    \sum_{n=1}^{\infty} [ \ln{n} - \ln{(n+1)} ]
                \] 
                Υπολογίζουμε την ακολουθία των μερικών αθροισμάτων και έχουμε
                \begin{align*}
                  S_{1} &= \ln{1} - \ln{2} = - \ln{2}  \\
                  S_{2} &= \ln{1} - \cancel{\ln{2}} + \cancel{\ln{2}} - \ln{3} 
                  = - \ln{3}  \\
                  S_{3} &= \ln{1} - \cancel{\ln{2}} + \cancel{\ln{2}} -
                  \bcancel{\ln{3}} + \bcancel{\ln{3}} - \ln{4} = - \ln{4} \\
                  \vdots \\
                  S_{n} &= - \ln{(n+1)} \\
                  \vdots
                \end{align*}
                επομένως $ \lim_{n \to \infty} S_{n} = 
                \lim_{n \to \infty} (- \ln{(n+1)}) = - \infty $ οπότε η σειρά αποκλίνει.
            \end{proof}
    \end{enumerate}
\end{examples}


\begin{example}
    Να υπολογιστεί το άθροισμα της σειράς $ \sum_{n=3}^{\infty} \frac{1}{n(n+1)} $.
\end{example}
\begin{proof}
  Σύμφωνα με την πρόταση~\ref{prop:diff}, έχουμε:
    \[
        \sum_{n=3}^{\infty} \frac{1}{n(n+1)} = \sum_{n=1}^{\infty} 
        \frac{1}{n(n+1)} - \sum_{n=1}^{2} \frac{1}{n(n+1)} = 1 - \left(\frac{1}{2} + 
        \frac{1}{6}\right) = \frac{2}{6} = \frac{1}{3}
    \] 
\end{proof}


\section{Γεωμετρική Σειρά}

Η σειρά $ \sum_{n=0}^{\infty} x^{n} $ λέγεται \textcolor{Col1}{Γεωμετρική} και ο αριθμός
$x$ λέγεται \textcolor{Col1}{λόγος} της.

\begin{mybox3}
\begin{prop}
Η σειρά $ \sum_{n=0}^{\infty} x^{n} $ συγκλίνει $ \Leftrightarrow \abs{x} < 1 $
και ισχύει ότι $ \sum_{n=0}^{\infty} x^{n} = \frac{1}{1-x} $
\end{prop}
\end{mybox3}
\begin{proof}
\item {}
    Ισχύει ότι $ S_{N} = \sum_{n=0}^{N} x^{n} =  1 + x + x^{2} + \cdots x^{N} = 
    \frac{1- x^{n+1}}{1-x} $.

    Όμως $ \lim_{n \to \infty} x^{n} = 0 \Leftrightarrow \abs{x} < 1 $, οπότε 
    $ \lim_{n \to \infty} S_{N} = \frac{1}{1-x} $.
\end{proof}

\begin{mybox3}
\begin{cor}
    Η σειρά $ \sum_{n= n_{0}}^{\infty} x^{n} $ συγκλίνει $ \Leftrightarrow \abs{x} <1 $
    και ισχύει ότι $ \sum_{n= n_{0}}^{\infty} x^{n} = \frac{x^{n_{0}}}{1-x} $.
\end{cor}
\end{mybox3}
\begin{proof}
\item {}
    Αν $ \abs{x} < 1 $ τότε η σειρά $ \sum_{n=0}^{\infty} x^{n} $ συγκλίνει ως 
    γεωμετρική και ισχύει $ \sum_{n=0}^{\infty} x^{n} = \frac{1}{1-x} $. 

    Οπότε έχουμε
    \begin{align*}
        \sum_{n=0}^{\infty} x^{n} - \sum_{n= n_{0}}^{\infty} x^{n} = 
        \sum_{n= 0}^{n_{0} -1} x^{n} = 1 + x + x^{2} + \cdots + x^{n_{0}-1} 
        \Rightarrow \\
        \sum_{n= n_{0}}^{\infty} x^{n} = \sum_{n=0}^{\infty} x^{n} - (1 + x + x^{2} + 
        \cdots + x^{n_{0}-1}) = \frac{1}{1-x} - \frac{1- x^{n_{0}}}{1-x} = 
        \frac{x^{n_{0}}}{1-x}  
    \end{align*} 
\end{proof}

\begin{examples}
\item Η σειρά $ \sum_{n=0}^{\infty} \frac{1}{2^{n}} = \sum_{n=0}^{\infty} 
    {\left(\frac{1}{2} \right)}^{n} $ είναι Γεωμετρική με λόγο $ x = \frac{1}{2} $ 
    και επειδή 
    $ \abs{x} = \abs{\frac{1}{2}} = \frac{1}{2} < 1  $, ισχύει ότι η σειρά συγκλίνει.
    Έχουμε
    \[
        \sum_{n=0}^{\infty} {\left(\frac{1}{2} \right)}^{n} = 
        \frac{1}{1 - \frac{1}{2}} = 1
    \] 

\item Η σειρά $ \sum_{n=0}^{\infty} \frac{9^{n}}{10^{n+1}} = 
    \sum_{n=0}^{\infty} \frac{9^{n}}{10 \cdot 10^{n}} = 
    \frac{1}{10} \sum_{n=0}^{\infty} {\left(\frac{9}{10} \right)}^{n} $ είναι
    Γεωμετρική με λόγο $ x = \frac{9}{10} $ και επειδή $ \abs{x} = 
    \abs{\frac{9}{10}} = \frac{9}{10} < 1$, ισχύει ότι η σειρά συγκλίνει.
    Έχουμε 
    \[
        \sum_{n=0}^{\infty}  \frac{9^{n}}{10^{n+1}} = 
        \frac{1}{10} \sum_{n=0}^{\infty} {\left(\frac{9}{10} \right)}^{n} = 
        \frac{1}{10} \cdot \frac{1}{1 - \frac{9}{10}} = \frac{1}{10} \cdot 10 = 1
    \]
\end{examples}

\begin{mybox3}
\begin{prop}
  Έστω $ \sum_{n=1}^{\infty} a_{n} = a $ και $ \sum_{n=1}^{\infty} b_{n} = b $ 
  συγκλίνουσες σειρές. Τότε:
  \begin{enumerate}[i)]
    \item $ \sum_{n=1}^{\infty} (a_{n}+b_{n}) = a+b $
    \item $ \sum_{n=1}^{\infty} (k\cdot a_{n}) = k\cdot a $ 
  \end{enumerate} \label{prop:sum1}
\end{prop}
\end{mybox3}

\begin{example}
    \begin{enumerate}
        \item Να βρεθεί το άθροισμα της σειράς $ \sum_{n=5}^{\infty} 
            \frac{2^{2n}+5^{n}}{3^{3n}} $
            \begin{proof}
            \item {} 
                Έχουμε
                \begin{align*}
                    \sum_{n=5}^{\infty} \frac{2^{2n}+5^{n}}{3^{3n}} = 
                    \sum_{n=5}^{\infty} \left( \frac{2^{2n}}{3^{3n}} + 
                    \frac{5^{n}}{3^{3n}}\right) = \sum_{n=5}^{\infty} 
                    \left[{\left(\frac{4}{27} \right)}^{n} +
                    {\left(\frac{5}{27} \right)}^{n}\right]
                \end{align*}
                Όμως
                \begin{align*}
                    \sum_{n=5}^{\infty} {\left(\frac{4}{27} \right)}^{n} 
                    &= \frac{4^{5}}{27^{5}} \cdot 
                    \frac{1}{1- \frac{4}{27}} = \frac{4^{5}}{27^{5}} \cdot 
                    \frac{1}{27-4} = \frac{4^{5}}{27^{4}\cdot 23} 
                    \intertext{και}
                    \sum_{n=5}^{\infty} \left(\frac{5}{27} \right)^{n} 
                    &= \frac{5^{5}}{27^{5}} \cdot \frac{1}{1 - \frac{5}{27}} = 
                    \frac{5^{5}}{27^{4}} \cdot \frac{1}{27-5} = 
                    \frac{5^{5}}{27^{4}\cdot 22} 
                \end{align*}
                Επειδή και οι δύο σειρές συγκλίνουν τότε από την προηγούμενη 
                πρόταση έχουμε ότι 
                \[
                    \sum_{n=5}^{\infty} \frac{2^{2n}+5^{n}}{3^{3n}} =  
                    \frac{4^{5}}{27^{4}\cdot 23} + \frac{5^{5}}{27^{4}\cdot 22} 
                \] 
            \end{proof}

        \item Να υπολογιστεί το άθροισμα της σειράς $ \sum_{n=1}^{\infty} 
            \frac{3^{n-1}-1}{6^{n-1}} $.
            \begin{proof}
            \item {}
                Έχουμε 
                \[ 
                    \sum_{n=1}^{\infty} \frac{3^{n-1}-1}{6^{n-1}} = \sum_{n=1}^{\infty} 
                    \left(\frac{3^{n-1}}{6^{n-1}} - \frac{1}{6^{n-1}}\right) = 
                    \sum_{n=1}^{\infty} \left[\left(\frac{3}{6} \right)^{n-1} -
                    \left( \frac{1}{6}\right) ^{n-1}\right] = \sum_{n=1}^{\infty} 
                    \left[\left(\frac{1}{2} \right)^{n-1} - \left(\frac{1}{3} 
                    \right)^{n-1}\right]
                \]
                Όμως 
                \begin{align*}
                    \sum_{n=1}^{\infty} \left(\frac{1}{2} \right)^{n-1} 
                    = \frac{1}{1- \frac{1}{2}} = 2
                    \quad \text{και} \quad
                    \sum_{n=1}^{\infty} \left(\frac{1}{6} \right)^{n-1} 
                    = \frac{1}{1- \frac{1}{6}} = \frac{6}{5} 
                \end{align*} 
                Επειδή και οι δύο σειρές συγκλίνουν τότε από την προηγούμενη 
                πρόταση έχουμε ότι 
                \[
                    \sum_{n=1}^{\infty} \frac{3^{n_1}-1}{6^{n-1}} =  
                    \sum_{n=1}^{\infty} \left[\left(\frac{1}{2} \right)^{n_1} - 
                    \left(\frac{1}{6} \right)^{n-1}\right] = 2 - \frac{6}{5} = 
                    \frac{4}{5} 
                \] 
            \end{proof}
    \end{enumerate}
\end{example}

\begin{mybox3}
\begin{prop}
  Αν $ \sum_{n=1}^{\infty} a_{n} $ συγκλίνει και $ \sum_{n=1}^{\infty} b_{n} $ 
  αποκλίνει, τότε η σειρά $ \sum_{n=1}^{\infty} (a_{n}+b_{n}) $ αποκλίνει.
\end{prop}
\end{mybox3}
\begin{proof}(Με άτοπο)

  Έστω ότι η $ \sum_{n=1}^{\infty} (a_{n}+b_{n}) $ συγκλίνει. Τότε, επειδή $
  \sum_{n=1}^{\infty} - a_{n} $ επίσης συγκλίνει, θα έχουμε, από την
  πρόταση~\ref{prop:sum1}, ότι $ \sum_{n=1}^{\infty} (a_{n}+b_{n} - a_{n}) = 
  \sum_{n=1}^{\infty} b_{n} $ συγκλίνει, άτοπο.
\end{proof}


\begin{rem}
    Έστω ότι $ \sum_{n=1}^{\infty} a_{n} = a $ και $ \sum_{n=1}^{\infty} b_{n} = b $, 
    με $ b_{n} \neq 0, \; \forall n \in \mathbb{N} $ και $ b \neq 0 $. Τότε η σειρά 
    $ \sum_{n=1}^{\infty} \frac{a_{n}}{b_{n}} $ μπορεί 
    \begin{enumerate}[i)]
        \item Να αποκλίνει 
        \item Να συγκλίνει, και μάλιστα $ \sum_{n=1}^{\infty} \frac{a_{n}}{b_{n}} \neq 
            \frac{a}{b} $
    \end{enumerate}
    \begin{proof}[Παράδειγματα]
    \item {}
        \begin{enumerate}[i)]
            \item Έστω η σειρά $ \sum_{n=1}^{\infty} 
                \underbrace{\frac{1}{n(n+1)}}_{a_{n}} = 1 $ και 
                $ \sum_{n=1}^{\infty} \underbrace{\frac{1}{(n+1)(n+2)}}_{b_{n}} = 
                \frac{1}{2} $. Τότε η σειρά 
                \[
                    \sum_{n=1}^{\infty} \frac{a_{n}}{b_{n}} = \sum_{n=1}^{\infty} 
                    \frac{\frac{1}{n(n+1)}}{\frac{1}{(n+1)(n+2)}} = 
                    \sum_{n=1}^{\infty} \frac{n+2}{n} 
                \]
                αποκλίνει γιατί $ \lim_{n \to \infty} \frac{n+2}{n} = 1 \neq 0 $

            \item Έστω οι σειρές 

                \begin{minipage}{0.70\textwidth}
                    \begin{myitemize}
                    \item 
                        $ \sum_{n=1}^{\infty} \underbrace{\frac{1}{2^{2n-1}}}_{a_{n}} 
                        = 2 \sum_{n=1}^{\infty} \frac{1}{4^{n}} = 
                        2 \cdot \sum_{n=1}^{\infty} 
                        \left(\frac{1}{4} \right)^{n} = 2 \cdot \frac{1}{4} \cdot  
                        \frac{1}{1 - \frac{1}{4}} = \frac{1}{2} \cdot \frac{4}{3} = 
                        \frac{2}{3} = a$ \hfill \tikzmark{a}

                    \item $ \sum_{n=1}^{\infty} \underbrace{\frac{3}{2^{n-1}}}_{b_{n}} 
                        = 3 \sum_{n=1}^{\infty} \frac{1}{2^{n-1}} = 
                        3 \cdot \sum_{n=1}^{\infty} 
                        \left(\frac{1}{2} \right)^{n-1} = 3 \cdot \frac{1}{1 - 
                        \frac{1}{2}} = 3 \cdot 2 = 6 = b$ \hfill \tikzmark{b}
                    \end{myitemize}
                \end{minipage}

                \mybrace{a}{b}[$ \frac{a}{b} = \frac{\frac{2}{3}}{6} = \frac{1}{9} $]

                Όμως 
                \[
                    \sum_{n=1}^{\infty} \frac{a_{n}}{b_{n}} = \sum_{n=1}^{\infty} 
                    \frac{\frac{1}{2^{2n-1}}}{\frac{3}{2^{n-1}}} = 
                    \frac{1}{3} \sum_{n=1}^{\infty} \frac{2^{n-1}}{2^{2n-1}} = 
                    \frac{1}{3} \sum_{n=1}^{\infty} \frac{2^{n}}{2^{2n}} = 
                    \frac{1}{3} \sum_{n=1}^{\infty} \frac{1}{2^{n}} = \frac{1}{3} \cdot 
                    \frac{\frac{1}{2}}{1 - \frac{1}{2}} = \frac{1}{3} \neq \frac{1}{9} 
                    = \frac{a}{b}
                \] 
        \end{enumerate}

    \end{proof}
\end{rem}

\begin{mybox3}
\begin{prop}
Έστω ότι η σειρά $ \sum_{n=1}^{\infty} a_{n} $ συγκλίνει. Τότε 
    $ \forall \varepsilon >0, \; \exists n_{0} \in \mathbb{N} :
\quad \abs{\sum_{n= n_{0}}^{\infty} a_{n}} < \varepsilon $
\end{prop}
\end{mybox3}

\begin{proof}
\item {}
    Έστω ότι η σειρά $ \sum_{n=1}^{\infty} a_{n} $ συγκλίνει. Τότε υπάρχει 
    $ S \in \mathbb{R} $ τέτοιος ώστε $ \lim_{n \to \infty} 
    {(S_{n})}_{n \in \mathbb{N}} = S \in \mathbb{R} $. Δηλαδή 
    \[
        \forall \varepsilon >0, \; \exists n_{0} \in \mathbb{N} \; : \; 
        \forall N \geq n_{0} \quad \abs{S_{N}-S} < \varepsilon 
    \] 
    Από την πρόταση~\ref{prop:diff}~\ref{prop:diff2} έχουμε ότι αφού 
    $ \sum_{n=1}^{\infty} a_{n} $ συγκλίνουσα ισχύει
    \begin{gather*}
        \sum_{n=1}^{\infty} a_{n} - \sum_{n= n_{0}+1}^{\infty} a_{n} = 
        \sum_{n= 1}^{n_{0}} a_{n} \Rightarrow \\ 
        \sum_{n= n_{0}+1}^{\infty} a_{n} = \sum_{n=1}^{\infty} a_{n} - 
        \sum_{n=1}^{n_{0}} a_{n} = S - S_{n} \Rightarrow \\
        \abs{\sum_{n= n_{0}}^{\infty} a_{n}} = \abs{S - S_{n}} < \varepsilon
    \end{gather*} 
\end{proof}

\begin{mybox3}
\begin{prop}
Η αρμονική σειρά $ \sum_{n=1}^{\infty} \frac{1}{n} $ δεν συγκλίνει.
\end{prop}
\end{mybox3}

\begin{proof}(Με άτοπο)
\item {}
    Έστω ότι η σειρά συγκλίνει στο $ S \in \mathbb{R} $. Τότε 
    $ \sum_{n=1}^{\infty} \frac{1}{n} 
    = S \in \mathbb{R} \Leftrightarrow \lim_{n \to \infty} {(S_{n})}_{n \in \mathbb{N}} 
    = S $, δηλαδή 
    για $ \varepsilon = \frac{1}{4}, \; \exists n_{0} \in \mathbb{N} \; : \; 
    \forall n \geq n_{0} \quad \abs{S_{n}-S} < \frac{1}{4} $, άρα  και 
    $ \abs{S_{2 n_{0}}-S} < \frac{1}{4}  $ επειδή και κάθε  υπακολουθία της 
    $ {(S_{n})}_{n \in \mathbb{N}} $ θα συγκλίνει στο ίδιο όριο.
    Επομένως
    \[
        \abs{S_{2 n_{0} }- S_{n_{0}}} = \abs{S_{2 n_{0} } -S + S - S_{n_{0}}} \leq 
        \abs{S_{2 n_{0}}-S} + \abs{S_{n_{0}}-S} < \frac{1}{4} + \frac{1}{4} = 
        \frac{1}{2}
    \] 
    Από την άλλη μεριά
    \begin{align*}
        \abs{S_{2 n_{0}} - S_{n_{0}}} 
        &= \abs{\left(1+ \frac{1}{2} + \cdots + 
                \frac{1}{2 n_{0}}\right) - \left(1 +
        \frac{1}{2} + \cdots + \frac{1}{n_{0}}\right)} = \frac{1}{n_{0}+1} + \cdots + 
        \frac{1}{2 n_{0}} \geq \underbrace{\frac{1}{2 n_{0}} + \cdots + 
        \frac{1}{2 n_{0}}}_{n_{0} \; \text{φορές}} = \\ 
        &= n_{0}\cdot \frac{1}{2 n_{0}} = \frac{1}{2} \quad \text{άτοπο, άρα 
        η σειρά δεν συγκλίνει.}
    \end{align*}
\end{proof}



\section{Κριτήρια Σύγκλισης}

\subsection{Κριτήριο Σύγκρισης}
Έστω ότι $ 0 \leq a_{n} \leq b_{n}, \; \forall n \geq n_{0} $.
\begin{enumerate}[i)]
    \item $ \sum_{n=1}^{\infty} b_{n} $ συγκλίνει $ \Rightarrow \sum_{n=1}^{\infty} 
        a_{n}$ συγκλίνει.
    \item $ \sum_{n=1}^{\infty} a_{n} $ αποκλίνει $ \Rightarrow \sum_{n=1}^{\infty} 
        b_{n} $ αποκλίνει.
\end{enumerate}

\begin{mybox3}
\begin{prop}
Η σειρά $ \sum_{n=1}^{\infty} \frac{1}{n^{2}} $ συγκλίνει.
\end{prop}
\end{mybox3}

\begin{proof}
\item {}
    Θέτουμε $ a_{n} = \frac{1}{(n+1)^{2}}, \; \forall n \in \mathbb{N} $ και 
    $ b_{n} = \frac{1}{n(n+1)}, \; \forall n \in \mathbb{N} $. Ισχύει, προφανώς ότι
    $ 0 \leq a_{n} \leq b_{n}, \; \forall n \in \mathbb{N} $. Επιπλέον έχουμε αποδείξει
    ότι $ \sum_{n=1}^{\infty} \frac{1}{n(n+1)} = 1 $, συγκλίνει, οπότε από κριτήριο 
    σύγκρισης και η σειρά 
    $ \sum_{n=1}^{\infty} a_{n} $ συγκλίνει. Όμως ισχύει ότι
    \[
        \sum_{n=1}^{\infty} a_{n} = \sum_{n=1}^{\infty} \frac{1}{(n+1)^{2}} = 
        \sum_{n=2}^{\infty} \frac{1}{n^{2}} 
    \] 
    δηλαδή η σειρά $ \sum_{n=2}^{\infty} \frac{1}{n^{2}} $ συγκλίνει και από γνωστή 
    πρόταση και η σειρά $ \sum_{n=1}^{\infty} \frac{1}{n^{2}} $ συγκλίνει.
\end{proof}

\begin{examples}
\item {}
    \begin{enumerate}
        \item Η σειρά $ \sum_{n=1}^{\infty} \frac{1}{n(n+1)} $ συγκλίνει. Πράγματι
            \[
                a_{n}= \frac{1}{n(n+1)} = \frac{1}{n^{2}+n} \leq \frac{1}{n^{2}} = 
                b_{n}, \; \forall n \in \mathbb{N}
            \] 
            και η $ \sum_{n=1}^{\infty} b_{n} = \sum_{n=1}^{\infty} \frac{1}{n^{2}} $ 
            συγκλίνει, οπότε από το κριτήριο 
            σύγκρισης και η $ \sum_{n=1}^{\infty} a_{n} = \sum_{n=1}^{\infty} 
            \frac{1}{n(n+1)} $ συγκλίνει. 

        \item Η σειρά $ \sum_{n=1}^{\infty} \frac{n^{2}+n+1}{n^{3}} $ αποκλίνει. 
            \[
                a_{n} = \frac{n^{2}+n+1}{n^{3}} \geq \frac{n^{2}}{n^{3}} = \frac{1}{n} 
                = b_{n}, \; \forall n \in \mathbb{N} 
            \] 
            και η $ \sum_{n=1}^{\infty} b_{n} = \frac{1}{n} $ αποκλίνει, οπότε από 
            το κριτήριο σύγκρισης και η $ \sum_{n=1}^{\infty} a_{n} = 
            \sum_{n=1}^{\infty} \frac{n^{2}+n+1}{n^{3}} $ αποκλίνει.
    \end{enumerate}
\end{examples}

\subsection{Κριτήριο Ορίου}
\begin{mybox2}
\begin{thm}
Έστω $ {(a_{n})}_{n \in \mathbb{N}} $ και $ {(b_{n})}_{n \in \mathbb{N}} $ δύο 
    ακολουθίες θετικών αριθμών, τέτοιες ώστε $ \lim_{n \to \infty} \frac{a_{n}}{b_{n}}
    \in \mathbb{R} $, τότε: 
    \begin{enumerate}[i)]
        \item $ \sum_{n=1}^{\infty} b_{n} $ συγκλίνει $ \Rightarrow 
            \sum_{n=1}^{\infty} a_{n} $ συγκλίνει.
        \item $ \sum_{n=1}^{\infty} a_{n} $ αποκλίνει (στο $+ \infty$) $ \Rightarrow 
            \sum_{n=1}^{\infty} b_{n} $ αποκλίνει (στο $+ \infty$).
\end{enumerate}
\end{thm}
\end{mybox2}

\begin{proof}
\item {}
    Έστω η ακολουθία $ {\left(\frac{a_{n}}{b_{n}}\right)}_{n \in \mathbb{N}} $ 
    συγκλίνει.  Τότε θα είναι και φραγμένη, επομένως
    \[
        \exists M>0 \; : \; \frac{a_{n}}{b_{n}} \leq M, \; \forall n \in \mathbb{N} 
    \] 
    Οπότε $ 0 \leq a_{n} \leq M\cdot b_{n}, \; \forall n \in \mathbb{N} $.
    \begin{enumerate}[i)]
        \item Η σειρά $ \sum_{n=1}^{\infty} b_{n} $ συγκλίνει, άρα και η σειρά 
            $ \sum_{n=1}^{\infty} M\cdot b_{n} $ συγκλίνει, οπότε από το κριτήριο 
            σύγκρισης συγκλίνει και η $ \sum_{n=1}^{\infty} a_{n} $.
        \item Η σειρά $ \sum_{n=1}^{\infty} a_{n} $ αποκλίνει στο $ + \infty $, 
            οπότε από το κριτήριο σύγκρισης η σειρά 
            $ \sum_{n=1}^{\infty} M \cdot b_{n} $ αποκλίνει
            στο $ + \infty $. Τελικά και η σειρά $ \sum_{n=1}^{\infty} b_{n} $ 
            αποκλίνει στο $ + \infty $.
    \end{enumerate}
\end{proof}

\begin{cor}
    Έστω $ {(a_{n})}_{n \in \mathbb{N}} $ και $ {(b_{n})}_{n \in \mathbb{N}} $ δύο 
    ακολουθίες θετικών αριθμών, τέτοιες ώστε $ \lim_{n \to \infty} \frac{a_{n}}{b_{n}} 
    \in \mathbb{R} \setminus \{ 0 \}$. Τότε η σειρά $ \sum_{n=1}^{\infty} b_{n} $ 
    συγκλίνει αν και μόνον αν η σειρά $ \sum_{n=1}^{\infty} a_{n} $ συγκλίνει.
\end{cor}
\begin{proof}
    Επειδή $ \lim_{n \to \infty} \frac{a_{n}}{b_{n}} \neq 0 \Rightarrow \lim_{n \to
    \infty} \frac{b_{n}}{a_{n}} \in \mathbb{R} $, οπότε από την προηγούμενη πρόταση 
    έχουμε το ζητούμενο.
\end{proof}

\begin{examples}
\item {}
    \begin{enumerate}
        \item Η σειρά $ \sum_{n=1}^{\infty} \frac{n^{5}+8n}{n^{6}+6} $ αποκλίνει.
            \begin{proof}
            \item {}
                \begin{minipage}{0.3\textwidth}
                    \begin{myitemize}
                    \item $ a_{n} = \frac{n^{5}+8n}{n^{6}+6}, \; 
                        \forall n \in \mathbb{N} $ \hfill \tikzmark{a}
                    \item $ b_{n} = \frac{1}{n}, \; \forall n \in \mathbb{N} $
                        \hfill \tikzmark{b}
                    \end{myitemize}    
                \end{minipage}

                \mybrace{a}{b}[$ \frac{a_{n}}{b_{n}} = 
                \frac{\frac{n^{5}+8n}{n^{6}+6}}{\frac{1}{n}} = 
                \frac{n^{6}+8n^{2}}{n^{6}+6} \to 1 $]
                και επειδή η $ \sum_{n=1}^{\infty} b_{n} = 
                \sum_{n=1}^{\infty} \frac{1}{n} $
                αποκλίνει, τότε και η $ \sum_{n=1}^{\infty} a_{n} = \sum_{n=1}^{\infty} 
                \frac{n^{5}+8n}{n^{6}+6}$ αποκλίνει.
            \end{proof}

        \item Η σειρά $ \sum_{n=1}^{\infty} \frac{1}{\sqrt[3]{n^{4}+1}} $ συγκλίνει.
            \begin{proof}
            \item {} 
                \begin{minipage}{0.3\textwidth}
                    \begin{myitemize}
                    \item $ a_{n} = \frac{1}{\sqrt[3]{n^{4}+1}}, \; 
                        \forall n \in \mathbb{N} $ \hfill \tikzmark{a}
                    \item $ b_{n} = \frac{1}{n^{\frac{4}{3}}}, \; 
                        \forall n \in \mathbb{N} $ \hfill \tikzmark{b}
                    \end{myitemize}    
                \end{minipage}

                \mybrace{a}{b}[$ \frac{a_{n}}{b_{n}} =
                \frac{\frac{1}{\sqrt[3]{n^{4}+1}}}{\frac{1}{\sqrt[3]{n^{4}}}} = 
                \sqrt[3]{\frac{n^{4}}{n^{4}+1}} \to 1$]
                και επειδή η $ \sum_{n=1}^{\infty} b_{n} = \sum_{n=1}^{\infty} 
                \frac{1}{n^{\frac{4}{3}}} $ συγκλίνει, τότε και η 
                $ \sum_{n=1}^{\infty} a_{n} = \sum_{n=1}^{\infty} 
                \frac{1}{\sqrt[3]{n^{4}+1}}$ συγκλίνει.
            \end{proof}
    \end{enumerate}
\end{examples}

\begin{mybox3}
  \begin{prop}
    $ \sum_{n=1}^{\infty} a_{n} $ συγκλίνει 
    $ \Rightarrow \lim_{n \to \infty} a_{n} = 0 $.
  \end{prop}
\end{mybox3}

\begin{proof}
\item {}
  Έστω ότι $ \sum_{n=1}^{\infty} a_{n} $ συγκλίνει. Τότε η ακολουθία $ (S_{n})_{n \in
  \mathbb{N}} $ συγκλίνει, και έστω ότι $ \lim_{n \to \infty} S_{n}=S $. Θα δείξουμε ότι
  $ \lim_{n \to \infty} a_{n}=0 $. Έστω $ \varepsilon >0 $. Επειδή $ \lim_{n \to \infty}
  S_{n} = S $, έχουμε ότι για $ \frac{\varepsilon}{2} >0, \; \exists n_{1} \in \mathbb{N}
  \; : \; \forall n \geq n_{1} \quad \abs{S_{n}-S} < \frac{\varepsilon}{2}$. Άρα 
  για το $ \varepsilon >0, \; \exists n_{0} = n_{1}+1 \in \mathbb{N} \; : \; \forall n
  \geq n_{0} $ να έχουμε:
  \begin{align*}
    \abs{a_{n} - 0} = \abs{a_{n}} = \abs{a_{1}+ a_{2}+ 
    \cdots + a_{n} - (a_{1}+ a_{2}+ \cdots + a_{n-1})} 
    &= \abs{S_{n}-S_{n-1}} = \abs{S_{n}-S+S-S_{n-1}} \\ 
    &\leq \abs{S_{n}-S} + \abs{S_{n-1}-S} < 
    \frac{\varepsilon}{2} + \frac{\varepsilon}{2} = \varepsilon 
  \end{align*} 
\end{proof}

\begin{cor}[Αντιθετοαντίστροφο]
    $ \lim_{n \to \infty} a_{n} \neq 0 $, ή $ \not \exists \lim_{n \to \infty} a_{n} 
    \Rightarrow \sum_{n=1}^{\infty} a_{n} $ αποκλίνει.
\end{cor}

\begin{examples}
    \begin{enumerate}
        \item Η σειρά $ \sum_{n=1}^{\infty} n^{2} $ δεν συγκλίνει, γιατί 
            $ \lim_{n \to \infty} n^{2} = \infty \neq 0$
        \item Η σειρά $ \sum_{n=1}^{\infty} \frac{n+1}{n} $ δεν συγκλίνει, γιατί 
            $ \lim_{n \to \infty} \frac{n+1}{n} = 1 \neq 0 $
        \item Η σειρά $ \sum_{n=1}^{\infty} (1 + \frac{1}{n} )^{n} $ δεν συγκλίνει, 
            γιατί $ \lim_{n \to \infty} (1+ \frac{1}{n} )^{n} = e \neq 0 $
        \item Η σειρά $ \sum_{n=1}^{\infty} \sin{\left(\frac{n \pi}{2}\right)} $ 
            δεν συγκλίνει, γιατί 

            \begin{minipage}{0.55\textwidth}
                \begin{myitemize}
                \item $ a_{2n} = \sin{(n \pi)} = 0, \; \forall n \in \mathbb{N} 
                    \Rightarrow a_{2n} \xrightarrow{n \to \infty} 0$ \hfill \tikzmark{a}
                \item $ a_{4n+1} = \sin{\left(2n \pi + \frac{\pi}{2}\right)}, \; 
                    \forall n \in \mathbb{N} \Rightarrow a_{4n+1} 
                    \xrightarrow{n \to  \infty} 1 $ \hfill \tikzmark{b}
                \end{myitemize}
            \end{minipage}

            \mybrace{a}{b}[$ {(a_{n})}_{n \in \mathbb{N}} $ δεν συγκλίνει]
            επομένως η $ \sum_{n=1}^{\infty} \sin{\left(\frac{n \pi}{2}\right) } $ 
            αποκλίνει.
    \end{enumerate}
\end{examples}

\section{Απόλυτη Σύγκλιση}

\begin{mybox3}
\begin{prop}
    $ \sum_{n=1}^{\infty} \abs{a_{n}} $ συγκλίνει 
    $ \Rightarrow \sum_{n=1}^{\infty} a_{n} $ συγκλίνει 
  \end{prop}
\end{mybox3}

\begin{proof}
  $ \sum_{n=1}^{\infty} \abs{a_{n}} $ συγκλίνει $ \Rightarrow \sum_{n=1}^{\infty} 2 
  \abs{a_{n}} $ συγκλίνει. Ισχύει
    \[
        0 \leq a_{n} + \abs{a_{n}} \leq 2 \abs{a_{n}}, \; \forall n \in \mathbb{N} 
    \] 
    οπότε από το κριτήριο σύγκρισης, έπεται ότι $ \sum_{n=1}^{\infty} (a_{n}+ 
    \abs{a_{n}}) $ συγκλίνει. Τότε
    \[
        S_{N} = \sum_{n=1}^{N} a_{n} = \sum_{n=1}^{N} (a_{n}+ \abs{a_{n}}) - 
        \sum_{n=1}^{\infty} \abs{a_{n}}   
    \] 
    Από άλγεβρα των ορίων η $ {(S_{n})}_{n \in \mathbb{N}} $ συγκλίνει και άρα 
    έχουμε το ζητούμενο.
\end{proof}

\begin{examples}
\item {}
    \begin{enumerate}
        \item Η σειρά $ \sum_{n=1}^{\infty} (-1)^{n} \frac{1}{n^{2}} $ συγκλίνει, 
            γιατί 
            \[ 
                \sum_{n=1}^{\infty} \abs{a_{n}} = \sum_{n=1}^{\infty} \abs{(-1)^{n}
                \frac{1}{n^{2}}} = \sum_{n=1}^{\infty} \abs{(-1)^{n}} 
                \cdot \abs{\frac{1}{n^{2}} } =  
                \sum_{n=1}^{\infty} \abs{-1}^{n} \cdot 
                \frac{1}{n^{2}} = \sum_{n=1}^{\infty} \frac{1}{n^{2}} 
            \]
            η οποία συγκλίνει.
    \end{enumerate}
\end{examples}


\section{Κριτήριο Λόγου}

\begin{mybox3}
\begin{prop}
Έστω $ {(a_{n})}_{n \in \mathbb{N}} $ ακολουθία αριθμών ώστε να υπάρχει το $ 
    \lim_{n \to \infty} \frac{\abs{a_{n+1}}}{\abs{a_{n}}} $. Τότε 
    \begin{enumerate}[i)]
        \item $ \lim_{n \to \infty} \frac{\abs{a_{n+1}}}{\abs{a_{n}}} <1 \Rightarrow 
            \sum_{n=1}^{\infty} a_{n}$ συγκλίνει 
        \item $ \lim_{n \to \infty} \frac{\abs{a_{n+1}}}{\abs{a_{n}}} >1 \Rightarrow 
            \sum_{n=1}^{\infty} a_{n}$ αποκλίνει
\end{enumerate}
\end{prop}
\end{mybox3}


\begin{examples}
\item {}
    \begin{enumerate}
        \item Η σειρά $ \sum_{n=1}^{\infty} \frac{1}{n!} $ συγκλίνει.
            \[
                \abs{\frac{a_{n+1}}{a_{n}}} = \frac{\frac{1}{(n+1)!}}{\frac{1}{n!}} = 
                \frac{n!}{(n+1)!} = \frac{1 \cdot 2 \cdots n }{1 \cdot 2 
                \cdots n \cdot(n+1)} = \frac{1}{n+1} \xrightarrow{n \to \infty } 0 < 1 
            \]
            επομένως από Κριτήριο Λόγου η σειρά $ \sum_{n=1}^{\infty} a_{n} = 
            \sum_{n=1}^{\infty} \frac{1}{n!}$ συγκλίνει.

        % \item Η σειρά $ \sum_{n=1}^{\infty} \frac{9^{n}}{(-2)^{n}\cdot n} $ αποκλίνει.
        %     \[
        %         \abs{\frac{a_{n+1}}{a_{n}}} = \frac{\abs{\frac{9^{n+1}}{(-2)^{n+1}\cdot
        %         (n+1)}}}{\abs{\frac{9^{n}}{(-2)^{n}\cdot n}}} = 
        %         \frac{\frac{9^{n+1}}{2^{n+1}\cdot (n+1)}}{\frac{9^{n}}{2^{n}\cdot n}} 
        %         = \frac{9^{n}\cdot 9}{9^{n}} \cdot \frac{2^{n}}{2^{n}\cdot 2}\cdot 
        %         \frac{n}{n+1} = \frac{9}{2} \cdot \frac{n}{n+1} \xrightarrow{n \to 
        %         \infty} \frac{9}{2} \cdot 1 = \frac{9}{2}> 1
        %     \] 
        %     επομένως από Κριτήριο Λόγου η σειρά 
        %     $ \sum_{n=1}^{\infty} \frac{9^{n}}{(-2)^{n}\cdot n} $ αποκλίνει

        \item Η σειρά $ \sum_{n=1}^{\infty} \frac{n!}{10^{n}} $ συγκλίνει.
            \[
                \abs{\frac{a_{n+1}}{a_{n}}} = 
                \frac{\frac{(n+1)!}{10^{n+1}}}{\frac{n!}{10^{n}}} =
                \frac{10^{n}}{10^{n+1}} \cdot \frac{(n+1)!}{n!} = 
                \frac{10^{n}}{10^{n} \cdot 10}
                \cdot \frac{1\cdot 2 \cdots n \cdot (n+1)}{1\cdot 2 \cdots n}  
                \frac{1}{10} \cdot (n+1) \xrightarrow{n \to \infty} + \infty > 1
            \] 
            επομένως από Κριτήριο Λόγου η σειρά 
            $ \sum_{n=1}^{\infty} \frac{n!}{10^{n}} $ αποκλίνει.

        \item Η σειρά $ \sum_{n=1}^{\infty} \frac{2n-1}{(\sqrt{3}) ^{n}} $ συγκλίνει.
            \[
                \abs{\frac{a_{n+1}}{a_{n}}} = \frac{\frac{2(n+1)-1}{(\sqrt{3}
                )^{n+1}}}{\frac{2n-1}{(\sqrt{3} )^{n}}} = \frac{2n+2-1}{2n-1} \cdot 
                \frac{(\sqrt{3} )^{n}}{(\sqrt{3} )^{n}\cdot \sqrt{3}} = 
                \frac{1}{(\sqrt{3})} \cdot \frac{2n+1}{2n-1} 
                \xrightarrow{n \to \infty} = \frac{1}{\sqrt{3}} \cdot 1 =
                \frac{1}{\sqrt{3}} < 1
            \] 
            επομένως από Κριτήριο Λόγου η σειρά 
            $ \sum_{n=1}^{\infty} \frac{2n-1}{(\sqrt{3}) ^{n}} $ συγκλίνει.
    \end{enumerate}
\end{examples}

\section{Κριτήριο Ρίζας}

\begin{mybox3}
\begin{prop}
Έστω $ {(a_{n})}_{n \in \mathbb{N}} $ ακολουθία αριθμών ώστε να υπάρχει το 
    $ \lim_{n \to \infty} \sqrt[n]{\abs{a_{n}}} $. Τότε
    \begin{enumerate}[i)]
        \item $ \lim_{n \to \infty} \sqrt[n]{\abs{a_{n}}} < 1 \Rightarrow 
            \sum_{n=1}^{\infty} a_{n}$ συγκλίνει
        \item $ \lim_{n \to \infty} \sqrt[n]{\abs{a_{n}}} > 1 \Rightarrow 
            \sum_{n=1}^{\infty} a_{n}$ αποκλίνει
\end{enumerate}
\end{prop}
\end{mybox3}

\begin{examples}
\item {}
    \begin{enumerate}
        \item Η σειρά $ \sum_{n=1}^{\infty} \left(\frac{n}{n+1} \right)^{n^{2}} $ 
            συγκλίνει
            \[
                \sqrt[n]{\abs{a_{n}}} = \sqrt[n]{\left(\frac{n}{n+1} \right)^{n^{2}}} 
                = \left(\frac{n}{n+1}\right)^{n} = \frac{1}{\left(\frac{n+1}{n} 
                \right)^{n}} = \frac{1}{(1+ \frac{1}{n})^{n}} \xrightarrow{n \to \infty}
                \frac{1}{e} <1
            \] 
            επομένως από Κριτήριο Ρίζας η σειρά 
            $ \sum_{n=1}^{\infty} \left(\frac{n}{n+1} \right)^{n^{2}} $ συγκλίνει.
    \end{enumerate}
\end{examples}

\section{Κριτήριο Dirichlet}

Έστω $ {(a_{n})}_{n \in \mathbb{N}} $ και $ (b_{n})_{n \in \mathbb{N}} $ ακολουθίες 
πραγματικών αριθμών τέτοιες ώστε:
\begin{enumerate}[i)]
    \item Η $ (b_{n})_{n \in \mathbb{N}} $ έχει θετικούς όρους, είναι φθίνουσα και 
        συγκλίνει στο 0.
    \item Η $ {(a_{n})}_{n \in \mathbb{N}} $ έχει φραγμένα μερικά αθροίσματα, δηλ.
        \[
            \exists M>0 \; : \; \abs{\sum_{n=1}^{N} a_{n}} \leq M, \; 
            \forall n \in \mathbb{N} 
        \] 
        Τότε η σειρά $ \sum_{n=1}^{\infty} a_{n}\cdot \mathbb{N} $ συγκλίνει.
\end{enumerate}

\begin{examples}
\item {}
    \begin{enumerate}
        \item Η σειρά $ \sum_{n=1}^{\infty} \frac{(-1)^{n}}{n} $ συγκλίνει.
            Θέτουμε $ a_{n}=(-1)^{n}, \; \forall n \in \mathbb{N} $ και $ b_{n}= 
            \frac{1}{n}, \; \forall n \in \mathbb{N}$ και έχουμε:
            Η $ (b_{n})_{n \in \mathbb{N}} $ έχει προφανώς θετικούς όρους, είναι 
            φθίνουσα και $ \lim_{n \to \infty} \frac{1}{n} = 0 $. 

            Για την $ {(a_{n})}_{n \in \mathbb{N}} $ ισχύει:
            \[
                \abs{\sum_{n=1}^{N} (-1)^{n}} = 0 \; \text{ή} \; 1, \; 
                \forall N \in \mathbb{N} 
            \] 
            δηλαδή σε κάθε περίπτωση 
            \[
                \abs{\sum_{n=1}^{N} (-1)^{n}} \leq 1 = M, \; \forall n \in \mathbb{N}
            \]
            Επομένως από Κριτήριο Dirichlet η σειρά 
            $ \sum_{n=1}^{\infty} \frac{(-1)^{n}}{n} $ συγκλίνει.

        \item Η σειρά $ \sum_{n=1}^{\infty} \frac{(-1)^{n}}{\sqrt{n}} $ συγκλίνει.

            Θέτουμε $ a_{n}=(-1)^{n}, \; \forall n \in \mathbb{N} $ και $ b_{n}= 
            \frac{1}{\sqrt{n}}, \; \forall n \in \mathbb{N}$ και έχουμε:
            Η $ (b_{n})_{n \in \mathbb{N}} $ έχει προφανώς θετικούς όρους, είναι 
            φθίνουσα και $ \lim_{n \to \infty} \frac{1}{\sqrt{n}} = 0 $. 

            Για την $ {(a_{n})}_{n \in \mathbb{N}} $ ισχύει όπως και στο 
            προηγούμενο παράδειγμα ότι έχει φραγμένα μερικά αθροίσματα.

            Επομένως από Κριτήριο Dirichlet η σειρά 
            $ \sum_{n=1}^{\infty} \frac{(-1)^{n}}{\sqrt{n} } $ συγκλίνει.
    \end{enumerate}
\end{examples}


\end{document}
