\input{../frontmatter/preamble_a4.tex}
\input{../frontmatter/definitions.tex}
\input{tikz.tex}


\pagestyle{vangelis}

\begin{document}

\chapter{Πραγματικοί Αριθμοί}

\section{Μέγιστο, Ελάχιστο, Άνω και Κάτω Φράγματα}

\begin{dfn}
    $ A \subseteq \mathbb{R}, \; A \neq \emptyset $ \textcolor{Col1}{άνω 
    φραγμένο} $ \Leftrightarrow \exists \textcolor{Col2}{M \in \mathbb{R}} \; 
    \text{(\textcolor{Col1}{άνω φράγμα})} \; : a \leq M, \; 
    \forall a \in A$ 
\end{dfn}

\begin{dfn}
    $ A \subseteq \mathbb{R}, \; A \neq \emptyset $ \textcolor{Col1}{κάτω 
    φραγμένο} $ \Leftrightarrow \exists \textcolor{Col2}{m \in \mathbb{R}} \; 
    \text{(\textcolor{Col1}{κάτω φράγμα})}  \; : a \geq m, \; 
    \forall a \in A$
\end{dfn}

\begin{dfn}
    $ A \subseteq \mathbb{R}, \; A \neq \emptyset $ \textcolor{Col1}{ 
    φραγμένο}   $ \Leftrightarrow A $ άνω και κάτω φραγμένο 
        $ \Leftrightarrow \exists m,M \in \mathbb{R} \; : \; m \leq a \leq M,
        \; \forall a \in A $
\end{dfn}

\begin{dfn}
        $ A \subseteq \mathbb{R}, \; A \neq \emptyset $ \textcolor{Col1}{ 
        απολύτως φραγμένο} 
        $\Leftrightarrow \exists M>0 \, : \, \abs{a} \leq M, \; \forall a \in A $ 
        $ \Leftrightarrow  \, -M \leq a \leq M, \; \forall a \in A $
\end{dfn}

\begin{prop}
    $ A \subseteq \mathbb{R}, \; A \neq \emptyset $ είναι φραγμένο 
    $ \Leftrightarrow $ $ A $ είναι απολύτως φραγμένο.
\end{prop}

\begin{rem}
\item {}
    \begin{itemize}[label=\textcolor{Col1}{\tiny$\blacksquare$}]
        \item Αν $ M $ άνω φράγμα του $A$ και $ \textcolor{Col2}{M \in A} $ τότε $M$ λέγεται 
    \textcolor{Col1}{μέγιστο} του $A$ και συμβολίζεται $ M = \max A $.
\item Αν $ m $ κάτω φράγμα του $A$ και $ \textcolor{Col2}{m \in A} $ τότε $m$ λέγεται \textcolor{Col1}{ελάχιστο} του $A$ και συμβολίζεται $ m = \min A $.  \end{itemize}
\end{rem}

\section{Supremum, Infimum}

\begin{dfn}
    Έστω $ A \subseteq \mathbb{R}, \; A \neq \emptyset $.
    \begin{itemize}[label=\textcolor{Col1}{\tiny$\blacksquare$}]
        \item $ s \in \mathbb{R} $ \textcolor{Col1}{supremum} του $A$ 
          (\textcolor{Col2}{$s = \sup A $}) 
            $ \Leftrightarrow $ 
            \begin{tabular}[t]{l}
                $s$ άνω φράγμα του $A \Leftrightarrow a \leq s, \; \forall a 
                \in A $ \\
                $s$ \textbf{ελάχιστο άνω φράγμα} του $A  \Leftrightarrow 
                \forall M \; \text{α.φ.\ του} A \Rightarrow s \leq M $
            \end{tabular} 

        \item $ s \in \mathbb{R} $ \textcolor{Col1}{infimum} του $A$ 
            ($ \textcolor{Col2}{s= \inf A} $) 
            $ \Leftrightarrow $ 
            \begin{tabular}[t]{l}
                $s$ κάτω φράγμα του $A \Leftrightarrow a \geq s, \; \forall a 
                \in A $ \\
                $s$ \textbf{μέγιστο κάτω φράγμα} του $A \Leftrightarrow 
                \forall m \; \text{κ.φ.\ του} A \Rightarrow s \geq m $
            \end{tabular} 
    \end{itemize}
\end{dfn}

\section{Χαρακτηριστική Ιδιότητα του Supremum, Infimum}

    \begin{itemize}[label=\textcolor{Col1}{\tiny$\blacksquare$}]
        % \item 
        %   Η \textcolor{Col1}{χαρακτηριστική ιδιότητα} που έχει το supremum ενός συνόλου $ A $, είναι
        %   ότι είναι το \textbf{ελάχιστο από τα άνω φράγματα} του $A$. Δηλαδή, με άλλα λόγια,
    % κάθε πραγματικός αριθμός μικρότερος του $M$ \textbf{δεν} είναι άνω φράγμα του 
    % $A$, επομένως, $ \forall k < M, \; \exists a \in A \; : \; k < a < M $, 
    % ή καλύτερα, $ \forall \varepsilon > 0, \; \exists a \in A \; : \; 
    % M- \varepsilon < a$

        \item 
          Η \textcolor{Col1}{χαρ.\ ιδιότητα} του sup ενός συνόλου $ A $, είναι
          ότι είναι το \textbf{ελάχιστο άνω φράγμα} του $A$. Δηλ, 
    κάθε αριθμός μικρότερος του $s$ \textbf{δεν} είναι α.φ.\ του 
    $A$, δηλ, $ \forall k < s, \; \exists a \in A \, : \, k < a < s $, 
    ή $ \forall \varepsilon > 0, \; \exists a \in A \, : \, s- \varepsilon < a.$

\item 
  Η \textcolor{Col1}{χαρ.\ ιδιότητα} του inf ενός συνόλου $ A $, είναι
  ότι είναι το \textbf{μέγιστο κάτω φράγμα} του $A$. Δηλ,
  κάθε αριθμός μεγαλύτερος του $s$ \textbf{δεν} είναι κ.φ.\ του 
    $A$, δηλ, $ \forall k > s, \; \exists a \in A \, : \, s < a < k $, 
    ή $ \forall \varepsilon > 0, \; \exists a \in A \, : \, 
    a < s + \varepsilon. $
    \end{itemize}
\begin{thm}[\textcolor{Col2}{Χαρακτηριστική ιδιότητα του sup και inf}]
\item {}
    \begin{itemize}[label=\textcolor{Col1}{\tiny$\blacksquare$}]
        \item Έστω $ A \subseteq \mathbb{R}, \; A \neq \emptyset $, $ A $ άνω φραγμένο με
          $s \in \mathbb{R}$ \textbf{άνω φράγμα} του A, τότε 
    \[
         \boxed{s = \sup A \Leftrightarrow \forall \varepsilon > 0, \; \exists 
         a \in A \; : \; s- \varepsilon  < a}
     \] 

 \item Έστω $ A \subseteq \mathbb{R}, \; A \neq \emptyset $, $ A $ κάτω
   φραγμένο με $s \in \mathbb{R}$ \textbf{κάτω φράγμα} του A, τότε 
    \[
         \boxed{s = \inf A \Leftrightarrow \forall \varepsilon > 0, \; \exists 
         a \in A \; : \; a < s+ \varepsilon}
     \] 
    \end{itemize}
    \end{thm}

% \begin{prop}
%         \item {}
%     \begin{itemize}[label=\textcolor{Col1}{\tiny$\blacksquare$}]
%         \item $ A \subseteq \mathbb{R}, \; A \neq \emptyset $ και $A$ όχι 
%             άνω φραγμένο $ \Rightarrow \sup A = + \infty $.
%         \item $ A \subseteq \mathbb{R}, \; A \neq \emptyset $ και $A$ όχι 
%             κάτω φραγμένο $ \Rightarrow \inf A = - \infty $.
%     \end{itemize}
% \end{prop}

    \vspace{\baselineskip}

    \twocolumnsides{
      \section{Supremum και infimum}
      \begin{myitemize}
        \item $ \sup (-A) = - \inf A \; \text{και} \; \inf (-A) = - \sup A $
        \item $ \sup (\lambda A) = \lambda \sup A \; \text{και} \; \inf (\lambda A) =
          \lambda \inf A, \; \textcolor{Col1}{\lambda >0} $
        \item $ \sup (\lambda A) = \lambda \inf A \; \text{και} \inf (\lambda A) =
          \lambda \sup A, \; \textcolor{Col1}{\lambda <0} $
        \item $ \sup {(A \cup B)} = \max \{ \sup A, \sup B \} $ 
        \item $ \inf {(A \cup B)} = \min \{ \inf A, \inf B \} $ 
        \item $ \sup {(A+B)} = \sup A + \sup B $
        \item $ \inf {(A+B)} = \inf A + \inf B $
        \item $ \sup {(A\cdot B)} = \sup A \cdot \sup B $ \tikzmark{a}
        \item $ \inf {(A\cdot B)} = \inf A \cdot \inf B $ \;\;\, \tikzmark{b}
      \end{myitemize}
      \mybrace{a}{b}[αν $ A,B \subseteq \mathbb{R}^{+} $]
    }{
    \section{Ρητοί και Άρρητοι}
    \begin{myitemize}
      \item ρητός + άρρητος = άρρητος 
      \item άρρητος + άρρητος = ρητός ή άρρητος 
      \item ρητός $ \cdot $ άρρητος = άρρητος (εκτός αν ο ρητός=0)
    \end{myitemize}
    \section{Ανισότητα Bernoulli}
    \begin{myitemize}
      \item $(1+a)^{n} \geq 1+ na, \quad \forall n \in \mathbb{N}, \; a>-1 $
    \end{myitemize}
    \section{Τριγωνική Ανισότητα}
    \begin{myitemize}
      \item $ \abs{\abs{a} - \abs{b}} \leq \abs{a \pm b} \leq \abs{a} + \abs{b}, \quad 
        \forall a,b \in \mathbb{R} $
    \end{myitemize}
    \section{Ακέραιο Μέρος}
    \begin{myitemize}
      \item $ [x] \leq x < [x]+1, \quad \forall x \in \mathbb{R} $ 
    \end{myitemize}
  }

\end{document}

