\input{preamble.tex}
\input{definitions.tex}
\input{tikz.tex}
\input{myboxes.tex}


\begin{document}
\setcounter{chapter}{3}

\chapter{Συναρτήσεις}

\section{Ορισμός συνάρτησης}

\begin{dfn}
  Έστω $ A, B \subseteq \mathbb{R} $. Καλούμε \textcolor{Col1}{συνάρτηση}, από το $A$ 
  στο $B$, κάθε κανόνα $f$ σύμφωνα με τον οποίο σε \textbf{κάθε} στοιχείο $ x \in A $ 
  αντιστοιχεί \textbf{ένα και μόνο ένα} στοιχείο $ y \in B $. 
  \begin{myitemize}
    \item Το σύνολο $A$ καλείται \textcolor{Col1}{Πεδίο Ορισμού} της $f$ και το σύνολο 
      $B$ καλείται \textcolor{Col1}{Πεδίο Τιμών} της $f$.
    \item Τό σύνολο $ f(A) = \{ y \in B \; : \; \exists x \in A \; \text{με} 
      \; f(x)=y \} \subseteq B $, καλείται \textcolor{Col1}{σύνολο τιμών} της $f$.
  \end{myitemize}

  Συμβολικά, γράφουμε $ f \colon A \to B $ ή $ A \xrightarrow{f} B $, 
  ή $ y=f(x), \; x \in A $.  
\end{dfn}

\begin{exercise}
  Να βρεθεί το Πεδίο Ορισμού των συναρτήσεων:
  \begin{enumerate}[i)]
    \item $ f(x) = \frac{\sqrt{x}}{(x-1)(x-3)} $
    \item $ g(x) = \ln{(x-1)} \cdot \mathrm{e}^{\frac{1}{x-4}} $
  \end{enumerate}
\end{exercise}
\begin{solution}
\item {}
  \begin{enumerate}[i)]
    \item Για τη συνάρτηση $f$, έχουμε:
      \begin{align*}
        A_{f} &= \{ x \in \mathbb{R} \; : \; x \geq 0 \quad \text{και} \quad (x-1)(x-3)
        \neq 0 \} \\ 
              &= \{ x \in \mathbb{R} \; : \; x \geq 0 \quad
              \text{και} \quad x \neq 1,3 \} \\ 
              &= [0,1) \cup (1,3) \cup (3, + \infty) 
        \end{align*} 
      \item Για τη συνάρτηση $g$, έχουμε:
        \begin{align*}
          A_{g} &= \{ x \in \mathbb{R} \; : \; x-1 > 0 \quad \text{και} \quad x \neq 4 \}
          \\
                &= \{ x \in \mathbb{R} \; : \; x > 1 \quad \text{και} \quad x \neq 4 \} 
             \\ 
                &= (1,4) \cup (4, + \infty)
        \end{align*} 
    \end{enumerate}
\end{solution}

\section{Ένα προς ένα και επί συναρτήση}

\begin{dfn}
  Έστω $ A, B \subseteq \mathbb{R} $ και έστω $ f \colon A \to B $ συνάρτηση. 
  \begin{myitemize}
    \item Η $f$ καλείται \textcolor{Col1}{επί}, όταν $ f(A) = B $ ή ισοδύναμα όταν 
      $ \forall y \in B, \; \exists x \in A \; : \; f(x)=y $. 
    \item Η $f$ καλείται \textcolor{Col1}{ένα προς ένα} όταν 
      $ \forall x_{1}, x_{2} \in A, \; \text{με} \; x_{1} \neq x_{2} $, έχουμε ότι 
      $ f(x_{1}) \neq f(x_{2}) $ ή ισοδύναμα όταν, 
      $ \forall x_{1}, x_{2} \in A, \; \text{με} \; f(x_{1}) = f(x_{2}) $, έχουμε ότι 
      $ x_{1}= x_{2} $.
  \end{myitemize}
\end{dfn}

\section{Μονοτονία συνάρτησης}

\begin{dfn}
  Έστω $ A, B \subseteq \mathbb{R} $ και έστω $ f \colon A \to B $ συνάρτηση. 
  \begin{myitemize}
    \item Η $f$ καλείται \textcolor{Col1}{αύξουσα} όταν 
      $ \forall x_{1}, x_{2} \in A, \; \text{με} \; x_{1} < x_{2} $, έχουμε ότι 
      $ f(x_{1}) \leq f(x_{2}) $.
    \item Η $f$ καλείται \textcolor{Col1}{γνησίως αύξουσα} όταν 
      $ \forall x_{1}, x_{2} \in A, \; \text{με} \; x_{1} < x_{2} $, έχουμε ότι 
      $ f(x_{1}) < f(x_{2}) $.
    \item Η $f$ καλείται \textcolor{Col1}{φθίνουσα} όταν 
      $ \forall x_{1}, x_{2} \in A, \; \text{με} \; x_{1} < x_{2} $, έχουμε ότι 
      $ f(x_{1}) \geq f(x_{2}) $.
    \item Η $f$ καλείται \textcolor{Col1}{γνησίως φθίνουσα} όταν 
      $ \forall x_{1}, x_{2} \in A, \; \text{με} \; x_{1} < x_{2} $, έχουμε ότι 
      $ f(x_{1}) > f(x_{2}) $.
  \end{myitemize}
\end{dfn}

\section{Φραγμένες συναρτήσεις}

\begin{dfn}
  Έστω $ A, B \subseteq \mathbb{R} $ και έστω $ f \colon A \to B $ συνάρτηση. 
  \begin{myitemize}
    \item Η $f$ καλείται \textcolor{Col1}{άνω φραγμένη} όταν υπάρχει $ M \in \mathbb{R} $ τέτοιο ώστε 
      $ f(x) \leq M, \quad \forall x \in A $. Ο αριθμός $M$ καλείται
      \textcolor{Col1}{άνω φράγμα} της $f$.
    \item Η $f$ καλείται \textcolor{Col1}{κάτω φραγμένη} όταν υπάρχει $ m \in \mathbb{R} $ τέτοιο ώστε 
      $ f(x) \geq M, \quad \forall x \in A $. Ο αριθμός $m$ καλείται
      \textcolor{Col1}{κάτω φράγμα} της $f$.
    \item Η $f$ καλείται \textcolor{Col1}{φραγμένη} όταν είναι άνω και κάτω φραγμένη.
  \end{myitemize}
\end{dfn}

\section{Άρτιες και περιττές συναρτήσεις}

\begin{dfn}
  Έστω $ A \subseteq \mathbb{R} $ και έστω $ f \colon A \to \mathbb{R} $ συνάρτηση. 
\item Η $f$ καλείται \textcolor{Col1}{άρτια} όταν $ -x \in A, \; \forall x \in A $ και $ f(-x) = f(x), \;
  \forall x \in A $.
\end{dfn}

\begin{rem}
  Αν η $f$ είναι άρτια, τότε η γραφική της παράσταση είναι συμμετρική ως προς τον
  \textbf{άξονα $\mathbf{yy'}$}.
\end{rem}

\begin{dfn}
  Έστω $ A \subseteq \mathbb{R} $ και έστω $ f \colon A \to \mathbb{R} $ συνάρτηση. 
\item Η $f$ καλείται \textcolor{Col1}{περιττή} όταν $ -x \in A, \; \forall x \in A $ και $ f(-x) = -f(x), \;
  \forall x \in A $.
\end{dfn}

\begin{rem}
  Αν η $f$ είναι περιττή, τότε η γραφική της παράσταση είναι συμμετρική ως προς την 
  \textbf{αρχή των αξόνων}.
\end{rem}

\section{Τοπικά και Ολικά ακρότατα}

\begin{dfn}
  Έστω $ A \subseteq \mathbb{R} $ και $ f \colon A \to \mathbb{R} $ συνάρτηση και 
  έστω $ x_{0} \in A $.
  \begin{myitemize}
  \item Η $f$ παρουσιάζει στο $ x_{0} \in A $, \textcolor{Col1}{ολικό μέγιστο} όταν $ f(x) \leq
      f(x_{0}), \; \forall x \in A $.
    \item Η $f$ παρουσιάζει στο $ x_{0} \in A $, \textcolor{Col1}{ολικό ελάχιστο} όταν $ f(x) \geq
      f(x_{0}), \; \forall x \in A $.
  \end{myitemize}
\end{dfn}

\begin{dfn}
  Έστω $ A \subseteq \mathbb{R} $ και $ f \colon A \to \mathbb{R} $ συνάρτηση και 
  έστω $ x_{0} \in A $.
  \begin{myitemize}
    \item Η $f$ παρουσιάζει στο $ x_{0} \in A $, \textcolor{Col1}{τοπικό μέγιστο} όταν $ \exists \delta >
      0 \; : \; f(x) \leq f(x_{0}), \; \forall x \in (x_{0}- \delta , x_{0}+ \delta)
      \cap A  $.
    \item Η $f$ παρουσιάζει στο $ x_{0} \in A $, \textcolor{Col1}{τοπικό ελάχιστο} όταν $ \exists \delta >
      0 \; : \; f(x) \geq f(x_{0}), \; \forall x \in (x_{0}- \delta , x_{0}+ \delta)
      \cap A  $.
  \end{myitemize}
\end{dfn}

\begin{rems}
\item {}
  \begin{myitemize}
    \item Τα τοπικά μέγιστα και τοπικά ελάχιστα, καλούνται \textcolor{Col1}{τοπικά
      ακρότατα}.
    \item Ολικό μέγιστο και ολικό ελάχιστο, όταν υπάρχουν, είναι \textbf{μοναδικά}.
  \end{myitemize}
\end{rems}

\section{Περιοδική συνάρτηση}

\begin{dfn}
  Έστω $ A \subseteq \mathbb{R} $ και έστω $ f \colon A \to \mathbb{R} $ συνάρτηση. 
  και $ f(x+p) = f(x), \; \forall x \in A $. Ο αριθμός $ p $ καλείται
  \textcolor{Col1}{περίοδος} της $f$. 
  Ο \textbf{ελάχιστος} αριθμός $ p $ καλείται \textcolor{Col1}{κύρια περίοδος} της $f$ (δεν υπάρχει πάντα
  ελάχιστος αριθμός $p$). 
\end{dfn}

\section{Σύνθεση συναρτήσεων}

Έστω $ A, B \subseteq \mathbb{R} $, $ f \colon A \to \mathbb{R} $ και $ g \colon
B \to \mathbb{R} $ με $ f(A) \cap B \neq \emptyset $. θεωρούμε το σύνολο $ A_{1} = \{
x \in A \; : \; f(x) \in B \} $. Η σύνθεση των $f$ και $g$ συμβολίζεται με $ g \circ f $
και είναι μια συνάρτηση με πεδίο ορισμού το $ A_{1} $, τέτοια ώστε $ (g \circ f)(x) = 
g(f(x)), \; \forall x \in A_{1} $.

\section{Αντίστροφη συνάρτηση}

\begin{dfn}
  Έστω $ A \subseteq \mathbb{R} $ και $ f \colon A \to \mathbb{R} $ μια ένα προς ένα 
  συνάρτηση (ώστε $ \forall y \in f(A), \; \exists! \, x \in A \; : \; f(x)=y $). Τότε
  η αντίστροφη συνάρτηση συμβολίζεται με $ f^{-1} $, έχει πεδίο ορισμού το $ f(A) $ 
  και ορίζεται σύμφωνα με τη σχέση $ f(x)=y \Leftrightarrow f^{-1}(y)=x $.
\end{dfn}

\begin{prop}[Ιδιότητες]
\item {}
  \begin{myitemize}
    \item $ f^{-1}(f(x))= x $
    \item $ f(f^{-1}(y))= y $
  \end{myitemize}
\end{prop}


\section{Πράξεις συναρτήσεων}

\begin{dfn}
  Έστω $ A, B \subseteq \mathbb{R} $ με $ A \cap B \neq \emptyset $ και έστω $ f
  \colon A \to \mathbb{R} $ και $ g \colon B \to \mathbb{R} $ συναρτήσεις. Τότε:
  \begin{myitemize}
    \item $ f+g \colon A \cap B \to \mathbb{R} $ με 
      $ (f+g)(x)= f(x)+g(x), \; \forall x \in A \cap B $
    \item $ \lambda f \colon A \to \mathbb{R} $ με 
      $ (\lambda f)(x)= \lambda f(x), \; \forall x \in A $
    \item $ f\cdot g \colon A \cap B \to \mathbb{R} $ με 
      $ (f\cdot g)(x)= f(x)\cdot g(x), \; \forall x \in A \cap B $
    \item  $ \frac{f}{g} \colon A_{1} \to \mathbb{R} $ με 
      $ \left(\frac{f}{g}\right)(x)= \frac{f(x)}{g(x)}, \; \forall x \in A_{1} $,
      όπου $ A_{1} = \{ x \in A \cap B \; : \; g(x) \neq 0 \} $.
  \end{myitemize}
\end{dfn}

\section{Ισότητα συναρτήσεων}

\begin{dfn}
  Έστω $ A, B \subseteq \mathbb{R} $, $ f \colon A \to \mathbb{R} $ και 
  $ g \colon B \to \mathbb{R} $ συναρτήσεις. Λέμε ότι οι συναρτήσεις $f$ και $g$ είναι
  ίσες και γράφουμε $ f=g $, όταν: 
  \begin{myitemize}
    \item $ A=B $
    \item $f(x) = g(x), \; \forall x \in A = B $
  \end{myitemize}
\end{dfn}

\begin{exercise}
  Έστω $ f \colon \mathbb{R} \to \mathbb{R} $ συνάρτηση, τέτοια ώστε $ f(f(x)) +
  (f(x))^{3} = 5x+4, \; \forall x \in \mathbb{R} $. Τότε:
  \begin{enumerate}[i)]
    \item Να δείξετε ότι η $f$ είναι ένα προς ένα.
    \item Να λυθεί η εξίσωση $ f(x^{2}-3x)=f(2x-6) $
  \end{enumerate}
\end{exercise}
\begin{solution}
\item {}
  \begin{enumerate}[i)]
    \item Έστω $ x_{1}, x_{2} \in \mathbb{R} $ τέτοια ώστε $ f(x_{1}) = f(x_{2}) $. 
      Θα δείξουμε ότι $ x_{1}= x_{2} $. Πράγματι, έχουμε:
      \[
        f(f(x_{1})) = f(f(x_{2})) \quad \text{και} \quad (f(x_{1}))^{3} = (f(x_{2}))^{3}
      \] 
      τότε με πρόσθεση, κατά μέλη, έχουμε:
      \[
        f(f(x_{1})) + (f(x_{1}))^{3} = f(f(x_{2})) + (f(x_{2}))^{3} \Leftrightarrow 
        5 x_{1} + 4 = 5 x_{2} + 4 \Leftrightarrow x_{1} = x_{2}
      \] 
    \item Επειδή η $f$ είναι ένα προς ένα, τότε ορίζεται η αντίστροφη συνάρτηση και 
      έχουμε:
      \[
        f(x^{2}-3x) = f(2x-6) \xRightarrow{f^{-1}} x^{2}-3x=2x-6 \Rightarrow x^{2}-5x+6=0
        \Rightarrow x=-3 \quad \text{ή} \quad x=-2 
       \] 
  \end{enumerate}
\end{solution}

%todo να γράψω παραδείγματα και σχήματα (γραφ. παραστάσεις ) για όλες τις συναρτήσεις


\end{document}
