\documentclass{book}

\usepackage{etex}
\usepackage{etoolbox}

%%%%%%%%%%%%%%%%%%%%%%%%%%%%%%%%%%%%%%
% Babel language package
%\usepackage[english,greek]{babel}
% Inputenc font encoding
%\usepackage[utf8]{inputenc}


% \usepackage{xltxtra} 
% \usepackage{xgreek} 
% \setmainfont[Mapping=tex-text]{GFS Didot} 

%\usepackage{kmath,kerkis} % The order of the packages matters; kmath changes the default text font
%\usepackage[T1]{fontenc}
\usepackage{ifxetex}
\ifxetex
    % IF XELATEX

% MINION new font
\usepackage{fontspec}
%\usepackage{mathspec}

\setmainfont[Extension=.ttf,UprightFont=*-Regular,BoldFont=*-Bold,ItalicFont=*-Italic,BoldItalicFont=*-Bold-Italic]{Minion-Pro}
\setsansfont[Extension=.ttf,UprightFont=*H,BoldFont=*HB,ItalicFont=*HI,BoldItalicFont=*HBI]{Vera}
\usepackage{unicode-math}
\setmathfont{latinmodern-math.otf}
%\setmathfont[range=\mathup]{Minion-Pro-Regular.ttf}
%\setmathfont[range=\mathit]{Minion-Pro-Italic.ttf}
%\setmathfont[range=\mathbf]{Minion-Pro-Bold.ttf}
%\setmathfont[range={0222B}]{Minion-Pro-Italic.ttf}
%\setmathfontface\mathfoo{Minion-Pro-Regular.ttf}
%\setoperatorfont\mathfoo
    \usepackage{polyglossia}
    \setdefaultlanguage{greek}
    \setotherlanguage{english}
    %\RequirePackage{unicode-math}
    %\setmathfont{Latin Modern Math}
    %\newcommand{\smblkcircle}{•}
\else
    % IF PDFLATEX
    %\usepackage{tgheros}
    %\renewcommand*\familydefault{\sfdefault}
    %\usepackage[eulergreek]{sansmath}
    %\sansmath
    \usepackage[T1]{fontenc}
    \usepackage[utf8]{inputenc}
    \usepackage[english,greek]{babel}
\fi


\usepackage{anyfontsize}
\newlength{\FONTmain}\setlength{\FONTmain}{9pt}
\newlength{\FONTmainbl}\setlength{\FONTmainbl}{1.2\FONTmain}
\renewcommand{\tiny}        {\fontsize{0.6\FONTmain}{0.6\FONTmainbl}\selectfont}
\renewcommand{\scriptsize}  {\fontsize{0.7\FONTmain}{0.7\FONTmainbl}\selectfont}
\renewcommand{\footnotesize}{\fontsize{0.8\FONTmain}{0.8\FONTmainbl}\selectfont}
\renewcommand{\small}       {\fontsize{0.9\FONTmain}{0.9\FONTmainbl}\selectfont}
\renewcommand{\normalsize}  {\fontsize{1.0\FONTmain}{1.0\FONTmainbl}\selectfont}
\renewcommand{\large}       {\fontsize{1.2\FONTmain}{1.2\FONTmainbl}\selectfont}
\renewcommand{\Large}       {\fontsize{1.4\FONTmain}{1.4\FONTmainbl}\selectfont}
\renewcommand{\LARGE}       {\fontsize{1.6\FONTmain}{1.6\FONTmainbl}\selectfont}
\renewcommand{\huge}        {\fontsize{1.8\FONTmain}{1.8\FONTmainbl}\selectfont}

%%%%%%%%%%%%%%%%%%%%%%%%%%%%%%%%%%%%%%
\usepackage[table,RGB]{xcolor}

\usepackage{geometry}
\geometry{a5paper,top=15mm,bottom=15mm,left=15mm,right=15mm}
\setlength{\parindent}{0pt}


%\usepackage{extsizes}
\usepackage{multicol}

%%%%% math packages %%%%%%%%%%%%%%%%%%
\usepackage[intlimits]{amsmath}
\usepackage{amssymb}
\usepackage{amsfonts}
\usepackage{amsthm}
\usepackage{proof}
\usepackage{mathtools}
\usepackage{extarrows}

\usepackage[italicdiff]{physics}
\usepackage{siunitx}
\usepackage{xfrac}

%%%%%%% symbols packages %%%%%%%%%%%%%%
\usepackage{bm} %for use \bm instead \boldsymbol in math mode
\usepackage{dsfont}
%\usepackage{stmaryrd}
%%%%%%%%%%%%%%%%%%%%%%%%%%%%%%%%%%%%%%%


%%%%%% graphics %%%%%%%%%%%%%%%%%%%%%%%
\usepackage{graphicx}
%\usepackage{color}
%\usepackage{xypic}
%\usepackage[all]{xy}
%\usepackage{calc}

%%%%%% tables %%%%%%%%%%%%%%%%%%%%%%%%%
\usepackage{array}
\usepackage{booktabs}
\usepackage{multirow}
\usepackage{makecell}
\usepackage{minibox}
\usepackage{systeme}
%%%%%%%%%%%%%%%%%%%%%%%%%%%%%%%%%%%%%%%

\usepackage{enumitem}
\usepackage{tikz}
\usetikzlibrary{shapes,angles,calc,arrows,arrows.meta,quotes,intersections}
\usetikzlibrary{decorations.pathmorphing}
\usetikzlibrary{decorations.pathreplacing} 
\usetikzlibrary{decorations.markings,patterns} 
\usepackage{pgfplots}
\pgfplotsset{compat=1.15}

\tikzset{dot/.style={ draw, fill, circle, inner sep=1pt, minimum size=3pt }}
\usepackage{fancyhdr}
%%%%% header and footer rule %%%%%%%%%
%\setlength{\headheight}{14pt}
\renewcommand{\headrulewidth}{0pt}
\renewcommand{\footrulewidth}{0pt}
\fancypagestyle{plain}{\fancyhf{}\rfoot{\thepage}}
\fancypagestyle{vangelis}{\fancyhf{}
    \fancyfootoffset[LE,RO]{10mm}
    \rfoot[]{\thepage}
    \lfoot[\thepage]{}
    \rhead[]{\tikz[remember picture,overlay]{\node[rotate=90,anchor=east] (text) at ([shift={(-5mm,-8mm)}]current page.north east) {\textcolor{Col\thechapter}{\small\strut\leftmark}};
    \fill[Col\thechapter] ([xshift={-2.5mm}]text.east) rectangle++(5mm,5mm);
    }}
    %\lhead[\textcolor{Col\thechapter}{\leftmark}]{}
}
%%%%%%%%%%%%%%%%%%%%%%%%%%%%%%%%%%%%%%%

\usepackage[space]{grffile}


% \definecolor{Col1}{HTML}{eb3b79}
% \definecolor{Col2}{HTML}{9a529f}
% \definecolor{Col3}{HTML}{775ba6}
% \definecolor{Col4}{HTML}{5a68b0}
% \definecolor{Col5}{HTML}{55a0d8}
% \definecolor{Col6}{HTML}{34b0e5}
% \definecolor{Col7}{HTML}{34c1d7}
% \definecolor{Col8}{HTML}{65bc6a}
% \definecolor{Col9}{HTML}{9acb62}
% \definecolor{Col10}{HTML}{d1dd5b}
% \definecolor{Col11}{HTML}{f9ec5d}
% \definecolor{Col12}{HTML}{fbc82a}
% \definecolor{Col13}{HTML}{faa725}
% \definecolor{Col14}{HTML}{f26f47}
% \definecolor{Col15}{HTML}{8e6d65}
% \definecolor{Col16}{HTML}{bdbcbc}
% \definecolor{Col17}{HTML}{79919d}

\definecolor{Col1}{rgp}{0.74, 0.2, 0.64}
\definecolor{Col2}{rgp}{0.0, 0.55, 0.55}
\definecolor{Col3}{rgp}{0.74, 0.2, 0.64}
\definecolor{Col4}{rgp}{0.0, 0.55, 0.55}
\definecolor{Col5}{rgp}{0.74, 0.2, 0.64}
\definecolor{Col6}{rgp}{0.0, 0.55, 0.55}
\definecolor{Col7}{rgp}{0.74, 0.2, 0.64}
\definecolor{Col8}{rgp}{0.0, 0.55, 0.55}
\definecolor{Col9}{rgp}{0.74, 0.2, 0.64}
\definecolor{Col10}{rgp}{0.0, 0.55, 0.55}

\everymath{\displaystyle}

\usepackage[most]{tcolorbox}

\usepackage[explicit]{titlesec}
%%%%%% titlesec settings %%%%%%%%%%%%%
% \titleformat{ command }[ shape ]{ format }{ label }{ sep }{ before-code }[ after-code 
% \titlespacing*{ command }{ left }{ before-sep }{ after-sep }[ right-sep ]
% Chapter


% \titleformat{\chapter}[block]{\huge\bfseries}{\begin{tcolorbox}[colback=Col\thechapter,left=3pt,right=3pt,top=18pt,bottom=18pt,sharp
% corners,boxrule=0pt]\centering\huge\bfseries\textcolor{white}{#1}\end{tcolorbox}}{0pt}{\markboth{#1}}[\clearpage]
% \titlespacing*{\chapter}{0cm}{6\baselineskip}{0\baselineskip}[0ex]
% % Section
% \titleformat{\section}[hang]{\pagestyle{plain}\Large\bfseries\centering}{\begin{tcolorbox}[colback=Col\thechapter!75!white,left=1pt,right=1pt,top=2pt,bottom=2pt,sharp
% corners,boxrule=0pt]\centering\strut\textcolor{white}{#1}\end{tcolorbox}}{0ex}{}
% \titlespacing*{\section}{0cm}{2\baselineskip}{\baselineskip}[0ex]
% % subsection
% \titleformat{\subsection}[hang]{\pagestyle{plain}\large\bfseries\centering}{\begin{tcolorbox}[colback=Col\thechapter!55!white,left=1pt,right=1pt,top=2pt,bottom=2pt,sharp
% corners,boxrule=0pt]\centering\strut\textcolor{white}{#1}\end{tcolorbox}}{0ex}{}
% \titlespacing*{\section}{0cm}{2\baselineskip}{\baselineskip}[0ex]
% % Subsubsection
% \titleformat{\subsubsection}[hang]{\normalsize\bfseries\centering}{}{0ex}{\color{Col\thechapter!45}{#1}}{}
% \titlespacing*{\subsubsection}{0cm}{\baselineskip}{\baselineskip}[0ex]

%% Subsection
%\titleformat{\subsection}[hang]{\large\bfseries\centering}{\tcbox[colback=Col\thechapter!50!white,left=1pt,right=1pt,top=1pt,bottom=1pt,sharp corners]{#1}}{0ex}{}
%\titlespacing*{\subsection}{0cm}{2\baselineskip}{\baselineskip}[0ex]
%% Subsubsection
%\titleformat{\subsubsection}[hang]{\normalsize\bfseries\centering}{}{0ex}{\color{Col\thechapter}{#1}}{}
%\titlespacing*{\subsubsection}{0cm}{\baselineskip}{\baselineskip}[0ex]
%%%%%%%%%%%%%%%%%%%%%%%%%%%%%%%%%%%%%%%




\AtBeginDocument{\pagestyle{vangelis}\normalsize\raggedright}


\newcommand{\twocolumnside}[2]{\begin{minipage}[t]{0.45\linewidth}\raggedright
#1
\end{minipage}\hfill{\color{Col\thechapter}{\vrule width 1pt}}\hfill\begin{minipage}[t]{0.45\linewidth}\raggedright
#2
\end{minipage}
}

\newcommand{\twocolumnsides}[2]{\begin{minipage}[t]{0.45\linewidth}\raggedright
#1
\end{minipage}\hfill\begin{minipage}[t]{0.45\linewidth}\raggedright
#2
\end{minipage}
}

\newcommand{\twocolumnsidesc}[2]{\begin{minipage}{0.45\linewidth}\raggedright
#1
\end{minipage}\hfill\begin{minipage}[c]{0.45\linewidth}\raggedright
#2
\end{minipage}
}

\newcommand{\twocolumnsidesp}[2]{\begin{minipage}[t]{0.35\linewidth}\raggedright
#1
\end{minipage}\hfill\begin{minipage}[t]{0.55\linewidth}\raggedright
#2
\end{minipage}
}

\newcommand{\twocolumnsidesl}[2]{\begin{minipage}[t]{0.55\linewidth}\raggedright
#1
\end{minipage}\hfill\begin{minipage}[t]{0.35\linewidth}\raggedright
#2
\end{minipage}
}


\usepackage{calc}
\usepackage{array}
\definecolor{TabLine}{RGB}{254,254,254}
\newcommand{\TabRowHead}{\rowcolor{TabHeadRow}}
\newcommand{\TabRowHeadCor}{\cellcolor{white}}
\newcommand{\TabRowHCol}{\color{white}\bfseries\boldmath}
\newcommand{\TabCellHead}{\cellcolor{TabHeadRow}\TabRowHCol}
\newenvironment{Mytable}%
    {\begingroup\setlength{\arrayrulewidth}{2pt}\arrayrulecolor{TabLine}
    \colorlet{TabHeadRow}{Col\thechapter}
    \colorlet{TabRowOdd}{Col\thechapter!50!white}
    \colorlet{TabRowEven}{Col\thechapter!25!white}
    \rowcolors{1}{TabRowOdd}{TabRowEven}
    }%
    {\endgroup

}

\usepackage{fancyhdr}
%%%%% header and footer rule %%%%%%%%%
\setlength{\headheight}{14pt}
\renewcommand{\headrulewidth}{0pt}
\renewcommand{\footrulewidth}{0pt}
\fancypagestyle{plain}{\fancyhf{}
\fancyhead{}
\lfoot{\small \hrule \vspace{5pt}\color{Col1} Βαγγέλης Σαπουνάκης}
\cfoot{\small \hrule \vspace{5pt}\color{Col2!75} Φοιτητικό Πρόσημο}
\rfoot{\small \hrule \vspace{5pt} \thepage}}
\fancypagestyle{vangelis}{\fancyhf{}
\lfoot{\small \hrule \vspace{5pt}\color{Col1} Βαγγέλης Σαπουνάκης}
\cfoot{\small \hrule \vspace{5pt}\color{Col2!75} Φοιτητικό Πρόσημο}
\rfoot{\small \hrule \vspace{5pt} \thepage}}

%%%%%%%%%%%%Watermark%%%%%%%%%%%%%%%%%%
 \usepackage[printwatermark]{xwatermark} 
 \newwatermark[allpages,color=blue!8,angle=45,scale=3,xpos=0,ypos=0]{ΠΡΟΣΗΜΟ}
%%%%%%%%%%%%%%%%%%%%%%%%%%%%%%%%%%%%%%

%%%%%%%%% mdframed theorem boxes, breakable and with ref support %%%%%%%%

\usepackage[framemethod=TikZ]{mdframed}

\mdfdefinestyle{mythm}{innertopmargin=0pt,linecolor=Col2!75,linewidth=2pt,
  backgroundcolor=Col2!15, %background color of the box
  shadow=false,shadowcolor=Col2,shadowsize=5pt,% shadows
  frametitleaboveskip=\dimexpr-1.3\ht\strutbox\relax, 
  frametitlealignment={\hspace*{0.03\linewidth}},%
}

\mdfdefinestyle{mydfn}{innertopmargin=0pt,linecolor=Col1!75,linewidth=2pt,
  backgroundcolor=Col1!15, %background color of the box
  shadow=false,shadowcolor=Col1,shadowsize=5pt,% shadows
  frametitleaboveskip=\dimexpr-1.3\ht\strutbox\relax, 
  frametitlealignment={\hspace*{0.03\linewidth}},%
}

\mdfdefinestyle{myprop}{innertopmargin=0pt,linecolor=blue!75,linewidth=2pt,
  backgroundcolor=blue!10, %background color of the box
  shadow=false,shadowcolor=blue,shadowsize=5pt,% shadows
  frametitleaboveskip=\dimexpr-1.3\ht\strutbox\relax, 
  frametitlealignment={\hspace*{0.03\linewidth}},%
}

\mdfdefinestyle{myboxs}{innertopmargin=0pt,linecolor=blue!75,linewidth=0pt,
  backgroundcolor=blue!15, %background color of the box
  shadow=false,shadowcolor=blue,shadowsize=5pt,% shadows
}

% \newcounter{theo}[section]
% \setcounter{theo}{0}
% \renewcommand{\thetheo}{\arabic{section}.\arabic{theo}}


\newenvironment{mythm}[2][]{%
  \refstepcounter{thm}
  % Code for box design goes here.
  \ifstrempty{#1}%
    % if condition (without title)
    {\mdfsetup{
  frametitle={%
    \tikz[baseline=(current bounding box.east),outer sep=0pt]
    \node[anchor=east,rectangle,fill=Col2!75,text=white]
  {\strut Θεώρημα~\thethm};},%
      }%
      % else condition (with title)
      }{\mdfsetup{
  frametitle={%
    \tikz[baseline=(current bounding box.east),outer sep=0pt]
    \node[anchor=east,rectangle,fill=Col2!75,text=white]
  {\strut Θεώρημα~\thethm~~({#1})};},%
      }%
    }%
    % Both conditions
    \mdfsetup{
      style=mythm
    }
    \begin{mdframed}[]\relax\label{#2}}{%
  \end{mdframed}}
  %%%%%%%%%%%%%%%%%%%%%%%%%%%%%%%%%%%%%%%%%%%%%%%%%%%%%%%%%%

\newenvironment{mydfn}[2][]{%
  \refstepcounter{thm}
  % Code for box design goes here.
  \ifstrempty{#1}%
    % if condition (without title)
    {\mdfsetup{
  frametitle={%
    \tikz[baseline=(current bounding box.east),outer sep=0pt]
    \node[anchor=east,rectangle,fill=Col1!75,text=white,draw=Col1!75]
  {\strut Ορισμός~\thethm};},%
      }%
      % else condition (with title)
      }{\mdfsetup{
  frametitle={%
    \tikz[baseline=(current bounding box.east),outer sep=0pt]
    \node[anchor=east,rectangle,fill=Col1!75,text=white,draw=Col1!75]
  {\strut Ορισμός~\thethm~~({#1})};},%
      }%
    }%
    % Both conditions
    \mdfsetup{
      style=mydfn
    }
    \begin{mdframed}[]\relax\label{#2}}{%
  \end{mdframed}}
  %%%%%%%%%%%%%%%%%%%%%%%%%%%%%%%%%%%%%%%%%%%%%%%%%%%%%%%%%%

\newenvironment{myprop}[2][]{%
  \refstepcounter{thm}
  % Code for box design goes here.
  \ifstrempty{#1}%
    % if condition (without title)
    {\mdfsetup{
  frametitle={%
    \tikz[baseline=(current bounding box.east),outer sep=0pt]
    \node[anchor=east,rectangle,fill=blue!50,text=white]
  {\strut Πρόταση~\thethm};},%
      }%
      % else condition (with title)
      }{\mdfsetup{
  frametitle={%
    \tikz[baseline=(current bounding box.east),outer sep=0pt]
    \node[anchor=east,rectangle,fill=blue!50,text=white]
  {\strut Πρόταση~\thethm~~({#1})};},%
      }%
    }%
    % both conditions
    \mdfsetup{
      style=myprop
    }
    \begin{mdframed}[]\relax\label{#2}}{%
  \end{mdframed}}
  %%%%%%%%%%%%%%%%%%%%%%%%%%%%%%%%%%%%%%%%%%%%%%%%%%%%%%%%%%
\newenvironment{myboxs}{%
  % Code for box design goes here.
    \mdfsetup{
      style=myboxs
    }
    \begin{mdframed}[]\relax}{%
\end{mdframed}}
  %%%%%%%%%%%%%%%%%%%%%%%%%%%%%%%%%%%%%%%%%%%%%%%%%%%%%%%%%%
% \renewcommand{\qedsymbol}{$\blacksquare$}

\newcommand{\comb}[2]{\lambda_{1}\vec{#1}_{1} + \cdots + \lambda_{#2}\vec{#1}_{#2}}
\newcommand{\combc}[3]{#2_{1}\vec{#1}_{1} + \cdots + #2_{#3}\vec{#1}_{#3}}
\newcommand{\combb}[2]{\lambda_{1}\vec{#1}_{1} + \lambda_{2}\vec{#1}_{2} + \cdots + 
\lambda_{#2}\vec{#1}_{#2}}

\newcommand{\me}{\mathrm{e}}


\newlist{myitemize}{itemize}{3}
\setlist[myitemize]{label=\textcolor{Col1}{\tiny$\blacksquare$},leftmargin=*}

\newlist{myitemize*}{itemize*}{3}
\setlist[myitemize*]{itemjoin=\hspace{2\baselineskip},label=\textcolor{Col1}{\tiny$\blacksquare$}}

\newlist{myenumerate}{enumerate}{3}
\setlist[enumerate,1]{label=\textcolor{Col1}{\theenumi.},leftmargin=*}
\setlist[enumerate,2]{label=\textcolor{Col1}{\roman*)},leftmargin=*}

\setlist[description]{labelindent=1em,widest=Ιανουα0000,labelsep*=1em,itemindent=0pt,leftmargin=*}

% %%%%%%%%%%%%%%%%%% fancy headings %%%%%%%%%%%%%%%%%%

% %%%%%%%%%%%%%%%%%%%%%%% my boxes %%%%%%%%%%%%%%%%%%%%%%%%%%%%
% \newcommand{\mythm}[1]{
%       \refstepcounter{thm}
%     \begin{tikzpicture}
%         \node[myboxthm] (box1) 
%         {
%             \begin{minipage}{0.9\textwidth}
%                 #1
%             \end{minipage}
%         } ;

%         \node[myboxtitlethm] at (box1.north west) {\strut Θεώρημα~\thethm} ;
%     \end{tikzpicture}
% }

% \newcommand{\mythmm}[2]{
%       \refstepcounter{thm}
%     \begin{tikzpicture}
%         \node[myboxthm] (box1) 
%         {
%             \begin{minipage}{0.9\textwidth}
%                 #2
%             \end{minipage}
%         } ;

%         \node[myboxtitlethm] at (box1.north west) {\strut Θεώρημα~\thethm \; (#1)} ;
%     \end{tikzpicture}
% }

% \newcommand{\mydfn}[1]{
%       \refstepcounter{dfn}
%     \begin{tikzpicture}
%         \node[myboxdfn] (box1) 
%         {
%             \begin{minipage}{0.9\textwidth}
%                 #1
%             \end{minipage}
%         } ;

%         \node[myboxtitledfn] at (box1.north west) {\strut Ορισμός~\thedfn} ;
%     \end{tikzpicture}
% }


% \newcommand{\myprop}[1]{
%       \refstepcounter{thm}
%     \begin{tikzpicture}
%         \node[myboxprop] (box1) 
%         {
%             \begin{minipage}{0.9\textwidth}
%                 #1
%             \end{minipage}
%         } ;

%         \node[myboxtitleprop] at (box1.north west) {\strut Πρόταση~\theprop} ;
%     \end{tikzpicture}
% }

% \newcommand{\mypropp}[2]{
%       \refstepcounter{thm}
%     \begin{tikzpicture}
%         \node[myboxprop] (box1) 
%         {
%             \begin{minipage}{0.9\textwidth}
%                 #2
%             \end{minipage}
%         } ;

%         \node[myboxtitleprop] at (box1.north west) {\strut Πρόταση~\theprop (#1)} ;
%     \end{tikzpicture}
% }

%%%\mybrace{<first>}{<second>}[<Optional text>]
\newcommand{\tikzmark}[1]{\tikz[baseline={(#1.base)},overlay,remember picture] \node[outer
sep=0pt, inner sep=0pt] (#1) {\phantom{A}};}
%% syntax
\NewDocumentCommand\mybrace{mmo}{%
  \IfValueTF {#3}{%
    \begin{tikzpicture}[overlay, remember picture,decoration={brace,amplitude=1ex}]
      \draw[decorate,thick] (#1.north east) -- (#2.south east) 
        node (b) [midway,xshift=13pt,label={right=of b}:{#3}] {};
    \end{tikzpicture}%
  }%
  {%
    \begin{tikzpicture}[overlay, remember picture,decoration={brace,amplitude=1ex}]
      \draw[decorate,thick] (#1.north east) -- (#2.south east);
    \end{tikzpicture}%
  }%
}%

%%%%%%%How to use this %%%%%%%%%%%%%%%%%%%%%%%%%
%use \tikzmark{a} and \tikzmark{b} at first and last \item where the brace is
%wanted
%use the following command after \end{enumerate}
%\mybrace{a}{b}[Text comes here to describe these to items and justify for your
%case]]



%%%%% label inline equations and don't allow reference
\newcommand\inlineeqno{\stepcounter{equation} (\theequation)}

%%%%%defines \inlineequation[<label name>]{<equation>}
%%%%%%%%format use \inlineequation[<label name>]{<equation>}%%%%%%%
\makeatletter
\newcommand*{\inlineequation}[2][]{%
    \begingroup
    % Put \refstepcounter at the beginning, because
    % package `hyperref' sets the anchor here.
    \refstepcounter{equation}%
    \ifx\\#1\\%
\else
    \label{#1}%
\fi
% prevent line breaks inside equation
\relpenalty=10000 %
\binoppenalty=10000 %
\ensuremath{%
    % \displaystyle % larger fractions, ...
    #2%
}%
\quad ~\@eqnnum
\endgroup
}
\makeatother


%%%%%%%%%%%%%%%%%% fancy enumitem cicled label %%%%%%%%%%%%%%%%%%
\newcommand*\circled[1]{\tikz[baseline=(char.base)]{
\node[shape=circle,draw,inner sep=0.3pt] (char) {#1};}}
% use it like \begin{enumerate}[label=\protect\circled{\Alph{enumi}}]
%%%\mybrace{<first>}{<second>}[<Optional text>]
%%% wrap with braces list environments




%%%%%%%%%%%%% puts brace under matrix
\newcommand\undermat[2]{%
  \makebox[0pt][l]{$\smash{\underbrace{\phantom{%
\begin{matrix}#2\end{matrix}}}_{\text{$#1$}}}$}#2}


%circle item inside array or matrix
\newcommand\Circle[1]{%
\tikz[baseline=(char.base)]\node[circle,draw,inner sep=2pt] (char) {#1};}


  %redeftine \eqref so that parenthesis () have the color the link
\makeatletter
\renewcommand*{\eqref}[1]{%
  \hyperref[{#1}]{\textup{\tagform@{\ref*{#1}}}}%
}
\makeatother

%removes qedsymbol and additional vertical space at the end 
\makeatletter
\renewenvironment{proof}[1][\proofname]{\par
  % \pushQED{\hfill\qedhere}% <--- remove the QED business
  \normalfont \topsep6\p@\@plus6\p@\relax
  \trivlist
  \item[\hskip\labelsep
        \itshape
        #1\@addpunct{.}]\ignorespaces
}{%
 % \popQED% <--- remove the QED business
  \endtrivlist\@endpefalse
}
\renewcommand\qedhere{$\blacksquare$} % to ensure code portability
\makeatother


\input{tikz.tex}
\input{myboxes.tex}


\begin{document}
\setcounter{chapter}{3}

\chapter{Συναρτήσεις}

\section{Ορισμός συνάρτησης}

\begin{dfn}
  Έστω $ A, B \subseteq \mathbb{R} $. Καλούμε \textcolor{Col1}{συνάρτηση}, από το $A$ 
  στο $B$, κάθε κανόνα $f$ σύμφωνα με τον οποίο σε \textbf{κάθε} στοιχείο $ x \in A $ 
  αντιστοιχεί \textbf{ένα και μόνο ένα} στοιχείο $ y \in B $. 
  \begin{myitemize}
    \item Το σύνολο $A$ καλείται \textcolor{Col1}{Πεδίο Ορισμού} της $f$ και το σύνολο 
      $B$ καλείται \textcolor{Col1}{Πεδίο Τιμών} της $f$.
    \item Τό σύνολο $ f(A) = \{ y \in B \; : \; \exists x \in A \; \text{με} 
      \; f(x)=y \} \subseteq B $, καλείται \textcolor{Col1}{σύνολο τιμών} της $f$.
  \end{myitemize}

  Συμβολικά, γράφουμε $ f \colon A \to B $ ή $ A \xrightarrow{f} B $, 
  ή $ y=f(x), \; x \in A $.  
\end{dfn}

\begin{exercise}
  Να βρεθεί το Πεδίο Ορισμού των συναρτήσεων:
  \begin{enumerate}[i)]
    \item $ f(x) = \frac{\sqrt{x}}{(x-1)(x-3)} $
    \item $ g(x) = \ln{(x-1)} \cdot \mathrm{e}^{\frac{1}{x-4}} $
  \end{enumerate}
\end{exercise}
\begin{solution}
\item {}
  \begin{enumerate}[i)]
    \item Για τη συνάρτηση $f$, έχουμε:
      \begin{align*}
        A_{f} &= \{ x \in \mathbb{R} \; : \; x \geq 0 \quad \text{και} \quad (x-1)(x-3)
        \neq 0 \} \\ 
              &= \{ x \in \mathbb{R} \; : \; x \geq 0 \quad
              \text{και} \quad x \neq 1,3 \} \\ 
              &= [0,1) \cup (1,3) \cup (3, + \infty) 
        \end{align*} 
      \item Για τη συνάρτηση $g$, έχουμε:
        \begin{align*}
          A_{g} &= \{ x \in \mathbb{R} \; : \; x-1 > 0 \quad \text{και} \quad x \neq 4 \}
          \\
                &= \{ x \in \mathbb{R} \; : \; x > 1 \quad \text{και} \quad x \neq 4 \} 
             \\ 
                &= (1,4) \cup (4, + \infty)
        \end{align*} 
    \end{enumerate}
\end{solution}

\section{Ένα προς ένα και επί συναρτήση}

\begin{dfn}
  Έστω $ A, B \subseteq \mathbb{R} $ και έστω $ f \colon A \to B $ συνάρτηση. 
  \begin{myitemize}
    \item Η $f$ καλείται \textcolor{Col1}{επί}, όταν $ f(A) = B $ ή ισοδύναμα όταν 
      $ \forall y \in B, \; \exists x \in A \; : \; f(x)=y $. 
    \item Η $f$ καλείται \textcolor{Col1}{ένα προς ένα} όταν 
      $ \forall x_{1}, x_{2} \in A, \; \text{με} \; x_{1} \neq x_{2} $, έχουμε ότι 
      $ f(x_{1}) \neq f(x_{2}) $ ή ισοδύναμα όταν, 
      $ \forall x_{1}, x_{2} \in A, \; \text{με} \; f(x_{1}) = f(x_{2}) $, έχουμε ότι 
      $ x_{1}= x_{2} $.
  \end{myitemize}
\end{dfn}

\section{Μονοτονία συνάρτησης}

\begin{dfn}
  Έστω $ A, B \subseteq \mathbb{R} $ και έστω $ f \colon A \to B $ συνάρτηση. 
  \begin{myitemize}
    \item Η $f$ καλείται \textcolor{Col1}{αύξουσα} όταν 
      $ \forall x_{1}, x_{2} \in A, \; \text{με} \; x_{1} < x_{2} $, έχουμε ότι 
      $ f(x_{1}) \leq f(x_{2}) $.
    \item Η $f$ καλείται \textcolor{Col1}{γνησίως αύξουσα} όταν 
      $ \forall x_{1}, x_{2} \in A, \; \text{με} \; x_{1} < x_{2} $, έχουμε ότι 
      $ f(x_{1}) < f(x_{2}) $.
    \item Η $f$ καλείται \textcolor{Col1}{φθίνουσα} όταν 
      $ \forall x_{1}, x_{2} \in A, \; \text{με} \; x_{1} < x_{2} $, έχουμε ότι 
      $ f(x_{1}) \geq f(x_{2}) $.
    \item Η $f$ καλείται \textcolor{Col1}{γνησίως φθίνουσα} όταν 
      $ \forall x_{1}, x_{2} \in A, \; \text{με} \; x_{1} < x_{2} $, έχουμε ότι 
      $ f(x_{1}) > f(x_{2}) $.
  \end{myitemize}
\end{dfn}

\section{Φραγμένες συναρτήσεις}

\begin{dfn}
  Έστω $ A, B \subseteq \mathbb{R} $ και έστω $ f \colon A \to B $ συνάρτηση. 
  \begin{myitemize}
    \item Η $f$ καλείται \textcolor{Col1}{άνω φραγμένη} όταν υπάρχει $ M \in \mathbb{R} $ τέτοιο ώστε 
      $ f(x) \leq M, \quad \forall x \in A $. Ο αριθμός $M$ καλείται
      \textcolor{Col1}{άνω φράγμα} της $f$.
    \item Η $f$ καλείται \textcolor{Col1}{κάτω φραγμένη} όταν υπάρχει $ m \in \mathbb{R} $ τέτοιο ώστε 
      $ f(x) \geq M, \quad \forall x \in A $. Ο αριθμός $m$ καλείται
      \textcolor{Col1}{κάτω φράγμα} της $f$.
    \item Η $f$ καλείται \textcolor{Col1}{φραγμένη} όταν είναι άνω και κάτω φραγμένη.
  \end{myitemize}
\end{dfn}

\section{Άρτιες και περιττές συναρτήσεις}

\begin{dfn}
  Έστω $ A \subseteq \mathbb{R} $ και έστω $ f \colon A \to \mathbb{R} $ συνάρτηση. 
\item Η $f$ καλείται \textcolor{Col1}{άρτια} όταν $ -x \in A, \; \forall x \in A $ και $ f(-x) = f(x), \;
  \forall x \in A $.
\end{dfn}

\begin{rem}
  Αν η $f$ είναι άρτια, τότε η γραφική της παράσταση είναι συμμετρική ως προς τον
  \textbf{άξονα $\mathbf{yy'}$}.
\end{rem}

\begin{dfn}
  Έστω $ A \subseteq \mathbb{R} $ και έστω $ f \colon A \to \mathbb{R} $ συνάρτηση. 
\item Η $f$ καλείται \textcolor{Col1}{περιττή} όταν $ -x \in A, \; \forall x \in A $ και $ f(-x) = -f(x), \;
  \forall x \in A $.
\end{dfn}

\begin{rem}
  Αν η $f$ είναι περιττή, τότε η γραφική της παράσταση είναι συμμετρική ως προς την 
  \textbf{αρχή των αξόνων}.
\end{rem}

\section{Τοπικά και Ολικά ακρότατα}

\begin{dfn}
  Έστω $ A \subseteq \mathbb{R} $ και $ f \colon A \to \mathbb{R} $ συνάρτηση και 
  έστω $ x_{0} \in A $.
  \begin{myitemize}
  \item Η $f$ παρουσιάζει στο $ x_{0} \in A $, \textcolor{Col1}{ολικό μέγιστο} όταν $ f(x) \leq
      f(x_{0}), \; \forall x \in A $.
    \item Η $f$ παρουσιάζει στο $ x_{0} \in A $, \textcolor{Col1}{ολικό ελάχιστο} όταν $ f(x) \geq
      f(x_{0}), \; \forall x \in A $.
  \end{myitemize}
\end{dfn}

\begin{dfn}
  Έστω $ A \subseteq \mathbb{R} $ και $ f \colon A \to \mathbb{R} $ συνάρτηση και 
  έστω $ x_{0} \in A $.
  \begin{myitemize}
    \item Η $f$ παρουσιάζει στο $ x_{0} \in A $, \textcolor{Col1}{τοπικό μέγιστο} όταν $ \exists \delta >
      0 \; : \; f(x) \leq f(x_{0}), \; \forall x \in (x_{0}- \delta , x_{0}+ \delta)
      \cap A  $.
    \item Η $f$ παρουσιάζει στο $ x_{0} \in A $, \textcolor{Col1}{τοπικό ελάχιστο} όταν $ \exists \delta >
      0 \; : \; f(x) \geq f(x_{0}), \; \forall x \in (x_{0}- \delta , x_{0}+ \delta)
      \cap A  $.
  \end{myitemize}
\end{dfn}

\begin{rems}
\item {}
  \begin{myitemize}
    \item Τα τοπικά μέγιστα και τοπικά ελάχιστα, καλούνται \textcolor{Col1}{τοπικά
      ακρότατα}.
    \item Ολικό μέγιστο και ολικό ελάχιστο, όταν υπάρχουν, είναι \textbf{μοναδικά}.
  \end{myitemize}
\end{rems}

\section{Περιοδική συνάρτηση}

\begin{dfn}
  Έστω $ A \subseteq \mathbb{R} $ και έστω $ f \colon A \to \mathbb{R} $ συνάρτηση. 
  και $ f(x+p) = f(x), \; \forall x \in A $. Ο αριθμός $ p $ καλείται
  \textcolor{Col1}{περίοδος} της $f$. 
  Ο \textbf{ελάχιστος} αριθμός $ p $ καλείται \textcolor{Col1}{κύρια περίοδος} της $f$ (δεν υπάρχει πάντα
  ελάχιστος αριθμός $p$). 
\end{dfn}

\section{Σύνθεση συναρτήσεων}

Έστω $ A, B \subseteq \mathbb{R} $, $ f \colon A \to \mathbb{R} $ και $ g \colon
B \to \mathbb{R} $ με $ f(A) \cap B \neq \emptyset $. θεωρούμε το σύνολο $ A_{1} = \{
x \in A \; : \; f(x) \in B \} $. Η σύνθεση των $f$ και $g$ συμβολίζεται με $ g \circ f $
και είναι μια συνάρτηση με πεδίο ορισμού το $ A_{1} $, τέτοια ώστε $ (g \circ f)(x) = 
g(f(x)), \; \forall x \in A_{1} $.

\section{Αντίστροφη συνάρτηση}

\begin{dfn}
  Έστω $ A \subseteq \mathbb{R} $ και $ f \colon A \to \mathbb{R} $ μια ένα προς ένα 
  συνάρτηση (ώστε $ \forall y \in f(A), \; \exists! \, x \in A \; : \; f(x)=y $). Τότε
  η αντίστροφη συνάρτηση συμβολίζεται με $ f^{-1} $, έχει πεδίο ορισμού το $ f(A) $ 
  και ορίζεται σύμφωνα με τη σχέση $ f(x)=y \Leftrightarrow f^{-1}(y)=x $.
\end{dfn}

\begin{prop}[Ιδιότητες]
\item {}
  \begin{myitemize}
    \item $ f^{-1}(f(x))= x $
    \item $ f(f^{-1}(y))= y $
  \end{myitemize}
\end{prop}


\section{Πράξεις συναρτήσεων}

\begin{dfn}
  Έστω $ A, B \subseteq \mathbb{R} $ με $ A \cap B \neq \emptyset $ και έστω $ f
  \colon A \to \mathbb{R} $ και $ g \colon B \to \mathbb{R} $ συναρτήσεις. Τότε:
  \begin{myitemize}
    \item $ f+g \colon A \cap B \to \mathbb{R} $ με 
      $ (f+g)(x)= f(x)+g(x), \; \forall x \in A \cap B $
    \item $ \lambda f \colon A \to \mathbb{R} $ με 
      $ (\lambda f)(x)= \lambda f(x), \; \forall x \in A $
    \item $ f\cdot g \colon A \cap B \to \mathbb{R} $ με 
      $ (f\cdot g)(x)= f(x)\cdot g(x), \; \forall x \in A \cap B $
    \item  $ \frac{f}{g} \colon A_{1} \to \mathbb{R} $ με 
      $ \left(\frac{f}{g}\right)(x)= \frac{f(x)}{g(x)}, \; \forall x \in A_{1} $,
      όπου $ A_{1} = \{ x \in A \cap B \; : \; g(x) \neq 0 \} $.
  \end{myitemize}
\end{dfn}

\section{Ισότητα συναρτήσεων}

\begin{dfn}
  Έστω $ A, B \subseteq \mathbb{R} $, $ f \colon A \to \mathbb{R} $ και 
  $ g \colon B \to \mathbb{R} $ συναρτήσεις. Λέμε ότι οι συναρτήσεις $f$ και $g$ είναι
  ίσες και γράφουμε $ f=g $, όταν: 
  \begin{myitemize}
    \item $ A=B $
    \item $f(x) = g(x), \; \forall x \in A = B $
  \end{myitemize}
\end{dfn}

\begin{exercise}
  Έστω $ f \colon \mathbb{R} \to \mathbb{R} $ συνάρτηση, τέτοια ώστε $ f(f(x)) +
  (f(x))^{3} = 5x+4, \; \forall x \in \mathbb{R} $. Τότε:
  \begin{enumerate}[i)]
    \item Να δείξετε ότι η $f$ είναι ένα προς ένα.
    \item Να λυθεί η εξίσωση $ f(x^{2}-3x)=f(2x-6) $
  \end{enumerate}
\end{exercise}
\begin{solution}
\item {}
  \begin{enumerate}[i)]
    \item Έστω $ x_{1}, x_{2} \in \mathbb{R} $ τέτοια ώστε $ f(x_{1}) = f(x_{2}) $. 
      Θα δείξουμε ότι $ x_{1}= x_{2} $. Πράγματι, έχουμε:
      \[
        f(f(x_{1})) = f(f(x_{2})) \quad \text{και} \quad (f(x_{1}))^{3} = (f(x_{2}))^{3}
      \] 
      τότε με πρόσθεση, κατά μέλη, έχουμε:
      \[
        f(f(x_{1})) + (f(x_{1}))^{3} = f(f(x_{2})) + (f(x_{2}))^{3} \Leftrightarrow 
        5 x_{1} + 4 = 5 x_{2} + 4 \Leftrightarrow x_{1} = x_{2}
      \] 
    \item Επειδή η $f$ είναι ένα προς ένα, τότε ορίζεται η αντίστροφη συνάρτηση και 
      έχουμε:
      \[
        f(x^{2}-3x) = f(2x-6) \xRightarrow{f^{-1}} x^{2}-3x=2x-6 \Rightarrow x^{2}-5x+6=0
        \Rightarrow x=-3 \quad \text{ή} \quad x=-2 
       \] 
  \end{enumerate}
\end{solution}

%todo να γράψω παραδείγματα και σχήματα (γραφ. παραστάσεις ) για όλες τις συναρτήσεις


\end{document}
