\documentclass[a4paper,table]{report}
\input{preamble_typ.tex}
\input{definitions_typ.tex}
\input{tikz.tex}


\pagestyle{vangelis}

\begin{document}

\chapter{Σειρές}

\section{Βασικές Προτάσεις}
\twocolumnside{
  \begin{myitemize}[itemsep=.5\parskip]
    \item $ \sum_{n= n_{0}}^{\infty} a_{n} $ συγκλίνει $ \Leftrightarrow
      \sum_{n=n_{1}}^{\infty} a_{n} $ συγκλίνει, $ \; \forall n_{1} > n_{0} $
    \item $ \sum_{n= n_{0}}^{\infty} a_{n} = \pm\infty $ $ \Leftrightarrow
      \sum_{n=n_{1}}^{\infty} a_{n} = \pm\infty $, $ \; \forall n_{1} > n_{0} $
    \item $ a_{n} \geq 0, \; \forall n \in \mathbb{N} \Rightarrow \sum_{n=1}^{\infty}
      a_{n} = S \in \mathbb{R_{+}} \; \text{ή} \; \sum_{n=1}^{\infty} a_{n} = + \infty 
      $
  \end{myitemize}
  }{
  \begin{myitemize}[itemsep=.5\parskip]
    \item $ \sum_{n= n_{0}}^{\infty} a_{n} $ συγκλίνει $ \Rightarrow \lim_{n \to \infty}
      a_{n} = 0 $
    \item $ \sum_{n=1}^{\infty} a_{n} = a, \; \sum_{n=1}^{\infty} b_{n} = b 
      \Rightarrow \sum_{n=1}^{\infty} (ka_{n}+ \lambda b_{n}) = ka + \lambda b $
    \item $ \sum_{n=1}^{\infty} a_{n} $ συγκλ. και $ \sum_{n=1}^{\infty} b_{n} $ 
      αποκλ.  $ \Rightarrow \sum_{n=1}^{\infty} (a_{n}+ b_{n}) $ αποκλ.
  \end{myitemize}
}

\section{Κριτήρια Σύγκλισης}

\subsection{Κριτήρια Σειρών Θετικών Όρων}

\twocolumnsides{
  \subsection{Κριτήριο Σύγκρισης}
  \begin{myitemize}
    \item $ a_{n} \leq b_{n}, \; \forall n \geq n_{0} \quad \text{και} \quad
      \sum_{n=1}^{\infty} b_{n} $ συγκλ. $ \Rightarrow \sum_{n=1}^{\infty} a_{n} $
      συγκλ.
    \item $ a_{n} \geq b_{n}, \; \forall n \geq n_{0} \quad \text{και} \quad
      \sum_{n=1}^{\infty} b_{n} $ αποκλ. $ \Rightarrow \sum_{n=1}^{\infty} a_{n} $
      αποκλ.
  \end{myitemize}
  }{
  \subsection{Γενικευμένη Αρμονική}
  \centering
  $ \sum_{n=1}^{\infty} \frac{1}{n^{p}} $ συγκλίνει $ \Leftrightarrow p > 1 $
  \subsection{Γεωμετρική Σειρά}
  \centering
  $ \sum_{n=1}^{\infty} \lambda ^{n} $ συγκλίνει $ \Leftrightarrow \abs{\lambda} < 1 $
}


\twocolumnsides{
  \subsection{Κριτήριο Λόγου (D' Alembert)}
    Αν $ \lim_{n \to \infty} \frac{a_{n+1}}{a_{n}} = l \geq 0 $, τότε
      $ 
      \begin{cases}
        l < 1 \Rightarrow \sum_{n=1}^{\infty} a_{n} & \text{συγκλίνει} \\
        l > 1 \Rightarrow \sum_{n=1}^{\infty} a_{n} & \text{αποκλίνει} \\
        l = 1 \Rightarrow \sum_{n=1}^{\infty} a_{n} & \text{δεν ξέρουμε} \\
      \end{cases}$
  }{
  \subsection{Κριτήριο Ρίζας (Cauchy)}
    Αν $ \lim_{n \to \infty} \sqrt[n]{a_{n}} = l \geq 0 $, τότε
      $ 
      \begin{cases}
        l < 1 \Rightarrow \sum_{n=1}^{\infty} a_{n} & \text{συγκλίνει} \\
        l > 1 \Rightarrow \sum_{n=1}^{\infty} a_{n} & \text{αποκλίνει} \\
        l = 1 \Rightarrow \sum_{n=1}^{\infty} a_{n} & \text{δεν ξέρουμε} \\
      \end{cases}$
}


\subsection{Κριτήριο Ορίου}
Έστω $ {(a_{n})}_{n \in \mathbb{N}} \geq 0, \; \forall n \in \mathbb{N} $ και $
{(b_{n})}_{n \in \mathbb{N}} > 0, \; \forall n \in \mathbb{N} $ 
και $ \lim_{n \to \infty} \frac{a_{n}}{b_{n}} = k  $, τότε: 
\begin{myitemize}
  \item Αν $ \textcolor{Col1}{0 < k < + \infty} $, τότε οι σειρές $ \sum_{n=1}^{\infty} a_{n} $ και $
    \sum_{n=1}^{\infty} b_{n} $ παρουσιάζουν \textbf{ίδια συμπεριφορά} ως προς τη σύγκλιση
  \item Αν $ \textcolor{Col1}{k=0} $, τότε $\sum_{n=1}^{\infty} b_{n} $ συγκλίνει 
    $ \Rightarrow  \sum_{n=1}^{\infty} a_{n} $ συγκλίνει, ή $ \sum_{n=1}^{\infty} a_{n} $
    αποκλίνει $ \Rightarrow \sum_{n=1}^{\infty} b_{n} $ αποκλίνει
  \item Αν $ \textcolor{Col1}{k=+ \infty} $, τότε $\sum_{n=1}^{\infty} a_{n} $ συγκλίνει 
    $ \Rightarrow  \sum_{n=1}^{\infty} b_{n} $ συγκλίνει, ή $ \sum_{n=1}^{\infty} b_{n} $
    αποκλίνει $ \Rightarrow \sum_{n=1}^{\infty} a_{n} $ αποκλίνει
\end{myitemize}

\section{Κριτήριο Leibniz (για Εναλλάσσουσες Σειρές)}

Έστω $ \sum_{n=1}^{\infty} (-1)^{n+1} a_{n} $ ή $ \sum_{n=1}^{\infty} (-1)^{n} a_{n} $ 
σειρές, με $ a_{n} > 0, \; \forall n \in \mathbb{N} $. Τότε, αν 

\parbox{5cm}{
  \begin{myitemize}
    \item $ (a_{n})_{n \in \mathbb{N}} $ \textbf{φθίνουσα} ακολουθία \hfill \tikzmark{a}
    \item $ \lim_{n \to \infty} a_{n} = 0 $ \hfill \tikzmark{b}
  \end{myitemize}
\mybrace{a}{b}[$ \sum_{n=1}^{\infty} (-1)^{n+1} a_{n} $ συγκλίνει]}


\section{Απόλυτη Σύγκλιση}

\begin{myitemize}
  \item $ \sum_{n=1}^{\infty} \abs{a_{n}} $ συγκλίνει $ \Rightarrow \sum_{n=1}^{\infty}
    a_{n} $ συγκλίνει \textcolor{Col1}{απόλυτα}
  \item $ \sum_{n=1}^{\infty} a_{n} $ συγκλίνει και $ \sum_{n=1}^{\infty} \abs{a_{n}}
    $ αποκλίνει, τότε λέμε $ \sum_{n=1}^{\infty} a_{n} $ συγκλίνει \textcolor{Col1}{υπό
    συνθήκη} $ \left( \text{πχ. η εναλ. αρμονική} \sum_{n=1}^{\infty} (-1)^{n-1} \frac{1}{n}\right) $
\end{myitemize}




\end{document}


