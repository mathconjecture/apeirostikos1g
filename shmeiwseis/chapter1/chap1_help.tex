%todo να μπουν σε κατάλληλες θέσεις στο chapter2
\begin{mypropbox}
  Έστω $ (a_{n})_{n \in \mathbb{N}} $ ακολουθία, και έστω $ a \in \mathbb{R} $, ώστε:
  \vspace{\baselineskip}
  \begin{minipage}{0.3\textwidth}
    \begin{enumerate}[(i)]
      \item $ 0 \leq a < a_{n}, \; \forall n \in \mathbb{N} $ \hfill \tikzmark{a}
      \item $ \lim_{n \to +\infty} a_{n} = 0 $ \hfill \tikzmark{b}
    \end{enumerate}
    \mybrace{a}{b}[$ a = 0 $]
  \end{minipage}
\end{mypropbox}

\begin{mypropbox}
  Έστω $ A \subseteq \mathbb{R} $, $ A \neq \emptyset $ και άνω φραγμένο. 
  Έστω $ s $ α.φ.\ του $A$. Τότε 
  \[
    s = \sup A \Leftrightarrow \forall n \in \mathbb{N} \; \exists a_{n} \in A
    \; : \; s - \frac{1}{n} < a_{n} 
  \]
\end{mypropbox}

\begin{mythmbox}
  Έστω $ A \subseteq \mathbb{R} $, $ A \neq \emptyset $ και άνω φραγμένο. 
  Έστω $ s $ α.φ.\ του $A$. Τότε 
  \[
    s = \sup A \Leftrightarrow \text{υπάρχει ακολουθία στοιχείων του $A$ με όριο το $s$} 
  \]
\end{mythmbox}
\begin{proof}
\item {}
  \begin{description}
    \item [$ (\Rightarrow) $]
      Έστω $ s = \sup A $. Τότε από τη χαρακτηριστική ιδιότητα του supremum έχουμε ότι:
      \begin{align*}
        \exists a_{1} \in A \; &: \; s - 1 < a_{1} \\
        \exists a_{2} \in A \; &: \; s - \frac{1}{2}  < a_{2} \\
                               &\vdots \\
        \exists a_{n} \in A \; &: \; s - \frac{1}{n}  < a_{n} 
      \end{align*} 

      Επομένως υπάρχει ακολουθία, η $ (a_{n})_{n \in \mathbb{N}} $, στοιχείων του $A$, 
      με την ιδιότητα $ s - \frac{1}{n} < a_{n} \leq s, \; \forall n \in \mathbb{N}$. 

      Παρατηρούμε ότι $ \lim_{n \to +\infty} s- \frac{1}{n} = \lim_{n \to +\infty} s = s $ 

      Οπότε από το Κριτήριο Παρεμβολής, έχουμε ότι $ \lim_{n \to +\infty} a_{n}= s $, 
      το ζητούμενο.

    \item [$ (\Rightarrow) $]
      Έστω $ s $ α.φ.\ του $A$ και έστω ότι υπάρχει ακολουθία $ (a_{n})_{n \in \mathbb{N}} $ 
      στοιχείων του $A$ ώστε $ \lim_{n \to +\infty} a_{n}= s$. Θ.δ.ο. $ s = \sup A $. 
      Πράγματι, 

      Έστω ότι $ s \neq \sup A $. Τότε $ \sup A < s \Rightarrow s - \sup A > 0 > 0 $. 

      Επιλέγω $ \varepsilon = s - \sup A >0 $. Οπότε από τον ορισμό του ορίου, έχουμε ότι 
      $ \exists n_{0} \in \mathbb{N} \; : \; \forall n \geq n_{0} \quad \abs{a_{n} - a} < 
      s - \sup A \Leftrightarrow \underbrace{-(s - \sup A) < a_{n} - s} < s - \sup A 
      \Leftrightarrow a_{n} > \sup A$, άτοπο, γιατί $ \sup A $ α.φ.\ του $A$.
  \end{description}
\end{proof}

