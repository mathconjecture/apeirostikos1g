\input{preamble_ask.tex}
%%%%%%%%% mdframed theorem boxes, breakable and with ref support %%%%%%%%

\usepackage[framemethod=TikZ]{mdframed}

\mdfdefinestyle{mythm}{innertopmargin=0pt,linecolor=Col2!75,linewidth=2pt,
  backgroundcolor=Col2!15, %background color of the box
  shadow=false,shadowcolor=Col2,shadowsize=5pt,% shadows
  frametitleaboveskip=\dimexpr-1.3\ht\strutbox\relax, 
  frametitlealignment={\hspace*{0.03\linewidth}},%
}

\mdfdefinestyle{mydfn}{innertopmargin=0pt,linecolor=Col1!75,linewidth=2pt,
  backgroundcolor=Col1!15, %background color of the box
  shadow=false,shadowcolor=Col1,shadowsize=5pt,% shadows
  frametitleaboveskip=\dimexpr-1.3\ht\strutbox\relax, 
  frametitlealignment={\hspace*{0.03\linewidth}},%
}

\mdfdefinestyle{myprop}{innertopmargin=0pt,linecolor=blue!75,linewidth=2pt,
  backgroundcolor=blue!10, %background color of the box
  shadow=false,shadowcolor=blue,shadowsize=5pt,% shadows
  frametitleaboveskip=\dimexpr-1.3\ht\strutbox\relax, 
  frametitlealignment={\hspace*{0.03\linewidth}},%
}

\mdfdefinestyle{myboxs}{innertopmargin=0pt,linecolor=blue!75,linewidth=0pt,
  backgroundcolor=blue!15, %background color of the box
  shadow=false,shadowcolor=blue,shadowsize=5pt,% shadows
}

% \newcounter{theo}[section]
% \setcounter{theo}{0}
% \renewcommand{\thetheo}{\arabic{section}.\arabic{theo}}


\newenvironment{mythm}[2][]{%
  \refstepcounter{thm}
  % Code for box design goes here.
  \ifstrempty{#1}%
    % if condition (without title)
    {\mdfsetup{
  frametitle={%
    \tikz[baseline=(current bounding box.east),outer sep=0pt]
    \node[anchor=east,rectangle,fill=Col2!75,text=white]
  {\strut Θεώρημα~\thethm};},%
      }%
      % else condition (with title)
      }{\mdfsetup{
  frametitle={%
    \tikz[baseline=(current bounding box.east),outer sep=0pt]
    \node[anchor=east,rectangle,fill=Col2!75,text=white]
  {\strut Θεώρημα~\thethm~~({#1})};},%
      }%
    }%
    % Both conditions
    \mdfsetup{
      style=mythm
    }
    \begin{mdframed}[]\relax\label{#2}}{%
  \end{mdframed}}
  %%%%%%%%%%%%%%%%%%%%%%%%%%%%%%%%%%%%%%%%%%%%%%%%%%%%%%%%%%

\newenvironment{mydfn}[2][]{%
  \refstepcounter{thm}
  % Code for box design goes here.
  \ifstrempty{#1}%
    % if condition (without title)
    {\mdfsetup{
  frametitle={%
    \tikz[baseline=(current bounding box.east),outer sep=0pt]
    \node[anchor=east,rectangle,fill=Col1!75,text=white,draw=Col1!75]
  {\strut Ορισμός~\thethm};},%
      }%
      % else condition (with title)
      }{\mdfsetup{
  frametitle={%
    \tikz[baseline=(current bounding box.east),outer sep=0pt]
    \node[anchor=east,rectangle,fill=Col1!75,text=white,draw=Col1!75]
  {\strut Ορισμός~\thethm~~({#1})};},%
      }%
    }%
    % Both conditions
    \mdfsetup{
      style=mydfn
    }
    \begin{mdframed}[]\relax\label{#2}}{%
  \end{mdframed}}
  %%%%%%%%%%%%%%%%%%%%%%%%%%%%%%%%%%%%%%%%%%%%%%%%%%%%%%%%%%

\newenvironment{myprop}[2][]{%
  \refstepcounter{thm}
  % Code for box design goes here.
  \ifstrempty{#1}%
    % if condition (without title)
    {\mdfsetup{
  frametitle={%
    \tikz[baseline=(current bounding box.east),outer sep=0pt]
    \node[anchor=east,rectangle,fill=blue!50,text=white]
  {\strut Πρόταση~\thethm};},%
      }%
      % else condition (with title)
      }{\mdfsetup{
  frametitle={%
    \tikz[baseline=(current bounding box.east),outer sep=0pt]
    \node[anchor=east,rectangle,fill=blue!50,text=white]
  {\strut Πρόταση~\thethm~~({#1})};},%
      }%
    }%
    % both conditions
    \mdfsetup{
      style=myprop
    }
    \begin{mdframed}[]\relax\label{#2}}{%
  \end{mdframed}}
  %%%%%%%%%%%%%%%%%%%%%%%%%%%%%%%%%%%%%%%%%%%%%%%%%%%%%%%%%%
\newenvironment{myboxs}{%
  % Code for box design goes here.
    \mdfsetup{
      style=myboxs
    }
    \begin{mdframed}[]\relax}{%
\end{mdframed}}
  %%%%%%%%%%%%%%%%%%%%%%%%%%%%%%%%%%%%%%%%%%%%%%%%%%%%%%%%%%
% \renewcommand{\qedsymbol}{$\blacksquare$}

\newcommand{\comb}[2]{\lambda_{1}\vec{#1}_{1} + \cdots + \lambda_{#2}\vec{#1}_{#2}}
\newcommand{\combc}[3]{#2_{1}\vec{#1}_{1} + \cdots + #2_{#3}\vec{#1}_{#3}}
\newcommand{\combb}[2]{\lambda_{1}\vec{#1}_{1} + \lambda_{2}\vec{#1}_{2} + \cdots + 
\lambda_{#2}\vec{#1}_{#2}}

\newcommand{\me}{\mathrm{e}}


\newlist{myitemize}{itemize}{3}
\setlist[myitemize]{label=\textcolor{Col1}{\tiny$\blacksquare$},leftmargin=*}

\newlist{myitemize*}{itemize*}{3}
\setlist[myitemize*]{itemjoin=\hspace{2\baselineskip},label=\textcolor{Col1}{\tiny$\blacksquare$}}

\newlist{myenumerate}{enumerate}{3}
\setlist[enumerate,1]{label=\textcolor{Col1}{\theenumi.},leftmargin=*}
\setlist[enumerate,2]{label=\textcolor{Col1}{\roman*)},leftmargin=*}

\setlist[description]{labelindent=1em,widest=Ιανουα0000,labelsep*=1em,itemindent=0pt,leftmargin=*}

% %%%%%%%%%%%%%%%%%% fancy headings %%%%%%%%%%%%%%%%%%

% %%%%%%%%%%%%%%%%%%%%%%% my boxes %%%%%%%%%%%%%%%%%%%%%%%%%%%%
% \newcommand{\mythm}[1]{
%       \refstepcounter{thm}
%     \begin{tikzpicture}
%         \node[myboxthm] (box1) 
%         {
%             \begin{minipage}{0.9\textwidth}
%                 #1
%             \end{minipage}
%         } ;

%         \node[myboxtitlethm] at (box1.north west) {\strut Θεώρημα~\thethm} ;
%     \end{tikzpicture}
% }

% \newcommand{\mythmm}[2]{
%       \refstepcounter{thm}
%     \begin{tikzpicture}
%         \node[myboxthm] (box1) 
%         {
%             \begin{minipage}{0.9\textwidth}
%                 #2
%             \end{minipage}
%         } ;

%         \node[myboxtitlethm] at (box1.north west) {\strut Θεώρημα~\thethm \; (#1)} ;
%     \end{tikzpicture}
% }

% \newcommand{\mydfn}[1]{
%       \refstepcounter{dfn}
%     \begin{tikzpicture}
%         \node[myboxdfn] (box1) 
%         {
%             \begin{minipage}{0.9\textwidth}
%                 #1
%             \end{minipage}
%         } ;

%         \node[myboxtitledfn] at (box1.north west) {\strut Ορισμός~\thedfn} ;
%     \end{tikzpicture}
% }


% \newcommand{\myprop}[1]{
%       \refstepcounter{thm}
%     \begin{tikzpicture}
%         \node[myboxprop] (box1) 
%         {
%             \begin{minipage}{0.9\textwidth}
%                 #1
%             \end{minipage}
%         } ;

%         \node[myboxtitleprop] at (box1.north west) {\strut Πρόταση~\theprop} ;
%     \end{tikzpicture}
% }

% \newcommand{\mypropp}[2]{
%       \refstepcounter{thm}
%     \begin{tikzpicture}
%         \node[myboxprop] (box1) 
%         {
%             \begin{minipage}{0.9\textwidth}
%                 #2
%             \end{minipage}
%         } ;

%         \node[myboxtitleprop] at (box1.north west) {\strut Πρόταση~\theprop (#1)} ;
%     \end{tikzpicture}
% }

%%%\mybrace{<first>}{<second>}[<Optional text>]
\newcommand{\tikzmark}[1]{\tikz[baseline={(#1.base)},overlay,remember picture] \node[outer
sep=0pt, inner sep=0pt] (#1) {\phantom{A}};}
%% syntax
\NewDocumentCommand\mybrace{mmo}{%
  \IfValueTF {#3}{%
    \begin{tikzpicture}[overlay, remember picture,decoration={brace,amplitude=1ex}]
      \draw[decorate,thick] (#1.north east) -- (#2.south east) 
        node (b) [midway,xshift=13pt,label={right=of b}:{#3}] {};
    \end{tikzpicture}%
  }%
  {%
    \begin{tikzpicture}[overlay, remember picture,decoration={brace,amplitude=1ex}]
      \draw[decorate,thick] (#1.north east) -- (#2.south east);
    \end{tikzpicture}%
  }%
}%

%%%%%%%How to use this %%%%%%%%%%%%%%%%%%%%%%%%%
%use \tikzmark{a} and \tikzmark{b} at first and last \item where the brace is
%wanted
%use the following command after \end{enumerate}
%\mybrace{a}{b}[Text comes here to describe these to items and justify for your
%case]]



%%%%% label inline equations and don't allow reference
\newcommand\inlineeqno{\stepcounter{equation} (\theequation)}

%%%%%defines \inlineequation[<label name>]{<equation>}
%%%%%%%%format use \inlineequation[<label name>]{<equation>}%%%%%%%
\makeatletter
\newcommand*{\inlineequation}[2][]{%
    \begingroup
    % Put \refstepcounter at the beginning, because
    % package `hyperref' sets the anchor here.
    \refstepcounter{equation}%
    \ifx\\#1\\%
\else
    \label{#1}%
\fi
% prevent line breaks inside equation
\relpenalty=10000 %
\binoppenalty=10000 %
\ensuremath{%
    % \displaystyle % larger fractions, ...
    #2%
}%
\quad ~\@eqnnum
\endgroup
}
\makeatother


%%%%%%%%%%%%%%%%%% fancy enumitem cicled label %%%%%%%%%%%%%%%%%%
\newcommand*\circled[1]{\tikz[baseline=(char.base)]{
\node[shape=circle,draw,inner sep=0.3pt] (char) {#1};}}
% use it like \begin{enumerate}[label=\protect\circled{\Alph{enumi}}]
%%%\mybrace{<first>}{<second>}[<Optional text>]
%%% wrap with braces list environments




%%%%%%%%%%%%% puts brace under matrix
\newcommand\undermat[2]{%
  \makebox[0pt][l]{$\smash{\underbrace{\phantom{%
\begin{matrix}#2\end{matrix}}}_{\text{$#1$}}}$}#2}


%circle item inside array or matrix
\newcommand\Circle[1]{%
\tikz[baseline=(char.base)]\node[circle,draw,inner sep=2pt] (char) {#1};}


  %redeftine \eqref so that parenthesis () have the color the link
\makeatletter
\renewcommand*{\eqref}[1]{%
  \hyperref[{#1}]{\textup{\tagform@{\ref*{#1}}}}%
}
\makeatother

%removes qedsymbol and additional vertical space at the end 
\makeatletter
\renewenvironment{proof}[1][\proofname]{\par
  % \pushQED{\hfill\qedhere}% <--- remove the QED business
  \normalfont \topsep6\p@\@plus6\p@\relax
  \trivlist
  \item[\hskip\labelsep
        \itshape
        #1\@addpunct{.}]\ignorespaces
}{%
 % \popQED% <--- remove the QED business
  \endtrivlist\@endpefalse
}
\renewcommand\qedhere{$\blacksquare$} % to ensure code portability
\makeatother


\input{tikz.tex}
\input{myboxes.tex}

\pagestyle{vangelis}

\begin{document}



\begin{center}
  \minibox{\large\bfseries\textcolor{Col1}{Πραγματικοί Αριθμοί}}
\end{center}

\vspace{\baselineskip}


\begin{lem}\label{lem:ineqs}
  Έστω $ x,y >0 $ και $ n \in \mathbb{N} $. Τότε
  \begin{enumerate*}[(i)]
    \item $ x \leq y \Leftrightarrow x^{n} \leq y^{n} $
    \item $ x < y \Leftrightarrow x^{n} < y^{n} $ \label{lem:ineqreal2}
    \item $ x = y \Leftrightarrow x^{n} = y^{n} $ \label{lem:ineqreal3}
  \end{enumerate*}
\end{lem}

\begin{mybox2}
\begin{thm}[Ύπαρξη $ \mathbf{\sqrt{2}} $]
  Υπάρχει μοναδικός θετικός αριθμός $ x \in \mathbb{R} $ τέτοιος ώστε $ x^{2}=2 $.
\end{thm}
\end{mybox2}
\begin{proof}
\item {}
  Έστω $ A = \{ a \in \mathbb{R} \; : \; a > 0, a^{2} < 2 \}  $. Έχουμε $ 1 \in A 
  \Rightarrow A \neq \emptyset $ 
  και αν $ a \in A \Rightarrow a^{2} < 2 \Rightarrow a^{2} < 4
  \overset{\ref{lem:ineqreal2}}{\Rightarrow} a <2,\; \forall a \in A $, 
  άρα το 2 είναι άνω φράγμα του $A$. Άρα υπάρχει το supremum του Α, έστω 
  $ x = \sup A $. 

  Θα δείξουμε ότι $ x > 0 $ και $ x^{2} = 2 $

  Πράγματι, $ 1 \in A \Rightarrow 1 \leq x $, γιατί $x$ α.φ.\ του $A$, οπότε $ x >0 $.

  Θα δείξουμε ότι $ x^{2} = 2 $ αποκλείοντας τις περιπτώσεις $ x^{2} <2 $ και 
  $ x^{2} > 2 $.
  \begin{myitemize}
    \item Έστω $ x^{2} > 2 $. Επιλέγω $ \varepsilon = \frac{1}{2} \min \{ x, 
      \frac{x^{2}-2}{2x}\} $.

      Καταρχάς $ \varepsilon > 0 $, γιατί $ x>0 $ και $ x^{2} -2 >0 $. 

      Επίσης από ορισμό του $ \varepsilon $ έχουμε ότι  $\varepsilon < x $ και 
      $ \varepsilon < \frac{x^{2}-2}{2x}$. 

      Άρα $ \varepsilon < \frac{x^{2}-2}{2x} \Rightarrow 
      2x \varepsilon < x^{2} - 2 \Leftrightarrow 2 < x^{2} -2x \varepsilon 
      \overset{\varepsilon ^{2}>0}{\leq}
      x^{2} -2x \varepsilon + \varepsilon ^{2} \Rightarrow 2 
      < (x- \varepsilon )^{2}   $

      Έστω $ a \in A \Rightarrow a^{2} <2 \Rightarrow a^{2}<2< 
      (x- \varepsilon )^{2} \Rightarrow a^{2}< (x- \varepsilon )^{2} 
      \Rightarrow a < x- \varepsilon, \; \forall a \in A$, 
      άρα $ x - \varepsilon $ α.φ.\ του $A$, άτοπο, γιατί $ x= \sup A $.

    \item Έστω $ x^{2}<2 $. Ομοίως.

      Οπότε $ x^{2}=2 $. 

      Τέλος, για να αποδείξουμε τη μοναδικότητα, έχουμε:

      Έστω ότι υπάρχουν $ 2 $ αριθμοί, έστω $ x,y $ τέτοιοι ώστε $ x^{2} =2 $ 
      και $ y^{2}=2 $. Τότε $ x^{2}=y^{2} \overset{\ref{lem:ineqreal3}}
      {\Rightarrow} x =y $.
  \end{myitemize}
\end{proof}

\begin{mybox2}
\begin{thm}[Ύπαρξη ν-οστής ρίζας]
  Για κάθε $ x_{0} >0 $ και για κάθε $ n \in \mathbb{N} $ υπάρχει μοναδικός, θετικός 
  πραγματικός αριθμός $ p $, τέτοιος ώστε $ p^{n}= x_{0} $.
\end{thm}
\end{mybox2}

\begin{cor}\label{cor:ineqs}
  Έστω $ x,y >0 $ και $ n \in \mathbb{N} $. Τότε
  \begin{enumerate}[label=(\roman*),itemjoin=\hspace{1cm}]
    \item $ x \leq y \Leftrightarrow x^{\frac{1}{n}} \leq y^{\frac{1}{n}} $
    \item $ x <y \Leftrightarrow x^{\frac{1}{n}} < y^{\frac{1}{n}} $
    \item $ x =y \Leftrightarrow x^{\frac{1}{n}} = y^{\frac{1}{n}} $
  \end{enumerate}
\end{cor}

\begin{proof}
\item {}
  Αν $ x,y>0 $, τότε από το θεώρημα ύπαρξης $n$-οστής ρίζας υπάρχουν $ \tilde{x}  
  = x^{\frac{1}{n}} $ και $ \tilde{y} =y^{\frac{1}{n}} $, οπότε από το 
  λήμμα~\ref{lem:ineqs} έχουμε $ \tilde{x} \leq \tilde{y}  \Leftrightarrow 
  \tilde{x} ^{n} \leq \tilde{y} ^{n} $,
  οπότε $ x^{\frac{1}{n}} \leq y^{\frac{1}{n}} \Leftrightarrow x \leq y $.
\end{proof}

\begin{lem}\label{lem:ineqq}
  Έστω $ x,y \geq 0 $ με $ x<y $ και $ q \in \mathbb{Q}, \; q >0 $. 
  Τότε: $ x^{q} <y^{q} $
\end{lem}

\begin{proof}
\item {}
  Έστω $ p \in \mathbb{Q} $. Τότε υπάρχουν $ n,m \in \mathbb{N} $, τέτοιοι ώστε 
  $ q = \frac{n}{m} $. Τότε 

  \[ x<y \overset{\ref{cor:ineqs}}{\Rightarrow} x^{\frac{1}{m}} 
    < y^{\frac{1}{m}} \overset{\ref{lem:ineqs}}{\Rightarrow} 
  x^{\frac{n}{m}} < y^{\frac{n}{m}} \Rightarrow x^{q} < y^{q}  \] 
\end{proof}

\begin{lem}
  Έστω $ a > 0 $ και $ q_{1}, q_{2} \in \mathbb{Q} $ με $ q_{1} < q_{2} $. 
  \begin{enumerate}[(i)]
    \item $ a>1 \Rightarrow a^{q_{1}} < a^{q_{2}} $
    \item $ a<1 \Rightarrow a^{q_{1}} > a^{q_{2}} $
  \end{enumerate}
\end{lem}

\begin{proof}
\item {}
  \begin{enumerate}[(i)]
    \item  $ a>1 \overset{\ref{lem:ineqq}}{\Rightarrow} 
      a^{q_{2}- q_{1}} > 1^{q_{2} - q_{1}} \overset{a^{q_{1}}>0}{\Rightarrow} 
      a^{q_{2}} > a^{q_{1} } $

    \item $ a<1 \Rightarrow \frac{1}{a} >1 \Rightarrow 
      (\frac{1}{a} )^{q_{1}} < (\frac{1}{a} )^{q_{2} } \Rightarrow a^{q_{1}} 
      > a^{q_{2}} $
  \end{enumerate}
\end{proof}

\end{document}
