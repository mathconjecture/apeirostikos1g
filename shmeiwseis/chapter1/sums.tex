\input{preamble_ask.tex}
\input{definitions.tex}
\input{tikz.tex}
\input{myboxes.tex}

\pagestyle{vangelis}

\begin{document}



\begin{center}
  \minibox{\large\bfseries\textcolor{Col1}{Αθροίσματα}}
\end{center}

\vspace{\baselineskip}

\begin{mybox1}
\begin{dfn}
Έστω $ a_{1}, a_{2}, \ldots, a_{n} \in \mathbb{R} $. Ορίζουμε 
$ \sum\limits_{k=1}^{n} a_{k} = a_{1} + a_{2} + \cdots + a_{n} $
\end{dfn}
\end{mybox1}

Το σύμβολο του αθροίσματος έχει τις παρακάτω ιδιότητες.
{
  \everymath{\displaystyle}
  \begin{myitemize}
    \item $ \sum_{n=k}^{N} a_{n} = \sum_{n=k- \rho }^{N- \rho } a_{n+ \rho} $ ή  
      $ \sum_{n=k}^{N} a_{n} = \sum_{n=k+ \rho }^{N+ \rho } a_{n- \rho} $
    \item $ \sum_{n=k}^{N} \lambda a_{n} = \lambda \sum_{n=k}^{N} a_{n} $ 
    \item $ \sum_{n=k}^{N} (a_{n}+b_{n}) = \sum_{n=k}^{N} a_{n} + \sum_{n=k}^{N} b_{n} $ 
  \end{myitemize}
}

\begin{rem}
  Αν ένα από τα όρια του συμβόλου του αθροίσματος είναι $ - \infty $ ή $ + \infty $, 
  τότε ή τρίτη ιδιότητα \textbf{δεν} ισχύει πάντα.
\end{rem}

Τα παρακάτω παραδείγματα, περιγράφουν μερικές από τις πιο βασικές ιδιότητες που 
έχουν τα αθροίσματα.

\begin{example}
  Να δείξετε ότι $ \sum_{n=3}^{6} n^{2} = \sum_{n=4}^{7} (n-1)^{2}   $
\end{example}
\begin{proof}
  \begin{align*}
    \sum_{n=4}^{7} (n-1)^{2} &= (4-1)^{2}+(5-1)^{2}+(6-1)^{2}+(7-1)^{2} 
    = 3^{2}+4^{2}+5^{2}+6^{2} = \sum_{n=3}^{6} n^{2} 
  \end{align*}
\end{proof}

\begin{example}
  Να δείξετε ότι $ \sum_{n=3}^{6} n^{2} = \sum_{n=1}^{4} (n+2)^{2}   $
\end{example}
\begin{proof}
  \begin{align*}
    \sum_{n=1}^{4} (n+2)^{2} = (1+2)^{2}+(2+2)^{2}+(3+2)^{2}+(4+2)^{2}=
    3^{2}+4^{2}+5^{2}+6^{2} = \sum_{n=3}^{6} n^{2} 
  \end{align*}
\end{proof}

\begin{example}
  Να δείξετε ότι $ \sum_{n=1}^{5} (2n+1) - \sum_{n=3}^{5} (2n+1)  =  
  \sum_{n=1}^{2} (2n+1) $
\end{example}
\begin{proof}
  \begin{align*}
    \sum_{n=1}^{5} (2n+1)- \sum_{n=3}^{5} (2n+1) = 
    (3+5+7+9+11) - (7+9+11) = 3 + 5 = \sum_{n=1}^{2} (2n+1) 
  \end{align*}
\end{proof}

\begin{example}
  Να δείξετε ότι $ \sum_{n=1}^{20} (2n+1) - \sum_{n=8}^{20} (2n+1) = \sum_{n=1}^{7} 
  (2n+1) $ 
\end{example}
\begin{proof}
\item 
  \[
    \sum_{n=1}^{20} (2n+1) - \sum_{n=8}^{20} (2n+1) = \sum_{n=1}^{7} (2n+1) + 
    \sum_{n=8}^{20} (2n+1) - \sum_{n=8}^{20} (2n+1) = 
    \sum_{n=1}^{7} (2n+1)
  \]
\end{proof}

\begin{example}[Τηλεσκοπικό Άθροισμα]
  Να δείξετε ότι $ \sum_{n=1}^{5} \frac{1}{n(n+1)} = 1 - \frac{1}{6}  $
\end{example}
\begin{proof}(Με ανάλυση σε απλά κλάσματα)
\item {}
  Έχουμε $ \frac{1}{n(n+1)} = \frac{A}{n} + \frac{B}{n+1} = 
  \frac{1}{n} - \frac{1}{n+1}$, οπότε
  \begin{equation*}
    \sum_{n=1}^{5} \frac{1}{n(n+1)} = \sum_{n=1}^{5} \left(\frac{1}{n} - 
    \frac{1}{n+1}\right)    
    = \left(\frac{1}{1} - \frac{1}{2}\right) + \left(\frac{1}{2} 
    - \frac{1}{3} \right) + \left(\frac{1}{3} - \frac{1}{4}\right) 
    + \left(\frac{1}{4} - \frac{1}{5}\right) +
    \left(\frac{1}{5} - \frac{1}{6}\right)  = 1 - \frac{1}{6}
  \end{equation*} 
\end{proof}

\begin{mybox3}
\begin{prop}[Γεωμετρική Πρόοδος]
  \label{prop:geom}
  Έστω $ \lambda \in \mathbb{R} \setminus \{ 1 \}  $. Τότε το άθροισμα 
  $ \sum_{k=0}^{n} \lambda ^{k} = 1 + \lambda + \lambda ^{2} + \lambda ^{3} + 
  \cdots + \lambda ^{n} = \frac{\lambda ^{n+1}-1}{\lambda -1}, 
  \quad \; \forall n \in \mathbb{N} $
\end{prop}
\end{mybox3}
\begin{proof} (Με Μαθ. Επαγωγή)
\item {}
  \begin{myitemize}
    \item Για $ n=1 $, έχω: $1+ \lambda = \frac{\lambda ^{2}-1}{\lambda -1} = 
      \frac{\cancel{(\lambda -1)}(\lambda +1)}{\cancel{\lambda -1}} = \lambda +1 $,
      ισχύει.
    \item Έστω ότι η ισότητα ισχύει για $n$, δηλ. 
      $\inlineequation[eq:epagex3]{1+ \lambda + \lambda ^{2} + \lambda ^{3} + 
      \cdots \lambda ^{n} = \frac{\lambda ^{n+1}-1}{\lambda -1}}$.
    \item Θα δείξουμε ότι ισχύει και για $ n+1 $. Πράγματι:
      \begin{align*}
        1+ \lambda + \lambda ^{2}+ \cdots + \lambda ^{n} + \lambda ^{n+1}
          &\overset{\eqref{eq:epagex3}}{=}
          \frac{\lambda ^{n+1}-1}{\lambda -1} + \lambda ^{n+1} = \frac{\lambda
            ^{n+1}-1 + \lambda ^{n+1}(\lambda -1)}{\lambda -1} = 
            \frac{\cancel{\lambda ^{n+1}} -1 + 
          \lambda ^{n+2} - \cancel{\lambda ^{n+1}}}{\lambda -1} \\ 
          &= \frac{\lambda ^{(n +1)+1}-1}{\lambda -1} 
      \end{align*}
  \end{myitemize}
\end{proof}

\begin{example} Να υπολογιστεί το άθροισμα $ \sum_{n=1}^{6} 2^{n} $
  \begin{solution}
    \[
      \sum_{n=1}^{6} 2^{n} = \sum_{n=0}^{5} 2^{n+1} = 2 \cdot \sum_{n=0}^{5} 2^{n} 
      \overset{\ref{prop:geom}}= 2 \cdot \frac{2^{6}-1}{2-1} = 2 \cdot 63 = 126 
    \]
  \end{solution}
\end{example}

\begin{example}
  Να δείξετε ότι $ \sum_{n=1}^{100} \frac{1}{n(n+1)} \leq \sum_{n=1}^{100} 
  \frac{1}{n^{2}} $
\end{example}
\begin{proof}
\item {} 
  Έχουμε $ n+1 > n $ , $ \forall n \in \mathbb{N} $, οπότε
  \[
    \frac{1}{n+1} < \frac{1}{n} \overset{\cdot \frac{1}{n} >0}{\Rightarrow}
    \frac{1}{n(n+1)} < \frac{1}{n^{2}} \Rightarrow \sum_{n=1}^{100} 
    \frac{1}{n(n+1)} \leq \sum_{n=1}^{100} \frac{1}{n^{2}} 
  \] 
\end{proof}

\begin{exercise}
  Να δείξετε ότι $ \sum_{n=1}^{100} \frac{1}{n(n+1)} \leq \sum_{n=2}^{101} 
  \frac{1}{n^{2}} $
\end{exercise} 
\begin{proof}
\item {} 
  Έχουμε $ n-1 < n, \; \forall n \in \mathbb{N} $ και 
  $ n-1>0, \; \forall n \geq 2 $, άρα
  \begin{equation}\label{eq:sumexam}
    \frac{1}{n-1} \leq \frac{1}{n} \overset{\cdot \frac{1}{n} >0}{\Rightarrow}
    \frac{1}{n(n-1)} \leq \frac{1}{n^{2}}, \quad \forall n \geq 2 \Rightarrow 
    \sum_{n=2}^{101} \frac{1}{n(n-1)} \leq 
    \sum_{n=2}^{101} \frac{1}{n^{2}}
  \end{equation} 
  Άρα, 
  \[
    \sum_{n=1}^{100} \frac{1}{n(n+1)} = \sum_{n=2}^{101} \frac{1}{(n-1)n} 
    \overset{\eqref{eq:sumexam}}{\leq} \sum_{n=2}^{101} \frac{1}{n^{2}} 
  \]
\end{proof}

\begin{example}
  Με Μαθηματική Επαγωγή, αποδεικνύονται εύκολα τα παρακάτω αθροίσματα.
  \begin{enumerate}[wide,labelindent=0pt]
    \item $ \sum_{k=1}^{n} k = 1+2+3+\cdots+n = \frac{n(n+1)}{2}, 
      \quad \forall n \in \mathbb{N}  $
    \item $ \sum_{k=1}^{n} k^{2} =1^{2}+2^{2}+3^{2}+\cdots+n^{2} = 
      \frac{n(n+1)(2n+1)}{6}, \quad \forall n \in \mathbb{N} $
    \item $ \sum_{k=1}^{n} k^{3} = 1^{3}+2^{3}+3^{3}+\cdots +n^{3} = 
      \left(\frac{n(n+1)}{2} \right)^{2}, \; \forall n \in \mathbb{N} $
  \end{enumerate}
\end{example}




\end{document}
