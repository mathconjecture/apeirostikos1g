\input{preamble_ask.tex}
%%%%%%%%% mdframed theorem boxes, breakable and with ref support %%%%%%%%

\usepackage[framemethod=TikZ]{mdframed}

\mdfdefinestyle{mythm}{innertopmargin=0pt,linecolor=Col2!75,linewidth=2pt,
  backgroundcolor=Col2!15, %background color of the box
  shadow=false,shadowcolor=Col2,shadowsize=5pt,% shadows
  frametitleaboveskip=\dimexpr-1.3\ht\strutbox\relax, 
  frametitlealignment={\hspace*{0.03\linewidth}},%
}

\mdfdefinestyle{mydfn}{innertopmargin=0pt,linecolor=Col1!75,linewidth=2pt,
  backgroundcolor=Col1!15, %background color of the box
  shadow=false,shadowcolor=Col1,shadowsize=5pt,% shadows
  frametitleaboveskip=\dimexpr-1.3\ht\strutbox\relax, 
  frametitlealignment={\hspace*{0.03\linewidth}},%
}

\mdfdefinestyle{myprop}{innertopmargin=0pt,linecolor=blue!75,linewidth=2pt,
  backgroundcolor=blue!10, %background color of the box
  shadow=false,shadowcolor=blue,shadowsize=5pt,% shadows
  frametitleaboveskip=\dimexpr-1.3\ht\strutbox\relax, 
  frametitlealignment={\hspace*{0.03\linewidth}},%
}

\mdfdefinestyle{myboxs}{innertopmargin=0pt,linecolor=blue!75,linewidth=0pt,
  backgroundcolor=blue!15, %background color of the box
  shadow=false,shadowcolor=blue,shadowsize=5pt,% shadows
}

% \newcounter{theo}[section]
% \setcounter{theo}{0}
% \renewcommand{\thetheo}{\arabic{section}.\arabic{theo}}


\newenvironment{mythm}[2][]{%
  \refstepcounter{thm}
  % Code for box design goes here.
  \ifstrempty{#1}%
    % if condition (without title)
    {\mdfsetup{
  frametitle={%
    \tikz[baseline=(current bounding box.east),outer sep=0pt]
    \node[anchor=east,rectangle,fill=Col2!75,text=white]
  {\strut Θεώρημα~\thethm};},%
      }%
      % else condition (with title)
      }{\mdfsetup{
  frametitle={%
    \tikz[baseline=(current bounding box.east),outer sep=0pt]
    \node[anchor=east,rectangle,fill=Col2!75,text=white]
  {\strut Θεώρημα~\thethm~~({#1})};},%
      }%
    }%
    % Both conditions
    \mdfsetup{
      style=mythm
    }
    \begin{mdframed}[]\relax\label{#2}}{%
  \end{mdframed}}
  %%%%%%%%%%%%%%%%%%%%%%%%%%%%%%%%%%%%%%%%%%%%%%%%%%%%%%%%%%

\newenvironment{mydfn}[2][]{%
  \refstepcounter{thm}
  % Code for box design goes here.
  \ifstrempty{#1}%
    % if condition (without title)
    {\mdfsetup{
  frametitle={%
    \tikz[baseline=(current bounding box.east),outer sep=0pt]
    \node[anchor=east,rectangle,fill=Col1!75,text=white,draw=Col1!75]
  {\strut Ορισμός~\thethm};},%
      }%
      % else condition (with title)
      }{\mdfsetup{
  frametitle={%
    \tikz[baseline=(current bounding box.east),outer sep=0pt]
    \node[anchor=east,rectangle,fill=Col1!75,text=white,draw=Col1!75]
  {\strut Ορισμός~\thethm~~({#1})};},%
      }%
    }%
    % Both conditions
    \mdfsetup{
      style=mydfn
    }
    \begin{mdframed}[]\relax\label{#2}}{%
  \end{mdframed}}
  %%%%%%%%%%%%%%%%%%%%%%%%%%%%%%%%%%%%%%%%%%%%%%%%%%%%%%%%%%

\newenvironment{myprop}[2][]{%
  \refstepcounter{thm}
  % Code for box design goes here.
  \ifstrempty{#1}%
    % if condition (without title)
    {\mdfsetup{
  frametitle={%
    \tikz[baseline=(current bounding box.east),outer sep=0pt]
    \node[anchor=east,rectangle,fill=blue!50,text=white]
  {\strut Πρόταση~\thethm};},%
      }%
      % else condition (with title)
      }{\mdfsetup{
  frametitle={%
    \tikz[baseline=(current bounding box.east),outer sep=0pt]
    \node[anchor=east,rectangle,fill=blue!50,text=white]
  {\strut Πρόταση~\thethm~~({#1})};},%
      }%
    }%
    % both conditions
    \mdfsetup{
      style=myprop
    }
    \begin{mdframed}[]\relax\label{#2}}{%
  \end{mdframed}}
  %%%%%%%%%%%%%%%%%%%%%%%%%%%%%%%%%%%%%%%%%%%%%%%%%%%%%%%%%%
\newenvironment{myboxs}{%
  % Code for box design goes here.
    \mdfsetup{
      style=myboxs
    }
    \begin{mdframed}[]\relax}{%
\end{mdframed}}
  %%%%%%%%%%%%%%%%%%%%%%%%%%%%%%%%%%%%%%%%%%%%%%%%%%%%%%%%%%
% \renewcommand{\qedsymbol}{$\blacksquare$}

\newcommand{\comb}[2]{\lambda_{1}\vec{#1}_{1} + \cdots + \lambda_{#2}\vec{#1}_{#2}}
\newcommand{\combc}[3]{#2_{1}\vec{#1}_{1} + \cdots + #2_{#3}\vec{#1}_{#3}}
\newcommand{\combb}[2]{\lambda_{1}\vec{#1}_{1} + \lambda_{2}\vec{#1}_{2} + \cdots + 
\lambda_{#2}\vec{#1}_{#2}}

\newcommand{\me}{\mathrm{e}}


\newlist{myitemize}{itemize}{3}
\setlist[myitemize]{label=\textcolor{Col1}{\tiny$\blacksquare$},leftmargin=*}

\newlist{myitemize*}{itemize*}{3}
\setlist[myitemize*]{itemjoin=\hspace{2\baselineskip},label=\textcolor{Col1}{\tiny$\blacksquare$}}

\newlist{myenumerate}{enumerate}{3}
\setlist[enumerate,1]{label=\textcolor{Col1}{\theenumi.},leftmargin=*}
\setlist[enumerate,2]{label=\textcolor{Col1}{\roman*)},leftmargin=*}

\setlist[description]{labelindent=1em,widest=Ιανουα0000,labelsep*=1em,itemindent=0pt,leftmargin=*}

% %%%%%%%%%%%%%%%%%% fancy headings %%%%%%%%%%%%%%%%%%

% %%%%%%%%%%%%%%%%%%%%%%% my boxes %%%%%%%%%%%%%%%%%%%%%%%%%%%%
% \newcommand{\mythm}[1]{
%       \refstepcounter{thm}
%     \begin{tikzpicture}
%         \node[myboxthm] (box1) 
%         {
%             \begin{minipage}{0.9\textwidth}
%                 #1
%             \end{minipage}
%         } ;

%         \node[myboxtitlethm] at (box1.north west) {\strut Θεώρημα~\thethm} ;
%     \end{tikzpicture}
% }

% \newcommand{\mythmm}[2]{
%       \refstepcounter{thm}
%     \begin{tikzpicture}
%         \node[myboxthm] (box1) 
%         {
%             \begin{minipage}{0.9\textwidth}
%                 #2
%             \end{minipage}
%         } ;

%         \node[myboxtitlethm] at (box1.north west) {\strut Θεώρημα~\thethm \; (#1)} ;
%     \end{tikzpicture}
% }

% \newcommand{\mydfn}[1]{
%       \refstepcounter{dfn}
%     \begin{tikzpicture}
%         \node[myboxdfn] (box1) 
%         {
%             \begin{minipage}{0.9\textwidth}
%                 #1
%             \end{minipage}
%         } ;

%         \node[myboxtitledfn] at (box1.north west) {\strut Ορισμός~\thedfn} ;
%     \end{tikzpicture}
% }


% \newcommand{\myprop}[1]{
%       \refstepcounter{thm}
%     \begin{tikzpicture}
%         \node[myboxprop] (box1) 
%         {
%             \begin{minipage}{0.9\textwidth}
%                 #1
%             \end{minipage}
%         } ;

%         \node[myboxtitleprop] at (box1.north west) {\strut Πρόταση~\theprop} ;
%     \end{tikzpicture}
% }

% \newcommand{\mypropp}[2]{
%       \refstepcounter{thm}
%     \begin{tikzpicture}
%         \node[myboxprop] (box1) 
%         {
%             \begin{minipage}{0.9\textwidth}
%                 #2
%             \end{minipage}
%         } ;

%         \node[myboxtitleprop] at (box1.north west) {\strut Πρόταση~\theprop (#1)} ;
%     \end{tikzpicture}
% }

%%%\mybrace{<first>}{<second>}[<Optional text>]
\newcommand{\tikzmark}[1]{\tikz[baseline={(#1.base)},overlay,remember picture] \node[outer
sep=0pt, inner sep=0pt] (#1) {\phantom{A}};}
%% syntax
\NewDocumentCommand\mybrace{mmo}{%
  \IfValueTF {#3}{%
    \begin{tikzpicture}[overlay, remember picture,decoration={brace,amplitude=1ex}]
      \draw[decorate,thick] (#1.north east) -- (#2.south east) 
        node (b) [midway,xshift=13pt,label={right=of b}:{#3}] {};
    \end{tikzpicture}%
  }%
  {%
    \begin{tikzpicture}[overlay, remember picture,decoration={brace,amplitude=1ex}]
      \draw[decorate,thick] (#1.north east) -- (#2.south east);
    \end{tikzpicture}%
  }%
}%

%%%%%%%How to use this %%%%%%%%%%%%%%%%%%%%%%%%%
%use \tikzmark{a} and \tikzmark{b} at first and last \item where the brace is
%wanted
%use the following command after \end{enumerate}
%\mybrace{a}{b}[Text comes here to describe these to items and justify for your
%case]]



%%%%% label inline equations and don't allow reference
\newcommand\inlineeqno{\stepcounter{equation} (\theequation)}

%%%%%defines \inlineequation[<label name>]{<equation>}
%%%%%%%%format use \inlineequation[<label name>]{<equation>}%%%%%%%
\makeatletter
\newcommand*{\inlineequation}[2][]{%
    \begingroup
    % Put \refstepcounter at the beginning, because
    % package `hyperref' sets the anchor here.
    \refstepcounter{equation}%
    \ifx\\#1\\%
\else
    \label{#1}%
\fi
% prevent line breaks inside equation
\relpenalty=10000 %
\binoppenalty=10000 %
\ensuremath{%
    % \displaystyle % larger fractions, ...
    #2%
}%
\quad ~\@eqnnum
\endgroup
}
\makeatother


%%%%%%%%%%%%%%%%%% fancy enumitem cicled label %%%%%%%%%%%%%%%%%%
\newcommand*\circled[1]{\tikz[baseline=(char.base)]{
\node[shape=circle,draw,inner sep=0.3pt] (char) {#1};}}
% use it like \begin{enumerate}[label=\protect\circled{\Alph{enumi}}]
%%%\mybrace{<first>}{<second>}[<Optional text>]
%%% wrap with braces list environments




%%%%%%%%%%%%% puts brace under matrix
\newcommand\undermat[2]{%
  \makebox[0pt][l]{$\smash{\underbrace{\phantom{%
\begin{matrix}#2\end{matrix}}}_{\text{$#1$}}}$}#2}


%circle item inside array or matrix
\newcommand\Circle[1]{%
\tikz[baseline=(char.base)]\node[circle,draw,inner sep=2pt] (char) {#1};}


  %redeftine \eqref so that parenthesis () have the color the link
\makeatletter
\renewcommand*{\eqref}[1]{%
  \hyperref[{#1}]{\textup{\tagform@{\ref*{#1}}}}%
}
\makeatother

%removes qedsymbol and additional vertical space at the end 
\makeatletter
\renewenvironment{proof}[1][\proofname]{\par
  % \pushQED{\hfill\qedhere}% <--- remove the QED business
  \normalfont \topsep6\p@\@plus6\p@\relax
  \trivlist
  \item[\hskip\labelsep
        \itshape
        #1\@addpunct{.}]\ignorespaces
}{%
 % \popQED% <--- remove the QED business
  \endtrivlist\@endpefalse
}
\renewcommand\qedhere{$\blacksquare$} % to ensure code portability
\makeatother


\input{tikz.tex}
\input{myboxes.tex}

\pagestyle{vangelis}

\begin{document}



\begin{center}
  \minibox{\large\bfseries\textcolor{Col1}{Αθροίσματα}}
\end{center}

\vspace{\baselineskip}

\begin{mybox1}
\begin{dfn}
Έστω $ a_{1}, a_{2}, \ldots, a_{n} \in \mathbb{R} $. Ορίζουμε 
$ \sum\limits_{k=1}^{n} a_{k} = a_{1} + a_{2} + \cdots + a_{n} $
\end{dfn}
\end{mybox1}

Το σύμβολο του αθροίσματος έχει τις παρακάτω ιδιότητες.
{
  \everymath{\displaystyle}
  \begin{myitemize}
    \item $ \sum_{n=k}^{N} a_{n} = \sum_{n=k- \rho }^{N- \rho } a_{n+ \rho} $ ή  
      $ \sum_{n=k}^{N} a_{n} = \sum_{n=k+ \rho }^{N+ \rho } a_{n- \rho} $
    \item $ \sum_{n=k}^{N} \lambda a_{n} = \lambda \sum_{n=k}^{N} a_{n} $ 
    \item $ \sum_{n=k}^{N} (a_{n}+b_{n}) = \sum_{n=k}^{N} a_{n} + \sum_{n=k}^{N} b_{n} $ 
  \end{myitemize}
}

\begin{rem}
  Αν ένα από τα όρια του συμβόλου του αθροίσματος είναι $ - \infty $ ή $ + \infty $, 
  τότε ή τρίτη ιδιότητα \textbf{δεν} ισχύει πάντα.
\end{rem}

Τα παρακάτω παραδείγματα, περιγράφουν μερικές από τις πιο βασικές ιδιότητες που 
έχουν τα αθροίσματα.

\begin{example}
  Να δείξετε ότι $ \sum_{n=3}^{6} n^{2} = \sum_{n=4}^{7} (n-1)^{2}   $
\end{example}
\begin{proof}
  \begin{align*}
    \sum_{n=4}^{7} (n-1)^{2} &= (4-1)^{2}+(5-1)^{2}+(6-1)^{2}+(7-1)^{2} 
    = 3^{2}+4^{2}+5^{2}+6^{2} = \sum_{n=3}^{6} n^{2} 
  \end{align*}
\end{proof}

\begin{example}
  Να δείξετε ότι $ \sum_{n=3}^{6} n^{2} = \sum_{n=1}^{4} (n+2)^{2}   $
\end{example}
\begin{proof}
  \begin{align*}
    \sum_{n=1}^{4} (n+2)^{2} = (1+2)^{2}+(2+2)^{2}+(3+2)^{2}+(4+2)^{2}=
    3^{2}+4^{2}+5^{2}+6^{2} = \sum_{n=3}^{6} n^{2} 
  \end{align*}
\end{proof}

\begin{example}
  Να δείξετε ότι $ \sum_{n=1}^{5} (2n+1) - \sum_{n=3}^{5} (2n+1)  =  
  \sum_{n=1}^{2} (2n+1) $
\end{example}
\begin{proof}
  \begin{align*}
    \sum_{n=1}^{5} (2n+1)- \sum_{n=3}^{5} (2n+1) = 
    (3+5+7+9+11) - (7+9+11) = 3 + 5 = \sum_{n=1}^{2} (2n+1) 
  \end{align*}
\end{proof}

\begin{example}
  Να δείξετε ότι $ \sum_{n=1}^{20} (2n+1) - \sum_{n=8}^{20} (2n+1) = \sum_{n=1}^{7} 
  (2n+1) $ 
\end{example}
\begin{proof}
\item 
  \[
    \sum_{n=1}^{20} (2n+1) - \sum_{n=8}^{20} (2n+1) = \sum_{n=1}^{7} (2n+1) + 
    \sum_{n=8}^{20} (2n+1) - \sum_{n=8}^{20} (2n+1) = 
    \sum_{n=1}^{7} (2n+1)
  \]
\end{proof}

\begin{example}[Τηλεσκοπικό Άθροισμα]
  Να δείξετε ότι $ \sum_{n=1}^{5} \frac{1}{n(n+1)} = 1 - \frac{1}{6}  $
\end{example}
\begin{proof}(Με ανάλυση σε απλά κλάσματα)
\item {}
  Έχουμε $ \frac{1}{n(n+1)} = \frac{A}{n} + \frac{B}{n+1} = 
  \frac{1}{n} - \frac{1}{n+1}$, οπότε
  \begin{equation*}
    \sum_{n=1}^{5} \frac{1}{n(n+1)} = \sum_{n=1}^{5} \left(\frac{1}{n} - 
    \frac{1}{n+1}\right)    
    = \left(\frac{1}{1} - \frac{1}{2}\right) + \left(\frac{1}{2} 
    - \frac{1}{3} \right) + \left(\frac{1}{3} - \frac{1}{4}\right) 
    + \left(\frac{1}{4} - \frac{1}{5}\right) +
    \left(\frac{1}{5} - \frac{1}{6}\right)  = 1 - \frac{1}{6}
  \end{equation*} 
\end{proof}

\begin{mybox3}
\begin{prop}[Γεωμετρική Πρόοδος]
  \label{prop:geom}
  Έστω $ \lambda \in \mathbb{R} \setminus \{ 1 \}  $. Τότε το άθροισμα 
  $ \sum_{k=0}^{n} \lambda ^{k} = 1 + \lambda + \lambda ^{2} + \lambda ^{3} + 
  \cdots + \lambda ^{n} = \frac{\lambda ^{n+1}-1}{\lambda -1}, 
  \quad \; \forall n \in \mathbb{N} $
\end{prop}
\end{mybox3}
\begin{proof} (Με Μαθ. Επαγωγή)
\item {}
  \begin{myitemize}
    \item Για $ n=1 $, έχω: $1+ \lambda = \frac{\lambda ^{2}-1}{\lambda -1} = 
      \frac{\cancel{(\lambda -1)}(\lambda +1)}{\cancel{\lambda -1}} = \lambda +1 $,
      ισχύει.
    \item Έστω ότι η ισότητα ισχύει για $n$, δηλ. 
      $\inlineequation[eq:epagex3]{1+ \lambda + \lambda ^{2} + \lambda ^{3} + 
      \cdots \lambda ^{n} = \frac{\lambda ^{n+1}-1}{\lambda -1}}$.
    \item Θα δείξουμε ότι ισχύει και για $ n+1 $. Πράγματι:
      \begin{align*}
        1+ \lambda + \lambda ^{2}+ \cdots + \lambda ^{n} + \lambda ^{n+1}
          &\overset{\eqref{eq:epagex3}}{=}
          \frac{\lambda ^{n+1}-1}{\lambda -1} + \lambda ^{n+1} = \frac{\lambda
            ^{n+1}-1 + \lambda ^{n+1}(\lambda -1)}{\lambda -1} = 
            \frac{\cancel{\lambda ^{n+1}} -1 + 
          \lambda ^{n+2} - \cancel{\lambda ^{n+1}}}{\lambda -1} \\ 
          &= \frac{\lambda ^{(n +1)+1}-1}{\lambda -1} 
      \end{align*}
  \end{myitemize}
\end{proof}

\begin{example} Να υπολογιστεί το άθροισμα $ \sum_{n=1}^{6} 2^{n} $
  \begin{solution}
    \[
      \sum_{n=1}^{6} 2^{n} = \sum_{n=0}^{5} 2^{n+1} = 2 \cdot \sum_{n=0}^{5} 2^{n} 
      \overset{\ref{prop:geom}}= 2 \cdot \frac{2^{6}-1}{2-1} = 2 \cdot 63 = 126 
    \]
  \end{solution}
\end{example}

\begin{example}
  Να δείξετε ότι $ \sum_{n=1}^{100} \frac{1}{n(n+1)} \leq \sum_{n=1}^{100} 
  \frac{1}{n^{2}} $
\end{example}
\begin{proof}
\item {} 
  Έχουμε $ n+1 > n $ , $ \forall n \in \mathbb{N} $, οπότε
  \[
    \frac{1}{n+1} < \frac{1}{n} \overset{\cdot \frac{1}{n} >0}{\Rightarrow}
    \frac{1}{n(n+1)} < \frac{1}{n^{2}} \Rightarrow \sum_{n=1}^{100} 
    \frac{1}{n(n+1)} \leq \sum_{n=1}^{100} \frac{1}{n^{2}} 
  \] 
\end{proof}

\begin{exercise}
  Να δείξετε ότι $ \sum_{n=1}^{100} \frac{1}{n(n+1)} \leq \sum_{n=2}^{101} 
  \frac{1}{n^{2}} $
\end{exercise} 
\begin{proof}
\item {} 
  Έχουμε $ n-1 < n, \; \forall n \in \mathbb{N} $ και 
  $ n-1>0, \; \forall n \geq 2 $, άρα
  \begin{equation}\label{eq:sumexam}
    \frac{1}{n-1} \leq \frac{1}{n} \overset{\cdot \frac{1}{n} >0}{\Rightarrow}
    \frac{1}{n(n-1)} \leq \frac{1}{n^{2}}, \quad \forall n \geq 2 \Rightarrow 
    \sum_{n=2}^{101} \frac{1}{n(n-1)} \leq 
    \sum_{n=2}^{101} \frac{1}{n^{2}}
  \end{equation} 
  Άρα, 
  \[
    \sum_{n=1}^{100} \frac{1}{n(n+1)} = \sum_{n=2}^{101} \frac{1}{(n-1)n} 
    \overset{\eqref{eq:sumexam}}{\leq} \sum_{n=2}^{101} \frac{1}{n^{2}} 
  \]
\end{proof}

\begin{example}
  Με Μαθηματική Επαγωγή, αποδεικνύονται εύκολα τα παρακάτω αθροίσματα.
  \begin{enumerate}[wide,labelindent=0pt]
    \item $ \sum_{k=1}^{n} k = 1+2+3+\cdots+n = \frac{n(n+1)}{2}, 
      \quad \forall n \in \mathbb{N}  $
    \item $ \sum_{k=1}^{n} k^{2} =1^{2}+2^{2}+3^{2}+\cdots+n^{2} = 
      \frac{n(n+1)(2n+1)}{6}, \quad \forall n \in \mathbb{N} $
    \item $ \sum_{k=1}^{n} k^{3} = 1^{3}+2^{3}+3^{3}+\cdots +n^{3} = 
      \left(\frac{n(n+1)}{2} \right)^{2}, \; \forall n \in \mathbb{N} $
  \end{enumerate}
\end{example}




\end{document}
