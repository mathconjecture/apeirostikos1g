\documentclass{book}

\usepackage{etex}
\usepackage{etoolbox}

%%%%%%%%%%%%%%%%%%%%%%%%%%%%%%%%%%%%%%
% Babel language package
%\usepackage[english,greek]{babel}
% Inputenc font encoding
%\usepackage[utf8]{inputenc}


% \usepackage{xltxtra} 
% \usepackage{xgreek} 
% \setmainfont[Mapping=tex-text]{GFS Didot} 

%\usepackage{kmath,kerkis} % The order of the packages matters; kmath changes the default text font
%\usepackage[T1]{fontenc}
\usepackage{ifxetex}
\ifxetex
    % IF XELATEX

% MINION new font
\usepackage{fontspec}
%\usepackage{mathspec}

\setmainfont[Extension=.ttf,UprightFont=*-Regular,BoldFont=*-Bold,ItalicFont=*-Italic,BoldItalicFont=*-Bold-Italic]{Minion-Pro}
\setsansfont[Extension=.ttf,UprightFont=*H,BoldFont=*HB,ItalicFont=*HI,BoldItalicFont=*HBI]{Vera}
\usepackage{unicode-math}
\setmathfont{latinmodern-math.otf}
%\setmathfont[range=\mathup]{Minion-Pro-Regular.ttf}
%\setmathfont[range=\mathit]{Minion-Pro-Italic.ttf}
%\setmathfont[range=\mathbf]{Minion-Pro-Bold.ttf}
%\setmathfont[range={0222B}]{Minion-Pro-Italic.ttf}
%\setmathfontface\mathfoo{Minion-Pro-Regular.ttf}
%\setoperatorfont\mathfoo
    \usepackage{polyglossia}
    \setdefaultlanguage{greek}
    \setotherlanguage{english}
    %\RequirePackage{unicode-math}
    %\setmathfont{Latin Modern Math}
    %\newcommand{\smblkcircle}{•}
\else
    % IF PDFLATEX
    %\usepackage{tgheros}
    %\renewcommand*\familydefault{\sfdefault}
    %\usepackage[eulergreek]{sansmath}
    %\sansmath
    \usepackage[T1]{fontenc}
    \usepackage[utf8]{inputenc}
    \usepackage[english,greek]{babel}
\fi


\usepackage{anyfontsize}
\newlength{\FONTmain}\setlength{\FONTmain}{9pt}
\newlength{\FONTmainbl}\setlength{\FONTmainbl}{1.2\FONTmain}
\renewcommand{\tiny}        {\fontsize{0.6\FONTmain}{0.6\FONTmainbl}\selectfont}
\renewcommand{\scriptsize}  {\fontsize{0.7\FONTmain}{0.7\FONTmainbl}\selectfont}
\renewcommand{\footnotesize}{\fontsize{0.8\FONTmain}{0.8\FONTmainbl}\selectfont}
\renewcommand{\small}       {\fontsize{0.9\FONTmain}{0.9\FONTmainbl}\selectfont}
\renewcommand{\normalsize}  {\fontsize{1.0\FONTmain}{1.0\FONTmainbl}\selectfont}
\renewcommand{\large}       {\fontsize{1.2\FONTmain}{1.2\FONTmainbl}\selectfont}
\renewcommand{\Large}       {\fontsize{1.4\FONTmain}{1.4\FONTmainbl}\selectfont}
\renewcommand{\LARGE}       {\fontsize{1.6\FONTmain}{1.6\FONTmainbl}\selectfont}
\renewcommand{\huge}        {\fontsize{1.8\FONTmain}{1.8\FONTmainbl}\selectfont}

%%%%%%%%%%%%%%%%%%%%%%%%%%%%%%%%%%%%%%
\usepackage[table,RGB]{xcolor}

\usepackage{geometry}
\geometry{a5paper,top=15mm,bottom=15mm,left=15mm,right=15mm}
\setlength{\parindent}{0pt}


%\usepackage{extsizes}
\usepackage{multicol}

%%%%% math packages %%%%%%%%%%%%%%%%%%
\usepackage[intlimits]{amsmath}
\usepackage{amssymb}
\usepackage{amsfonts}
\usepackage{amsthm}
\usepackage{proof}
\usepackage{mathtools}
\usepackage{extarrows}

\usepackage[italicdiff]{physics}
\usepackage{siunitx}
\usepackage{xfrac}

%%%%%%% symbols packages %%%%%%%%%%%%%%
\usepackage{bm} %for use \bm instead \boldsymbol in math mode
\usepackage{dsfont}
%\usepackage{stmaryrd}
%%%%%%%%%%%%%%%%%%%%%%%%%%%%%%%%%%%%%%%


%%%%%% graphics %%%%%%%%%%%%%%%%%%%%%%%
\usepackage{graphicx}
%\usepackage{color}
%\usepackage{xypic}
%\usepackage[all]{xy}
%\usepackage{calc}

%%%%%% tables %%%%%%%%%%%%%%%%%%%%%%%%%
\usepackage{array}
\usepackage{booktabs}
\usepackage{multirow}
\usepackage{makecell}
\usepackage{minibox}
\usepackage{systeme}
%%%%%%%%%%%%%%%%%%%%%%%%%%%%%%%%%%%%%%%

\usepackage{enumitem}
\usepackage{tikz}
\usetikzlibrary{shapes,angles,calc,arrows,arrows.meta,quotes,intersections}
\usetikzlibrary{decorations.pathmorphing}
\usetikzlibrary{decorations.pathreplacing} 
\usetikzlibrary{decorations.markings,patterns} 
\usepackage{pgfplots}
\pgfplotsset{compat=1.15}

\tikzset{dot/.style={ draw, fill, circle, inner sep=1pt, minimum size=3pt }}
\usepackage{fancyhdr}
%%%%% header and footer rule %%%%%%%%%
%\setlength{\headheight}{14pt}
\renewcommand{\headrulewidth}{0pt}
\renewcommand{\footrulewidth}{0pt}
\fancypagestyle{plain}{\fancyhf{}\rfoot{\thepage}}
\fancypagestyle{vangelis}{\fancyhf{}
    \fancyfootoffset[LE,RO]{10mm}
    \rfoot[]{\thepage}
    \lfoot[\thepage]{}
    \rhead[]{\tikz[remember picture,overlay]{\node[rotate=90,anchor=east] (text) at ([shift={(-5mm,-8mm)}]current page.north east) {\textcolor{Col\thechapter}{\small\strut\leftmark}};
    \fill[Col\thechapter] ([xshift={-2.5mm}]text.east) rectangle++(5mm,5mm);
    }}
    %\lhead[\textcolor{Col\thechapter}{\leftmark}]{}
}
%%%%%%%%%%%%%%%%%%%%%%%%%%%%%%%%%%%%%%%

\usepackage[space]{grffile}


% \definecolor{Col1}{HTML}{eb3b79}
% \definecolor{Col2}{HTML}{9a529f}
% \definecolor{Col3}{HTML}{775ba6}
% \definecolor{Col4}{HTML}{5a68b0}
% \definecolor{Col5}{HTML}{55a0d8}
% \definecolor{Col6}{HTML}{34b0e5}
% \definecolor{Col7}{HTML}{34c1d7}
% \definecolor{Col8}{HTML}{65bc6a}
% \definecolor{Col9}{HTML}{9acb62}
% \definecolor{Col10}{HTML}{d1dd5b}
% \definecolor{Col11}{HTML}{f9ec5d}
% \definecolor{Col12}{HTML}{fbc82a}
% \definecolor{Col13}{HTML}{faa725}
% \definecolor{Col14}{HTML}{f26f47}
% \definecolor{Col15}{HTML}{8e6d65}
% \definecolor{Col16}{HTML}{bdbcbc}
% \definecolor{Col17}{HTML}{79919d}

\definecolor{Col1}{rgp}{0.74, 0.2, 0.64}
\definecolor{Col2}{rgp}{0.0, 0.55, 0.55}
\definecolor{Col3}{rgp}{0.74, 0.2, 0.64}
\definecolor{Col4}{rgp}{0.0, 0.55, 0.55}
\definecolor{Col5}{rgp}{0.74, 0.2, 0.64}
\definecolor{Col6}{rgp}{0.0, 0.55, 0.55}
\definecolor{Col7}{rgp}{0.74, 0.2, 0.64}
\definecolor{Col8}{rgp}{0.0, 0.55, 0.55}
\definecolor{Col9}{rgp}{0.74, 0.2, 0.64}
\definecolor{Col10}{rgp}{0.0, 0.55, 0.55}

\everymath{\displaystyle}

\usepackage[most]{tcolorbox}

\usepackage[explicit]{titlesec}
%%%%%% titlesec settings %%%%%%%%%%%%%
% \titleformat{ command }[ shape ]{ format }{ label }{ sep }{ before-code }[ after-code 
% \titlespacing*{ command }{ left }{ before-sep }{ after-sep }[ right-sep ]
% Chapter


% \titleformat{\chapter}[block]{\huge\bfseries}{\begin{tcolorbox}[colback=Col\thechapter,left=3pt,right=3pt,top=18pt,bottom=18pt,sharp
% corners,boxrule=0pt]\centering\huge\bfseries\textcolor{white}{#1}\end{tcolorbox}}{0pt}{\markboth{#1}}[\clearpage]
% \titlespacing*{\chapter}{0cm}{6\baselineskip}{0\baselineskip}[0ex]
% % Section
% \titleformat{\section}[hang]{\pagestyle{plain}\Large\bfseries\centering}{\begin{tcolorbox}[colback=Col\thechapter!75!white,left=1pt,right=1pt,top=2pt,bottom=2pt,sharp
% corners,boxrule=0pt]\centering\strut\textcolor{white}{#1}\end{tcolorbox}}{0ex}{}
% \titlespacing*{\section}{0cm}{2\baselineskip}{\baselineskip}[0ex]
% % subsection
% \titleformat{\subsection}[hang]{\pagestyle{plain}\large\bfseries\centering}{\begin{tcolorbox}[colback=Col\thechapter!55!white,left=1pt,right=1pt,top=2pt,bottom=2pt,sharp
% corners,boxrule=0pt]\centering\strut\textcolor{white}{#1}\end{tcolorbox}}{0ex}{}
% \titlespacing*{\section}{0cm}{2\baselineskip}{\baselineskip}[0ex]
% % Subsubsection
% \titleformat{\subsubsection}[hang]{\normalsize\bfseries\centering}{}{0ex}{\color{Col\thechapter!45}{#1}}{}
% \titlespacing*{\subsubsection}{0cm}{\baselineskip}{\baselineskip}[0ex]

%% Subsection
%\titleformat{\subsection}[hang]{\large\bfseries\centering}{\tcbox[colback=Col\thechapter!50!white,left=1pt,right=1pt,top=1pt,bottom=1pt,sharp corners]{#1}}{0ex}{}
%\titlespacing*{\subsection}{0cm}{2\baselineskip}{\baselineskip}[0ex]
%% Subsubsection
%\titleformat{\subsubsection}[hang]{\normalsize\bfseries\centering}{}{0ex}{\color{Col\thechapter}{#1}}{}
%\titlespacing*{\subsubsection}{0cm}{\baselineskip}{\baselineskip}[0ex]
%%%%%%%%%%%%%%%%%%%%%%%%%%%%%%%%%%%%%%%




\AtBeginDocument{\pagestyle{vangelis}\normalsize\raggedright}


\newcommand{\twocolumnside}[2]{\begin{minipage}[t]{0.45\linewidth}\raggedright
#1
\end{minipage}\hfill{\color{Col\thechapter}{\vrule width 1pt}}\hfill\begin{minipage}[t]{0.45\linewidth}\raggedright
#2
\end{minipage}
}

\newcommand{\twocolumnsides}[2]{\begin{minipage}[t]{0.45\linewidth}\raggedright
#1
\end{minipage}\hfill\begin{minipage}[t]{0.45\linewidth}\raggedright
#2
\end{minipage}
}

\newcommand{\twocolumnsidesc}[2]{\begin{minipage}{0.45\linewidth}\raggedright
#1
\end{minipage}\hfill\begin{minipage}[c]{0.45\linewidth}\raggedright
#2
\end{minipage}
}

\newcommand{\twocolumnsidesp}[2]{\begin{minipage}[t]{0.35\linewidth}\raggedright
#1
\end{minipage}\hfill\begin{minipage}[t]{0.55\linewidth}\raggedright
#2
\end{minipage}
}

\newcommand{\twocolumnsidesl}[2]{\begin{minipage}[t]{0.55\linewidth}\raggedright
#1
\end{minipage}\hfill\begin{minipage}[t]{0.35\linewidth}\raggedright
#2
\end{minipage}
}


\usepackage{calc}
\usepackage{array}
\definecolor{TabLine}{RGB}{254,254,254}
\newcommand{\TabRowHead}{\rowcolor{TabHeadRow}}
\newcommand{\TabRowHeadCor}{\cellcolor{white}}
\newcommand{\TabRowHCol}{\color{white}\bfseries\boldmath}
\newcommand{\TabCellHead}{\cellcolor{TabHeadRow}\TabRowHCol}
\newenvironment{Mytable}%
    {\begingroup\setlength{\arrayrulewidth}{2pt}\arrayrulecolor{TabLine}
    \colorlet{TabHeadRow}{Col\thechapter}
    \colorlet{TabRowOdd}{Col\thechapter!50!white}
    \colorlet{TabRowEven}{Col\thechapter!25!white}
    \rowcolors{1}{TabRowOdd}{TabRowEven}
    }%
    {\endgroup

}

\usepackage{fancyhdr}
%%%%% header and footer rule %%%%%%%%%
\setlength{\headheight}{14pt}
\renewcommand{\headrulewidth}{0pt}
\renewcommand{\footrulewidth}{0pt}
\fancypagestyle{plain}{\fancyhf{}
\fancyhead{}
\lfoot{\small \hrule \vspace{5pt}\color{Col1} Βαγγέλης Σαπουνάκης}
\cfoot{\small \hrule \vspace{5pt}\color{Col2!75} Φοιτητικό Πρόσημο}
\rfoot{\small \hrule \vspace{5pt} \thepage}}
\fancypagestyle{vangelis}{\fancyhf{}
\lfoot{\small \hrule \vspace{5pt}\color{Col1} Βαγγέλης Σαπουνάκης}
\cfoot{\small \hrule \vspace{5pt}\color{Col2!75} Φοιτητικό Πρόσημο}
\rfoot{\small \hrule \vspace{5pt} \thepage}}

%%%%%%%%%%%%Watermark%%%%%%%%%%%%%%%%%%
 \usepackage[printwatermark]{xwatermark} 
 \newwatermark[allpages,color=blue!8,angle=45,scale=3,xpos=0,ypos=0]{ΠΡΟΣΗΜΟ}
%%%%%%%%%%%%%%%%%%%%%%%%%%%%%%%%%%%%%%

%%%%%%%%% mdframed theorem boxes, breakable and with ref support %%%%%%%%

\usepackage[framemethod=TikZ]{mdframed}

\mdfdefinestyle{mythm}{innertopmargin=0pt,linecolor=Col2!75,linewidth=2pt,
  backgroundcolor=Col2!15, %background color of the box
  shadow=false,shadowcolor=Col2,shadowsize=5pt,% shadows
  frametitleaboveskip=\dimexpr-1.3\ht\strutbox\relax, 
  frametitlealignment={\hspace*{0.03\linewidth}},%
}

\mdfdefinestyle{mydfn}{innertopmargin=0pt,linecolor=Col1!75,linewidth=2pt,
  backgroundcolor=Col1!15, %background color of the box
  shadow=false,shadowcolor=Col1,shadowsize=5pt,% shadows
  frametitleaboveskip=\dimexpr-1.3\ht\strutbox\relax, 
  frametitlealignment={\hspace*{0.03\linewidth}},%
}

\mdfdefinestyle{myprop}{innertopmargin=0pt,linecolor=blue!75,linewidth=2pt,
  backgroundcolor=blue!10, %background color of the box
  shadow=false,shadowcolor=blue,shadowsize=5pt,% shadows
  frametitleaboveskip=\dimexpr-1.3\ht\strutbox\relax, 
  frametitlealignment={\hspace*{0.03\linewidth}},%
}

\mdfdefinestyle{myboxs}{innertopmargin=0pt,linecolor=blue!75,linewidth=0pt,
  backgroundcolor=blue!15, %background color of the box
  shadow=false,shadowcolor=blue,shadowsize=5pt,% shadows
}

% \newcounter{theo}[section]
% \setcounter{theo}{0}
% \renewcommand{\thetheo}{\arabic{section}.\arabic{theo}}


\newenvironment{mythm}[2][]{%
  \refstepcounter{thm}
  % Code for box design goes here.
  \ifstrempty{#1}%
    % if condition (without title)
    {\mdfsetup{
  frametitle={%
    \tikz[baseline=(current bounding box.east),outer sep=0pt]
    \node[anchor=east,rectangle,fill=Col2!75,text=white]
  {\strut Θεώρημα~\thethm};},%
      }%
      % else condition (with title)
      }{\mdfsetup{
  frametitle={%
    \tikz[baseline=(current bounding box.east),outer sep=0pt]
    \node[anchor=east,rectangle,fill=Col2!75,text=white]
  {\strut Θεώρημα~\thethm~~({#1})};},%
      }%
    }%
    % Both conditions
    \mdfsetup{
      style=mythm
    }
    \begin{mdframed}[]\relax\label{#2}}{%
  \end{mdframed}}
  %%%%%%%%%%%%%%%%%%%%%%%%%%%%%%%%%%%%%%%%%%%%%%%%%%%%%%%%%%

\newenvironment{mydfn}[2][]{%
  \refstepcounter{thm}
  % Code for box design goes here.
  \ifstrempty{#1}%
    % if condition (without title)
    {\mdfsetup{
  frametitle={%
    \tikz[baseline=(current bounding box.east),outer sep=0pt]
    \node[anchor=east,rectangle,fill=Col1!75,text=white,draw=Col1!75]
  {\strut Ορισμός~\thethm};},%
      }%
      % else condition (with title)
      }{\mdfsetup{
  frametitle={%
    \tikz[baseline=(current bounding box.east),outer sep=0pt]
    \node[anchor=east,rectangle,fill=Col1!75,text=white,draw=Col1!75]
  {\strut Ορισμός~\thethm~~({#1})};},%
      }%
    }%
    % Both conditions
    \mdfsetup{
      style=mydfn
    }
    \begin{mdframed}[]\relax\label{#2}}{%
  \end{mdframed}}
  %%%%%%%%%%%%%%%%%%%%%%%%%%%%%%%%%%%%%%%%%%%%%%%%%%%%%%%%%%

\newenvironment{myprop}[2][]{%
  \refstepcounter{thm}
  % Code for box design goes here.
  \ifstrempty{#1}%
    % if condition (without title)
    {\mdfsetup{
  frametitle={%
    \tikz[baseline=(current bounding box.east),outer sep=0pt]
    \node[anchor=east,rectangle,fill=blue!50,text=white]
  {\strut Πρόταση~\thethm};},%
      }%
      % else condition (with title)
      }{\mdfsetup{
  frametitle={%
    \tikz[baseline=(current bounding box.east),outer sep=0pt]
    \node[anchor=east,rectangle,fill=blue!50,text=white]
  {\strut Πρόταση~\thethm~~({#1})};},%
      }%
    }%
    % both conditions
    \mdfsetup{
      style=myprop
    }
    \begin{mdframed}[]\relax\label{#2}}{%
  \end{mdframed}}
  %%%%%%%%%%%%%%%%%%%%%%%%%%%%%%%%%%%%%%%%%%%%%%%%%%%%%%%%%%
\newenvironment{myboxs}{%
  % Code for box design goes here.
    \mdfsetup{
      style=myboxs
    }
    \begin{mdframed}[]\relax}{%
\end{mdframed}}
  %%%%%%%%%%%%%%%%%%%%%%%%%%%%%%%%%%%%%%%%%%%%%%%%%%%%%%%%%%
% \renewcommand{\qedsymbol}{$\blacksquare$}

\newcommand{\comb}[2]{\lambda_{1}\vec{#1}_{1} + \cdots + \lambda_{#2}\vec{#1}_{#2}}
\newcommand{\combc}[3]{#2_{1}\vec{#1}_{1} + \cdots + #2_{#3}\vec{#1}_{#3}}
\newcommand{\combb}[2]{\lambda_{1}\vec{#1}_{1} + \lambda_{2}\vec{#1}_{2} + \cdots + 
\lambda_{#2}\vec{#1}_{#2}}

\newcommand{\me}{\mathrm{e}}


\newlist{myitemize}{itemize}{3}
\setlist[myitemize]{label=\textcolor{Col1}{\tiny$\blacksquare$},leftmargin=*}

\newlist{myitemize*}{itemize*}{3}
\setlist[myitemize*]{itemjoin=\hspace{2\baselineskip},label=\textcolor{Col1}{\tiny$\blacksquare$}}

\newlist{myenumerate}{enumerate}{3}
\setlist[enumerate,1]{label=\textcolor{Col1}{\theenumi.},leftmargin=*}
\setlist[enumerate,2]{label=\textcolor{Col1}{\roman*)},leftmargin=*}

\setlist[description]{labelindent=1em,widest=Ιανουα0000,labelsep*=1em,itemindent=0pt,leftmargin=*}

% %%%%%%%%%%%%%%%%%% fancy headings %%%%%%%%%%%%%%%%%%

% %%%%%%%%%%%%%%%%%%%%%%% my boxes %%%%%%%%%%%%%%%%%%%%%%%%%%%%
% \newcommand{\mythm}[1]{
%       \refstepcounter{thm}
%     \begin{tikzpicture}
%         \node[myboxthm] (box1) 
%         {
%             \begin{minipage}{0.9\textwidth}
%                 #1
%             \end{minipage}
%         } ;

%         \node[myboxtitlethm] at (box1.north west) {\strut Θεώρημα~\thethm} ;
%     \end{tikzpicture}
% }

% \newcommand{\mythmm}[2]{
%       \refstepcounter{thm}
%     \begin{tikzpicture}
%         \node[myboxthm] (box1) 
%         {
%             \begin{minipage}{0.9\textwidth}
%                 #2
%             \end{minipage}
%         } ;

%         \node[myboxtitlethm] at (box1.north west) {\strut Θεώρημα~\thethm \; (#1)} ;
%     \end{tikzpicture}
% }

% \newcommand{\mydfn}[1]{
%       \refstepcounter{dfn}
%     \begin{tikzpicture}
%         \node[myboxdfn] (box1) 
%         {
%             \begin{minipage}{0.9\textwidth}
%                 #1
%             \end{minipage}
%         } ;

%         \node[myboxtitledfn] at (box1.north west) {\strut Ορισμός~\thedfn} ;
%     \end{tikzpicture}
% }


% \newcommand{\myprop}[1]{
%       \refstepcounter{thm}
%     \begin{tikzpicture}
%         \node[myboxprop] (box1) 
%         {
%             \begin{minipage}{0.9\textwidth}
%                 #1
%             \end{minipage}
%         } ;

%         \node[myboxtitleprop] at (box1.north west) {\strut Πρόταση~\theprop} ;
%     \end{tikzpicture}
% }

% \newcommand{\mypropp}[2]{
%       \refstepcounter{thm}
%     \begin{tikzpicture}
%         \node[myboxprop] (box1) 
%         {
%             \begin{minipage}{0.9\textwidth}
%                 #2
%             \end{minipage}
%         } ;

%         \node[myboxtitleprop] at (box1.north west) {\strut Πρόταση~\theprop (#1)} ;
%     \end{tikzpicture}
% }

%%%\mybrace{<first>}{<second>}[<Optional text>]
\newcommand{\tikzmark}[1]{\tikz[baseline={(#1.base)},overlay,remember picture] \node[outer
sep=0pt, inner sep=0pt] (#1) {\phantom{A}};}
%% syntax
\NewDocumentCommand\mybrace{mmo}{%
  \IfValueTF {#3}{%
    \begin{tikzpicture}[overlay, remember picture,decoration={brace,amplitude=1ex}]
      \draw[decorate,thick] (#1.north east) -- (#2.south east) 
        node (b) [midway,xshift=13pt,label={right=of b}:{#3}] {};
    \end{tikzpicture}%
  }%
  {%
    \begin{tikzpicture}[overlay, remember picture,decoration={brace,amplitude=1ex}]
      \draw[decorate,thick] (#1.north east) -- (#2.south east);
    \end{tikzpicture}%
  }%
}%

%%%%%%%How to use this %%%%%%%%%%%%%%%%%%%%%%%%%
%use \tikzmark{a} and \tikzmark{b} at first and last \item where the brace is
%wanted
%use the following command after \end{enumerate}
%\mybrace{a}{b}[Text comes here to describe these to items and justify for your
%case]]



%%%%% label inline equations and don't allow reference
\newcommand\inlineeqno{\stepcounter{equation} (\theequation)}

%%%%%defines \inlineequation[<label name>]{<equation>}
%%%%%%%%format use \inlineequation[<label name>]{<equation>}%%%%%%%
\makeatletter
\newcommand*{\inlineequation}[2][]{%
    \begingroup
    % Put \refstepcounter at the beginning, because
    % package `hyperref' sets the anchor here.
    \refstepcounter{equation}%
    \ifx\\#1\\%
\else
    \label{#1}%
\fi
% prevent line breaks inside equation
\relpenalty=10000 %
\binoppenalty=10000 %
\ensuremath{%
    % \displaystyle % larger fractions, ...
    #2%
}%
\quad ~\@eqnnum
\endgroup
}
\makeatother


%%%%%%%%%%%%%%%%%% fancy enumitem cicled label %%%%%%%%%%%%%%%%%%
\newcommand*\circled[1]{\tikz[baseline=(char.base)]{
\node[shape=circle,draw,inner sep=0.3pt] (char) {#1};}}
% use it like \begin{enumerate}[label=\protect\circled{\Alph{enumi}}]
%%%\mybrace{<first>}{<second>}[<Optional text>]
%%% wrap with braces list environments




%%%%%%%%%%%%% puts brace under matrix
\newcommand\undermat[2]{%
  \makebox[0pt][l]{$\smash{\underbrace{\phantom{%
\begin{matrix}#2\end{matrix}}}_{\text{$#1$}}}$}#2}


%circle item inside array or matrix
\newcommand\Circle[1]{%
\tikz[baseline=(char.base)]\node[circle,draw,inner sep=2pt] (char) {#1};}


  %redeftine \eqref so that parenthesis () have the color the link
\makeatletter
\renewcommand*{\eqref}[1]{%
  \hyperref[{#1}]{\textup{\tagform@{\ref*{#1}}}}%
}
\makeatother

%removes qedsymbol and additional vertical space at the end 
\makeatletter
\renewenvironment{proof}[1][\proofname]{\par
  % \pushQED{\hfill\qedhere}% <--- remove the QED business
  \normalfont \topsep6\p@\@plus6\p@\relax
  \trivlist
  \item[\hskip\labelsep
        \itshape
        #1\@addpunct{.}]\ignorespaces
}{%
 % \popQED% <--- remove the QED business
  \endtrivlist\@endpefalse
}
\renewcommand\qedhere{$\blacksquare$} % to ensure code portability
\makeatother


\input{tikz.tex}
\input{myboxes.tex}

\pagestyle{vangelis}

\begin{document}


\chapter{Πραγματικοί Αριθμοί}

\section{Μαθηματική Επαγωγή}


\begin{mybox2}
\begin{thm}
  Έστω $ S \subseteq \mathbb{N} $ τέτοιο ώστε:
  \begin{minipage}[t]{0.25\textwidth}
    \begin{enumerate}[(i)]
      \item  $ 1 \in S $ \hfill \tikzmark{a}
      \item  $ n \in S \Rightarrow n + 1 \in S $ \hfill \tikzmark{b}
    \end{enumerate} 
  \end{minipage}
  \mybrace{a}{b}[$ S = \mathbb{N} $]
\end{thm}
\end{mybox2}

\begin{example}
  Να αποδείξετε ότι $ 4^{n} \geq n^{2}, \; \forall n \in \mathbb{N} $.
\end{example}
\begin{proof}
\item {}
  \begin{myitemize}
    \item Για $ n=1 $, έχω: $ 4^{1} \geq 1^{2} $, ισχύει.
    \item Έστω ότι η ανισότητα ισχύει για $n$, δηλ. 
      $\inlineequation[eq:epagex1]{4^{n} \geq n^{2}}$
    \item Θα δείξουμε ότι ισχύει και για $ n+1 $. Πράγματι:
      \begin{align*}
        4^{n+1} = 4^{n}\cdot 4 \overset{\eqref{eq:epagex1}}{\geq}
        n^{2}\cdot 4 
        = 4n^{2} = n^{2} + 2n^{2} + n^{2} \geq n^{2} + 2n + 1 = (n+1)^{2}.
      \end{align*}
  \end{myitemize}
\end{proof}

\begin{example}
  Να αποδείξετε ότι $ 2^{n} \geq n^{3}, \; \forall n \geq 10 $.
\end{example}
\begin{proof}
\item {}
  \begin{myitemize}
    \item Για $ n=10 $, έχω: $ 2^{10} = 1024 \geq 1000 = 10^{3} $, 
      ισχύει.
    \item Έστω ότι η ανισότητα ισχύει για $n$, δηλ. 
      $\inlineequation[eq:epagex2]{2^{n} \geq n^{3}}$.
    \item Θα δείξουμε ότι ισχύει και για $ n+1 $. Πράγματι:
      \begin{align*}
        2^{n+1} = 2^{n} \cdot 2 \overset{\eqref{eq:epagex2}}{\geq} n^{3} 
        \cdot 2 = 2n^{3} = n^{3} + n^{3} = n^{3} + nn^{2} \overset{n\geq 10}
        {\geq} n^{3}+10n^{2} > n^{3}+7n^{2} &= n^{3} + 3n^{2} + 3n^{2} + n^{2} \\ 
                                            &\geq  n^{3} + 3n^{2} + 3n + 1 \\ 
                                            &= (n+1)^{3}
      \end{align*} 
  \end{myitemize}
\end{proof}

\section{Ανισότητα Bernoulli}
\[
  \boxed{(1+a)^{n} \geq 1 + na, \quad \forall a \geq -1, \; \forall n \in
  \mathbb{N}}
\] 
\begin{proof}
\item {}
  \begin{myitemize}
    \item Για $ n=1 $, έχω: $ (1+a)^{1} = 1+a \geq 1+a = 1 + 1 \cdot a $, ισχύει.
    \item Έστω ότι η ανισότητα ισχύει, για $ n $, δηλ. $\inlineequation[eq:bern]
      {(1+a)^{n} \geq 1 + na}$
    \item Θα δείξουμε ότι ισχύει και για $ n+1 $. Πράγματι
      \begin{align*}
        (1+a)^{n+1} = (1+a)^{n}(1+a) \overset{\eqref{eq:bern}}
        {\underset{a \geq -1}{\geq}} (1+na)(1+a) &= 1 + a + na + na^{2} 
        = 1 + (n+1)a + na^{2} \geq 1 + (n+1)a
      \end{align*}
  \end{myitemize}
\end{proof}

\begin{rem}
  Αν $ n=2,3,4,\ldots $ και $ a>-1 $, τότε ισχύει $ (1+a)^{n}>1+na $.
\end{rem}

\section{Απόλυτη Τιμή}

\begin{dfn}
  Για κάθε $ a \in \mathbb{R} $, θέτουμε
  \[ \abs{a} = \begin{cases} a, & a \geq 0 \\ -a, & a < 0 \end{cases}  \]
\end{dfn}

\begin{rem}
  Άμεση συνέπεια του ορισμού είναι οι σχέσεις:
  \begin{myitemize}
    \item $ \abs{a} \geq 0, \;  \forall a \in \mathbb{R} $
    \item $ -\abs{a} \leq a \leq \abs{a}, \; \forall a \in \mathbb{R} $
  \end{myitemize}
\end{rem}

\begin{mybox3}
\begin{prop}
  \label{prop:absprop1}
  Έστω $ \theta > 0 $. Τότε 
  \[
    \abs{a} \leq \theta \Leftrightarrow - \theta \leq a \leq \theta  
  \]
\end{prop}
\end{mybox3}
\begin{proof}
\item {}
  \begin{description}
    \item [$(  \Rightarrow ) $] 
      Έστω ότι $ \abs{a} \leq \theta, \; a \in \mathbb{R} $ και 
      $ \theta >0 $. 
      \begin{myitemize}
        \item Έστω $ a \geq 0 $. Τότε $ \abs{a} = a $, οπότε:
          $
          0 \leq \abs{a} \leq \theta \Rightarrow 
          0 \leq a \leq \theta \Rightarrow - \theta 
          \leq a \leq \theta
          $ 
        \item Έστω $ a < 0 $. Τότε $ \abs{a} = -a $, οπότε:
          $
          0 \leq \abs{a} \leq \theta \Rightarrow 
          0 < -a \leq \theta \Rightarrow 
          - \theta \leq a < 0 \Rightarrow 
          - \theta \leq a \leq \theta 
          $ 
      \end{myitemize}
    \item [$(\Leftarrow)$] Έστω $ - \theta \leq a \leq \theta $ 
      για $ a \in \mathbb{R} $ και $ \theta >0 $.
      \begin{myitemize}
        \item Έστω $ a \geq 0 $. Τότε $ \abs{a} = a $. Οπότε 
          $ \abs{a} = a \leq \theta $.
        \item Έστω $ a <0 $. Τότε $ \abs{a} = -a $. Οπότε
          $ \abs{a} = -a \leq \theta $.
      \end{myitemize}
  \end{description}
\end{proof}


\begin{mybox3}
\begin{prop}[Τριγωνική Ανισότητα]
  \label{prop:trigineq}
  Για κάθε $ a, b \in \mathbb{R} $, ισχύει:
  \begin{enumerate}[(i)]
    \item $ \abs{a+b} \leq \abs{a} + \abs{b}   $
    \item $ \abs{a} - \abs{b} \leq \abs{a+b}  $
  \end{enumerate}
\end{prop}
\end{mybox3}
\begin{proof}
\item {}
  \begin{enumerate}[(i)]
    \item Έχουμε ότι 
      \[ 
        a \leq \abs{a} \Rightarrow - \abs{a} \leq a \leq \abs{a}
      \] 
      \[
        b \leq \abs{b} \Rightarrow - \abs{b} \leq b \leq \abs{b} 
      \] 
      Με πρόσθεση, προκύπτει
      \[
        - (\abs{a} + \abs{b} ) \leq a + b \leq \abs{a} + \abs{b} 
      \] 
      Οπότε από την πρόταση~\ref{prop:absprop1}, ισχύει:
      $ \abs{a+b} \leq \abs{a} + \abs{b} $
    \item Θέτω $ x = a+b $ και $ y = -b $. Για τους $ x,y $ από το $ \rm{i}) $ 
      ερώτημα, έχουμε:
      \begin{equation}\label{eq:absnew1}
        \abs{x+y} \leq \abs{x} + \abs{y} \Rightarrow \abs{a} \leq 
        \abs{a+b} + \abs{b} \Rightarrow \abs{a+b} \geq \abs{a} - \abs{b}
      \end{equation} 
  \end{enumerate} 
\end{proof} 

\begin{rem}
  Αν θέσουμε $ x = a+b $ και $ y = -a $, τότε για τους $ x,y $ από το $ \rm{i)} $ 
  υποερώτημα της προηγούμενης πρότασης, έχουμε:
  \begin{equation}\label{eq:absnew2}
    \abs{x+y} \leq \abs{x} + \abs{y} \Rightarrow \abs{b} \leq 
    \abs{a+b} + \abs{a} \Rightarrow \abs{a+b} \geq \abs{b} - \abs{a} 
  \end{equation} 
  Άρα, από τις~\eqref{eq:absnew1} και~\eqref{eq:absnew2}, έχουμε
  $ - \abs{a+b} \leq \abs{a} - \abs{b} \leq \abs{a+b} $, οπότε τελικά, 
  από την πρόταση~\ref{prop:absprop1} ισχύει ότι
  \begin{equation}\label{eq:trigs} 
    \abs{\abs{a} - \abs{b}} \leq \abs{a+b}
  \end{equation}

  Τώρα, χρησιμοποιώντας το $ \rm{i)} $ υποερώτημα της πρότασης~\ref{prop:absprop1}, 
  για το $ -b $, έχουμε:
  \[
    \abs{a-b} \leq \abs{a} + \abs{b} 
  \] 
  και από την σχέση~\eqref{eq:trigs}, για $-b$ έχουμε:
  \[
    \abs{\abs{a} - \abs{b}} \leq \abs{a-b} 
  \]
  Οπότε, τελικά έχουμε:
  \[
    \boxed{\abs{\abs{a} - \abs{b}} \leq \abs{a \pm b} \leq \abs{a} + \abs{b}  }
  \]
\end{rem}



\section{Μέγιστο και Ελάχιστο}

\begin{mybox1}
\begin{dfn}
  Έστω $ A \subseteq \mathbb{R} $. Λέμε ότι το $A$ έχει \textbf{μέγιστο} στοιχείο, 
  αν υπάρχει $ \bm{x_{0} \in A} $ τέτοιο ώστε $ a \leq x_{0}, \; \forall a \in A $.
\end{dfn}
\end{mybox1}

\begin{mybox1}
\begin{dfn}
  Έστω $ A \subseteq \mathbb{R} $. Λέμε ότι το $A$ έχει \textbf{ελάχιστο} στοιχείο, 
  αν υπάρχει $ \bm{x_{0} \in A} $ τέτοιο ώστε $ a \geq x_{0}, \; \forall a \in A$.
\end{dfn}
\end{mybox1}

\begin{example}
  Έστω $ A = [0,3] = \{ x \in \mathbb{R} \; : \; 0 \leq x \leq 3 \} $. 
  Το $ x_{0}= 3 $ είναι μέγιστο του $A$, γιατί $ 3 \in A $ και 
  $ a \leq 3, \; \forall a \in A $. Ομοίως $ x_{0}= 0 $ είναι ελάχιστο 
  του $A$, γιατί $ 0 \in A $ και $ a \geq 0, \forall a \in A $.
\end{example}

%todo Να γράψω και άλλα παραδείγματα
\begin{example}
  Το σύνολο $ A=(- \infty, 3] $ έχει μέγιστο στοιχείο το 3, ενώ προφανώς δεν έχει 
  ελάχιστο στοιχείο.
\end{example}

\begin{example}
  Να δείξετε ότι το σύνολο $ A = (0,3) $ δεν έχει μέγιστο στοιχείο.
\end{example}
\begin{proof}(Με άτοπο)
\item {}
  $ A = (0,3) = \{ x \in \mathbb{R} \; : \; 0 < x < 3 \} $. 
  Έστω ότι $ x_{0} = \max A $. Άρα $ x_{0} \in A \Rightarrow  x_{0} 
  < 3$. Άρα  $ (x_{0}, 3) \neq \emptyset \Rightarrow (x_{0},3) \cap A \neq 
  \emptyset \Rightarrow \exists a \in (x_{0},3) \cap A $. Δηλαδή $ 
  \exists a \in A$ τέτοιο ώστε $ a > x_{0} $. Άτοπο, γιατί $ x_{0}= \max A $.
\end{proof}

\begin{mybox3}
\begin{prop}
  Αν $ A \subseteq \mathbb{R} $ έχει μέγιστο στοιχείο, τότε αυτό είναι 
  μοναδικό και συμβολίζεται με $ \bm{\max A} $.
\end{prop}
\end{mybox3}
\begin{proof}
  Έστω ότι το $A$ έχει δυο μέγιστα στοιχεία,  $ x_{0}, {x_{0}}' $.

  Τότε 
  \begin{myitemize}
    \item $ x_{0} $ μέγιστο του $A \Rightarrow a \leq x_{0}, \; \forall a \in A \xRightarrow{{x_{0}}' 
      \in A } {x_{0}}'  \leq x_{0} $ \tikzmark{a}
    \item $ x_{0}' $ μέγιστο του $ A \Rightarrow a \leq {x_{0}}', \; \forall a \in A \xRightarrow{x_{0} 
      \in A } x_{0} \leq {x_{0}}' $ \tikzmark{b}
      \mybrace{a}{b}[$ x_{0} = {x_{0}}' $.] 
  \end{myitemize}
\end{proof}

\begin{mybox3}
\begin{prop}
  Αν $ A \subseteq \mathbb{R} $ έχει ελάχιστο στοιχείο, τότε αυτό είναι 
  μοναδικό και συμβολίζεται με $ \bm{\min A} $.
\end{prop}
\end{mybox3}

\begin{proof}
  Ομοίως 
\end{proof}

\begin{mybox1}
\begin{dfn}
  Έστω $ A \subseteq \mathbb{R}, \; A \neq \emptyset $. Λέμε ότι 
  $A$ είναι \textbf{άνω φραγμένο}, αν $ \bm{\exists x_{0} \in \mathbb{R}}$ τέτοιο ώστε 
  $ a \leq x_{0}, \; \forall a \in A$.
\end{dfn}
\end{mybox1}

\begin{mybox1}
\begin{dfn}
  Έστω $ A \subseteq \mathbb{R}, \; A \neq \emptyset $. Λέμε ότι 
  $A$ είναι \textbf{κάτω φραγμένο}, αν $ \bm{\exists x_{0} \in \mathbb{R}}$ τέτοιο ώστε 
  $ a \geq x_{0}, \; \forall a \in A $.
\end{dfn}
\end{mybox1}

\begin{rems}
\item {}
  \begin{myitemize}
    \item Το μέγιστο (ελάχιστο) στοιχείο ενός συνόλου, όταν υπάρχει, αποτελεί άνω
      (κάτω) φράγμα του συνόλου.
    \item Ένα  άνω (κάτω) φράγμα, ενός συνόλου, όταν ανήκει στο σύνολο, είναι 
      το μέγιστο (ελάχιστο) στοιχείο του συνόλου.
    \item Το άνω (κάτω) φράγμα ενός συνόλου, δεν είναι μοναδικό.
  \end{myitemize}
\end{rems}

\begin{example}
\item {}
  \begin{enumerate}[(i)]
    \item Το  σύνολο $ A = (0,3) $  έχει ως άνω φράγματα τους αριθμούς 
      $ 3, 4, 144, \ldots$ και κάτω φράγματα τους αριθμούς
      $ 0, -1, -2, -128, \ldots $. Συγκεκριμένα, το σύνολο των άνω φραγμάτων του 
      $A$ είναι το $ \{ x \in \mathbb{R} \; : \; x \geq 3 \} = [3,+\infty) $ και 
      το σύνολο των κάτω φραγμάτων είναι το 
      $ \{ x \in \mathbb{R} \; : \; x \leq 0 \} = (-\infty,0] $.

    \item Το σύνολο $ B = [-1,+\infty) $ δεν είναι άνω φραγμένο, ενώ το σύνολο των 
      κάτω φραγμάτων του είναι το $ \{ x \in \mathbb{R} \; : \; x \leq -1 \} $. 
      Παρατηρούμε ότι $ \min B =-1 $ 

    \item Αν $ C= (-\infty,2) \cup {3} $, τότε παρατηρούμε ότι $ \max C = 3 $ και ότι 
      το σύνολο των άνω φραγμάτων είναι το $[3,+\infty) $, ενώ το σύνολο $C$ δεν είναι 
      κάτω φραγμένο.
  \end{enumerate}
\end{example}

\begin{rem}
\item {}
  \begin{myitemize}
    \item Υπάρχουν σύνολα που δεν είναι άνω ή κάτω φραγμένα.
    \item Το  $ 3 $ είναι το ελάχιστο από τα άνω φράγματα του $A$, ενώ το $ 0
      $ είναι το μέγιστο από τα κάτω φράγματα του $A$. 
  \end{myitemize}
\end{rem}


\begin{mybox1}
\begin{dfn}
  Έστω $ A \subseteq \mathbb{R}, \; A \neq \emptyset $. Λέμε ότι το $A$ 
  είναι \textbf{φραγμένο}, αν είναι άνω και κάτω φραγμένο. 
  Τότε υπάρχουν $ m,M \in \mathbb{R} $ ώστε $ m \leq a \leq M, \quad a \in A $ 
\end{dfn}
\end{mybox1}

\begin{mybox3}
\begin{prop}
  Ένα σύνολο $ A \subseteq \mathbb{R}, \; A \neq \emptyset $ είναι 
  φραγμένο αν και μόνο αν $ \exists M>0 \; : \; \abs{a} \leq M, \; 
  \forall a \in A$.
\end{prop}
\end{mybox3}
\begin{proof}
\item {}
  \begin{description}
    \item [$ (\Rightarrow) $] Έστω $A$ φραγμένο. Τότε 
      $ \exists m',M' \in \mathbb{R} \; : \; m' \leq a \leq M', \; 
      \forall a \in A $.

      Επιλέγω $ M = \max \{ \abs{m'}, \abs{M'} \} $. Τότε $ M >0 $ και 
      \begin{gather*}
        -M \leq - \abs{m'} \leq m' \leq a \leq M' \leq \abs{M'} 
        \leq M, \quad \forall a \in A \\
        -M \leq a \leq M, \quad \forall a \in A \\
        \abs{a} \leq M, \quad \forall a \in A
      \end{gather*}

    \item [$ (\Leftarrow) $]
      Έστω ότι $ \exists M>0 \; : \; \abs{a} \leq M, \; \forall a \in 
      A \Leftrightarrow -M \leq a \leq M, \; \forall a \in A$. 

      Τότε προφανώς έχουμε ότι $-M $ και $ M $, 
      είναι αντίστοιχα κάτω  και άνω  φράγματα του $A$, και άρα $A$ 
      φραγμένο.
  \end{description} 
\end{proof}

\begin{mybox3}
\begin{prop}
  Έστω $ A \subseteq \mathbb{R}$, $A \neq \emptyset $ και έστω 
  $ -A = \{ x \in \mathbb{R} \; : \; x = -a, \; a \in A \} = \{ 
  -a \; : \; a \in A\} $. Αν $M$ είναι άνω φράγμα 
  του $A$, τότε το $ -M $ είναι κάτω φράγμα του συνόλου $ -A $.
\end{prop}
\end{mybox3}
\begin{proof}
  Έστω $M$ α.φ.\ του $A$. Τότε 
  \begin{align*}
    a &\leq M, \; \forall a \in A \Rightarrow  \\
    -a &\geq -M, \; \forall a \in A \Rightarrow \\
    -a &\geq -M, \; \forall (-a) \in -A \Rightarrow \\
    x &\geq -M, \; \forall x \in -A, \quad \text{όπου θέσαμε } x=-a 
  \end{align*}
  οπότε $ -M $  κ.φ.\ του $ -A $.
\end{proof}

\begin{mybox3}
\begin{prop}
  Έστω $ A \subseteq \mathbb{R}$, $A \neq \emptyset $ και έστω 
  $ -A = \{ x \in \mathbb{R} \; : \; x = -a, \; a \in A \} = \{ 
  -a \; : \; a \in A\} $. Αν $m$ είναι κάτω  φράγμα 
  του $A$, τότε το $ -m $ είναι άνω  φράγμα του συνόλου $ -A $.
\end{prop}
\end{mybox3}
\begin{proof}
Ομοίως  
\end{proof}


\section{Supremum και Infimum}

\begin{mybox1}
\begin{dfn}
  Έστω $ A \subseteq \mathbb{R}, A \neq \emptyset $ και $A$ άνω 
  φραγμένο. Αν υπάρχει άνω φράγμα $s$ του $A$ τέτοιο ώστε 
  \[
    s \leq M, \quad \text{για κάθε M άνω φράγμα του A}, 
  \] 
  τότε το $s$ ονομάζεται \textbf{supremum} του $A$ και συμβολίζεται $ s=\sup A $.
\end{dfn}
\end{mybox1}

\begin{mybox1}
\begin{dfn}
  Έστω $ A \subseteq \mathbb{R}, A \neq \emptyset $ και $A$ κάτω 
  φραγμένο. Αν υπάρχει κάτω φράγμα $s$ του $A$ τέτοιο ώστε 
  \[
    s \geq m, \quad \text{για κάθε m κάτω φράγμα του A}, 
  \] 
  τότε το $s$ ονομάζεται \textbf{infimum} του $A$ και συμβολίζεται $ s=\inf A $.
\end{dfn}
\end{mybox1}

\begin{example}
  Αν $ A= \{ -1,-2 \} \cup (1,4] $, τότε παρατηρούμε ότι $ \max A = 4, \; \min A =-2 $,
  το σύνολο των άνω φραγμάτων είναι το $ [4,+\infty] $ ενώ το σύνολο των κάτω φραγμάτων
  είναι το $ (-\infty,-2) $. Επομένως, $ \sup A= \max A=4 $ και 
  $ \inf A = \min A= -2 $
\end{example}

\begin{example}
  Έστω $ A = (0,3) $. Θα δείξουμε ότι $ \sup A = 3 $. Πράγματι, 
  το $ 3 $ προφανώς είναι α.φ.\ του $A$. Θα δείξουμε ότι είναι το
  ελάχιστο από τα άνω φράγματα του $A$. Έστω ότι δεν είναι, δηλαδή 
  $ \exists M $ α.φ.\ του $A$, ώστε $ M < 3 \Rightarrow (M,3) \neq 
  \emptyset \Rightarrow (M,3) \cap A \neq \emptyset \Rightarrow \exists 
  a \in (M,3) \cap A $. Τότε, έχουμε ότι $ a \in A $ και $ a > M $, 
  άτοπο, γιατί $M$ είναι α.φ.\ του $A$. Ομοίως αποδεικνύουμε ότι $ 
  \inf A = 0$.
\end{example}

\begin{mybox3}
\begin{prop}
  Έστω $ A \subseteq \mathbb{R}, \; A \neq \emptyset $ και το $A$ έχει μέγιστο 
  στοιχείο. Τότε $ \sup A = \max A $.
\end{prop}
\end{mybox3}
\begin{proof}
\item {}
  Έστω $ x_{0} = \max A \Rightarrow a \leq x_{0}, \; \forall a \in A $, άρα $ x_{0} $ α.φ.\
  του $A$. Άρα το $A$ είναι άνω φραγμένο και επειδή $ A \neq \emptyset $, από το αξίωμα
  πληρότητας υπάρχει το $ \sup A $. 
  Ισχύει $ x_{0} \leq \sup A $, γιατί $ \sup A $ α.φ.\ του $A$ και $ x_{0} \in A $.
  Όμως $ x_{0} $ επίσης α.φ.\ του $A$, άρα $ \sup A \leq x_{0} $, γιατί το $ \sup A $
  είναι το ελάχιστο άνω φράγμα.
  Άρα $ x_{0}= \sup A $.
\end{proof}

\begin{mybox3}
\begin{prop}
  Έστω $ A \subseteq \mathbb{R}, \; A \neq \emptyset $ και το $A$ έχει ελάχιστο
  στοιχείο. Τότε $ \inf A = \min A $.
\end{prop}
\end{mybox3}
\begin{proof}
  Ομοίως 
\end{proof}


\section{Αξίωμα Πληρότητας}

\begin{myitemize}
  \item Κάθε μη κενό και άνω φραγμένο υποσύνολο των πραγματικών αριθμών έχει 
    supremum, δηλαδή έχει ελάχιστο άνω φράγμα.
  \item Κάθε μη κενό και κάτω  φραγμένο υποσύνολο των πραγματικών αριθμών έχει 
    infimum, δηλαδή έχει μέγιστο κάτω φράγμα.
\end{myitemize}

\begin{rem}
\item {}
  \begin{myitemize}
    \item Αν το $A$ δεν είναι άνω φραγμένο, τότε καταχρηστικά  γράφουμε ότι 
      $ \sup A = + \infty $.
    \item Αν το $A$ δεν είναι κάτω φραγμένο, τότε καταχρηστικά  γράφουμε ότι 
      $ \inf A = - \infty $.
  \end{myitemize}
\end{rem}

\begin{mybox3}
\begin{prop}
  \label{prop:epsilonprot}
  Έστω ότι για τον $x \in \mathbb{R}$ ισχύει ότι 
  \[
    0 \leq x < \varepsilon, \quad \forall \varepsilon >0 \Rightarrow x =0 
  \]
\end{prop}
\end{mybox3}
\begin{proof}
\item {}
  Έστω ότι $ 0 \leq x < \varepsilon, \quad \varepsilon >0 $ και 
  $ x \neq 0 \xRightarrow{x \geq 0} x > 0$. Τότε για $ \varepsilon = 
  x > 0$, έχουμε ότι $ 0 \leq x < x $, άτοπο.
\end{proof}

\begin{mybox3}
\begin{prop}[Αρχιμήδεια Ιδιότητα]
\item {}
  \begin{enumerate}[(i)]
    \item Το $ \mathbb{N} $ δεν είναι άνω φραγμένο.
    \item $ \forall y > 0, \; \exists n \in \mathbb{N} \; ; \; n > y$
    \item $ \forall \varepsilon >0, \; \exists n \in \mathbb{N} \; : 
      \; \frac{1}{n} < \varepsilon$
  \end{enumerate}
\end{prop}
\end{mybox3}
\begin{proof}
\item {}
  \begin{enumerate}[(i)]
    \item Έστω ότι το $ \mathbb{N} $ είναι άνω φραγμένο. Λόγω ότι 
      είναι και μη κενό ($ 1 \in \mathbb{N} $), από το αξίωμα 
      Πληρότητας, έχουμε ότι υπάρχει το $ \sup \mathbb{N} $. 
      Τότε από τη χαρακτηριστική ιδιότητα του supremum, έχουμε 
      ότι για $ \varepsilon = 1 >0, \; \exists n \in \mathbb{N} 
      \; : \; \sup \mathbb{N}-1 < n \Leftrightarrow n+1 > \sup
      \mathbb{N} $, άτοπο.

    \item Έστω ότι δεν ισχύει η πρόταση. Τότε  $ \exists y >0, \; 
      \forall n \in \mathbb{N} \; : \; n \leq y$. Δηλαδή
      το  $y$  είναι α.φ.\ του $\mathbb{N}$, άτοπο.

    \item Έστω ότι δεν ισχύει η πρόταση. Τότε $ \exists 
      \varepsilon >0 , \; \forall n \in \mathbb{N} \; : \; 
      \frac{1}{n} \geq \varepsilon  \Leftrightarrow \ n \leq 
      \frac{1}{\varepsilon} $, δηλαδή $ \frac{1}{\varepsilon} $ 
      α.φ.\ του $ \mathbb{N} $, άτοπο. 
  \end{enumerate}
\end{proof}

\section{Χαρακτηριστική Ιδιότητα του Supremum και Infimum}

\begin{mybox3}
\begin{prop}
  Έστω $ A \subseteq \mathbb{R} $, $ A \neq \emptyset $ και άνω φραγμένο. 
  Έστω $ s $ α.φ.\ του $A$. Τότε 
  \[
    s= \sup A \Leftrightarrow \forall \varepsilon >0, \; \exists a \in A \; 
    : \; s - \varepsilon < a
  \]
\end{prop}
\end{mybox3}
\begin{proof}
\item {}
  \begin{description}
    \item[$ (\Rightarrow) $] 
      Έστω $ s = \sup A $. Έστω $ \varepsilon >0 $. Έχουμε $ 
      s - \varepsilon < s $ άρα το $ s- \varepsilon $ δεν είναι 
      α.φ.\ του $A$, άρα $ \exists a \in A $ τέτοιο ώστε $ 
      a > s- \varepsilon$. 

    \item [$ (\Leftarrow) $] 
      Έστω ότι $ \forall \varepsilon >0, \; \exists a \in A \; : 
      \; s- \varepsilon < a$. Θ.δ.ο. $ s = \sup A $. 

      \begin{minipage}{0.18\textwidth}
        \begin{myitemize}
          \item $ A \neq \emptyset $ \hfill \tikzmark{a}
          \item $ A $ άνω φραγμένο \hfill  \tikzmark{b}
        \end{myitemize}
      \end{minipage}

      \mybrace{a}{b}[$ \exists $ το $ \sup A $]

      Έστω ότι $ s \neq \sup A $, και λόγω ότι $ s $ α.φ.\ του $A$ 
      έχουμε ότι $sup A < s $. 

      Επιλέγουμε $ \varepsilon = s - \sup A > 0 $

      Τότε από την υπόθεση έχουμε ότι 
      υπάρχει $ a \in A \; : \; s - \varepsilon < a \Rightarrow s 
      - s + \sup A < a \Rightarrow \sup A < a $, άτοπο, γιατί 
      $ \sup A $ α.φ.\ του $A$.  
  \end{description} 
\end{proof}

\begin{mybox3}
\begin{prop}
  Έστω $ A \subseteq \mathbb{R} $, $ A \neq \emptyset $ και κάτω  φραγμένο. 
  Έστω $ s $ κ.φ.\ του $A$. Τότε 
  \[
    s= \inf A \Leftrightarrow \forall \varepsilon >0, \; \exists a \in A \; 
    : \; a < s + \varepsilon  
  \]
\end{prop}
\end{mybox3}

\begin{mybox3}
\begin{prop}
  Αν $ A, B $ μη-κενά, φραγμένα υποσύνολα του 
  $ \mathbb{R} $ με $ A \subseteq B $ να δείξετε ότι 
  $ \inf B \leq \inf A \leq \sup A \leq \sup B $.
\end{prop}
\end{mybox3}
\begin{proof}
\item {} 
  \begin{myitemize}
    \item Προφανώς $ \inf A \leq \sup A $, γιατί $ \forall x \in A, \; 
      \inf A \leq x \leq \sup A $.
    \item Θα δείξουμε ότι $ \inf B \leq \inf A $. 

      Αρκεί να δείξουμε ότι $ \inf B $ κ.φ.\ του $A$. Πράγματι:

      Έστω $ x \in A \overset{A \subseteq B}{\Rightarrow} x \in B \Rightarrow 
      \inf B \leq x, \; \forall x \in A $. Άρα $ \inf B $ κ.φ.\ του $A$.
    \item Θα δείξουμε οτι $ \sup A \leq \sup B $. 

      Άρκεί να δείξουμε ότι $ \sup B $ α.φ.\ του $A$. Πράγματι:

      Έστω $ x \in A \overset{A \subseteq B}{\Rightarrow} x \in B \Rightarrow 
      x \leq \sup B, \; \forall x \in A $.  Άρα $ \sup B $ α.φ.\ του $A$.
  \end{myitemize}
\end{proof}

\begin{mybox3}
\begin{prop}
  Έστω $ A $ μη κενό και φραγμένο υποσύνολο του $ \mathbb{R} $ και έστω 
  $ -A = \{ -a \; : \; a \in A \} $ Τότε:
  \begin{enumerate}
    \item $ \exists $ το $ \sup (-A) $ και το $ \inf (-A) $.
    \item $ \sup (-A) = - \inf A $
    \item $ \inf (-A) = - \sup A $
  \end{enumerate}
\end{prop}
\end{mybox3}
\begin{proof}
\item {}
  \begin{enumerate}
    \item $ A $ φραγμένο, άρα $ \exists m,M \in \mathbb{R} $ ώστε 
      \begin{gather*} 
        m  \leq  a \leq M, \quad \forall a \in A \\
        -M  \leq - a \leq -m, \quad \forall a \in A \\
        -M  \leq - a \leq -m, \quad \forall (-a) \in -A \\
      \end{gather*}
      άρα, $ -m,-M $ είναι άνω και κάτω φράγματα, αντίστοιχα, του $ -A $. Άρα 
      $ -A $ είναι άνω και κάτω φραγμένο, είναι επίσης και μη κενό (γιατί $A$ μη κενό), 
      επομένως από το αξίωμα πληρότητας υπάρχουν τα $ \sup (-A) $ και $ \inf (-A) $.
    \item Θα δείξουμε ότι $ \sup (-A) \leq - \inf A $ και $ - \inf A \leq \sup (-A) $
      Πράγματι:
      \begin{align*}
        -a &\leq \sup (-A), \quad \forall (-a) \in -A \\
        a &\geq -\sup (-A), \quad \forall (-a) \in -A \\
        a &\geq -\sup (-A), \quad \forall a \in A 
      \end{align*}
      οπότε το $ -\sup(-A) $ είναι κάτω φράγμα του $A$, άρα $ - \sup (-A) \leq \inf A
      \Rightarrow \sup (-A) \geq -\inf A $. Ομοίως 
      \begin{align*}
        \inf A &\leq a, \quad \forall a \in A \\
        - a &\leq - \inf A, \quad \forall a \in A \\
        - a &\leq - \inf A, \quad \forall (-a) \in -A 
      \end{align*}
      οπότε το $ - \inf A $ είναι άνω φράγμα του $ -A $, άρα $ \sup (-A) \leq - \inf
      A $
    \item Από το προηγούμενο ερώτημα, έχουμε $ - \sup A = - \sup (-(-A)) = - (- \inf (-A)) = \inf (-A) $ 
  \end{enumerate}
\end{proof}



\begin{mybox3}
\begin{prop}
  Έστω $ A \subseteq \mathbb{R}, A \neq \emptyset $ και έστω $ \lambda A = 
  \{ x \in \mathbb{R} \; : \; x = \lambda a, \; a \in A \} = \{ \lambda a \; : \; 
  a \in A\} $. Τότε
  \begin{enumerate}[(i)]
    \item $ A $ άνω φραγμένο και $ \lambda >0 \Rightarrow \exists $ το $ \sup A $
      και $ \sup (\lambda A) = \lambda \sup A $
    \item $ A $ κάτω φραγμένο και $ \lambda <0 \Rightarrow \exists $ το $ \sup A $
      και $ \sup (\lambda A) = \lambda \inf A $
  \end{enumerate}
\end{prop}
\end{mybox3}

\begin{proof}
\item {}
  \begin{enumerate}[(i)]
    \item $A$ άνω φραγμένο $ \Rightarrow \exists M \in \mathbb{R} $ τέτοιο ώστε
      \begin{align*}
        a \leq M, \; \forall a \in A
                &\Rightarrow \lambda a \leq \lambda M, \; \forall a \in A \\
                &\Rightarrow \lambda a \leq \lambda M, \; \forall \lambda a \in 
                \lambda A \\
                & \Rightarrow x \leq \lambda M, \; \forall x \in \lambda A \quad
                \text{όπου θέσαμε } x = \lambda a
      \end{align*}
      δηλαδή, το  $ \lambda M $ είναι α.φ.\ του $ \lambda A $.

      Άρα το $ \lambda A $ είναι άνω φραγμένο και μη-κενό (γιατί $ A \neq 
      \emptyset $), άρα υπάρχει το $ \sup (\lambda A) $.
      Θα δείξουμε ότι 
      \[ 
        \sup (\lambda A) \leq \lambda \sup A  \; \text{και} \; \sup (\lambda A) 
        \geq \lambda \sup A 
      \]
      \begin{myitemize}
        \item Για την πρώτη σχέση αρκεί να δείξουμε ότι $ \lambda \sup A 
          $ είναι α.φ.\ του $ \lambda A $. 
          Πράγματι
          \begin{align*}
            a \leq \sup A, \; \forall a \in A 
                        & \Rightarrow \lambda a \leq \lambda \sup A, \; \forall a \in A \\
                        & \Rightarrow \lambda a \leq \lambda \sup A, \; \forall \lambda
                        a \in \lambda A \\
                        & \Rightarrow x \leq \lambda \sup A, \; \forall x \in \lambda A
          \end{align*}
          άρα το $ \lambda \sup A $ είναι α.φ.\ του $ \lambda A $.

        \item Αποδεικνύουμε ότι $ \lambda \sup A $ είναι το ελάχιστο άνω φράγμα 
          του $ \lambda A$.  Πράγματι 

          Έστω $M$ άνω φράγμα του $ \lambda A $ με $ M < \lambda \sup A 
          \overset{\lambda >0} {\Rightarrow} \frac{M}{\lambda} < \sup A $, 
          άτοπο, γιατί $ \frac{ M}{\lambda} $ α.φ.\ του $A$.

          Πράγματι, αφού $ M $ α.φ.\ του $ \lambda A $, τότε
          \begin{align*}
            x \leq M, \; \forall x \in \lambda A 
            &\Rightarrow \lambda a \leq M, \; \forall \lambda a \in \lambda A \\
            &\Rightarrow \lambda a \leq M, \; \forall a \in A \\
            &\Rightarrow a \leq \frac{M}{\lambda}, \; \forall a \in A
          \end{align*} 
          άρα $ \frac{M}{\lambda} $ είναι α.φ.\ του Α.
      \end{myitemize}

    \item Ομοίως
  \end{enumerate}
\end{proof}

\begin{mybox3}
\begin{prop}
  Έστω $ A, B $, μη-κενά και φραγμένα υποσύνολα του $ \mathbb{R} $. Τότε: 
  \begin{enumerate}
    \item Υπάρχουν τα $ \sup {(A \cup B)} $ και $ \inf {(A \cup B)} $
    \item $ \sup {(A \cup B)} = \max \{ \sup A, \sup B \} $
    \item $ \inf {(A \cup B)} = \min \{ \inf A, \inf B \} $
  \end{enumerate}
\end{prop}
\end{mybox3}
\begin{proof}
\item {}
  Έστω $ x \in A \cup B \Rightarrow 
  \begin{cases} 
    x \in A  \\
    x \in B  
  \end{cases} \Rightarrow 
  \begin{cases} 
    x \leq \sup A, \quad \forall x \in A \\
    x \leq \sup B,  \quad \forall x \in B     
  \end{cases}
  $  
  Άρα $ x \leq \max \{ \sup A, \sup B \} , \quad \forall x \in A \cup B $.

  Δηλαδή, το $ \max \{ \sup A, \sup B \} $ είναι άνω φράγμα του συνόλου 
  $ A \cup B $. Άρα $ A \cup B $ είναι άνω φραγμένο κ μη κενό 
  (γιατί $A,B$ μη κενά), οπότε από το αξίωμα πληρότητας υπάρχει το 
  $ \sup {(A \cup B)} $. 
  Προφανώς ισχύει ότι $ \sup {(A \cup B)} \leq \max \{ \sup A, \sup B \} $.
  Επομένως αρκεί να δείξουμε ότι $ \max \{ \sup A, \sup B \} \leq \sup {(A \cup B)}
  $. Πράγματι, 
  \[
    \begin{rcases} 
      A \subseteq A \cup B \\
      B \subseteq A \cup B \\
    \end{rcases} \Rightarrow 
    \begin{rcases} 
      \sup A \leq \sup {(A \cup B)} \\
      \sup B \leq \sup {(A \cup B)} \\
    \end{rcases} \Rightarrow 
    \max \{ \sup A, \sup B \} \leq \sup {(A \cup B)}
  \] 
  Ομοίως για το infimum.
\end{proof}

\begin{lem}
  \label{lem:vare2}
  Έστω $ x,y \in \mathbb{R} $ και $ x<y+ \varepsilon \quad \forall \varepsilon >0 $. 
  Τότε $ x \leq y $.
\end{lem}
\begin{proof}(Με άτοπο)
\item {}
  Έστω $ x < y+ \varepsilon \quad \forall \varepsilon >0 $ και $ x>y $. Τότε 
  έχουμε ότι $ x-y>0 $ και άν θέσουμε  $ \varepsilon = x-y $, τότε από την 
  υπόθεση έχουμε ότι $ x< y+ x-y \Rightarrow x<x $, άτοπο.
\end{proof}

\begin{mybox3}
\begin{prop}
  Έστω $ A,B $ μη κενά και φραγμένα υποσύνολα του $ \mathbb{R} $ και έστω 
  $ A+B= \{ a+b \; : \; a \in A \; \text{και} \; b \in B \} $.
  Τότε:
  \begin{enumerate}
    \item Υπάρχει το $ \sup (A+B) $ και το $ \inf (A+B) $.
    \item $ \sup {(A+B)}= \sup A + \sup B $
    \item $ \inf {(A+B)}= \inf A + \inf B $
  \end{enumerate}
\end{prop}
\end{mybox3}
\begin{proof}
\item {}
  \begin{enumerate}
    \item $ A \neq \emptyset $ και φραγμένο, άρα από αξίωμα πληρότητας υπάρχουν τα 
      $ \sup A, \inf A $. Ομοίως και για το σύνολο $B$.
      \[
        \left.
          \begin{aligned}
            a \leq \sup A, \quad \forall a \in A \\
            b \leq \sup B, \quad \forall b \in B 
          \end{aligned}
        \right\} \Rightarrow a+b \leq \sup A + \sup B, \quad \forall a \in A 
        \; \text{και} \; \forall b \in B
      \]
      Άρα $ \sup A + \sup B $ είναι άνω φράγμα του $ A+B $. Άρα το $ A+B $ είναι 
      άνω φραγμένο και μη κενό (γιατί $A,B$ μη κενά), οπότε από το αξίωμα πληρότητας, 
      υπάρχει το $ \sup {(A+B)} $. Ομοίως αποδεικνύεται ότι υπάρχει και το 
      $ \inf {(A+B)} $. 
    \item Αφού $ \sup A + \sup B $ άνω φράγμα του $ A+B $, έχουμε:
      \begin{equation}\label{eq:supsum1}
        \sup {(A+B)} \leq \sup A + \sup B 
      \end{equation} 
      Αρκεί να δείξουμε ότι και $ \sup A + \sup B \leq \sup {(A+B)} $. Πράγματι, έστω 
      $ \varepsilon > 0 $. Τότε:
      \[
        \left.
          \begin{aligned} 
            \exists a \in A \; : \; \sup A - \frac{\varepsilon}{2} < a < \sup A 
            \quad \text{(από χαρ. ιδιοτ.\ του $\sup A$)} \\
            \exists b \in B \; : \; \sup B - \frac{\varepsilon}{2} < b < \sup B 
            \quad \text{(από χαρ. ιδιοτ.\ του $\sup B$)} 
          \end{aligned} 
        \right\} \Rightarrow \sup A + \sup B - \varepsilon < a+b \leq \sup {(A+B)} 
      \] 
      Άρα 
      \[
        \sup A + \sup B < \sup {(A+B)} + \varepsilon , \quad \forall \varepsilon >0
      \] 
      Άρα από την πρόταση~\ref{lem:vare2} έχουμε ότι 
      \begin{equation}\label{eq:supsum2}
        \sup A + \sup B \leq \sup {(A+B)}
      \end{equation}
      Οπότε από τις σχέσεις~\eqref{eq:supsum1} και~\eqref{eq:supsum2}, έπεται ότι 
      $ \sup {(A+B)} = \sup A + \sup B $. 
    \item Ομοίως
  \end{enumerate}
\end{proof}

\begin{exercise}
  Έστω $ A,B $ μη κενά και φραγμένα υποσύνολα του $ \mathbb{R} $ και έστω 
  \[
    A-B= \{ a-b \; : \; a \in A \; \text{και} \; b \in B \} . 
    A\cdot B= \{ a\cdot b \; : \; a \in A \; \text{και} \; b \in B \} . 
  \]
  Τότε να εξετάσετε αν ισχύουν οι παρακάτω σχέσεις:
  \begin{enumerate}
    \item $ \sup {(A-B)}= \sup A - \sup B $
    \item $ \inf {(A-B)}= \inf A - \inf B $
    \item $ \sup {(A\cdot B)}= \sup A \cdot  \sup B $
    \item $ \inf {(A\cdot B)}= \inf A \cdot  \inf B $
  \end{enumerate}
\end{exercise}
\begin{solution}
  Όχι, αν θεωρήσουμε σύνολα $ A = \{ 0,1 \} , B = \{ 1,2,3 \} $ για τις 
  δύο πρώτες σχέσεις, και $ A = \{ 1,2 \} , B = \{ -1,-2 \} $, για τις δύο 
  τελευταίες.
\end{solution}

\begin{rem}
  Για το supremum και το infimum του συνόλου $ A-B $ ισχύουν τα εξής:

  \twocolumnsides{
    \begin{align*}
      \sup {(A-B)}&= \sup {(A)+(-B)} \\ 
                  &= \sup A + \sup (-B) \\
                  &= \sup A + (- \inf B) \\
                  &= \sup A - \inf B
    \end{align*}
    }{
    \begin{align*}
      \inf {(A-B)}&= \inf {(A)+(-B)} \\ 
                  &= \inf A + \inf (-B) \\
                  &= \inf A + (- \sup B) \\
                  &= \inf A - \sup B
  \end{align*}}
\end{rem}

\begin{lem}
  \label{lem:vare3}
  Έστω $ x,y \in \mathbb{R}^{+} $, σταθεροί και έστω $ \varepsilon \cdot y >x, \quad 
  \forall \varepsilon >1$. Τότε $ x \leq y $
\end{lem}
\begin{proof}(Με άτοπο)
  Έστω $ \varepsilon \cdot y > x, \quad \forall \varepsilon >1 $  και $ x>y $.
  Τότε, $ \frac{x}{y} > 1$, οπότε αν θέσουμε $ \varepsilon = \frac{x}{y} > 1 $,
  τότε από υπόθεση έχουμε ότι $ x < \varepsilon \cdot y \Rightarrow x <
  \frac{x}{y} \cdot y = x \Rightarrow x<x $, άτοπο.
\end{proof}

\begin{mybox3}
\begin{prop}
Έστω $ A,B $ μη κενά και άνω φραγμένα σύνολα θετικών, πραγματικών αριθμών και έστω 
$ A \cdot B = \{ a\cdot b \; : \; a \in A, b \in B \} $. Τότε υπάρχει το $ \sup {(A
\cdot B)} $ και ισχύει ότι $ \sup {(A\cdot B)} = \sup A \cdot \sup B $.
\end{prop}
\end{mybox3}
\begin{proof}
  Έχουμε 
  \[
  \begin{rcases}
    a \leq \sup A, \quad \forall a \in A \\ 
    b \leq \sup B, \quad \forall b \in B  
  \end{rcases} \Rightarrow a\cdot b \leq \sup A \cdot \sup B, 
  \quad \forall a \in A \; \text{και} \; b \in B
  \]
  Άρα ο αριθμός $ \sup A \cdot \sup B $ είναι άνω φράγμα του συνόλου $ A\cdot B $. 
  Άρα το $ A \cdot B $ είναι άνω φραγμένο, και μη κενό (γιατί $A$, $B$ μη κενά), 
  οπότε από το αξίωμα πληρότητας, υπάρχει το $ \sup (A\cdot B) $. Προφανώς ισχύει ότι 
  $ \sup {(A\cdot B)} \leq \sup A \cdot \sup B $. Οπότε, αρκεί να δείξουμε ότι 
  $ \sup A \cdot \sup B \leq \sup {(A\cdot B)} $. Πράγματι, έστω $ \varepsilon > 1 $, 
  τότε:
  \[
  \begin{rcases}
    \exists a \in A \; : \; \frac{\sup A}{\sqrt{\varepsilon}} < a \leq \sup A \\
    \exists b \in B \; : \; \frac{\sup B}{\sqrt{\varepsilon}} < b \leq \sup B 
  \end{rcases} \Rightarrow 
  \frac{\sup A \cdot \sup B}{\sqrt{\varepsilon}} < a \cdot b \leq \sup {(A\cdot B)},
  \quad \forall a \in A \; \text{και} \; b \in B
\] 
Άρα 
$
  \frac{\sup A \cdot \sup B}{\sqrt{\varepsilon}} < \sup {(A\cdot B)}, 
  \quad \forall \varepsilon > 1 \Rightarrow  
  \sup A \cdot \sup B < \varepsilon \cdot \sup {(A\cdot B)}, 
  \quad \forall \varepsilon > 1 \overset{\ref{lem:vare3}}{\Rightarrow}  
  \sup A \cdot \sup B \leq \sup {(A\cdot B)} 
$ 
\end{proof}

\section{Ακέραιο Μέρος}

\begin{mybox3}
\begin{prop}
  Έστω $ x \in \mathbb{R} $. Τότε υπάρχει μοναδικός αριθμός $ m \in \mathbb{Z} $ τέτοιος ώστε $\inlineequation[eq:akermer]{m \leq x < m +1}$.
\end{prop}
\end{mybox3}

% \begin{proof}
%   Έστω $ x \in \mathbb{R} $. Επειδή $ \mathbb{N} $ όχι άνω φραγμένο, τότε από την 
%   Αρχιμήδεια Ιδιότητα, $ \exists n \in \mathbb{N} $ με $ n > x $. Έστω $ S = \{ 
%   n \in \mathbb{N} \; : \; n > x\} \Rightarrow S \neq \emptyset $ (υπάρχει το $n$). 
%   Άρα το $S$ έχει ελάχιστο στοιχείο, έστω $ n_{0} $. Άρα $ n_{0}-1 \leq x \leq 
%   n_{0} \Rightarrow m \leq n \leq m+1$, δηλαδή $ m= n_{0}-1 = [x] $.
%   Αποδεικνύουμε την μοναδικότητα:
%   Έστω $ m' $ με $ m' \leq x \leq m'+1 $ και έστω $ m' \leq m $. Τότε 
%   $ m' +1 \leq m \leq x $, άτοπο.
% \end{proof}

\begin{mybox1}
\begin{dfn}
  Έστω $ x \in \mathbb{R} $. Ο μοναδικός ακέραιος που ικανοποιεί την 
σχέση~\eqref{eq:akermer} ονομάζεται \textbf{ακέραιο μέρος} του $x$ και συμβολίζεται $ [x] $.
\end{dfn}
\end{mybox1}

Ουσιαστικά, το ακέραιο μέρος ενός πραγματικού αριθμού $ x $ είναι ο \textbf{μεγαλύτερος} ακέραιος που δεν υπερβαίνει τον $x$, ή που είναι ίσος με τον $x$,
αν ο $x$ είναι ακέραιος.

Το ακέραιο μέρος ενός πραγματικού αριθμού ικανοποιεί τις παρακάτω ιδιότητες.
\begin{myitemize}
  \item $ x = [x]+ \theta, \quad x \in \mathbb{R}, \; \theta \in \left[0,1\right) $
  \item $ [x+a]= [x] + a, \quad x \in \mathbb{R}, \; a \in \mathbb{Z} $
  \item $ [x]+[-x] = 
    \begin{cases} 
      0, & x \in \mathbb{Z} \\
      -1, & x \not \in \mathbb{Z}
    \end{cases}$
\end{myitemize}

\begin{example}
\item {}
  \begin{enumerate}[(i)]
    \item $ [3]=3 $
    \item $ [3,14] = 3  $
    \item $ [-3,14] =-4 $
  \end{enumerate}
\end{example}


\section{Ρητοί και Άρρητοι}

\begin{lem}
\item {}
  \begin{enumerate}[(i)]
    \item $n$ άρτιος $ \Leftrightarrow n^{2} $ άρτιος.
    \item $ n $ περιττός $ \Leftrightarrow n^{2} $ περιττός.
  \end{enumerate}
\end{lem}

\begin{proof}
\item {}
  \begin{enumerate}[(i)]
    \item 
      \begin{description}
        \item [($ \Rightarrow $ )] 
          Έστω $ n $ άρτιος $ \Rightarrow n =2k, \; k \in \mathbb{Z} 
          \Rightarrow n^{2} = (2k)^{2} = 4k^{2} = 2\cdot (2k)^{2} $ άρτιος. 
        \item [($ \Leftarrow $)] Έστω $ n^{2} $ άρτιος και $n$ περιττός. Τότε 
          $ n \cdot n = n^{2} $ περιττός. Άτοπο.
      \end{description}

    \item Ομοίως
  \end{enumerate}
\end{proof}

\begin{rem}
  Στις αποδείξεις τους παραπάνω λήμματος, χρησιμοποιήσαμε ότι 
  \begin{enumerate}[(i)]
    \item άρτιος $ \cdot $ άρτιος = άρτιος
    \item περιττός $ \cdot $ περιττός = περιττός
    \item άρτιος $ \cdot $ περιττός = άρτιος
  \end{enumerate}
\end{rem}

\begin{mybox2}
\begin{thm}
  Ο $ \sqrt{2} $ είναι άρρητος.
\end{thm}
\end{mybox2}

\begin{proof}
  Έστω $ \sqrt{2} $ όχι άρρητος. Άρα $ \sqrt{2} $ ρητός, δηλαδή $ \exists m,n 
  \in \mathbb{Z} $, με $ (m,n)=1 $, δηλαδή $ m,n $ πρώτοι μεταξύ τους
  τ.ω. $ \sqrt{2} = \frac{m}{n} \Rightarrow 2 = \frac{m^{2}}{n^{2}} \Rightarrow 
  m^{2} = 2n^{2} \Rightarrow m^{2}$ είναι άρτιος $ \Rightarrow m $ άρτιος 
  $ \Rightarrow m = 2k, \; k \in \mathbb{Z}$. 

  Άρα $ (2k)^{2} = 2n^{2} \Rightarrow 4k^{2}=2n^{2} \Rightarrow n^{2} = 2k^{2} 
  \Rightarrow n^{2} $ άρτιος $ \Rightarrow n $ άρτιος. Άτοπο, γιατί $ (m,n)=1 $.
\end{proof}

\begin{example}
  Ο $ \sqrt{3} $ είναι άρρητος.
\end{example}

\begin{proof}
  Έστω $ \sqrt{3} $ όχι άρρητος. Άρα $ \sqrt{3} $ ρητός, δηλαδή $ \exists m,n 
  \in \mathbb{Z} $, με $ (m,n)=1 $, δηλαδή $ m,n $ πρώτοι μεταξύ τους
  τ.ω. $ \sqrt{3} = \frac{m}{n} \Rightarrow 3 = \frac{m^{2}}{n^{2}} \Rightarrow 
  m^{2} = 3n^{2} \Rightarrow 3 \mid m^{2} \Rightarrow 3 \mid m  \Rightarrow m 
  \Rightarrow m = 3k, \; k \in \mathbb{Z}$. 

  Άρα $ (3k)^{2} = 3n^{2} \Rightarrow 9k^{2}=3n^{2} \Rightarrow n^{2} = 3k^{2} 
  \Rightarrow 3 \mid n^{2} \Rightarrow  3 \mid n$,  άτοπο, γιατί $ (m,n)=1 $.
\end{proof}

\begin{lem}
  $ 3 \mid m^{2} \Rightarrow 3 \mid m $
\end{lem}

\begin{proof}
  Έστω ότι $ 3 \nmid m \Rightarrow m = 3n +1 $ ή $ m = 3n+2 $. 
  Αν $ m=3n+1 \Rightarrow m^{2} = (3n+1)^{2} = \ldots 3(3n^{2}+2n)+1 \Rightarrow 
  3 \nmid m^{2}$, άτοπο και αν $ m =3n+2 \Rightarrow m^{2}=(3n+2)^{2} = \ldots = 
  3(3n^{2}+4n+1)+1 \Rightarrow 3 \nmid m^{2}$ άτοπο.
\end{proof}

\begin{rem}
  Ομοίως αποδεικνύονται και ότι οι $ \sqrt{5}$ και  $ \sqrt{6} $ είναι άρρητοι, 
  και γενικότερα ισχύει η επόμενη πρόταση.
\end{rem}

\begin{mybox3}
\begin{prop}
  $ \sqrt{n} $ άρρητος $ \Leftrightarrow n $ όχι τετράγωνο κάποιου φυσικού αριθμού.
\end{prop}
\end{mybox3}

\begin{example}
  Να δείξετε ότι ο αριθμός $ \sqrt[3]{2} $ είναι άρρητος.
\end{example}

\begin{proof}
  Έστω $ \sqrt[3]{2} $ όχι άρρητος. Άρα $ \sqrt[3]{2} $ ρητός, δηλαδή $ \exists m,n 
  \in \mathbb{Z} $, με $ (m,n)=1 $, δηλαδή $ m,n $ πρώτοι μεταξύ τους
  τ.ω. $ \sqrt[3]{2} = \frac{m}{n} \Rightarrow 2 = \frac{m^{3}}{n^{3}} \Rightarrow 
  m^{3} = 2n^{3} \Rightarrow m^{3} $ άρτιος $ \Rightarrow m $ άρτιος. 

  Άρα $ (2k)^{3} = 2n^{3} \Rightarrow 2k^{3}=2n^{3} \Rightarrow n^{3} = 4k^{3} 
  \Rightarrow n $ άρτιος,  άτοπο, γιατί $ (m,n)=1 $.
\end{proof}

\begin{example}
  Να δείξετε ότι $ \sqrt{2} + \sqrt{3} $ είναι άρρητος.
\end{example}

\begin{proof}
  Έστω ότι $ \sqrt{2} + \sqrt{3} $ είναι ρητός. Τότε και ο $ (\sqrt{2} + \sqrt{3} )
  ^{2} = 2 + 2 \sqrt{6} + 3 = 2 \sqrt{6} + 5 $ είναι ρητός, δηλαδή ο $ \sqrt{6} $ 
  είναι ρητός, άτοπο.
\end{proof}

\begin{mybox3}
\begin{prop}
  $
  \left.
    \begin{tabular}{l}
      $q$ ρητός \\
      $r$ άρρητος
    \end{tabular}
  \right\}  \Rightarrow (q+r) $ άρρητος
\end{prop}
\end{mybox3}

\begin{proof}
\item {}
  Έστω $(q+r)$ ρητός. Τότε ο $ r = (q+r)-q $ είναι ρητός ως άθροισμα δύο 
  ρητών. Άτοπο.
\end{proof}

\begin{mybox3}
\begin{prop}
  $
  \left.
    \begin{tabular}{l}
      $r$ άρρητος \\
      $s$ άρρητος
    \end{tabular}
  \right\}  \Rightarrow (r+s) $ ρητός ή άρρητος.
\end{prop}
\end{mybox3}

\begin{proof}
  Για παράδειγμα, ο αριθμός $ \sqrt{2} + \sqrt{3} $ είναι άρρητος, ενώ ο 
  $ \sqrt{2} - \sqrt{2} = 0 $ ρητός, και γενικά ισχύει ότι ο αριθμός
  $ \underbrace{r}_{\text{άρρητος}}+ \underbrace{(q-r)}_{\text{άρρητος}} =q $ 
  είναι ρητός, αν ο $q$ είναι ρητός.
\end{proof}

\begin{mybox3}
\begin{prop}
  $
  \left.
    \begin{tabular}{l}
      $q$ ρητός \\
      $r$ άρρητος
    \end{tabular}
  \right\}  \Rightarrow  $ \begin{tabular}{l}
    $ (q \cdot r) $ ρητός $ \Leftrightarrow q =0 $ \\
    $(q \cdot r)$   άρρητος $ \Leftrightarrow q \neq 0 $
  \end{tabular}
\end{prop}
\end{mybox3}
\begin{proof}
  Πράγματι, αν $ q=0 $ (ρητός) και $ r $ άρρητος, τότε $ q \cdot r =0 $ είναι ρητός, 
  αλλά αν $ q \neq 0 $ τότε $ (q \cdot r) $ είναι άρρητος, γιατί αλλιώς ο 
  $ r = \underbrace{(q \cdot r)}_{\text{ρητός}} \cdot \underbrace{q^{-1}}_{\text{
  ρητός}} $ είναι ρητός ως γινόμενο ρητών. Άτοπο.
\end{proof}

\begin{example}
  Υπάρχει αριθμός $ a \in \mathbb{R} $ τέτοιος ώστε $ a^{2} $ άρρητος και $ a^{4} $ 
  ρητός; 
\end{example}

\begin{proof}
  Ναι, ο $ a= \sqrt[4]{2} $. Πράγματι, $ a^{2} = \sqrt{2} $ άρρητος, και $ 
  a^{4} = 2$ ρητός.
\end{proof}

\begin{example}
  Υπάρχουν αριθμοί, $ a,b $ άρρητοι, ώστε $ a+b, a\cdot b $ να είναι ρητοί;
\end{example}

\begin{proof}
  Ναι οι $ a= \sqrt{2} $ και $ b= - \sqrt{2} $, οι οποίοι είναι και οι δύο άρρητοι 
  και $ a+b= \sqrt{2} - \sqrt{2} = 0 $ ρητός, και $ a\cdot b = \sqrt{2} \cdot (- 
  \sqrt{2}) = -2 $, επίσης ρητός.
\end{proof}


\section{Πυκνότητα Ρητών και Άρρητων}

\begin{mybox3}
\begin{prop}
  Σε κάθε ανοιχτό διάστημα πραγματικών αριθμών, υπάρχει ρητός.
\end{prop}
\end{mybox3}

\begin{proof}
\item {}
  Θα δείξουμε ότι $ (a,b) $ διάστημα στο $ \mathbb{R} \Rightarrow \exists q 
  \in \mathbb{Q} $, τ.ω. $ a < q < b $. Πράγματι:

  Αν $ a<b \Rightarrow b-a >0 \overset{\text{Αρχ.Ιδιοτ.}}{\Rightarrow} \exists 
  n_{0} \in \mathbb{N} \; : \; \inlineequation[eq:pykn]{\frac{1}{n_{0}} < b-a} $. Τότε, θεωρούμε:
  \begin{myitemize}
    \item $ x = \frac{[na]+1}{n} > \frac{na}{n} = a $
    \item $ x = \frac{[na]+1}{n} \leq \frac{na + 1}{n} = a+ \frac{1}{n} 
      \overset{\eqref{eq:pykn}}{<} a+ b - a=b $

      Άρα ο $ x= \frac{[na]+1}{n} \in \mathbb{Q} \in (a,b) $.
  \end{myitemize}
\end{proof}

\begin{mybox3}
\begin{prop}
  Σε κάθε ανοιχτό διάστημα πραγματικών αριθμών, υπάρχει άρρητος.
\end{prop}
\end{mybox3}

\begin{proof}
\item {}
  Θα δείξουμε ότι αν $ (a,b) $ τυχαίο διάστημα στο $ \mathbb{R} $ τότε 
  $ \exists r \in \mathbb{R} \setminus \mathbb{Q} $ τ.ω. $a < r < b$. Πράγματι.
  Έστω $ \sqrt{2} $ (τυχαίος) άρρητος. Τότε
  $ a < b \Leftrightarrow a - \sqrt{2} < b- \sqrt{2} \overset{\text{Πυκν. Ρητών}}{\Rightarrow} \exists q \in \mathbb{Q} \; : \;  a - \sqrt{2} < q < b - 
  \sqrt{2} \Leftrightarrow  a < \underbrace{q + \sqrt{2}}_{\text{άρρητος}} < b $ 
\end{proof}



\end{document}
