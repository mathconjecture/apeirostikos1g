\documentclass{book}

\usepackage{etex}
\usepackage{etoolbox}

%%%%%%%%%%%%%%%%%%%%%%%%%%%%%%%%%%%%%%
% Babel language package
%\usepackage[english,greek]{babel}
% Inputenc font encoding
%\usepackage[utf8]{inputenc}


% \usepackage{xltxtra} 
% \usepackage{xgreek} 
% \setmainfont[Mapping=tex-text]{GFS Didot} 

%\usepackage{kmath,kerkis} % The order of the packages matters; kmath changes the default text font
%\usepackage[T1]{fontenc}
\usepackage{ifxetex}
\ifxetex
    % IF XELATEX

% MINION new font
\usepackage{fontspec}
%\usepackage{mathspec}

\setmainfont[Extension=.ttf,UprightFont=*-Regular,BoldFont=*-Bold,ItalicFont=*-Italic,BoldItalicFont=*-Bold-Italic]{Minion-Pro}
\setsansfont[Extension=.ttf,UprightFont=*H,BoldFont=*HB,ItalicFont=*HI,BoldItalicFont=*HBI]{Vera}
\usepackage{unicode-math}
\setmathfont{latinmodern-math.otf}
%\setmathfont[range=\mathup]{Minion-Pro-Regular.ttf}
%\setmathfont[range=\mathit]{Minion-Pro-Italic.ttf}
%\setmathfont[range=\mathbf]{Minion-Pro-Bold.ttf}
%\setmathfont[range={0222B}]{Minion-Pro-Italic.ttf}
%\setmathfontface\mathfoo{Minion-Pro-Regular.ttf}
%\setoperatorfont\mathfoo
    \usepackage{polyglossia}
    \setdefaultlanguage{greek}
    \setotherlanguage{english}
    %\RequirePackage{unicode-math}
    %\setmathfont{Latin Modern Math}
    %\newcommand{\smblkcircle}{•}
\else
    % IF PDFLATEX
    %\usepackage{tgheros}
    %\renewcommand*\familydefault{\sfdefault}
    %\usepackage[eulergreek]{sansmath}
    %\sansmath
    \usepackage[T1]{fontenc}
    \usepackage[utf8]{inputenc}
    \usepackage[english,greek]{babel}
\fi


\usepackage{anyfontsize}
\newlength{\FONTmain}\setlength{\FONTmain}{9pt}
\newlength{\FONTmainbl}\setlength{\FONTmainbl}{1.2\FONTmain}
\renewcommand{\tiny}        {\fontsize{0.6\FONTmain}{0.6\FONTmainbl}\selectfont}
\renewcommand{\scriptsize}  {\fontsize{0.7\FONTmain}{0.7\FONTmainbl}\selectfont}
\renewcommand{\footnotesize}{\fontsize{0.8\FONTmain}{0.8\FONTmainbl}\selectfont}
\renewcommand{\small}       {\fontsize{0.9\FONTmain}{0.9\FONTmainbl}\selectfont}
\renewcommand{\normalsize}  {\fontsize{1.0\FONTmain}{1.0\FONTmainbl}\selectfont}
\renewcommand{\large}       {\fontsize{1.2\FONTmain}{1.2\FONTmainbl}\selectfont}
\renewcommand{\Large}       {\fontsize{1.4\FONTmain}{1.4\FONTmainbl}\selectfont}
\renewcommand{\LARGE}       {\fontsize{1.6\FONTmain}{1.6\FONTmainbl}\selectfont}
\renewcommand{\huge}        {\fontsize{1.8\FONTmain}{1.8\FONTmainbl}\selectfont}

%%%%%%%%%%%%%%%%%%%%%%%%%%%%%%%%%%%%%%
\usepackage[table,RGB]{xcolor}

\usepackage{geometry}
\geometry{a5paper,top=15mm,bottom=15mm,left=15mm,right=15mm}
\setlength{\parindent}{0pt}


%\usepackage{extsizes}
\usepackage{multicol}

%%%%% math packages %%%%%%%%%%%%%%%%%%
\usepackage[intlimits]{amsmath}
\usepackage{amssymb}
\usepackage{amsfonts}
\usepackage{amsthm}
\usepackage{proof}
\usepackage{mathtools}
\usepackage{extarrows}

\usepackage[italicdiff]{physics}
\usepackage{siunitx}
\usepackage{xfrac}

%%%%%%% symbols packages %%%%%%%%%%%%%%
\usepackage{bm} %for use \bm instead \boldsymbol in math mode
\usepackage{dsfont}
%\usepackage{stmaryrd}
%%%%%%%%%%%%%%%%%%%%%%%%%%%%%%%%%%%%%%%


%%%%%% graphics %%%%%%%%%%%%%%%%%%%%%%%
\usepackage{graphicx}
%\usepackage{color}
%\usepackage{xypic}
%\usepackage[all]{xy}
%\usepackage{calc}

%%%%%% tables %%%%%%%%%%%%%%%%%%%%%%%%%
\usepackage{array}
\usepackage{booktabs}
\usepackage{multirow}
\usepackage{makecell}
\usepackage{minibox}
\usepackage{systeme}
%%%%%%%%%%%%%%%%%%%%%%%%%%%%%%%%%%%%%%%

\usepackage{enumitem}
\usepackage{tikz}
\usetikzlibrary{shapes,angles,calc,arrows,arrows.meta,quotes,intersections}
\usetikzlibrary{decorations.pathmorphing}
\usetikzlibrary{decorations.pathreplacing} 
\usetikzlibrary{decorations.markings,patterns} 
\usepackage{pgfplots}
\pgfplotsset{compat=1.15}

\tikzset{dot/.style={ draw, fill, circle, inner sep=1pt, minimum size=3pt }}
\usepackage{fancyhdr}
%%%%% header and footer rule %%%%%%%%%
%\setlength{\headheight}{14pt}
\renewcommand{\headrulewidth}{0pt}
\renewcommand{\footrulewidth}{0pt}
\fancypagestyle{plain}{\fancyhf{}\rfoot{\thepage}}
\fancypagestyle{vangelis}{\fancyhf{}
    \fancyfootoffset[LE,RO]{10mm}
    \rfoot[]{\thepage}
    \lfoot[\thepage]{}
    \rhead[]{\tikz[remember picture,overlay]{\node[rotate=90,anchor=east] (text) at ([shift={(-5mm,-8mm)}]current page.north east) {\textcolor{Col\thechapter}{\small\strut\leftmark}};
    \fill[Col\thechapter] ([xshift={-2.5mm}]text.east) rectangle++(5mm,5mm);
    }}
    %\lhead[\textcolor{Col\thechapter}{\leftmark}]{}
}
%%%%%%%%%%%%%%%%%%%%%%%%%%%%%%%%%%%%%%%

\usepackage[space]{grffile}


% \definecolor{Col1}{HTML}{eb3b79}
% \definecolor{Col2}{HTML}{9a529f}
% \definecolor{Col3}{HTML}{775ba6}
% \definecolor{Col4}{HTML}{5a68b0}
% \definecolor{Col5}{HTML}{55a0d8}
% \definecolor{Col6}{HTML}{34b0e5}
% \definecolor{Col7}{HTML}{34c1d7}
% \definecolor{Col8}{HTML}{65bc6a}
% \definecolor{Col9}{HTML}{9acb62}
% \definecolor{Col10}{HTML}{d1dd5b}
% \definecolor{Col11}{HTML}{f9ec5d}
% \definecolor{Col12}{HTML}{fbc82a}
% \definecolor{Col13}{HTML}{faa725}
% \definecolor{Col14}{HTML}{f26f47}
% \definecolor{Col15}{HTML}{8e6d65}
% \definecolor{Col16}{HTML}{bdbcbc}
% \definecolor{Col17}{HTML}{79919d}

\definecolor{Col1}{rgp}{0.74, 0.2, 0.64}
\definecolor{Col2}{rgp}{0.0, 0.55, 0.55}
\definecolor{Col3}{rgp}{0.74, 0.2, 0.64}
\definecolor{Col4}{rgp}{0.0, 0.55, 0.55}
\definecolor{Col5}{rgp}{0.74, 0.2, 0.64}
\definecolor{Col6}{rgp}{0.0, 0.55, 0.55}
\definecolor{Col7}{rgp}{0.74, 0.2, 0.64}
\definecolor{Col8}{rgp}{0.0, 0.55, 0.55}
\definecolor{Col9}{rgp}{0.74, 0.2, 0.64}
\definecolor{Col10}{rgp}{0.0, 0.55, 0.55}

\everymath{\displaystyle}

\usepackage[most]{tcolorbox}

\usepackage[explicit]{titlesec}
%%%%%% titlesec settings %%%%%%%%%%%%%
% \titleformat{ command }[ shape ]{ format }{ label }{ sep }{ before-code }[ after-code 
% \titlespacing*{ command }{ left }{ before-sep }{ after-sep }[ right-sep ]
% Chapter


% \titleformat{\chapter}[block]{\huge\bfseries}{\begin{tcolorbox}[colback=Col\thechapter,left=3pt,right=3pt,top=18pt,bottom=18pt,sharp
% corners,boxrule=0pt]\centering\huge\bfseries\textcolor{white}{#1}\end{tcolorbox}}{0pt}{\markboth{#1}}[\clearpage]
% \titlespacing*{\chapter}{0cm}{6\baselineskip}{0\baselineskip}[0ex]
% % Section
% \titleformat{\section}[hang]{\pagestyle{plain}\Large\bfseries\centering}{\begin{tcolorbox}[colback=Col\thechapter!75!white,left=1pt,right=1pt,top=2pt,bottom=2pt,sharp
% corners,boxrule=0pt]\centering\strut\textcolor{white}{#1}\end{tcolorbox}}{0ex}{}
% \titlespacing*{\section}{0cm}{2\baselineskip}{\baselineskip}[0ex]
% % subsection
% \titleformat{\subsection}[hang]{\pagestyle{plain}\large\bfseries\centering}{\begin{tcolorbox}[colback=Col\thechapter!55!white,left=1pt,right=1pt,top=2pt,bottom=2pt,sharp
% corners,boxrule=0pt]\centering\strut\textcolor{white}{#1}\end{tcolorbox}}{0ex}{}
% \titlespacing*{\section}{0cm}{2\baselineskip}{\baselineskip}[0ex]
% % Subsubsection
% \titleformat{\subsubsection}[hang]{\normalsize\bfseries\centering}{}{0ex}{\color{Col\thechapter!45}{#1}}{}
% \titlespacing*{\subsubsection}{0cm}{\baselineskip}{\baselineskip}[0ex]

%% Subsection
%\titleformat{\subsection}[hang]{\large\bfseries\centering}{\tcbox[colback=Col\thechapter!50!white,left=1pt,right=1pt,top=1pt,bottom=1pt,sharp corners]{#1}}{0ex}{}
%\titlespacing*{\subsection}{0cm}{2\baselineskip}{\baselineskip}[0ex]
%% Subsubsection
%\titleformat{\subsubsection}[hang]{\normalsize\bfseries\centering}{}{0ex}{\color{Col\thechapter}{#1}}{}
%\titlespacing*{\subsubsection}{0cm}{\baselineskip}{\baselineskip}[0ex]
%%%%%%%%%%%%%%%%%%%%%%%%%%%%%%%%%%%%%%%




\AtBeginDocument{\pagestyle{vangelis}\normalsize\raggedright}


\newcommand{\twocolumnside}[2]{\begin{minipage}[t]{0.45\linewidth}\raggedright
#1
\end{minipage}\hfill{\color{Col\thechapter}{\vrule width 1pt}}\hfill\begin{minipage}[t]{0.45\linewidth}\raggedright
#2
\end{minipage}
}

\newcommand{\twocolumnsides}[2]{\begin{minipage}[t]{0.45\linewidth}\raggedright
#1
\end{minipage}\hfill\begin{minipage}[t]{0.45\linewidth}\raggedright
#2
\end{minipage}
}

\newcommand{\twocolumnsidesc}[2]{\begin{minipage}{0.45\linewidth}\raggedright
#1
\end{minipage}\hfill\begin{minipage}[c]{0.45\linewidth}\raggedright
#2
\end{minipage}
}

\newcommand{\twocolumnsidesp}[2]{\begin{minipage}[t]{0.35\linewidth}\raggedright
#1
\end{minipage}\hfill\begin{minipage}[t]{0.55\linewidth}\raggedright
#2
\end{minipage}
}

\newcommand{\twocolumnsidesl}[2]{\begin{minipage}[t]{0.55\linewidth}\raggedright
#1
\end{minipage}\hfill\begin{minipage}[t]{0.35\linewidth}\raggedright
#2
\end{minipage}
}


\usepackage{calc}
\usepackage{array}
\definecolor{TabLine}{RGB}{254,254,254}
\newcommand{\TabRowHead}{\rowcolor{TabHeadRow}}
\newcommand{\TabRowHeadCor}{\cellcolor{white}}
\newcommand{\TabRowHCol}{\color{white}\bfseries\boldmath}
\newcommand{\TabCellHead}{\cellcolor{TabHeadRow}\TabRowHCol}
\newenvironment{Mytable}%
    {\begingroup\setlength{\arrayrulewidth}{2pt}\arrayrulecolor{TabLine}
    \colorlet{TabHeadRow}{Col\thechapter}
    \colorlet{TabRowOdd}{Col\thechapter!50!white}
    \colorlet{TabRowEven}{Col\thechapter!25!white}
    \rowcolors{1}{TabRowOdd}{TabRowEven}
    }%
    {\endgroup

}

\usepackage{fancyhdr}
%%%%% header and footer rule %%%%%%%%%
\setlength{\headheight}{14pt}
\renewcommand{\headrulewidth}{0pt}
\renewcommand{\footrulewidth}{0pt}
\fancypagestyle{plain}{\fancyhf{}
\fancyhead{}
\lfoot{\small \hrule \vspace{5pt}\color{Col1} Βαγγέλης Σαπουνάκης}
\cfoot{\small \hrule \vspace{5pt}\color{Col2!75} Φοιτητικό Πρόσημο}
\rfoot{\small \hrule \vspace{5pt} \thepage}}
\fancypagestyle{vangelis}{\fancyhf{}
\lfoot{\small \hrule \vspace{5pt}\color{Col1} Βαγγέλης Σαπουνάκης}
\cfoot{\small \hrule \vspace{5pt}\color{Col2!75} Φοιτητικό Πρόσημο}
\rfoot{\small \hrule \vspace{5pt} \thepage}}

%%%%%%%%%%%%Watermark%%%%%%%%%%%%%%%%%%
 \usepackage[printwatermark]{xwatermark} 
 \newwatermark[allpages,color=blue!8,angle=45,scale=3,xpos=0,ypos=0]{ΠΡΟΣΗΜΟ}
%%%%%%%%%%%%%%%%%%%%%%%%%%%%%%%%%%%%%%

%%%%%%%%% mdframed theorem boxes, breakable and with ref support %%%%%%%%

\usepackage[framemethod=TikZ]{mdframed}

\mdfdefinestyle{mythm}{innertopmargin=0pt,linecolor=Col2!75,linewidth=2pt,
  backgroundcolor=Col2!15, %background color of the box
  shadow=false,shadowcolor=Col2,shadowsize=5pt,% shadows
  frametitleaboveskip=\dimexpr-1.3\ht\strutbox\relax, 
  frametitlealignment={\hspace*{0.03\linewidth}},%
}

\mdfdefinestyle{mydfn}{innertopmargin=0pt,linecolor=Col1!75,linewidth=2pt,
  backgroundcolor=Col1!15, %background color of the box
  shadow=false,shadowcolor=Col1,shadowsize=5pt,% shadows
  frametitleaboveskip=\dimexpr-1.3\ht\strutbox\relax, 
  frametitlealignment={\hspace*{0.03\linewidth}},%
}

\mdfdefinestyle{myprop}{innertopmargin=0pt,linecolor=blue!75,linewidth=2pt,
  backgroundcolor=blue!10, %background color of the box
  shadow=false,shadowcolor=blue,shadowsize=5pt,% shadows
  frametitleaboveskip=\dimexpr-1.3\ht\strutbox\relax, 
  frametitlealignment={\hspace*{0.03\linewidth}},%
}

\mdfdefinestyle{myboxs}{innertopmargin=0pt,linecolor=blue!75,linewidth=0pt,
  backgroundcolor=blue!15, %background color of the box
  shadow=false,shadowcolor=blue,shadowsize=5pt,% shadows
}

% \newcounter{theo}[section]
% \setcounter{theo}{0}
% \renewcommand{\thetheo}{\arabic{section}.\arabic{theo}}


\newenvironment{mythm}[2][]{%
  \refstepcounter{thm}
  % Code for box design goes here.
  \ifstrempty{#1}%
    % if condition (without title)
    {\mdfsetup{
  frametitle={%
    \tikz[baseline=(current bounding box.east),outer sep=0pt]
    \node[anchor=east,rectangle,fill=Col2!75,text=white]
  {\strut Θεώρημα~\thethm};},%
      }%
      % else condition (with title)
      }{\mdfsetup{
  frametitle={%
    \tikz[baseline=(current bounding box.east),outer sep=0pt]
    \node[anchor=east,rectangle,fill=Col2!75,text=white]
  {\strut Θεώρημα~\thethm~~({#1})};},%
      }%
    }%
    % Both conditions
    \mdfsetup{
      style=mythm
    }
    \begin{mdframed}[]\relax\label{#2}}{%
  \end{mdframed}}
  %%%%%%%%%%%%%%%%%%%%%%%%%%%%%%%%%%%%%%%%%%%%%%%%%%%%%%%%%%

\newenvironment{mydfn}[2][]{%
  \refstepcounter{thm}
  % Code for box design goes here.
  \ifstrempty{#1}%
    % if condition (without title)
    {\mdfsetup{
  frametitle={%
    \tikz[baseline=(current bounding box.east),outer sep=0pt]
    \node[anchor=east,rectangle,fill=Col1!75,text=white,draw=Col1!75]
  {\strut Ορισμός~\thethm};},%
      }%
      % else condition (with title)
      }{\mdfsetup{
  frametitle={%
    \tikz[baseline=(current bounding box.east),outer sep=0pt]
    \node[anchor=east,rectangle,fill=Col1!75,text=white,draw=Col1!75]
  {\strut Ορισμός~\thethm~~({#1})};},%
      }%
    }%
    % Both conditions
    \mdfsetup{
      style=mydfn
    }
    \begin{mdframed}[]\relax\label{#2}}{%
  \end{mdframed}}
  %%%%%%%%%%%%%%%%%%%%%%%%%%%%%%%%%%%%%%%%%%%%%%%%%%%%%%%%%%

\newenvironment{myprop}[2][]{%
  \refstepcounter{thm}
  % Code for box design goes here.
  \ifstrempty{#1}%
    % if condition (without title)
    {\mdfsetup{
  frametitle={%
    \tikz[baseline=(current bounding box.east),outer sep=0pt]
    \node[anchor=east,rectangle,fill=blue!50,text=white]
  {\strut Πρόταση~\thethm};},%
      }%
      % else condition (with title)
      }{\mdfsetup{
  frametitle={%
    \tikz[baseline=(current bounding box.east),outer sep=0pt]
    \node[anchor=east,rectangle,fill=blue!50,text=white]
  {\strut Πρόταση~\thethm~~({#1})};},%
      }%
    }%
    % both conditions
    \mdfsetup{
      style=myprop
    }
    \begin{mdframed}[]\relax\label{#2}}{%
  \end{mdframed}}
  %%%%%%%%%%%%%%%%%%%%%%%%%%%%%%%%%%%%%%%%%%%%%%%%%%%%%%%%%%
\newenvironment{myboxs}{%
  % Code for box design goes here.
    \mdfsetup{
      style=myboxs
    }
    \begin{mdframed}[]\relax}{%
\end{mdframed}}
  %%%%%%%%%%%%%%%%%%%%%%%%%%%%%%%%%%%%%%%%%%%%%%%%%%%%%%%%%%
% \renewcommand{\qedsymbol}{$\blacksquare$}

\newcommand{\comb}[2]{\lambda_{1}\vec{#1}_{1} + \cdots + \lambda_{#2}\vec{#1}_{#2}}
\newcommand{\combc}[3]{#2_{1}\vec{#1}_{1} + \cdots + #2_{#3}\vec{#1}_{#3}}
\newcommand{\combb}[2]{\lambda_{1}\vec{#1}_{1} + \lambda_{2}\vec{#1}_{2} + \cdots + 
\lambda_{#2}\vec{#1}_{#2}}

\newcommand{\me}{\mathrm{e}}


\newlist{myitemize}{itemize}{3}
\setlist[myitemize]{label=\textcolor{Col1}{\tiny$\blacksquare$},leftmargin=*}

\newlist{myitemize*}{itemize*}{3}
\setlist[myitemize*]{itemjoin=\hspace{2\baselineskip},label=\textcolor{Col1}{\tiny$\blacksquare$}}

\newlist{myenumerate}{enumerate}{3}
\setlist[enumerate,1]{label=\textcolor{Col1}{\theenumi.},leftmargin=*}
\setlist[enumerate,2]{label=\textcolor{Col1}{\roman*)},leftmargin=*}

\setlist[description]{labelindent=1em,widest=Ιανουα0000,labelsep*=1em,itemindent=0pt,leftmargin=*}

% %%%%%%%%%%%%%%%%%% fancy headings %%%%%%%%%%%%%%%%%%

% %%%%%%%%%%%%%%%%%%%%%%% my boxes %%%%%%%%%%%%%%%%%%%%%%%%%%%%
% \newcommand{\mythm}[1]{
%       \refstepcounter{thm}
%     \begin{tikzpicture}
%         \node[myboxthm] (box1) 
%         {
%             \begin{minipage}{0.9\textwidth}
%                 #1
%             \end{minipage}
%         } ;

%         \node[myboxtitlethm] at (box1.north west) {\strut Θεώρημα~\thethm} ;
%     \end{tikzpicture}
% }

% \newcommand{\mythmm}[2]{
%       \refstepcounter{thm}
%     \begin{tikzpicture}
%         \node[myboxthm] (box1) 
%         {
%             \begin{minipage}{0.9\textwidth}
%                 #2
%             \end{minipage}
%         } ;

%         \node[myboxtitlethm] at (box1.north west) {\strut Θεώρημα~\thethm \; (#1)} ;
%     \end{tikzpicture}
% }

% \newcommand{\mydfn}[1]{
%       \refstepcounter{dfn}
%     \begin{tikzpicture}
%         \node[myboxdfn] (box1) 
%         {
%             \begin{minipage}{0.9\textwidth}
%                 #1
%             \end{minipage}
%         } ;

%         \node[myboxtitledfn] at (box1.north west) {\strut Ορισμός~\thedfn} ;
%     \end{tikzpicture}
% }


% \newcommand{\myprop}[1]{
%       \refstepcounter{thm}
%     \begin{tikzpicture}
%         \node[myboxprop] (box1) 
%         {
%             \begin{minipage}{0.9\textwidth}
%                 #1
%             \end{minipage}
%         } ;

%         \node[myboxtitleprop] at (box1.north west) {\strut Πρόταση~\theprop} ;
%     \end{tikzpicture}
% }

% \newcommand{\mypropp}[2]{
%       \refstepcounter{thm}
%     \begin{tikzpicture}
%         \node[myboxprop] (box1) 
%         {
%             \begin{minipage}{0.9\textwidth}
%                 #2
%             \end{minipage}
%         } ;

%         \node[myboxtitleprop] at (box1.north west) {\strut Πρόταση~\theprop (#1)} ;
%     \end{tikzpicture}
% }

%%%\mybrace{<first>}{<second>}[<Optional text>]
\newcommand{\tikzmark}[1]{\tikz[baseline={(#1.base)},overlay,remember picture] \node[outer
sep=0pt, inner sep=0pt] (#1) {\phantom{A}};}
%% syntax
\NewDocumentCommand\mybrace{mmo}{%
  \IfValueTF {#3}{%
    \begin{tikzpicture}[overlay, remember picture,decoration={brace,amplitude=1ex}]
      \draw[decorate,thick] (#1.north east) -- (#2.south east) 
        node (b) [midway,xshift=13pt,label={right=of b}:{#3}] {};
    \end{tikzpicture}%
  }%
  {%
    \begin{tikzpicture}[overlay, remember picture,decoration={brace,amplitude=1ex}]
      \draw[decorate,thick] (#1.north east) -- (#2.south east);
    \end{tikzpicture}%
  }%
}%

%%%%%%%How to use this %%%%%%%%%%%%%%%%%%%%%%%%%
%use \tikzmark{a} and \tikzmark{b} at first and last \item where the brace is
%wanted
%use the following command after \end{enumerate}
%\mybrace{a}{b}[Text comes here to describe these to items and justify for your
%case]]



%%%%% label inline equations and don't allow reference
\newcommand\inlineeqno{\stepcounter{equation} (\theequation)}

%%%%%defines \inlineequation[<label name>]{<equation>}
%%%%%%%%format use \inlineequation[<label name>]{<equation>}%%%%%%%
\makeatletter
\newcommand*{\inlineequation}[2][]{%
    \begingroup
    % Put \refstepcounter at the beginning, because
    % package `hyperref' sets the anchor here.
    \refstepcounter{equation}%
    \ifx\\#1\\%
\else
    \label{#1}%
\fi
% prevent line breaks inside equation
\relpenalty=10000 %
\binoppenalty=10000 %
\ensuremath{%
    % \displaystyle % larger fractions, ...
    #2%
}%
\quad ~\@eqnnum
\endgroup
}
\makeatother


%%%%%%%%%%%%%%%%%% fancy enumitem cicled label %%%%%%%%%%%%%%%%%%
\newcommand*\circled[1]{\tikz[baseline=(char.base)]{
\node[shape=circle,draw,inner sep=0.3pt] (char) {#1};}}
% use it like \begin{enumerate}[label=\protect\circled{\Alph{enumi}}]
%%%\mybrace{<first>}{<second>}[<Optional text>]
%%% wrap with braces list environments




%%%%%%%%%%%%% puts brace under matrix
\newcommand\undermat[2]{%
  \makebox[0pt][l]{$\smash{\underbrace{\phantom{%
\begin{matrix}#2\end{matrix}}}_{\text{$#1$}}}$}#2}


%circle item inside array or matrix
\newcommand\Circle[1]{%
\tikz[baseline=(char.base)]\node[circle,draw,inner sep=2pt] (char) {#1};}


  %redeftine \eqref so that parenthesis () have the color the link
\makeatletter
\renewcommand*{\eqref}[1]{%
  \hyperref[{#1}]{\textup{\tagform@{\ref*{#1}}}}%
}
\makeatother

%removes qedsymbol and additional vertical space at the end 
\makeatletter
\renewenvironment{proof}[1][\proofname]{\par
  % \pushQED{\hfill\qedhere}% <--- remove the QED business
  \normalfont \topsep6\p@\@plus6\p@\relax
  \trivlist
  \item[\hskip\labelsep
        \itshape
        #1\@addpunct{.}]\ignorespaces
}{%
 % \popQED% <--- remove the QED business
  \endtrivlist\@endpefalse
}
\renewcommand\qedhere{$\blacksquare$} % to ensure code portability
\makeatother


\input{tikz.tex}
\input{myboxes.tex}


\pagestyle{vangelis}

\zexternaldocument*{chapter1}

\begin{document}

\refstepcounter{chapter}
\chapter{Ακολουθίες}

\section{Ορισμός Ακολουθίας}

\begin{mybox1}
  \begin{dfn}
    \textcolor{Col1}{Ακολουθία} πραγματικών αριθμών ονομάζεται 
    κάθε συνάρτηση με πεδίο ορισμού τους φυσικούς αριθμούς. 
    \begin{align*}
      a \colon &\mathbb{N} \to \mathbb{R} \\
               &n \mapsto a(n)=a_{n} \quad (\text{ν-οστός όρος})
    \end{align*} 
    Οι ακολουθίες συμβολίζονται ως $ (a_{n})_{n \in \mathbb{N}} $ 
    ή $ ( a_{n} ) _{n=1}^{+\infty}$  ή $ \{ a_{n} \} _{n \in \mathbb{N}} $ ή 
    $ \{ a_{n} \} _{n=1}^{+\infty}$ , κλπ.
  \end{dfn}
\end{mybox1}

\begin{mybox1}
  \begin{dfn}
    \textcolor{Col1}{Σύνολο Τιμών} (Σ.Τ.) της ακολουθίας 
    $ (a_{n})_{n \in \mathbb{N}} $, ονομάζουμε το \textbf{σύνολο των όρων της}, 
    δηλαδή το σύνολο $ \{ a_{1}, a_{2}, \ldots, a_{n} \} $ το οποίο μπορεί να 
    είναι πεπερασμένο ή άπειρο.
  \end{dfn}
\end{mybox1}

\begin{examples}
\item {}
  \begin{enumerate}[i)]
    \item Η ακολουθία $ Η ακολουθία a_{n} = n, \; \forall n \in \mathbb{N} $. 
      Έχει Σ.Τ.\  το σύνολο $  \{ 1,2,3, \ldots \} $.
    \item Η ακολουθία $ a_{n}=\left(\frac{1}{n}\right)_{n \in \mathbb{N}} $. 
      Έχει Σ.Τ.\ το σύνολο $  \left\{ 1, \frac{1}{2}, \frac{1}{3}, \ldots \right\} $.
    \item Η ακολουθία $ a_{n}= \{(-1)^{n}\}_{n=1}^{+ \infty}, $. Έχει Σ.Τ.\ 
      το σύνολο $ \{ -1,1 \} $.
    \item Η ακολουθία $ a_{n} = c, \; \forall n \in \mathbb{N}, c \in \mathbb{R} $.
      Έχει Σ.Τ.\ το σύνολο $ \{ c \} $ και ονομάζεται 
      \textcolor{Col1}{σταθερή ακολουθία}.
    \item Η ακολουθία $ a_{n}=2n, \; \forall n \in \mathbb{N} $. Έχει Σ.Τ.\ το 
      σύνολο $ \{ 2,4,6, \ldots, 2n, \ldots \} $, δηλαδή όλους τους
      \textcolor{Col1}{άρτιους} φυσικούς αριθμούς.
    \item Η ακολουθία $ a_{n}= 2n-1, \; \forall n \in \mathbb{N} $. Έχει Σ.Τ.\ το 
      σύνολο $ \{ 1,3,5, \ldots, 2n-1, \ldots \} $, δηλαδή όλους τους περιττούς
      φυσικούς αριθμούς.
    \item \label{ex:anadr} Η ακολουθία $ a_{1}= a_{2} = 1 $ και $ a_{n+2}=a_{n+1}
      +a_{n}, \; \forall n \in \mathbb{N}$. Έχει Σ.Τ.\ το σύνολο 
      $ \{ 1,1,2,3,5,8, 13,21,34, \ldots\} $.  Πρόκειται για την 
      \textcolor{Col1}{ακολουθία Fibonacci}. 
  \end{enumerate}
\end{examples}

\begin{rem}
\item {}
  \begin{enumerate}[i)]
    \item Ουσιαστικά οι ακολουθίες είναι \textbf{λίστες} πραγματικών αριθμών.
    \item Η ακολουθία~\ref{ex:anadr}, όπου κάθε επόμενος όρος, ορίζεται με 
      τη βοήθεια του προηγούμενου, λέγεται
      \textcolor{Col1}{αναδρομική ακολουθία}.  Προτάσεις που αφορούν 
      αναδρομικές ακολουθίες, αποδεικνύονται με \textbf{Μαθηματική Επαγωγή}.
  \end{enumerate}
\end{rem}

\begin{mybox1}
  \begin{dfn}
    Δυο ακολουθίες, $(a_{n})_{n \in \mathbb{N}}$  και $ (b_{n})_{n \in \mathbb{N}} $ 
    ονομάζονται \textcolor{Col1}{ίσες}, αν 
    $ a_{n} = b_{n}, \; \forall n \in \mathbb{N} $.
  \end{dfn}
\end{mybox1}

\begin{mybox1}
  \begin{dfn}
    Οι πράξεις μεταξύ ακολουθιών, ορίζονται όπως ακριβώς και για τις συναρτήσεις.
  \end{dfn}
\end{mybox1}


\section{Μονοτονία Ακολουθιών}

\begin{mybox1}
  \begin{dfn}
    Μια ακολουθία $ (a_{n})_{n \in \mathbb{N}} $ ονομάζεται:
    \begin{enumerate}[i)]
      \twocolumnsides{
      \item \textcolor{Col1}{γνησίως αύξουσα} 
        $ \overset{\text{ορ.}}{\Leftrightarrow} a_{n} < a_{n+1}, 
        \quad \forall n \in \mathbb{N}$
      \item \textcolor{Col1}{γνησίως φθίνουσα} 
        $ \overset{\text{ορ.}}{\Leftrightarrow} a_{n} > a_{n+1}, 
        \quad \forall n \in \mathbb{N}$
        }{
      \item \textcolor{Col1}{αύξουσα} 
        $ \overset{\text{ορ.}}{\Leftrightarrow} a_{n} \leq a_{n+1}, 
        \quad \forall n \in \mathbb{N}  $.
      \item \textcolor{Col1}{φθίνουσα} 
        $ \overset{\text{ορ.}}{\Leftrightarrow} a_{n} \geq a_{n+1}, 
        \quad \forall n \in \mathbb{N}  $.
      }
  \end{enumerate}
  \end{dfn}
\end{mybox1}

\begin{rems}
\item {}
  \begin{enumerate}[i)]
    \item $ (a_{n})_{n \in \mathbb{N}} $ γνησίως αύξουσα (γνησίως φθίνουσα) $ 
      \Rightarrow (a_{n})_{n \in \mathbb{N}} $ αύξουσα (φθίνουσα) 
    \item $ (a_{n})_{n \in \mathbb{N}} $ φθίνουσα  $ 
      \Rightarrow (a_{n})_{n \in \mathbb{N}} $ άνω φραγμένη, με 
      α.φ.\ το $ a_{1} $  
    \item $ (a_{n})_{n \in \mathbb{N}} $ αύξουσα  $ 
      \Rightarrow (a_{n})_{n \in \mathbb{N}} $ κάτω φραγμένη, με 
      κ.φ.\ το $ a_{1} $  
  \end{enumerate}
\end{rems}


\begin{rem}\label{dfn:isodmono}
\item {}
  Αν μια ακολουθία $ (a_{n})_{n \in \mathbb{N}} $ \textbf{διατηρεί σταθερό πρόσημο}, 
  τότε έχουμε:

  \twocolumnsidee{
    Αν $ a_{n}>0, \; \forall n \in \mathbb{N} $, τότε:
    \begin{myitemize}
      \item $ \frac{a_{n+1}}{a_{n}} > 1, \; \forall n \in \mathbb{N} \Rightarrow
        (a_{n})_{n \in \mathbb{N}} $ είναι \textcolor{Col1}{γνησίως αύξουσα}  
    \item $ \frac{a_{n+1}}{a_{n}} < 1, \; \forall n \in \mathbb{N} \Rightarrow 
      (a_{n})_{n \in \mathbb{N}} $ είναι \!\textcolor{Col1}{γνησίως φθίνουσα}  
    \item $ \frac{a_{n+1}}{a_{n}} \geq 1, \; \forall n \in \mathbb{N} \Rightarrow 
      (a_{n})_{n \in \mathbb{N}} $ είναι \textcolor{Col1}{αύξουσα} 
    \item $ \frac{a_{n+1}}{a_{n}} \leq 1, \; \forall n \in \mathbb{N} \Rightarrow 
      (a_{n})_{n \in \mathbb{N}} $ είναι \textcolor{Col1}{φθίνουσα}
    \end{myitemize}
  }{
    Αν $ a_{n}<0, \; \forall n \in \mathbb{N} $, τότε:
    \begin{myitemize}
      \item $ \frac{a_{n+1}}{a_{n}} > 1, \; \forall n \in \mathbb{N} \Rightarrow
        (a_{n})_{n \in \mathbb{N}} $ είναι \!\textcolor{Col1}{γνησίως φθίνουσα}  
      \item $ \frac{a_{n+1}}{a_{n}} < 1, \; \forall n \in \mathbb{N} \Rightarrow
        (a_{n})_{n \in \mathbb{N}} $ είναι \textcolor{Col1}{γνησίως αύξουσα}  
    \item $ \frac{a_{n+1}}{a_{n}} \geq 1, \; \forall n \in \mathbb{N} \Rightarrow 
      (a_{n})_{n \in \mathbb{N}} $ είναι \textcolor{Col1}{φθίνουσα}  
    \item $ \frac{a_{n+1}}{a_{n}} \leq 1, \; \forall n \in \mathbb{N} \Rightarrow 
      (a_{n})_{n \in \mathbb{N}} $ είναι \textcolor{Col1}{αύξουσα}  
    \end{myitemize}
}
\end{rem}

\section{Μεθοδολογία εύρεσης μονοτονίας μιας ακολουθίας}
\begin{myitemize}
  \item Σχηματίζουμε τη διαφορά $ a_{n+1} - a_n $ και ελέγχουμε το πρόσημό της. 
    Αν $ a_{n+1}-a_{n}>0, \; (<0), \; \forall n \in \mathbb{N} $ τότε 
    $ (a_{n})_{n \in \mathbb{N}}$ γνησίως αύξουσα (γνησίως φθίνουσα). 
    Αν για τουλάχιστον ένα $ n \in \mathbb{N} $, στις παραπάνω ανισότητες, 
    έχω ισότητα, τότε  $ (a_{n})_{n \in \mathbb{N}} $ είναι αύξουσα (φθίνουσα).
  \item Αν οι όροι της ακολουϑίας διατηρούν πρόσημο, $ \; \forall n \in \mathbb{N} $ 
    τότε συγκρίνουμε το πηλίκο δυο διαδοχικών όρων της ακολουθίας με τη μονάδα, 
    και βγάζουμε τα συμπεράσματά μας σύμφωνα με την παρατήρηση~\ref{dfn:isodmono}
  \item Αν η ακολουθία δίνεται με μη-αναδρομικό τύπο, και είναι (αρκετά) σύνθετη, 
    τότε μετατρέπω την ακολουθία στην αντίστοιχη συνάρτηση και μελετάμε τη 
    μονοτονία της αντίστοιχης συνάρτησης, συνήθως με τη βοήθεια της παραγώγου.
  \item Αν η $ (a_{n})_{n \in \mathbb{N}} $ δίνεται με αναδρομικό τύπο 
    $ (a_{n+1}= f(a_{n}), \; \forall n \in \mathbb{N}) $ τότε συνήθως η απόδειξή 
    της γίνεται με Μαθηματική Επαγωγή.
\end{myitemize}

\begin{examples}
\item {}
  \begin{enumerate}[i)]
    \item Η $ a_{n} = 2n-1, \; \forall n \in \mathbb{N} $ είναι γνησίως 
      αύξουσα. Πράγματι,

      \twocolumnsides{%
        \begin{description}
          \item[Α᾽ τρόπος]
            \begin{align*}
              n+1 &\geq n, \; \forall n \in \mathbb{N} \\
              2(n+1) &\geq 2n, \; \forall n \in \mathbb{N} \\
              2(n+1) -1 &\geq 2n-1, \; \forall n \in 
              \mathbb{N} \\
              a_{n+1} &\geq a_{n}, \; \forall n \in \mathbb{N}
            \end{align*}
        \end{description}
        }{%
        \begin{description}
          \item[Β᾽ τρόπος] 
            \begin{align*}
              a_{n+1}-a_{n} &= 2(n+1)-1 - (2n-1) \\
                            &= 2 >0, \; \forall n \in 
                            \mathbb{N} 
            \end{align*}
        \end{description}
      }

    \item Η $ a_{n} = \frac{(n-1)!}{n^{n}}, \; \forall n \in 
      \mathbb{N} $ 
      είναι γνησίως φθίνουσα. Πράγματι, επειδή όλοι οι όροι της ακολουθίας 
      είναι \textbf{θετικοί}, επομένως διατηρεί πρόσημο, έχουμε:
      \[
        \frac{a_{n+1}}{a_n} =
        \frac{\frac{(n+1-1)!}{(n+1)^{n+1}}}{\frac{(n-1)!}{n^{n}}} 
        = \frac{n^{n}}{(n+1)^{n+1}}\cdot \frac{n!}{(n-1)!} 
        = \frac{n^{n}}{(n+1)^{n+1}}\cdot \frac{1\cdot 2\cdots (n-1)\cdot n}{1 \cdot 2
        \cdot (n-1)}  
        = \left(\frac{n}{n+1} \right)^{n+1} < 1, \; \forall n \in \mathbb{N} 
      \] 

    \item Η $ a_{n}= \frac{4^{n}}{n^{2}}, \; \forall n \in \mathbb{N} $ 
      είναι αύξουσα. Πράγματι, επειδή όλοι οι όροι της ακολουθίας 
      είναι \textbf{θετικοί}, επομένως διατηρεί πρόσημο, έχουμε:
      \[
        \frac{a_{n+1}}{a_{n}} 
        = \frac{\frac{4^{n+1}}{(n+1)^{2}}}{\frac{4^{n}}{n^{2}}} 
        = \frac{4^{n+1}}{4^{n}}\cdot \frac{n^{2}}{(n+1)^{2}}  
        = \frac{4\cdot \cancel{4^{n}}}{\cancel{4^{n}}} \cdot \frac{n^{2}}{(n+1)^{2}}
        = \left( \frac{2n}{n+1} \right)^{2} \geq 1, 
        \; \forall n \in \mathbb{N} 
      \]
      Πράγματι, η ακολουθία δεν είναι γνησίως αύξουσα, γιατί $ a_{1}= a_{2}=4$.

    \item Η $ a_{n+1}=2 - \frac{1}{a_{n}}, \; \forall n \in \mathbb{N}
      $ με $ a_{1} = 2 $ είναι γνησίως φθίνουσα. Πράγματι, επειδή η ακολουθία είναι 
      \textbf{αναδρομική}, με μαθηματική επαγωγή, έχουμε:
      \begin{myitemize}[labelindent=1em]
        \item Για $ n=1 $, έχω: $ a_{1}= 2 >
          \frac{3}{2} = 2 - \frac{1}{2} = a_{2}$, ισχύει.
        \item Έστω ότι ισχύει για $n$, δηλ.
          $\inlineequation[eq:epag]{a_{n+1}<a_{n}}$.
        \item Θ.δ.ο.\ ισχύει για $ n+1 $. Πράγματι, από τη σχέση~\eqref{eq:epag} 
          έχουμε:
          \begin{align*}
            a_{n+1} < a_{n} \Leftrightarrow \frac{1}{a_{n+1}} > \frac{1}{a_{n}}
            \Leftrightarrow - \frac{1}{a_{n+1}} < - \frac{1}{a_{n}} \Leftrightarrow 2-
            \frac{1}{a_{n+1}} < 2 - \frac{1}{a_{n}} \Leftrightarrow a_{(n+1)+1} 
            < a_{n+1}
          \end{align*} 
      \end{myitemize}

    \item Να δείξετε ότι ακολουθία $ a_{n} = (-1)^{n} \frac{1}{n^{2}}, 
      \; \forall n \in \mathbb{N} $ 
      δεν είναι μονότονη.

      \begin{proof}
        Θα δείξουμε ότι η ακολουθία δεν διατηρεί πρόσημο. Πράγματι, 
        $ \forall n \in \mathbb{N} $
        \begin{align*}
          a_{n+1}- a_{n} = \frac{(-1)^{n+1}}{(n+1)^{2}} - 
          \frac{(-1)^{n}} {n^{2}} = \frac{(-1)^{n+1}}{(n+1)^{2}} + 
                        \frac{(-1)^{n+1}}{n^{2}} 
                        = (-1)^{n+1}\Biggl[\underbrace{\frac{1}{(n+1)^{2}} + 
                            \frac{1}{n^{2}}}_{b_{n} > 0, \; \forall n \in 
                        \mathbb{N}}\Biggr] 
                        = \begin{cases}
                          b_{n}, & n \; \text{περιττός} \\
                          -b_{n}, & n \; \text{άρτιος} 
                        \end{cases}
        \end{align*} 
      \end{proof}
  \end{enumerate}
\end{examples}

% \begin{mybox3}
% \begin{prop}
% Η ακολουθία $ \left(1+ \frac{1}{n}\right)^{n} $ είναι γνησίως αύξουσα.
% \end{prop}
% \end{mybox3}
% \begin{proof}
% \item {}
%   Έστω $ n_{0} \in \mathbb{N} $. Τότε 
%   \begin{align*}
%     \left(1+ \frac{1}{n_{0}+1} \right)^{n_{0}+1} > 
%     \left(1+ \frac{1}{n_{0}} \right)^{n_{0}} 
%         &\Leftrightarrow \left(1+ \frac{1}{n_{0}+1} \right)^{n_{0}} 
%         \cdot \left(1 + \frac{1}{n_{0} +1} \right) > \left(1+ \frac{1}{n_{0}+1} 
%         \right) \\
%         & \Leftrightarrow \frac{(n_{0}+2)^{n_{0}}}{(n_{0}+1)^{n_{0}}} \cdot 
%         \frac{n_{0}+2}{n_{0}+1} > \frac{(n_{0}+1)^{n_{0}}}{n_{0}^{n_{0}}} \\
%         & \Leftrightarrow \frac{n_{0}^{n_{0}}(n_{0}+2)^{n_{0}}}{(n_{0}+1)^{2n0}} > 
%         \frac{n_{0}+1}{n_{0}+2} \\
%         & \Leftrightarrow \left(\frac{n_{0}(n_{0}+2)}{(n_{0}+1)^{2}}\right)^{n_{0}} > 
%         \frac{n_{0}+2-2+1}{n_{0}+2} \\
%         & \Leftrightarrow \left(\frac{n_{0}^{2}+2 n_{0}+1-1}{(n_{0}+1)^{2}}\right)
%         ^{n_{0}} > 1-\frac{1}{n_{0}+2} \\
%         & \Leftrightarrow \left(\frac{(n_{0}+1)^{2}-1}{(n_{0}+1)^{2}} \right)^{n_{0}}
%         > 1 - \frac{1}{n_{0}+2} \\
%         & \Leftrightarrow \left(1 - \frac{1}{(n_{0}+1)^{2}} \right)^{n_{0} } > 1 - 
%         \frac{1}{n_{0}+2} \\
%   \end{align*} 
%   Άρα αρκεί να δείξουμε αυτή την ανισότητα. Πράγματι,
%   για $ a = - \frac{1}{(n_{0}+1)^{2}} > -1 $ από ανισότητα Bernoulli: 
%   \begin{align*}
%     \left(1- \frac{1}{(n_{0}+1)^{2}}\right)^{n_{0}} > 1 - n_{0}\cdot 
%     \frac{1}{(n_{0}+1)^{2}} > 1 - \frac{1}{n_{0}+2} 
%   \end{align*}
%   όπου χρησιμοποιήσαμε το γεγονός ότι $ \frac{n_{0}}{(n_{0}+1)^{2}} < 
%   \frac{1}{n_{0}+2}  $, το οποίο ισχύει, γιατί 
%   \[
%     \frac{n_{0}}{(n_{0}+1)^{2}} < \frac{1}{n_{0}+2} 
%     \Leftrightarrow n_{0}(n_{0}+2) < (n_{0}+1)^{2} 
%     \Leftrightarrow n_{0}^{2}+2 n_{0} < n_{0}^{2} + 2 n_{0}+1 
%     \Leftrightarrow 0 < 1
%   \]
% \end{proof}

\section{Φραγμένες Ακολουθίες}

\begin{mybox1}
  \begin{dfn}
    Μια ακολουθία $ (a_{n})_{n \in \mathbb{N}} $ ονομάζεται:
    \begin{enumerate}[i)]
      \item \textcolor{Col1}{άνω φραγμένη} 
        $ \overset{\text{ορ.}}{\Leftrightarrow} \exists M \in 
        \mathbb{R} \; : \; a_{n} \leq M, \; \forall n \in \mathbb{N}$.
      \item \textcolor{Col1}{κάτω φραγμένη} 
        $ \overset{\text{ορ.}}{\Leftrightarrow} \exists m \in 
        \mathbb{R} \; : \; m \leq a_{n}, \; \forall n \in \mathbb{N}  $
      \item \textcolor{Col1}{φραγμένη} 
        $ \overset{\text{ορ.}}{\Leftrightarrow} $ είναι άνω και κάτω φραγμένη.
    \end{enumerate}
  \end{dfn}
\end{mybox1}

\begin{mybox3}
  \begin{prop}\label{prop:apolfragm}
    {$ (a_{n})_{n \in \mathbb{N}} $ φραγμένη $ \Leftrightarrow \exists M>0, 
      \; M \in \mathbb{R} \; : \; \abs{a_{n}} \leq M, \; \forall n \in \mathbb{N} $ 
    (\textbf{απολύτως φραγμένη}).}
  \end{prop}
\end{mybox3}

\begin{proof}
\item {}
  \begin{description}
    \item [($ \Rightarrow $)] Έστω $ (a_{n})_{n \in \mathbb{N}} $ φραγμένη. Τότε η 
      $ (a_{n})_{n \in \mathbb{N}} $ είναι άνω και κάτω φραγμένη, δηλαδή υπάρχουν 
      $ m,M \in \mathbb{R} $ ώστε $ m \leq a_{n} \leq M, \; \forall n \in \mathbb{N} $.
      Θεωρούμε $ \mu = \max \{ \abs{m} , \abs{M} \} $. Τότε, η παραπάνω διπλή 
      ανισότητα, λαμβάνοντας υπόψιν και τις γνωστές ιδιότητες της απόλυτης τιμής, 
      γίνεται
      \[
        - \mu \leq - \abs{m} \leq m \leq a_{n} \leq M \leq \abs{M} \leq \mu , 
        \quad \forall n \in \mathbb{N}
      \]
      Επομένως $ \exists \mu > 0 $ ώστε 
      $ \abs{a_{n}} \leq \mu, \; \forall n \in \mathbb{N} $.
    \item [($ \Leftarrow$)] Προφανώς, αν $( a_{n})_{n \in \mathbb{N}} $ είναι 
      απολύτως φραγμένη, τότε $ \exists M>0 $ ώστε $ -M \leq a_{n} \leq M, \; \forall n
      \in \mathbb{N} $, και άρα η ακολουθία είναι φραγμένη. 
  \end{description}
\end{proof}

\begin{rem}
  Πολλές φορές στη βιβλιογραφία, η πρόταση~\ref{prop:apolfragm}, 
  δίνεται και ως ορισμός της φραγμένης ακολουθίας.
\end{rem}

\begin{examples}
\item {}  
  \begin{enumerate}[i)]
    \item Η ακολουθία $ a_{n}= \frac{1}{n}, \; \forall n \in \mathbb{N} $, είναι 
      φραγμένη.
      Πράγματι, προφανώς, ισχύει ότι $ 0 \leq \frac{1}{n} \leq 1, \; \forall n \in 
      \mathbb{N} $. 

      Επίσης, $ \abs{\frac{1}{n}} = \frac{1}{n} \leq 1, \; \forall n \in \mathbb{N} $, 
      άρα είναι και απολύτως φραγμένη.
    \item Η ακολουθία $ a_{n}=(-1)^{n} \frac{1}{n}, \; \forall n \in \mathbb{N} $ 
      είναι απολύτως φραγμένη. Πράγματι,
      \[
        \abs{a_{n}} = \abs{(-1)^{n} \frac{1}{n}} = \abs{(-1)^{n}} 
        \cdot \abs{\frac{1}{n}} = \abs{-1}^{n} \cdot \frac{1}{n}
        = 1 \cdot \frac{1}{n} = \frac{1}{n} \leq 1, \; \forall n 
        \in \mathbb{N}
      \] 

    \item Η ακολουθία $ a_{n}= \frac{(n-1)!}{n^{n}}, \; \forall n \in \mathbb{N} $
      είναι φραγμένη. Πράγματι, 

      $ a_{n} > 0, \; \forall n \in 
      \mathbb{N}$, άρα 0 κ.φ.\ της $( a_{n})_{n \in 
      \mathbb{N}} $. 
      Επίσης 
      \[
        a_{n}= \frac{(n-1)!}{n^{n}} = \frac{1 \cdot 2 
          \cdots (n-1)}{n^{n}} < \frac{\smash{\overbrace{n 
              \cdot n \cdots n} ^{n-1 \; 
        \text{φορές}}}}{n^{n}} = \frac{n^{n-1}}{n^{n}} =
        \frac{1}{n} \leq 1, \; \forall n \in \mathbb{N},
      \]
      άρα το 1 είναι α.φ.\ της $(a_{n})_{n \in \mathbb{N}}$. 

    \item Η ακολουθία $ a_{n}= 1 + \left(- \frac{1}{2} \right) + \left(- 
        \frac{1}{2}\right)^{2} + \cdots + \left(-\frac{1}{2} 
      \right) ^{n}, 
      \; \forall n \in \mathbb{N} $ είναι απολύτως φραγμένη. Πράγματι,

      Πρόκειται για το άθροισμα των $ n $ πρώτων όρων \textbf{γεωμετρικής προόδου} με 
      λόγο $ -\frac{1}{2} $. Έτσι
      \[ a_{n} = 1 + \left(- \frac{1}{2}\right) + \left(- \frac{1}{2} 
        \right)^{2} + \cdots + \left(- \frac{1}{2} \right)^{n} = 
        \frac{(- \frac{1}{2} )^{n+1}-1}{- \frac{1}{2}-1} = 
      \frac{2}{3} \left[1 - \left(- \frac{1}{2} \right)^{n}\right] \]
      Επομένως
      \[
        \abs{a_{n}} = \abs{\frac{2}{3} \left[1-\left(- \frac{1}{2} \right)^{n}
            \right]} = \frac{2}{3} \abs{1 - \left(- 
        \frac{1}{2}\right)^{n}} \leq 
        \frac{2}{3} \left(1 + \abs{-\frac{1}{2} }^{n} \right) = 
        \frac{2}{3} \left(1+ \frac{1}{2^{n}}\right) < \frac{2}{3}
        (1+1) = \frac{4}{3} 
      \] 

    \item Η ακολουθία $ a_{n}= 2n+5, \; \forall n \in \mathbb{N} $ είναι κάτω 
      φραγμένη.
      Πράγματι, $ 7 \leq 2n+5, \; \forall n \in \mathbb{N} $, άρα το 
      7 είναι κ.φ.\ της $ (a_{n} )_{n \in \mathbb{N}} $.

      Προσοχή, η ακολουθία $ a_{n}= 2n+5, \; \forall n \in \mathbb{N} $ δεν είναι 
      άνω φραγμένη, γιατί αν υποθέσουμε ότι είναι, τότε $ \exists M>0 $ ώστε 
      $ a_{n} \leq M, \; \forall n \in \mathbb{N} 
      \Leftrightarrow 2n+5 \leq M, \; \forall n \in \mathbb{N} 
      \Leftrightarrow 2n \leq M-5, \; \forall n \in \mathbb{N}
      \Leftrightarrow n \leq \frac{M-5}{2}, \; \forall n \in \mathbb{N} $, άτοπο, 
      γιατί το $ \mathbb{N} $ δεν είναι άνω φραγμένο.

    \item Η ακολουθία $ a_{1}=2, \; a_{n+1}=2 - \frac{1}{a_{n}}, \forall n \in 
      \mathbb{N}$
      είναι φραγμένη. Προφανώς, $ a_{n}\leq 2, \; \forall n \in \mathbb{N} $.
      Θα δείξουμε ότι $ a_{n}>1, \; \forall n \in \mathbb{N} $. Επειδή η ακολουθία 
      είναι \textbf{αναδρομική}, με επαγωγή, έχουμε:
      \begin{myitemize}[labelindent=1em]
        \item Για $ n=1 $, $ a_{1}=2>1 $, ισχύει. 
        \item Έστω ότι ισχύει για $n$, δηλ. $\inlineequation[eq:
          anadepag1]{a_{n}>1}$.
        \item Θ.δ.ο.\ ισχύει και για $ n+1 $. Πράγματι, από τη 
          σχέση~\eqref{eq: anadepag1}, έχουμε
          \[
            a_{n}>1 \Rightarrow \frac{1}{a_{n}} 
            < 1 \Rightarrow - \frac{1}{a_{n}} > 
            -1 \Rightarrow 2 - \frac{1}{a_{n}} 
            > 2-1 \Rightarrow a_{n+1} > 1.
          \] 
      \end{myitemize}

    \item Να δείξετε ότι η ακολουθία $ a_{n} = 
      \frac{n^{2}+1}{3n+ \sin^{3}{n}} $ δεν είναι άνω φραγμένη. 

      \begin{proof}
      \item {}
        Έστω ότι η $ a_{n} = \frac{n^{2}+1}{3n+ \sin^{3}{n}} $ είναι άνω 
        φραγμένη. Τότε επειδή είναι και κάτω φραγμένη, από το 0, έχουμε ότι
        $ \exists M>0 \; : \; \abs{a_{n}} \leq M, \; \forall n \in 
        \mathbb{N} $, δηλαδή
        \begin{align*}
          \abs{\frac{n^{2}+1}{3n + \sin^{3}{n}}} \leq M \Leftrightarrow 
          \frac{n^{2}+1}{\abs{3n + \sin^{3}{n}}} \leq M \Leftrightarrow 
          n^{2}+1
               &\leq M \cdot \abs{3n + \sin^{3}{n}} \\ 
               &\leq 3nM + M \cdot \abs{\sin{n}} ^{3} \\
               &\leq 3n M +M, \; \forall n \in \mathbb{N}
        \end{align*} 
        Δηλαδή, 
        \[
          n^{2}-3nM \leq M-1 \Rightarrow n^{2}-3nM < M, \; \forall n \in 
          \mathbb{N}
        \] 
        και συμπληρώνοντας το τετράγωνο έχουμε
        \[
          \left(n - \frac{3}{2} M\right)^{2} < M + \frac{9}{4} M^{2}
          \Rightarrow \abs{n - \frac{3}{2} M} < 
          \sqrt{M + \frac{9}{4} M^{2}}, \; \forall n \in \mathbb{N}
        \]
        οπότε
        \[
          - \sqrt{M + \frac{9}{4} M^{2}}< \underbrace{n- \frac{3}{2} M 
          < \sqrt{M + \frac{9}{4} M^{2}}}, \; \forall n \in \mathbb{N} 
          \Rightarrow n < \frac{3}{2} M + \sqrt{M + \frac{9}{4} M^{2}}, 
          \; \forall n \in \mathbb{N} 
        \] 
        άτοπο, γιατί $ \mathbb{N} $ όχι άνω φραγμένο.
      \end{proof}
  \end{enumerate}
\end{examples}


\section{Σύγκλιση Ακολουθίας}

\begin{mybox1}
  \begin{dfn}
    \textcolor{Col1}{Περιοχή} ενός πραγματικού αριθμού $ x_{0} $ 
    ονομάζεται κάθε ανοιχτό διάστημα $(a,b)$ που περιέχει το $ x_{0} $. 
  \end{dfn}
\end{mybox1}

\begin{rem}
\item {}
  \begin{enumerate}[i)]
    \item 
      Αν $ \varepsilon > 0 $, τότε περιοχές του $ x_{0} $ της μορφής 
      $ (x_{0}- \varepsilon , x_{0} + \varepsilon) $ έχουν 
      \textcolor{Col1}{ακτίνα} $ \varepsilon $
      και \textcolor{Col1}{κέντρο} το $ x_{0} $. 

    \item $ x \in (x_{0}- \varepsilon, x_{0} + \varepsilon) 
      \Leftrightarrow x_{0}- \varepsilon < x < x_{0}+ \varepsilon 
      \Leftrightarrow - \varepsilon < x - x_{0} < \varepsilon 
      \Leftrightarrow \abs{x- x_{0}} < \varepsilon  $ 
  \end{enumerate}
\end{rem}

\begin{mybox1}
  \begin{dfn}[Ορίου Ακολουθίας]
    Μια ακολουθία $ (a_{n})_{n \in \mathbb{N}} $ \textcolor{Col1}
    {συγκλίνει} στον πραγματικό 
    αριθμό $ a \in \mathbb{R} $ (έχει όριο το $ a \in \mathbb{R} $ ή 
    τείνει στο $ a \in \mathbb{R} $), και συμβολίζουμε 
    $ \lim\limits_{n\to \infty} a_{n}=a $ (ή $ a_{n} \xrightarrow{n \to 
    \infty} a $) αν 
    \[
      \forall \varepsilon >0, \; \exists n_{0} \in \mathbb{N} \; : 
      \; \forall n \in \mathbb{N} \; \text{με} \; n \geq n_{0} 
      \Rightarrow \abs{a_{n}-a} < \varepsilon
    \] 
  \end{dfn}
\end{mybox1}

\begin{rem}
  Γενικά το $ n_{0} $ εξαρτάται από το $ \varepsilon $ και ισχύει ότι
  $ n_{0} = n_{0}(\varepsilon) $.
\end{rem}

\begin{rem}
  Ισοδύναμα, ο ορισμός του ορίου, μας λέει, ότι 
  \[
    \forall \varepsilon >0, \; \exists n_{0} \in \mathbb{N} \; : \; \forall n \in
    \mathbb{N} \; \text{με} \; n \geq n_{0} \quad a_{n} \in (a - \varepsilon , a +
    \varepsilon ) 
  \] 
\end{rem}

\begin{mybox1}
  \begin{dfn}
    Μια ακολουθία λέμε ότι \textcolor{Col1}{αποκλίνει} (ή είναι αποκλίνουσα), αν 
    \begin{myitemize}
      \item δεν υπάρχει το όριό της, για παράδειγμα αν η ακολουθία ταλαντεύεται
      \item απειρίζεται, θετικά ή αρνητικά.
    \end{myitemize}
  \end{dfn}
\end{mybox1}

\begin{mybox1}
  \begin{dfn}
    Η ακολουθία $ (a_{n})_{n \in \mathbb{N}}$ λέγεται
    \textcolor{Col1}{μηδενική ακολουθία}
    αν $ \lim\limits_{n\to \infty} = 0 $
  \end{dfn}
\end{mybox1}

\begin{examples}
\item {}
  \begin{enumerate}[i)]
    \item $ \lim\limits_{n \to \infty} \frac{1}{n} = 0 $.
      \begin{proof}
      \item {}
        \begin{description}
          \item[Δοκιμή:] $ \abs{\frac{1}{n} -0} < \varepsilon
            \Leftrightarrow \abs{\frac{1}{n}} < \varepsilon 
            \Leftrightarrow \frac{1}{n} < \varepsilon 
            \Leftrightarrow n > \frac{1}{\varepsilon}$
        \end{description}
            Έστω $ \varepsilon >0 $. Τότε $ \exists n_{0} \in
            \mathbb{N} $ με $\inlineequation[eq:1n1]{n_{0} > \frac{1}{\varepsilon}}$ 
            (Αρχ. Ιδιοτ.) τέτοιο ώστε $\inlineequation[eq:1n2]{\forall n \geq n_{0}}$
            \[
              \abs{\frac{1}{n} -0} = \abs{\frac{1}{n}} =
              \frac{1}{n} \overset{\eqref{eq:1n2}}{\leq}
              \frac{1}{n_{0}} \overset{\eqref{eq:1n1}}{<} 
              \frac{1}{\frac{1}{\varepsilon}} = \varepsilon 
            \]
      \end{proof}

    \item $ \lim\limits_{n \to \infty} \frac{1}{n^{4}} = 0 $. 
      \begin{proof}
      \item {}
        \begin{description}
          \item[Δοκιμή:]$ \abs{\frac{1}{n^{4}} - 0} < \varepsilon 
            \Leftrightarrow \abs{\frac{1}{n^{4}}} < \varepsilon 
            \Leftrightarrow \frac{1}{n^{4}} < \varepsilon
            \Leftrightarrow n^{4} > \frac{1}{\varepsilon}
            \Leftrightarrow n > \sqrt[4]{\frac{1}{\varepsilon}}$
        \end{description}
        Έστω $ \varepsilon >0 $. Τότε $ \exists n_{0}  \in 
        \mathbb{N}$ με $\inlineequation[eq:limexn41]{n_{0} >
        \sqrt[4]{\frac{1}{\varepsilon}}} $ (Αρχ. Ιδιοτ.)
        τέτοιο ώστε $\inlineequation[eq:limexn42]{\forall n 
        \geq n_{0}}$ 
        \[
          \abs{\frac{1}{n^{4}} - 0 } = \abs{\frac{1}{n^{4}}} 
          = \frac{1}{n^{4}} \overset{\eqref{eq:limexn42}}\leq 
          \frac{1}{n_{0}^{4}} \overset{\eqref{eq:limexn41}}{<}
          \frac{1}{\left(\sqrt[4]{\frac{1}{\varepsilon}
          }\right)^{4}} = \frac{1}{\frac{1}{\varepsilon}} =  
          \varepsilon
        \] 
      \end{proof}

    \item $ \lim\limits_{n \to \infty} \frac{1}{\sqrt{n}} = 0$.
      \begin{proof}
      \item {}
        \begin{description}
          \item[Δοκιμή:] $ \abs{\frac{1}{\sqrt{n}}-0 } 
            < \varepsilon 
            \Leftrightarrow \abs{\frac{1}{\sqrt{n}} } < 
            \varepsilon 
            \Leftrightarrow \frac{1}{\sqrt{n}} < 
            \varepsilon \Leftrightarrow \sqrt{n} >
            \frac{1}{\varepsilon} \Leftrightarrow n >
            \left(\frac{1}{\varepsilon}\right)^{2}
            $
        \end{description}
        Έστω $ \varepsilon > 0 $. Τότε $ \exists n_{0} \in 
        \mathbb{N} $
        με $\inlineequation[eq:limexsqrt1]{n_{0} > \left(\frac{1}
        {\varepsilon}\right)^{2}} $ (Αρχ. Ιδιοτ.) τέτοιο ώστε 
        $\inlineequation[eq:limexsqrt2]{\forall n \geq n_{0}}$
        \[
          \abs{\frac{1}{\sqrt{n}} -0} = \abs{\frac{1}{\sqrt{n}}} =
          \frac{1}{\sqrt{n}} \overset{\eqref{eq:limexsqrt2}}{\leq}
          \frac{1}{\sqrt{n_{0}}} \overset{\eqref{eq:limexsqrt1}}
          {<} \frac{1}{\sqrt{\left(\frac{1}{\varepsilon}\right)
          ^{2}}} = \frac{1}{\frac{1}{\varepsilon}} = \varepsilon
        \] 
      \end{proof}

    \item $ \lim_{n \to \infty} \frac{n}{n+1} = 1$

      \begin{proof}
      \item {}
        \begin{description}
          \item[Δοκιμή:] $ \abs{\frac{n}{n+1} - 1} < 
            \varepsilon \Leftrightarrow \abs{\frac{n-(n+1)}{n+1}}
            < \varepsilon \Leftrightarrow \abs{\frac{-1}{n+1}} < \varepsilon
            \Leftrightarrow \frac{\abs{-1}}{n+1} < \varepsilon 
            \Leftrightarrow \frac{1}{n+1} < \varepsilon 
            \Leftrightarrow n+1 > \frac{1}{\varepsilon} 
            \Leftrightarrow n > \frac{1}{\varepsilon} - 1 $
        \end{description}

        Έστω $ \varepsilon >0 $. Τότε $ \exists n_{0} \in \mathbb{N}
        $ με $\inlineequation[eq:limexfrac1]{n_{0} >
        \frac{1}{\varepsilon}-1} $ (Αρχ. Ιδιοτ.) τέτοιο ώστε
        $\inlineequation[eq:limexfrac2]{\forall n \geq n_{0}}$
        \[
          \abs{\frac{n}{n+1} - 1} =  \abs{\frac{n-(n+1)}{n+1}} =
          \frac{\abs{-1}}{n+1} = \frac{1}{n+1} \overset{\eqref{eq:limexfrac2}}{\leq}
          \frac{1}{n_{0}+1} \overset{\eqref{eq:limexfrac1}}{\leq}  
          \frac{1}{\frac{1}{\varepsilon} -1+1} = \frac{1}{\frac{1}{\varepsilon}
          } = \varepsilon 
        \] 
      \end{proof}

    \item $ \lim_{n \to \infty} \frac{\sin{n}}{n} = 0 $

      \begin{proof}
      \item {}
        \begin{description}
          \item[Δοκιμή:] $ \abs{\frac{\sin{n}}{n} - 0} < 
            \varepsilon \Leftrightarrow \abs{\frac{\sin{n}}{n}}
            < \varepsilon \Leftrightarrow 
            \inlineequation[eq:limexsin1]{\frac{\abs{\sin{n}}}
            {n} < \varepsilon} $

            Όμως
            $\inlineequation[eq:limexsin2]{\frac{\abs{\sin{n}}}
            {n} \leq \frac{1}{n}, \; \forall n \in \mathbb{N}} $

            Οπότε από τις σχέσεις \eqref{eq:limexsin1} και 
            \eqref{eq:limexsin2} αρκεί $ \frac{1}{n} < 
            \varepsilon \Leftrightarrow n > \frac{1}{
            \varepsilon} $
        \end{description}

        Έστω $ \varepsilon >0 $. Τότε $ \exists n_{0} \in \mathbb{N}
        $ με $ \inlineequation[eq:limexsin3]{n_{0} >
        \frac{1}{\varepsilon}} $ (Αρχ. Ιδιοτ.) τέτοιο ώστε
        $ \inlineequation[eq:limexsin4]{\forall n \geq n_{0}} $
        \[
          \abs{\frac{\sin{n}}{n} - 0} =  \abs{\frac{\sin{n}}{n}} =
          \frac{\abs{\sin{n}}}{n} \leq \frac{1}{n}
          \overset{\eqref{eq:limexsin4}}{\leq}  \frac{1}{n_{0}}
          \overset{\eqref{eq:limexsin3}}{<}
          \frac{1}{\frac{1}{\varepsilon}
          } = \varepsilon 
        \] 
      \end{proof}

    \item $ \lim_{n \to \infty} \frac{2n^{2}}{n^{2}+1} = 2 $.

      \begin{proof}
      \item {}
        \begin{description}
          \item[Δοκιμή:] $ \abs{\frac{2n^{2}}{n^{2}+1} - 2} < \varepsilon 
            \Leftrightarrow \abs{\frac{2n^{2}-2(n^{2}+1)}{n^{2}+1}} < \varepsilon 
            \Leftrightarrow \abs{\frac{-2}{n^{2}+1}} < \varepsilon 
            \Leftrightarrow \frac{\abs{-2}}{n^{2}+1} < \varepsilon 
            \Leftrightarrow \inlineequation[eq:limexsq1]{\frac{2}{n^{2}+1} 
            < \varepsilon} $

            Όμως
            $ \inlineequation[eq:limexsq2]{\frac{2}{n^{2}+1} < \frac{2}{n^{2}} \leq 
            \frac{2}{n}, \; \forall n \in \mathbb{N}} $

            Οπότε από τις σχέσεις \eqref{eq:limexsq1} και 
            \eqref{eq:limexsq2} αρκεί $ \frac{2}{n} < \varepsilon 
            \Leftrightarrow n > \frac{2}{ \varepsilon} $
        \end{description}

        Έστω $ \varepsilon >0 $. Τότε $ \exists n_{0} \in \mathbb{N}
        $ με $\inlineequation[eq:limexsq3]{n_{0} > \frac{2}{\varepsilon}}$ 
        (Αρχ. Ιδιοτ.) τέτοιο ώστε
        $\inlineequation[eq:limexsq4]{\forall n \geq n_{0}}$
        \[
          \abs{\frac{2n^{2}}{n^{2}+1} - 2} = \abs{\frac{2n^{2}-2(n^{2}+1)}{n^{2}+1}} =
          \abs{\frac{-2}{n^{2}+1}} = \frac{\abs{-2}}{n^{2}+1} =
          \frac{2}{n^{2}+1} < \frac{2}{n^{2}} \leq \frac{2}{n} 
          \overset{\eqref{eq:limexsq4}}{\leq} \frac{2}{n_{0}} 
          \overset{\eqref{eq:limexsq3}}{<} \frac{2}{\frac{2}{\varepsilon}} =
          \varepsilon 
        \] 
      \end{proof}

    \item $ \lim_{n \to \infty} \frac{n+1}{n^{2}} =0 $

      \begin{proof}
      \item {}
        \begin{description}
          \item[Δοκιμή:] $ \abs{\frac{n+1}{n^{2}} - 0} < \varepsilon 
            \Leftrightarrow \abs{\frac{n+1}{n^{2}}} < \varepsilon 
            \Leftrightarrow \frac{n+1}{n^{2}} < \varepsilon 
            \Leftrightarrow \frac{n}{n^{2}} + \frac{1}{n^{2}} < \varepsilon 
            \Leftrightarrow \inlineequation[eq:limexhard1]{\frac{1}{n} + \frac{1}{n^{2}}
            < \varepsilon} $

            Όμως
            $ \inlineequation[eq:limexhard2]{\frac{1}{n} + \frac{1}{n^{2}} \leq 
            \frac{1}{n} + \frac{1}{n} = \frac{2}{n}, \; \forall n \in \mathbb{N}} $

            Οπότε από τις σχέσεις \eqref{eq:limexhard1} και 
            \eqref{eq:limexhard2} αρκεί $ \frac{2}{n} < \varepsilon 
            \Leftrightarrow n > \frac{2}{ \varepsilon} $
        \end{description}

        Έστω $ \varepsilon >0 $. Τότε $ \exists n_{0} \in \mathbb{N}
        $ με $\inlineequation[eq:limexhard3]{n_{0} > \frac{2}{\varepsilon}}$ 
        (Αρχ. Ιδιοτ.) τέτοιο ώστε
        $\inlineequation[eq:limexhard4]{\forall n \geq n_{0}}$
        \[
          \abs{\frac{n+1}{n^{2}} - 0} = \abs{\frac{n+1}{n^{2}}} =
          \frac{n+1}{n^{2}} = \frac{n}{n^{2}} + \frac{1}{n^{2}} =
          \frac{1}{n} + \frac{1}{n^{2}} \leq \frac{1}{n} + \frac{1}{n} =
          \frac{2}{n} \overset{\eqref{eq:limexhard4}}{\leq} \frac{2}{n_{0}} 
          \overset{\eqref{eq:limexhard3}}{<} \frac{2}{\frac{2}{\varepsilon}} =
          \varepsilon 
        \] 
      \end{proof}
  \end{enumerate}
\end{examples}


\begin{mybox3}
  \begin{prop}
    Η ακολουθία $ \{ (-1)^{n} \}_{n \in \mathbb{N}} $ δεν συγκλίνει.
  \end{prop}
\end{mybox3}

\begin{proof}
\item {}
  Έστω ότι η $ a_{n}= (-1)^{n} $ συγκλίνει και έστω $ \lim_{n \to +
  \infty}(-1)^{n} = a $. 

  Τότε από τον ορισμό του ορίου, έχουμε ότι για 
  \[ 
    \varepsilon = 1, \; \exists n_{0} \in \mathbb{N} \; : \; \forall 
    n \geq n_{0} \quad \abs{(-1)^{n}-a} < 1 \Leftrightarrow -1 < 
    (-1)^{n} -a < 1 \Leftrightarrow a-1 < (-1)^{n} < a+1
  \]

  Όμως για $ n_{1} \geq n_{0} $ με $ n_{1} $ άρτιος, έχουμε:

  \[
    a-1 <  (-1)^{n_{1}} < a+1 \Leftrightarrow a-1 < 
    \underbrace{1 < a+1}_{a>0} 
  \] 

  Όμως για $ n_{2} \geq n_{0} $ με $ n_{2} $ περιττός, έχουμε:

  \[
    a-1 <  (-1)^{n_{2}} < a+1 \Leftrightarrow 
    \underbrace{a-1 < -1}_{a<0} < a+1
  \] 

  Οπότε καταλήγουμε σε άτοπο.
\end{proof}


\begin{mybox2}
  \begin{thm}
    Το όριο μιας ακολουθίας, όταν υπάρχει είναι μοναδικό.
  \end{thm}
\end{mybox2}
\begin{proof}
\item {}
  Έστω ότι μια ακολουθία $ (a_{n})_{n \in \mathbb{N}} $ συγκλίνει και έστω ότι 
  $ \exists a,b \in \mathbb{R} $ ώστε $ \lim_{n \to \infty} a_{n} = a $  και $ \lim_{n \to
  \infty} a_{n} = b $. Θα δείξουμε ότι $ a=b $. Έστω $ \varepsilon >0 $, τότε

  \[ 
    \lim_{n \to \infty} a_{n} = a \Leftrightarrow \forall 
    \varepsilon > 0, \; \exists n_{0} \in \mathbb{N} \; : \: 
    \forall n \geq n_{0} \quad \abs{a_{n}- a} \leq \varepsilon 
  \]

  Άρα και για $ \varepsilon = \frac{\varepsilon}{2}, \; \exists n_{1} \in 
  \mathbb{N} \; : \; \forall n \geq n_{1} \quad \abs{a_{n_{1}} - a} 
  < \frac{\varepsilon}{2}  $

  \[ 
    \lim_{n \to \infty} a_{n} = b \Leftrightarrow \forall 
    \varepsilon > 0, \; \exists n_{0} \in \mathbb{N} \; : \: 
    \forall n \geq n_{0} \quad \abs{a_{n}- b} \leq \varepsilon 
  \]

  Άρα και για $ \varepsilon = \frac{\varepsilon}{2}, \; \exists n_{2} \in 
  \mathbb{N} \; : \; \forall n \geq n_{2}  \quad 
  \abs{a_{n_{2}} - b} < \frac{\varepsilon}{2}  $

  Θέτουμε $ n_{0} = \max \{ n_{1}, n_{2} \} $, ώστε να ισχύουν οι 
  κ οι δύο παραπάνω ανισότητες ταυτόχρονα. 

  Επομένως, έχουμε 
  \[
    0 \leq \abs{a-b} = \abs{a- a_{n}+ a_{n}- b} \leq 
    \abs{a- a_{n}} + \abs{a_{n}- b} = \abs{a_{n}-a} + \abs{a_{n}-b} < \frac{\varepsilon}{2} + 
    \frac{\varepsilon}{2} = \varepsilon 
  \] 
  άρα από την πρόταση~\ref{prop:epsilonprot} έχουμε ότι 
  $ \abs{a-b} = 0 \Rightarrow a=b $.
\end{proof}

%todo idiothtes oriwn

\begin{mybox2}
  \begin{thm}
    Κάθε συγκλίνουσα ακολουθία είναι φραγμένη.
  \end{thm}
\end{mybox2}

\begin{proof}
  Έστω $ (a_{n})_{n \in \mathbb{Ν}} $ συγκλίνουσα $ \Rightarrow 
  \exists a \in \mathbb{R} $ τέτοιο ώστε $ \lim_{n \to +\infty} a_{n}
  =a $, άρα 
  \[
    \forall \varepsilon > 0, \; \exists n_{0} \in \mathbb{N} \; 
    : \; \forall n \geq n_{0}\quad \abs{a_{n}-a} < \varepsilon  
  \] 
  άρα και για $ \varepsilon =1>0, \; \exists n_{0} \in \mathbb{N} \; 
  : \; \forall n \geq n_{0} \quad 
  \inlineequation[eq:sygkfrag]{\abs{a_{n}-a} < 1} $. 

  Δηλαδή $ \forall n \geq n_{0} $ έχουμε ότι 
  \[
    \abs{a_{n}} = \abs{a_{n}-a + a} \leq \abs{a_{n} - a} + \abs{a} 
    \overset{\eqref{eq:sygkfrag}}{<} 1 + \abs{a}  
  \] 

  Δηλαδή, καταφέραμε να φράξουμε τους όρους της ακολουθίας με δείκτες 
  $n \geq n_{0} $.

  Τώρα, θέτουμε $ M = \max \{ \abs{a_{1}} , \abs{a_{2}} , \ldots, 
  \abs{a_{n-1}} , 1 + \abs{a}\} $ και έχουμε ότι $ \abs{a_{n}} 
  < M, \; \forall n \in \mathbb{N} $, επομένως η 
  $ (a_{n})_{n \in \mathbb{N}} $ είναι φραγμένη.
\end{proof}

\begin{rem}
\item {}
  \begin{enumerate}[i)]
    \item Προσοχή, το αντίστροφο της παραπάνω πρότασης, δεν ισχύει. Πράγματι, 
      για την  ακολουθία $ a_{n} = (-1)^{n}, \; \forall n \in \mathbb{N} $ 
      έχουμε ότι είναι φραγμένη, όχι όμως και συγκλίνουσα.
    \item Ενδιαφέρον έχει το αντιθετοαντίστροφο της παραπάνω πρότασης, που 
      λεέι ότι αν μια ακολουθία δεν είναι φραγμένη, τότε δεν συγκλίνει.
  \end{enumerate}
\end{rem}

\begin{mybox3}
  \begin{prop}
    Έστω $ (a_{n})_{n \in \mathbb{N}} $ ακολουθία πραγματικών αριθμών και 
    $ l \in \mathbb{R} $. Τότε ισχύουν οι ισοδυναμίες:
    \[ \lim_{n \to \infty} a_{n} = l \Leftrightarrow \lim_{n \to \infty} (
      a_{n}- l) = 0 
    \Leftrightarrow \lim_{n \to \infty} \abs{a_{n}-l} = 0\]
  \end{prop}
\end{mybox3}

\begin{proof}
\item {}
  \begin{align*} 
    \lim_{n \to \infty} a_{n}= l & \Leftrightarrow \forall 
    \varepsilon > 0 \; \exists n_{0} \in \mathbb{N} \; : \; 
    \abs{a_{n}-l} < \varepsilon, \; \forall n \geq n_{0} \\ 
    \lim_{n \to \infty} (a_{n}-l) = 0 & \Leftrightarrow \forall 
    \varepsilon >0 \; \exists n_{0} \in \mathbb{N} \; : \; 
    \abs {(a_{n}-l) - 0} < \varepsilon , \; \forall n \geq n_{0} \\
    \lim_{n \to \infty} \abs{a_{n}-l}=0 < \varepsilon & \Leftrightarrow 
    \forall \varepsilon > 0 \; \exists n_{0} \in \mathbb{N} 
    \; : \; \abs {\abs{a_{n}- l } - 0} < \varepsilon, \; \forall n \geq 
    n_{0} 
  \end{align*}
  και προφανώς ισχύει ότι 
  $ \abs{a_{n}-l} = \abs{(a_{n}-l)-0} = \abs{\abs{a_{n}- l} -0} $
\end{proof}

\begin{mybox3}
  \begin{prop}
    Έστω $ \lim_{n \to \infty} a_{n}=a \in \mathbb{R} $ και έστω $ l<a<k $. Τότε 
    $ \exists n_{0} \in \mathbb{N} \; : \; \forall n \geq n_{0} \quad l < a_{n} < k $.
  \end{prop}
\end{mybox3}
\begin{proof}
  Αν $ l<a<k $ τότε $l-a<0<k-a $.
  Έστω $ \varepsilon >0 $ τέτοιο ώστε $ \varepsilon < k-a $ και $ l-a < - 
  \varepsilon $. Τότε για αυτό το $ \varepsilon >0 $, επειδή $ \lim_{n \to \infty}
  a_{n}=a $, έχουμε ότι $ \exists n_{0} \in \mathbb{N} \; : \; \forall n \geq n_{0}
  \quad \abs{a_{n}- a} < \varepsilon \Leftrightarrow - \varepsilon < a_{n}-a <
  \varepsilon $. 

  Άρα $ \forall n \geq n_{0} $, έχουμε ότι
  \begin{gather*}
    l-a < -\varepsilon < a_{n}-a < \varepsilon < k-a \\
    l-a < a_{n}- a< k-a \\
    l < a_{n} < k
  \end{gather*} 
\end{proof}

  %todo να γράψω (ίσως) πρόταση $ \lim_{n \to \infty} a_{n}=a \neq 0 \Rightarrow
  %\exists n_{0} \in \mathbb{N} \; : \; \forall n \geq n_{0} \quad a_{n} \neq 0 $.  

\section{Υπακολουθίες}

\begin{mybox1}
  \begin{dfn}
    Έστω $ (a_{n})_{n \in \mathbb{N}} $ ακολουθία, και έστω $ 
    (k_{n})_{n \in \mathbb{N}} $ \textbf{γνησίως αύξουσα} ακολουθία φυσικών αριθμών $ (k_{1}<k_{2}<k_{3}<\cdots) $. 
    Τότε η ακολουθία $ b_{n} = (a_{k_{n}}), \; n \in \mathbb{N} $ 
    ονομάζεται \textcolor{Col1}{υπακολουθία} της 
    $ (a_{n})_{n \in \mathbb{N}} $.
  \end{dfn}
\end{mybox1}

%todo παραδειγματα υπακολουθιων

\begin{mybox3}
  \begin{prop}
    Αν μια ακολουθία είναι φραγμένη τότε και \textbf{κάθε} υπακολουθία της 
    είναι φραγμένη.
  \end{prop}
\end{mybox3}

\begin{rem}
\item {}
  \begin{enumerate}[i)]
    \item Ανάλογες προτάσεις ισχύουν και για την περίπτωση όπου η 
      ακολουθία είναι άνω, ή κάτω φραγμένη.
    \item 
      Ενδιαφέρον παρουσιάζει το αντιθετοαντίστροφο της παραπάνω 
      πρότασης, όπου λέει ότι αν τουλάχιστον μία υπακολουθία μιας 
      ακολουθίας δεν είναι φραγμένη, τότε κ η ίδια η ακολουθία 
      δεν θα είναι φραγμένη.
  \end{enumerate}
\end{rem}

\begin{mybox3}
  \begin{prop}
    Αν μια ακολουθία είναι γνησίως αύξουσα (αντίστοιχα γνησίως φθίνουσα) 
    τότε και \textbf{κάθε} υπακολουθία της θα είναι γνησίως αύξουσα (αντίστοιχα 
    γνησίως φθίνουσα).
    %todo υπολοιπες προτασεις εδω
  \end{prop}
\end{mybox3}

\begin{lem}\label{lem:kn}
  Έστω $ (a_{n})_{n \in \mathbb{N}} $ ακολουθία και $ (a_{k_{n}})_
  {n \in \mathbb{N}} $ υπακολουθία της. Τότε $ k_{n} \geq n, \; 
  \forall n \in \mathbb{N} $.
\end{lem}
\begin{proof}
\item {}
  \begin{myitemize}[labelindent=1em]
    \item Για $ n=1 $, έχω: Προφανώς $ k_{n} \in \mathbb{N}, \; \forall n 
      \in \mathbb{N} \Rightarrow k_{n} \geq 1 $, ισχύει:
    \item Έστω ότι ισχύει για $ n $, δηλ.  
      $ \inlineequation[eq:propkn]{k_{n} \geq n} $.  
    \item θ.δ.ο.\ ισχύει για $ n+1 $. Πράγματι, 
      \[ 
        k_{n+1} 
        \overset{k_{n} \; \text{γν.αυξ.}}{>} k_{n} 
        \overset{\eqref{eq:propkn}}{\geq} n \Rightarrow k_{n+1} > n \Rightarrow k_{n+1} 
        \geq n+1
      \]
  \end{myitemize}
\end{proof}

\begin{mybox3}
  \begin{prop}
    Αν μια ακολουθία $ (a_{n})_{n \in \mathbb{N}} $ συγκλίνει στο $ a 
    \in \mathbb{R} $, τότε και \textbf{κάθε} υπακολουθία της συγκλίνει επίσης 
    στο $ a \in \mathbb{R} $.
  \end{prop}
\end{mybox3}

\begin{proof}
\item {}
  Έστω $ (a_{k_{n}})_{n \in \mathbb{N}} $ υπακολουθία της 
  $ (a_{n})_{n \in \mathbb{N}} $. 

  Έστω $ \varepsilon >0 $. Τότε, επειδή $ \lim_{n \to \infty} a_{n}=a $, έχουμε ότι 
  \begin{equation} \label{eq:ypaksygk}
    \exists n_{0} \in \mathbb{N} \; : \; \forall n \geq n_{0} 
    \quad \abs{a_{n} - a} < \varepsilon 
  \end{equation}
  Δηλαδή,
  $ \exists n_{0} \in \mathbb{N} \, : \, k_{n} \overset{\ref{lem:kn}}{\geq} n 
  \geq n_{0}  $ και άρα από τη σχέση~\eqref{eq:ypaksygk}, έχουμε 
  $\abs{a_{k_{n}} - a} < \varepsilon  $, δηλαδή 
  $ \lim_{n \to +\infty} a_{k_{n}} = a$.
\end{proof}

\begin{rem}
  Το αντιθετοαντίστροφο της παραπάνω πρότασης, μας λέει, ότι αν δυο 
  υπακολουθίες μιας ακολουθίας $ (a_{n})_{n \in \mathbb{N}} $ συγκλίνουν σε 
  διαφορετικά όρια, τότε δεν υπάρχει το όριο της ακολουθίας 
  $ (a_{n})_{n \in \mathbb{N}} $.
\end{rem}

\begin{example}
  Να δείξετε ότι η ακολουθία $ a_{n} = (-1)^{n}\frac{n}{n+1} $ δεν συγκλίνει. 
\end{example}

\begin{proof}
\item {}
  Θεωρούμε τις υπακολουθίες:
  \begin{myitemize}[labelindent=1em]
    \item $ a_{2n} = (-1)^{2n} \frac{2n}{2n+1} = \frac{2n}{2n+1} = \frac{2n}{n(2+
      \frac{1}{n})} = \frac{2}{2+ \frac{1}{n}} \xrightarrow{n \to \infty} 1 $
    \item $ a_{2n-1} = (-1)^{2n-1} \frac{2n-1}{2n-1+1} = -\frac{2n-1}{2n} = 
      -\left(1 - \frac{1}{2n}\right) \xrightarrow{n \to \infty} -1  $
  \end{myitemize}
  Άρα, η ακολουθία $ a_{n}= (-1)^{n} \frac{n}{n+1} $ δεν συγκλίνει.
\end{proof}



\section{Ιδιότητες Ορίου Ακολουθίας}

\begin{mybox3}
  \begin{prop}
    Έστω $ \lim_{n \to +\infty} a_{n} = a$ και $ \lim_{n \to +\infty} 
    b_{n} = b $, όπου $ a,b \in \mathbb{R} $. Τότε:
    \begin{enumerate}[i)]
      \item $ \lim_{n \to +\infty} (a_{n} + b_{n}) = a+b $
      \item $ \lim_{n \to +\infty} \lambda a_{n}= \lambda a $
      \item $ \lim_{n \to +\infty} (a_{n}\cdot b_{n}) = a\cdot b $
      \item $ \lim_{n \to +\infty} \frac{1}{a_{n}} = \frac{1}{a},
        \; a \neq 0, \; a_{n} \neq 0, \; \forall n \in \mathbb{N}  $
      \item $ \lim_{n \to +\infty} \frac{a_{n}}{b_{n}} = \frac{a}{b},
         \; b \neq 0, \; b_{n} \neq 0, \forall n \in \mathbb{N} $
    \end{enumerate}
  \end{prop}
\end{mybox3}

\begin{proof}
\item {}
  \begin{enumerate}[i)]
    Έστω $ \varepsilon > 0 $.
  \item $ \lim_{n \to +\infty} a_{n} = a \Leftrightarrow \forall 
    \varepsilon > 0, \; \exists n_{1} \in \mathbb{N} \; : \; 
    \forall n \geq n_{1} \quad \abs{a_{n}-a} < \varepsilon$.

    Άρα και για $ \frac{\varepsilon}{ 2} > 0, \; 
    \exists n_{1} \in \mathbb{N} \; : \; \forall n \geq n_{1} 
    \quad \inlineequation[eq:idiot1]{\abs{a_{n}-a} <
    \frac{\varepsilon}{2}} $

    $ \lim_{n \to +\infty} b_{n} = b \Leftrightarrow \forall 
    \varepsilon > 0, \; \exists n_{2} \in \mathbb{N} \; : \; 
    \forall n \geq n_{2} \quad \abs{b_{n}-b} < \varepsilon$.

    Άρα και για $ \frac{\varepsilon}{ 2} > 0, \; 
    \exists n_{2} \in \mathbb{N} \; : \; \forall n \geq n_{2} 
    \quad \inlineequation[eq:idiot2]{\abs{b_{n}-b} <
    \frac{\varepsilon}{2}} $

    Θέτουμε $ n_{0}= \max \{ n_{1}, n_{2} \} $, ώστε να ισχύουν 
    και οι δύο παραπάνω ανισότητες ταυτόχρονα.

    Τότε για $ n \geq n_{0} $, έχουμε 
    \[
      \abs{(a_{n}+ b_{n}) - (a+b)} = \abs{(a_{n}- a) + 
      (b_{n}-b)} \leq \abs{a_{n}- a} + \abs{b_{n}-b} 
      \overset{\eqref{eq:idiot1}}
      {\underset{\eqref{eq:idiot2}}{<}} 
      \frac{\varepsilon }{2} + \frac{\varepsilon}{ 2} = 
      \varepsilon
    \] 

  \item 
    Αν $ \lambda =0 $, τότε η σχέση είναι προφανής.

    Έστω $ \lambda  \neq 0 $. Έστω $ \varepsilon > 0 $.

    $ \lim_{n \to +\infty} a_{n} = a \Leftrightarrow \forall 
    \varepsilon >0, \; \exists 
    n_{0} \in \mathbb{N} \; : \; \forall n \geq n_{0} \quad 
    \abs{a_{n}- a} < \varepsilon$. 

    Άρα και για $ \frac{\varepsilon}{\abs{\lambda }} >0$,
    έχουμε ότι $ \exists n_{0} \in \mathbb{N} \; : \; 
    \forall n \geq n_{0} \quad \inlineequation[eq:idiot3]
    {\abs{a_{n}- a} < \frac{\varepsilon}{\abs{\lambda }}}$. 

    Τότε $ \forall n \geq n_{0} $, 
    έχουμε ότι 
    \[
      \abs{\lambda  a_{n}- \lambda a} = \abs{\lambda (a_{n}- a)} = 
      \abs{\lambda } \cdot 
      \abs{a_{n}- a} \overset{\eqref{eq:idiot3}}{<} \abs{\lambda }
      \cdot \frac{\varepsilon}{\abs{\lambda }} = \varepsilon 
    \] 

  \item Έχουμε ότι $ (a_{n})_{n \in \mathbb{N}} $ συγκλίνουσα $ 
    \Rightarrow (a_{n})_{n \in \mathbb{N}} $ είναι φραγμένη. 
    Άρα 
    \begin{equation}
      \label{eq:idiot4}
      \exists M >0, \; M \in \mathbb{R} \; : \; \forall n \in 
      \mathbb{N} \quad 
      \abs{a_{n}} \leq M
    \end{equation} 
    Τότε 
    \begin{align*}
      \abs{a_{n} b_{n} - ab} = \abs{a_{n}b_{n} - a_{n}b + 
      a_{n}b -ab} &= \abs{a_{n}(b_{n}-b)+b(a_{n}-a) } \\
                  &\leq \abs{a_{n}}\cdot \abs{b_{n}-b} + \abs{b} 
                  \cdot \abs{a_{n}-a} \\
                  &\overset{\eqref{eq:idiot4}}{\leq} 
                  M \cdot  \abs{b_{n}-b} + \abs {b}\cdot 
                  \abs{a_{n}-a}
    \end{align*}

    Επειδή $ \lim_{n \to \infty} a_{n}=a $, τότε για 
    $ \frac{\varepsilon}{2M} > 0 $  
    \begin{equation} \label{eq:idiotm1}
     \exists n_{1} \in \mathbb{N} \; : \; \forall n \geq
      n_{1} \quad \abs{a_{n}- a} < \frac{\varepsilon}{2M}
    \end{equation}  
    Ομοίως, επειδή $ \lim_{n \to \infty} b_{n} =b $, τότε για 
    $ \frac{\varepsilon}{2 \abs{b}} $ 
    \begin{equation} \label{eq:idiotm2}
    \exists n_{2} \in \mathbb{N} \; : \; \forall 
      n \geq n_{2} \quad \abs{b_{n}-b} < \frac{\varepsilon}{2 \abs{b}} 
    \end{equation} 
    Επιλέγουμε $ n_{0}= \max \{ n_{1}, n_{2}\} $ και έχουμε ότι 
    $ \forall n \geq n_{0}$,
    \[
      \abs{a_{n} b_{n} - ab} \leq M \cdot \abs{b_{n}-b} + 
      \abs{b} \cdot \abs{a_{n}- a} 
    \overset{\eqref{eq:idiotm1}}{\underset{\eqref{eq:idiotm2}}{<}} M \cdot 
      \frac{\varepsilon}{2 M} + \abs {b} \cdot 
      \frac{\varepsilon}{2 \abs{b}} =
      \frac{\varepsilon}{2} + \frac{\varepsilon}{2} = 
      \varepsilon
    \]
  \item 
    \begin{lem}
      $ b \neq 0 \Rightarrow \exists n_{0} \in \mathbb{N} \; 
      : \; b_{n} \neq 0, \; n \geq n_{0} $ 
    \end{lem}

    \begin{proof}
    \item {}
      Έστω $ b \neq 0 \Rightarrow \frac{\abs{b}}{2} >0 $. 

      Για $ \varepsilon = \frac{\abs{b}}{2} >0 $ και λόγω 
      ότι $ \lim_{n \to +\infty} b_{n} = b $
      έχουμε ότι $ \exists n_{0} \in \mathbb{N} \; : \; 
      \forall n \geq n_{0} \quad \abs{b_{n}-b} 
      < \frac{\abs{b}}{2} $ 

      Άρα για $ n \geq n_{0} $, έχουμε 
      \begin{equation}\label{eq:idiot1/b}
        \abs{b} = \abs{b - b_{n} + b_{n}} \leq 
        \abs{b - b_{n}} + \abs{b_{n}} <
        \frac{\abs{b}}{2} + \abs{b_{n}} \Leftrightarrow 
        \abs{b_{n}} > \abs{b} -
        \frac{\abs{b}}{2} \Leftrightarrow \abs{b_{n}}  > 
        \frac{\abs{b}}{2} 
      \end{equation} 
      δηλαδή, $ b_{n} \neq 0, \; \forall n \geq n_{0} $.
      \qed 

      Οπότε, για $ n \geq n_{0} $ έχουμε ότι:

      \[ 
        \abs{\frac{1}{b_{n}} - \frac{1}{b}} = \abs{\frac{b-b_{n}}{b_{n}\cdot b}}
        = \frac{\abs{b-b_{n}} }{\abs{b_{n}} \cdot \abs{b}}
        \overset{\eqref{eq:idiot1/b}}{<} 
        \frac{2 \abs{b - b_{n}}}{\abs{b}^{2}} 
      \]

      Έστω $ \varepsilon >0 $, τότε επειδή 
      $ \lim_{n \to +\infty} b_{n} =b $, για 
      $ \frac{\varepsilon \abs{b} ^{2}}{2} > 0 $, έχουμε ότι
      \begin{equation}\label{eq:idiot1/b2}
        \exists n_{1} \in \mathbb{N} \; : \; \forall n 
        \geq n_{1} \quad \abs{b - b_{n}} <
        \frac{\varepsilon \abs{b}^{2}}{2}
      \end{equation} 

      Επιλέγουμε $ n_{2} = \max \{ n_{0}, n_{1} \} $. 
      Τότε $ \forall n \geq n_{2} $ ισχύει ότι 
      \[
        \abs{\frac{1}{b_{n}} - \frac{1}{b}} < 
        \frac{2 \abs{b -b_{n}}}{\abs{b} ^{2}} 
        \overset{\eqref{eq:idiot1/b2}}{<}  \varepsilon 
      \] 
    \end{proof}

  \item 
    $ \lim_{n \to +\infty} \frac{a_{n}}{b_{n}} = 
    \lim_{n \to +\infty} \left(a_{n}\cdot \frac{1}{b_{n}}
    \right) = \lim_{n \to +\infty} a_{n} \cdot \lim_{n \to +\infty} 
    \frac{1}{b_{n}} = a \cdot \frac{1}{b} = \frac{a}{b} $
\end{enumerate}
\end{proof}

\begin{mybox3}
  \begin{prop}[Μηδενική επί Φραγμένη]
    Έστω $ (a_{n})_{n \in \mathbb{N}} $ \textbf{μηδενική} ακολουθία και 
    $ (b_{n})_{n \in \mathbb{N}} $ 
    \textbf{φραγμένη}. Τότε, $ \lim_{n \to +\infty} (a_{n}\cdot b_{n}) = 0 $.
  \end{prop}
\end{mybox3}

\begin{proof}
\item {}
  Έστω $ \varepsilon >0$, τότε
  \begin{equation} \label{eq:qq1}
    (b_{n})_{n \in \mathbb{N}} \;\; \text{φραγμένη} \; \Rightarrow 
    \exists M>0, \; : \; \abs{b_{n}} \leq M \; \forall n \in \mathbb{N}
  \end{equation}
  \begin{equation} \label{eq:qq2}
    (a_{n})_{n \in \mathbb{N}}  \;\; \text{μηδενική} \; \Rightarrow  
    \text{για} \;\; \frac{\varepsilon}{M} > 0, \;  \exists n_{0} \in 
    \mathbb{N} \; : \; \forall n \geq n_{0} \quad \abs{a_{n}-0} 
    < \frac{\varepsilon}{M} \Leftrightarrow \abs{a_{n}} < \frac{\varepsilon}{M}
  \end{equation}
  Άρα για κάθε $ n \geq n_{0} $ έχουμε
  \[
    \abs{a_{n}\cdot b_{n} - 0} = \abs{a_{n}\cdot b_{n} } 
    = \abs{a_{n}} \cdot \abs{b_{n}} 
    \overset{\eqref{eq:qq1}}{\leq} \abs{a_{n}} \cdot M 
    \overset{\eqref{eq:qq2}}{<} 
    \frac{\varepsilon}{M} \cdot M = \varepsilon
  \]
  Άρα $ \lim_{n \to +\infty} a_{n} \cdot b_{n} = 0 $.
\end{proof}

%todo παραδείγματα (μηδενικη επι φραγμενη)

\begin{mybox3}
  \begin{prop}[Κριτήριο Παρεμβολής]
    Έστω $ (a_{n})_{n \in \mathbb{N}}, (b_{n})_{n \in \mathbb{N}} $ και 
    $ (c_{n})_{n \in \mathbb{N}} $, τρεις ακολουθίες, τέτοιες ώστε:
  \end{prop}

  \vspace{\baselineskip}

  \begin{minipage}{0.35\textwidth}
    \begin{enumerate}[i)]
      \item $ a_{n} \leq b_{n} \leq c_{n}, \; \forall n \in 
        \mathbb{N} $ \hfill \tikzmark{a} 
      \item $ \lim_{n \to +\infty} a_{n} = \lim_{n \to +\infty} 
        c_{n} = l $ \hfill \tikzmark{b}
    \end{enumerate}
  \end{minipage}
  \mybrace{a}{b}[$ \lim_{n \to +\infty} b_{n} = l$]
\end{mybox3}

\begin{proof}
\item {}
  Έστω $ \lim_{n \to +\infty} a_{n} = \lim_{n \to +\infty} c_{n} = l $ και 
  έστω $ \varepsilon >0 $, τότε
  \[ \lim_{n \to +\infty} a_{n} = l \Leftrightarrow \forall 
    \varepsilon >0, \; \exists n_{1} \in \mathbb{N} \; : \; \forall n 
    \geq n_{1} \quad \abs{a_{n} - l} < \varepsilon \Leftrightarrow 
  \overbrace{l - \varepsilon < a_{n}} < l + \varepsilon \] 

  \[ \lim_{n \to +\infty} c_{n} = l \Leftrightarrow \forall 
    \varepsilon >0, \; \exists n_{2} 
    \in \mathbb{N} \; : \; \forall n \geq n_{2} \quad \abs{c_{n} - l} 
    < \varepsilon \Leftrightarrow 
  l - \varepsilon < \underbrace{c_{n} < l + \varepsilon} \]

  Επιλέγουμε $ n_{0} = \max \{ n_{1}, n_{2} \} $, οπότε 
  $ \exists n_{0} \in \mathbb{N} \; : \; 
  \forall n \geq n_{0} $
  \begin{gather*}
    l - \varepsilon < a_{n} \leq b_{n} \leq c_{n} < l + 
    \varepsilon \Leftrightarrow \\
    l - \varepsilon < b_{n} < l + \varepsilon \Leftrightarrow \\
    - \varepsilon < b_{n} - l < \varepsilon \Leftrightarrow \\
    \abs{b_{n}-l} < \varepsilon, \; \forall n \geq n_{0}
  \end{gather*}
  Άρα $ \lim_{n \to +\infty} b_{n} = l $.
\end{proof}

\begin{cor}
  Έστω $ (a_{n})_{n \in \mathbb{N}} $ και $ 
  (b_{n})_{n \in \mathbb{N}} $ ακολουθίες, τέτοιες ώστε: 

  \vspace{\baselineskip}

  \begin{minipage}{0.25\textwidth}
    \begin{enumerate}[i)]
      \item $ \abs{a_{n}} \leq \abs{b_{n}}, \; \forall n \in 
        \mathbb{N} $ \hfill \tikzmark{a}
      \item $ \lim_{n \to +\infty} b_{n} = 0$ \hfill \tikzmark{b}
    \end{enumerate}
  \end{minipage}

  \mybrace{a}{b}[$ \lim_{n \to +\infty} a_{n} = 0 $]
\end{cor}

\begin{mybox3}
  \begin{prop}\label{prop:aleqb}
    Έστω $ \lim_{n \to +\infty} a_{n} = a $ και $ 
    \lim_{n \to +\infty} b_{n} = b $ και
    $ a_{n} \leq b_{n}, \; \forall n \in \mathbb{N} $, τότε $ a \leq b $.
  \end{prop}
\end{mybox3}

\begin{proof}(Με άτοπο)
\item {}
  Έστω $ a>b \Rightarrow a-b>0 $. Θέτουμε $ \varepsilon = 
  \frac{a-b}{2} $.

  Από τον ορισμό του ορίου για τις δύο ακολουθίες, έχουμε ότι:
  \begin{align*}
    \exists n_{1} \in \mathbb{N} \; : \; \forall n \geq n_{1} 
    \quad \abs{a_{n}-a} < \frac{a-b}{2} \Leftrightarrow 
    \overbrace{- \frac{a-b}{2} < a_{n} - a} < \frac{a+b}{2} \Rightarrow a_{n} > a -
    \frac{a-b}{2} = \frac{a+b}{2} \\
    \exists n_{2} \in \mathbb{N} \; : \; \forall n \geq n_{2} 
    \quad \abs{b_{n}-b} < \frac{a-b}{2} \Leftrightarrow 
    - \frac{a-b}{2} < \underbrace{b_{n} - b < \frac{a-b}{2}} \Rightarrow b_{n} < b +
    \frac{a-b}{2} = \frac{a+b}{2} 
  \end{align*}

  Δηλαδή, έχουμε ότι

  \[ a_{n} >  \frac{a+b}{2}, \quad \forall n \geq n_{1} \; \text{και} \;  
  b_{n} <  \frac{a+b}{2}, \quad \forall n \geq n_{2}  \]

  Άρα για $ n_{0} = \max \{ n_{1}, n_{2} \} $ έχουμε ότι 
  $ b_{n} < \frac{a+b}{2} < a_{n}, \quad \forall n \geq n_{0} $, 
  δηλαδή $a_{n} > b_{n}, \quad \forall n \geq n_{0} $, άτοπο.
\end{proof}

\begin{mybox3}
  \begin{prop}
    Έστω $ \lim_{n \to +\infty} a_{n} = a $ και $ \lim_{n \to +\infty} 
    b_{n} = b $ και 
    $ a_{n} < b_{n}, \; \forall n \in \mathbb{N} $, τότε $ a \leq b $ 
  \end{prop}
\end{mybox3}

\begin{proof}
  $ a_{n}< b_{n}, \; \forall n \in \mathbb{N} \Rightarrow a_{n} \leq 
  b_{n}, \; \forall n \in \mathbb{N} \overset{\ref{prop:aleqb}}{\Rightarrow} 
  a \leq b $
\end{proof}

\begin{example}
\item {}
  Αν $ a_{n}=0, \; \forall n \in \mathbb{N} $
  και $ b_{n}= \frac{1}{n}, \; \forall n \in \mathbb{N} $, τότε έχουμε ότι 
  $ \lim_{n \to +\infty} a_{n} = 0 = a $ και 
  $ \lim_{n \to +\infty} b_{n} =  0 = b $ , δηλαδή $ a=b=0  $.
\end{example}

\begin{cor}
  Έστω $ (a_{n})_{n \in \mathbb{N}} $ ακολουθία, τέτοια ώστε 
  $ a_{n} \geq 0, \; \forall n \in \mathbb{N} $ και 
  $ \lim_{n \to +\infty} a_{n} = a$, τότε $ a \geq 0 $.
\end{cor}

\begin{proof}
  Έστω $ b_{n} = 0, \; \forall n \in \mathbb{N} $. Τότε προφανώς 
  $ a_{n} \geq b_{n}, \; \forall n
  \in \mathbb{N} \overset{\ref{prop:aleqb}}{\Rightarrow} 
  a = \lim_{n \to +\infty} a_{n} \geq \lim_{n \to +\infty} b_{n} = 0$.
\end{proof}

\begin{mybox2}
  \begin{thm}
    \label{thm:aukssygk}
    Έστω $ {(a_{n})}_{n \in \mathbb{N}} $ \textbf{αύξουσα (ή γνησίως αύξουσα)} 
    ακολουθία, πραγματικών αριθμών. Τότε:
    \begin{enumerate}
      \item Αν η $ {(a_{n})}_{n \in \mathbb{N}} $ είναι \textbf{άνω φραγμένη}, 
        τότε συγκλίνει στο supremum του συνόλου των όρων της.
      \item Αν η $ {(a_{n})}_{n \in \mathbb{N}} $ δεν είναι άνω φραγμένη, τότε 
        $ \lim_{n \to \infty} a_{n}=+\infty $.
    \end{enumerate}
  \end{thm}
\end{mybox2}

\begin{proof}
\item {}
  \begin{description}
    \item [1.] Έστω $ (a_{n})_{n \in \mathbb{N}} $ ακολουθία πραγματικών αριθμών, 
      τέτοια ώστε:
      \begin{enumerate}[i)]
        \item $ (a_{n})_{n \in \mathbb{N}} $ αύξουσα $ \Leftrightarrow 
          a_{n+1} \geq a_{n}, \; \forall n \in \mathbb{N}$ 
        \item $ (a_{n})_{n \in \mathbb{N}} $ άνω  φραγμένη $ 
          \Leftrightarrow a_{n} \leq M, \; \forall n \in 
          \mathbb{N}$, με $ M \in \mathbb{R} $.
      \end{enumerate}

      Η ακολουθία $ a_{n} $ είναι άνω φραγμένη, δηλαδή το σύνολο των όρων της 
      $ \{ a_{n}\; :\; n \in \mathbb{N} \} $ είναι άνω φραγμένο, και μη κενό,
      άρα από το αξίωμα πληρότητας υπάρχει το 
      $ \sup \{ a_{n}\; :\; n \in \mathbb{N} \} $. Έστω $ s = \sup \{ a_{n}\;
      :\; n \in \mathbb{N}\} $. Θα δείξουμε ότι $ \lim_{n \to \infty} a_{n}=s $, 
      δηλαδή ότι $ \forall \varepsilon >0, \; \exists n_{0} \in \mathbb{N} \; : \; 
      \forall n \geq n_{0} \quad s- \varepsilon < a_{n}< s+ \varepsilon $.

      Έστω $ \varepsilon >0 $. Από τη χαρακτηριστική ιδιότητα του 
      supremum, έχουμε ότι $ \exists n_{0} \in \mathbb{N} $ ώστε για τον 
      $ a_{n_{0}} \in \{ a_{n_{0}}\; :\; n \in \mathbb{N} \} $ να ισχύει 
      $ s- \varepsilon < a_{n_{0}} $. Επειδή $ a_{n} $ 
      αύξουσα, έχουμε ότι $ \forall n \geq n_{0}, \; a_{n} \geq a_{n_{0}} $. Επομένως
      \[
        s- \varepsilon < a_{n_{0}} \leq a_{n}, \quad \forall n \geq n_{0}
      \] 
      Επίσης
      \[ a_{n} \leq s < s + \varepsilon, \; \forall n \in \mathbb{N} \] 
    \item [2.] Έστω $ \varepsilon >0 $. Η ακολουθία $ {(a_{n})}_{n \in \mathbb{N}} $ 
      δεν είναι άνω φραγμένη, δηλαδή το σύνολο των όρων της 
      $ \{ a_{n}\; :\; n \in \mathbb{N} \} $ δεν είναι άνω φραγμένο, άρα για το 
      $ \varepsilon >0, \; \exists n_{0} \in \mathbb{N} $ ώστε για τον 
      $ a_{n_{0}} \in \{ a_{n}\; :\; n \in \mathbb{N} \} $ να ισχύει 
      $ a_{n_{0}} > \varepsilon $. Επειδή $ {(a_{n})}_{n \in \mathbb{N}} $
      αύξουσα, έχουμε ότι $ \forall n \geq n_{0}, \; a_{n} \geq a_{n_{0}} $. Επομένως
      \[
        a_{n} \geq a_{n_{0}} > \varepsilon , \quad \forall n \geq n_{0}  
      \] 
      Άρα $ \lim_{n \to \infty} a_{n}=+\infty $.
  \end{description}
\end{proof}

\begin{mybox2}
  \begin{thm}
    Έστω $ {(a_{n})}_{n \in \mathbb{N}} $ \textbf{φθίνουσα (ή γνησίως φθίνουσα)} 
    ακολουθία, πραγματικών αριθμών. Τότε:
    \begin{enumerate}
      \item Αν η $ {(a_{n})}_{n \in \mathbb{N}} $ είναι \textbf{κάτω φραγμένη}, 
        τότε συγκλίνει στο infimum του συνόλου των όρων της.
      \item Αν η $ {(a_{n})}_{n \in \mathbb{N}} $ δεν είναι κάτω φραγμένη, τότε 
        $ \lim_{n \to \infty} a_{n}=-\infty $.
    \end{enumerate}
  \end{thm}
\end{mybox2}
\begin{proof}
  Ομοίως %todo να γράψω την απόδειξη
\end{proof}

\begin{mybox3}
  \begin{prop}
    Έστω $ {(a_{n})}_{n \in \mathbb{N}} $ και $ {(b_{n})}_{n \in \mathbb{N}} $ ακολουθίες 
    πραγματικών αριθμών, με $ \lim_{n \to \infty} a_{n}= +\infty $. Τότε:
    \begin{enumerate}[i)]
      \item $ \lim_{n \to \infty} b_{n} = +\infty \Rightarrow \lim_{n \to +\infty}
        (a_{n}+b_{n})= +\infty$
      \item $ \lim_{n \to \infty} b_{n} = +\infty \Rightarrow \lim_{n \to +\infty}
        (a_{n}\cdot b_{n})= +\infty$
      \item $ \lim_{n \to \infty} b_{n} = -\infty \Rightarrow \lim_{n \to +\infty}
        (a_{n}\cdot b_{n})= -\infty$
      \item Αν $ {(b_{n})}_{n \in \mathbb{N}} $ είναι κάτω φραγμένη τότε $ \lim_{n \to
        \infty} (a_{n}+b_{n}) = +\infty $
      \item $ \lim_{n \to \infty} b_{n} = b \in \mathbb{R} 
        \Rightarrow \lim_{n \to \infty} (a_{n}+b_{n})= +\infty$
      \item $ \lim_{n \to \infty} b_{n} = b \neq 0 
        \Rightarrow \lim_{n \to +\infty} (a_{n}\cdot b_{n})= 
        \begin{cases} 
          +\infty, & b >0 \\
          -\infty, & b <0 
        \end{cases} $
    \end{enumerate}
  \end{prop}
\end{mybox3}
\begin{proof}
  \begin{enumerate}[i)]
    \item Έστω $ \varepsilon >0 $. Αρκεί να βρούμε $ n_{0} \in \mathbb{N} $ ώστε 
      $ \forall n \geq n_{0} \quad a_{n}+b_{n} > \varepsilon $. Πράγματι:
      \begin{myitemize}
        \item $ \lim_{n \to \infty} a_{n}=+\infty \Rightarrow $ για 
          $ \frac{\varepsilon}{2} > 0, \; \exists n_{1} \in \mathbb{N} \; : \; 
          \forall n \geq n_{1} \quad a_{n} > \frac{\varepsilon}{2} $
        \item $ \lim_{n \to \infty} b_{n}=+\infty \Rightarrow $ για 
          $ \frac{\varepsilon}{2} >0, \; \exists n_{2} \in \mathbb{N} \; : \; 
          \forall n \geq n_{2} \quad b_{n} > \frac{\varepsilon}{2} $
      \end{myitemize}
      Επομένως για $ n_{0}= \max \{ n_{1}, n_{2} \} $, έχουμε ότι $ \forall n \geq n_{0}
      \quad a_{n}+ b_{n} > \frac{\varepsilon}{2} + \frac{\varepsilon}{2} = 
      \varepsilon $. 
    \item Έστω $ \varepsilon >0 $. Αρκεί να βρούμε $ n_{0} \in \mathbb{N} $ ώστε 
      $ \forall n \geq n_{0} \quad a_{n}\cdot b_{n} > \varepsilon $. Πράγματι:
      \begin{myitemize}
        \item $ \lim_{n \to \infty} a_{n}=+\infty \Rightarrow $ για $ \varepsilon =
          1 >0, \; \exists n_{1} \in \mathbb{N} \; : \; \forall n \geq
          n_{1} \quad a_{n} > 1 $
        \item $ \lim_{n \to \infty} b_{n}=+\infty \Rightarrow $ για $ \varepsilon > 0,
          \; \exists n_{2} \in \mathbb{N} \; : \; \forall n \geq
          n_{2} \quad b_{n} > \varepsilon $
      \end{myitemize}
      Επομένως για $ n_{0}= \max \{ n_{1}, n_{2} \} $, έχουμε ότι $ \forall n \geq n_{0}
      \quad a_{n}\cdot b_{n} > 1 \cdot \varepsilon = \varepsilon $. 
    \item Έστω $ \varepsilon >0 $. Αρκεί να βρούμε $ n_{0} \in \mathbb{N} $ ώστε 
      $ \forall n \geq n_{0} \quad a_{n}\cdot b_{n} < - \varepsilon $. Πράγματι:
      \begin{myitemize}
        \item $ \lim_{n \to \infty} a_{n}=+\infty \Rightarrow $ για $ \varepsilon =
          1 >0, \; \exists n_{1} \in \mathbb{N} \; : \; \forall n \geq
          n_{1} \quad a_{n} > 1 $
        \item $ \lim_{n \to \infty} b_{n}=-\infty \Rightarrow $ για $ \varepsilon > 0,
          \; \exists n_{2} \in \mathbb{N} \; : \; \forall n \geq
          n_{2} \quad b_{n} < - \varepsilon \Leftrightarrow -b_{n} > \varepsilon $
      \end{myitemize}
      Επομένως για $ n_{0}= \max \{ n_{1}, n_{2} \} $, έχουμε ότι $ \forall n \geq n_{0}
      \quad   a_{n} \cdot (-b_{n}) >  1 \cdot \varepsilon = \varepsilon \Leftrightarrow 
      a_{n}\cdot b_{n} < - \varepsilon $
    \item Έστω $ \varepsilon >0 $. Αρκεί να βρούμε $ n_{0} \in \mathbb{N} $ ώστε 
      $ \forall n \geq n_{0} \quad a_{n} + b_{n} > \varepsilon $. Πράγματι:
      \begin{myitemize}
        \item $ b_{n} $ κάτω  φραγμένη, άρα $ \exists m \in \mathbb{R} 
          \; : \; b_{n} \geq m, \; \forall n \in \mathbb{N} $
        \item $ \lim_{n \to \infty} a_{n}=+\infty \Rightarrow $ για 
          $ \varepsilon + \abs{m} >0, \; \exists n_{1} \in \mathbb{N} \; : \; 
          \forall n \geq n_{1} \quad a_{n} > \varepsilon + \abs{m} $.
      \end{myitemize}
      Επομένως για το $ n_{0} $, έχουμε ότι $ \forall n \geq n_{0}
      \quad a_{n} + b_{n} > \varepsilon + \abs{m} + m \geq \varepsilon $. 
    \item Επειδή $ \lim_{n \to \infty} b_{n} = b $, έχουμε ότι η $ {(b_{n})}_{n \in
      \mathbb{N}} $ είναι φραγμένη, άρα και κάτω φραγμένη, οπότε απο το προηγούμενο 
      ερώτημα, έχουμε ότι $ \lim_{n \to \infty} (a_{n} + b_{n}) = +\infty $
    \item Έστω $ b>0 $. Θα δείξουμε ότι $ \lim_{n \to \infty} (a_{n}\cdot b_{n}) =
      +\infty $. Έστω $ \varepsilon >0 $. Αρκεί να βρούμε $ n_{0} \in \mathbb{N} \; : \;
      \forall n \geq n_{0} \quad a_{n}+b_{n} > \varepsilon $. Πράγματι:
      \begin{myitemize}
        \item $ \lim_{n \to \infty} b_{n}= b $ για $ \varepsilon =
          \frac{b}{2} \; \exists n_{1} \in \mathbb{N} \; : \; \forall n \geq
          n_{1} \quad \abs{b_{n}-b} < \frac{b}{2} \Leftrightarrow 
          \underbrace{- \frac{b}{2} < b_{n}-b} < \frac{b}{2} 
          \Rightarrow b_{n} > b- \frac{b}{2} = \frac{b}{2} $
        \item $ \lim_{n \to \infty} a_{n}=+\infty \Rightarrow $ για $ \varepsilon =
          \frac{2 \varepsilon}{b} >0 \; \exists n_{2} \in \mathbb{N} \; : \; 
          \forall n \geq n_{2} \quad a_{n} > \frac{2 \varepsilon}{b} $
      \end{myitemize}
      Επομένως για το $ n_{0} = \max \{ n_{1}, n_{2} \}  $, έχουμε ότι 
      $ \forall n \geq n_{0} \quad a_{n} \cdot b_{n} > \frac{2 \varepsilon}{b}
      \cdot \frac{b}{2} = \varepsilon $. Ομοίως για $ b<0 $.
  \end{enumerate}
\end{proof}

\begin{mybox3}
  \begin{prop}
    Έστω $ {(a_{n})}_{n \in \mathbb{N}} $ και $ {(b_{n})}_{n \in \mathbb{N}} $ ακολουθίες 
    πραγματικών αριθμών, με $ \lim_{n \to \infty} a_{n}= -\infty $. Τότε:
    \begin{enumerate}[i)]
      \item $ \lim_{n \to \infty} b_{n} = -\infty \Rightarrow \lim_{n \to +\infty}
        (a_{n}+b_{n})= -\infty$
      \item $ \lim_{n \to \infty} b_{n} = -\infty \Rightarrow \lim_{n \to +\infty}
        (a_{n}\cdot b_{n})= +\infty$
      \item $ \lim_{n \to \infty} b_{n} = +\infty \Rightarrow \lim_{n \to +\infty}
        (a_{n}\cdot b_{n})= -\infty$
      \item Αν $ {(b_{n})}_{n \in \mathbb{N}} $ είναι άνω φραγμένη τότε $ \lim_{n \to
        \infty} (a_{n}+b_{n}) = -\infty $
      \item $ \lim_{n \to \infty} b_{n} = b \in \mathbb{R} 
        \Rightarrow \lim_{n \to \infty} (a_{n}+b_{n})= -\infty$
      \item $ \lim_{n \to \infty} b_{n} = b \neq 0 
        \Rightarrow \lim_{n \to +\infty} (a_{n}\cdot b_{n})= 
        \begin{cases} 
          +\infty, & b <0 \\
          -\infty, & b >0 
        \end{cases} $
    \end{enumerate}
  \end{prop}
\end{mybox3}
\begin{proof}
  \begin{enumerate}
    \item Έστω $ \varepsilon >0 $. Αρκεί να βρούμε $ n_{0} \in \mathbb{N} $ ώστε 
      $ \forall n \geq n_{0} \quad a_{n}+b_{n} < - \varepsilon $. Πράγματι:
      \begin{myitemize}
        \item $ \lim_{n \to \infty} a_{n}=+\infty \Rightarrow $ για $ \varepsilon =
          \frac{\varepsilon }{2} \; \exists n_{1} \in \mathbb{N} \; : \; \forall n \geq
          n_{1} \quad a_{n} < - \frac{\varepsilon}{2} $
        \item $ \lim_{n \to \infty} b_{n}=-\infty \Rightarrow $ για $ \varepsilon =
          \frac{\varepsilon }{2} \; \exists n_{2} \in \mathbb{N} \; : \; \forall n \geq
          n_{2} \quad b_{n} < - \frac{\varepsilon}{2} $
      \end{myitemize}
      Επομένως για $ n_{0}= \max \{ n_{1}, n_{2} \} $, έχουμε ότι $ \forall n \geq n_{0}
      \quad a_{n}+ b_{n} < - \frac{\varepsilon}{2} - \frac{\varepsilon}{2} = 
      \varepsilon $
    \item Έστω $ \varepsilon >0 $. Αρκεί να βρούμε $ n_{0} \in \mathbb{N} $ ώστε 
      $ \forall n \geq n_{0} \quad a_{n}\cdot b_{n} > \varepsilon $. Πράγματι:
      \begin{myitemize}
        \item $ \lim_{n \to \infty} a_{n}=-\infty \Rightarrow $ για $ \varepsilon =
          1 >0 \; \exists n_{1} \in \mathbb{N} \; : \; \forall n \geq
          n_{1} \quad a_{n} < - 1 \Leftrightarrow - a_{n} > 1 $
        \item $ \lim_{n \to \infty} b_{n}=-\infty \Rightarrow $ για $ \varepsilon > 0
          \; \exists n_{2} \in \mathbb{N} \; : \; \forall n \geq
          n_{2} \quad b_{n} < - \varepsilon \Leftrightarrow - b_{n} > \varepsilon $
      \end{myitemize}
      Επομένως για $ n_{0}= \max \{ n_{1}, n_{2} \} $, έχουμε ότι $ \forall n \geq n_{0}
      \quad (- a_{n}) \cdot (-b_{n}) >  1 \cdot \varepsilon = \varepsilon 
      \Leftrightarrow a_{n}\cdot b_{n} > \varepsilon $. 
    \item Έστω $ \varepsilon >0 $. Αρκεί να βρούμε $ n_{0} \in \mathbb{N} $ ώστε 
      $ \forall n \geq n_{0} \quad a_{n}\cdot b_{n} < - \varepsilon $. Πράγματι:
      \begin{myitemize}
        \item $ \lim_{n \to \infty} a_{n}=-\infty \Rightarrow $ για $ \varepsilon =
          1 >0 \; \exists n_{1} \in \mathbb{N} \; : \; \forall n \geq
          n_{1} \quad a_{n} < - 1 \Leftrightarrow - a_{n} > 1 $
        \item $ \lim_{n \to \infty} b_{n}=+\infty \Rightarrow $ για $ \varepsilon > 0
          \; \exists n_{2} \in \mathbb{N} \; : \; \forall n \geq
          n_{2} \quad b_{n} > \varepsilon $
      \end{myitemize}
      Επομένως για $ n_{0}= \max \{ n_{1}, n_{2} \} $, έχουμε ότι $ \forall n \geq n_{0}
      \quad  - a_{n} \cdot b_{n} >  1 \cdot \varepsilon = \varepsilon \Leftrightarrow 
      a_{n}\cdot b_{n} < - \varepsilon $
    \item Έστω $ \varepsilon >0 $. Αρκεί να βρούμε $ n_{0} \in \mathbb{N} $ ώστε 
      $ \forall n \geq n_{0} \quad a_{n} + b_{n} < - \varepsilon $. Πράγματι:
      \begin{myitemize}
        \item $ b_{n} $ άνω φραγμένη, άρα $ \exists M \in \mathbb{R} 
          \; : \; b_{n} \leq M, \; \forall n \in \mathbb{N} $
        \item $ \lim_{n \to \infty} a_{n}=-\infty \Rightarrow $ για $ \varepsilon =
          \varepsilon + \abs{M} >0 \; \exists n_{1} \in \mathbb{N} \; : \; 
          \forall n \geq n_{1} \quad a_{n} < - (\varepsilon + \abs{M})$.
      \end{myitemize}
      Επομένως για το $ n_{0} $, έχουμε ότι $ \forall n \geq n_{0}
      \quad a_{n} + b_{n} < -\varepsilon - \abs{M} + M \leq - \varepsilon $. 
    \item Επειδή $ \lim_{n \to \infty} b_{n} = b $, έχουμε ότι η $ {(b_{n})}_{n \in
      \mathbb{N}} $ είναι φραγμένη, άρα και άνω φραγμένη, οπότε απο το προηγούμενο 
      ερώτημα, έχουμε ότι $ \lim_{n \to \infty} (a_{n} + b_{n}) = -\infty $
    \item Έστω $ b>0 $. Θα δείξουμε ότι $ \lim_{n \to \infty} (a_{n}\cdot b_{n}) =
      -\infty $. Έστω $ \varepsilon >0 $. Αρκεί να βρούμε $ n_{0} \in \mathbb{N} \; : \;
      \forall n \geq n_{0} \quad a_{n}+b_{n} < - \varepsilon $. Πράγματι:
      \begin{myitemize}
        \item $ \lim_{n \to \infty} b_{n}= b $ για $ \varepsilon =
          \frac{b}{2} > 0 \; \exists n_{1} \in \mathbb{N} \; : \; \forall n \geq
          n_{1} \quad \abs{b_{n}-b} < \frac{b}{2} \Leftrightarrow 
          \underbrace{- \frac{b}{2} < b_{n}-b} < \frac{b}{2} 
          \Rightarrow b_{n} > b- \frac{b}{2} = \frac{b}{2} $
        \item $ \lim_{n \to \infty} a_{n}=-\infty \Rightarrow $ για $ \varepsilon =
          \frac{2 \varepsilon}{b} >0 \; \exists n_{2} \in \mathbb{N} \; : \; 
          \forall n \geq n_{2} \quad a_{n} < - \frac{2 \varepsilon}{b} 
          \Leftrightarrow - a_{n} > \frac{2 \varepsilon}{b} $
      \end{myitemize}
      Επομένως για το $ n_{0} = \max \{ n_{1}, n_{2} \}  $, έχουμε ότι 
      $ \forall n \geq n_{0} \quad - a_{n} \cdot b_{n} > \frac{2 \varepsilon}{b}
      \cdot \frac{b}{2} = \varepsilon \Leftrightarrow a_{n}\cdot b_{n} < 
      - \varepsilon $. Ομοίως για $ b<0 $.
  \end{enumerate}
\end{proof}


\section{Άπειρο Όριο Ακολουθίας}


\begin{mybox1}
  \begin{dfn}
    Μια ακολουθία πραγματικών αριθμών $ (a_{n})_{n \in \mathbb{N}} $ 
    \textcolor{Col1}{αποκλίνει} στο $ +\infty $ (συμβ.: 
    $ \lim_{n \to \infty} a_{n} = + 
    \infty $), αν $ \forall M>0, \; \exists n_{0} \in 
    \mathbb{N} \; : \; a_{n} > M, \; \forall n \geq n_{0}$
  \end{dfn}
\end{mybox1}

\begin{mybox1}
  \begin{dfn}
    Μια ακολουθία πραγματικών αριθμών $ (a_{n})_{n \in \mathbb{N}} $ 
    \textcolor{Col1}{αποκλίνει} στο $ -\infty $ (συμβ.: 
    $ \lim_{n \to \infty} a_{n} = - 
    \infty $), αν $ \forall M>0, \; \exists n_{0} \in 
    \mathbb{N} \; : \; a_{n} < -M, \; \forall n \geq n_{0}$
  \end{dfn}
\end{mybox1}

\begin{mybox3}
  \begin{prop}\label{prop:infzero}
    Έστω $ (a_{n})_{n \in \mathbb{N}} $ ακολουθία \textbf{θετικών} πραγματικών αριθμών. 
    Τότε 
    \[
      \lim_{n \to \infty} a_{n}= +\infty \Leftrightarrow \lim_{n \to \infty} \frac{1}{a_{n}}
      =0  
    \] 
  \end{prop}
\end{mybox3}
\begin{proof}
\item {}
  \begin{description}
    \item[$ (\Rightarrow) $] 
      Έστω $ \varepsilon >0 $.
      Επειδή $ \lim_{n \to \infty} a_{n}= +\infty $, για το 
      $ \frac{1}{\varepsilon} >0, \; \exists n_{0} \in \mathbb{N} \; : \; 
      \forall n \geq n_{0} \quad a_{n} > \frac{1}{\varepsilon} 
      \Leftrightarrow \frac{1}{a_{n}} < \varepsilon $. 
      Άρα έχουμε, ότι $ \exists n_{0} \in \mathbb{N} \; : \; \forall n
      \geq n_{0} \quad \abs{\frac{1}{a_{n}} - 0} = \abs{\frac{1}{a_{n}}} =
      \frac{1}{\abs{a_{n}}} \overset{a_{n}>0}{=} \frac{1}{a_{n}} < \varepsilon $.
      Δηλαδή $ \lim_{n \to +\infty} \frac{1}{a_{n}} =0$.
    \item [$ (\Leftarrow) $]
      Έστω $ \varepsilon >0 $.
      Επειδή $ \lim_{n \to \infty} \frac{1}{a_{n}} = 0 $, έχουμε για το  
      $ \frac{1}{\varepsilon} > 0, \; \exists n_{0} \in \mathbb{N} \; : \; 
      \forall n \geq n_{0} \quad \abs{\frac{1}{a_{n}} -0} < \frac{1}{\varepsilon} 
      \Leftrightarrow \abs{\frac{1}{a_{n}}} < \frac{1}{\varepsilon} 
      \Leftrightarrow \frac{1}{\abs{a_{n}}} < \frac{1}{\varepsilon} 
      \Leftrightarrow \frac{1}{a_{n}} < \frac{1}{\varepsilon} 
      \Leftrightarrow a_{n} > \varepsilon $. Δηλαδή, $ \lim_{n \to \infty}
      a_{n}=+\infty $.
  \end{description}
\end{proof}


\begin{example}
  Έχουμε ότι $ \lim_{n \to \infty} (n^{2}+1) = +\infty $. Άρα 
  $ \lim_{n \to \infty} \frac{1}{n^{2}+1} = 0 $
\end{example}

\begin{mybox3}
  \begin{prop}
    $ \lim_{n \to \infty} x^{n} = 
    \begin{cases} 
      0, & \abs{x} < 1 \\
      1, & x=1 \\
      +\infty, & x > 1
    \end{cases} $ 
  \end{prop}
\end{mybox3}
\begin{proof}
\item {}
  \begin{myitemize}
    \item Αν $ x>1 $ τότε $ x-1>0 $. Θέτουμε $ a = x-1>0 $, άρα $ x = 1+a 
      \Rightarrow x^{n} = (1+a)^{n} \geq 1+na, \; \; \forall n \in \mathbb{N} $.
      Επομένως 
      \[
        \lim_{n \to +\infty} x^{n} = \lim_{n \to +\infty} (1+na) 
        \overset{a>0}{=} +\infty 
      \]
    \item Αν $ x=1 $ τότε $ \lim_{n \to \infty} x^{n} = \lim_{n \to \infty} 1^{n} = 1 $
    \item Αν $ \abs{x} < 1 $ τότε: 
      \begin{enumerate}[i)]
        \item Αν $ x=0 $ τότε $ \lim_{n \to \infty} x^{n} = 
          \lim_{n \to \infty} 0^{n} = 0 $
        \item Αν $ x \neq 0 $ τότε $ \abs{x} < 1 \Rightarrow \frac{1}{\abs{x}} > 1$. 
          Επομένως, από την 1η περίπτωση
          \[
            \lim_{n \to \infty} \biggl(\frac{1}{\abs{x}}\biggr)^{n} = + \infty
          \] 
          και άρα, από την πρόταση~ref{prop:infzero} 
          $ \lim_{n \to \infty} \abs{x} ^{n} = 0 $. Όμως, ισχύει ότι
          \[
            - \abs{x} ^{n} \leq x^{n} \leq \abs{x^{n}}
          \]
          και επειδή προφανώς και $ \lim_{n \to \infty} - \abs{x} ^{n} = 0 $, από 
          το κριτήριο Παρεμβολής, έπεται ότι $ \lim_{n \to \infty} x^{n} = 0 $.
      \end{enumerate}
  \end{myitemize}
\end{proof}


\begin{mybox3}
  \begin{prop}
    $ \lim_{n \to \infty} \sqrt[n]{n} = 1, \; \forall n \in \mathbb{N}  $
  \end{prop}
\end{mybox3}

\begin{proof}
  Έστω $ n \in \mathbb{N} \Rightarrow n \geq 1, \; \forall n \in 
  \mathbb{N} \Rightarrow n ^{\frac{1}{2n}} \geq 1^{\frac{1}{2n}}, \; 
  \forall n \in \mathbb{N} \Rightarrow \sqrt[2n]{n} \geq 1, \; \forall n 
  \in \mathbb{N} $

  Θέτουμε $ a = \sqrt[2n]{n} -1 \geq 0 $. Τότε $ \sqrt[2n]{n} = 1 + a 
  \Rightarrow \sqrt{n} = (1+a)^{n} \overset{\text{Bernoulli}}{\geq} 1 
  + na \Rightarrow na \leq \sqrt{n} - 1  $. Οπότε 
  \[ 
    0 \leq a \leq \frac{\sqrt{n} -1}{n}, \; \forall n \in \mathbb{N}
  \] 
  Όμως
  \[ 
    \lim_{n \to \infty} \frac{\sqrt{n} -1}{n} = \lim_{n \to \infty} 
    \left( \frac{\sqrt{n}}{n} - \frac{1}{n}\right) = 0 - 0 = 0 
  \] 
  και άρα από το Κριτήριο Παρεμβολής, έχουμε ότι 
  \[ 
    \lim_{n \to \infty} a = 0 
    \Rightarrow \lim_{n \to \infty} \sqrt[2n]{n} = 1 \Rightarrow \lim_{n \to
    \infty} \sqrt[n]{n} = 1^{2} = 1 
  \]
\end{proof}

\begin{mybox3}
  \begin{prop}
    Έστω $ a>0 $. Τότε $ \lim_{n \to \infty} \sqrt[n]{a}=1, \; \forall n \in
    \mathbb{N} $.
  \end{prop}
\end{mybox3}

\begin{proof}
\item {}
  \begin{myitemize}[labelindent=1em]
    \item $ a>1 \Rightarrow a^{\frac{1}{n}} > 1^{\frac{1}{n}}, \; 
      \forall n \in \mathbb{N} \Rightarrow \sqrt[n]{a} >1, \; 
      \forall n \in \mathbb{N} $

      Για κάθε $ n \in \mathbb{N} $ θέτουμε $ b_{n} = \sqrt[n]{a} -1 
      > 0 \Rightarrow \sqrt[n]{a} = b_{n} + 1 \Rightarrow a = (
      b_{n}+1)^{n} \geq 1 + n b_{n} \Rightarrow n b_{n} \leq a-1
      \Rightarrow 0 \leq b_{n} \leq \frac{a-1}{n}, \; \forall n \in
      \mathbb{N} $

      Επειδή $ \lim_{n \to \infty} \frac{a-1}{n} = \lim_{n \to \infty}
      \left(\frac{a}{n} - \frac{1}{n}\right) = 0 - 0 = 0 $, άρα από το 
      Κριτήριο Παρεμβολής, έχουμε ότι $ \lim_{n \to \infty} b_{n} = 
      \lim_{n \to \infty} (\sqrt[n]{a}-1) = 0 \Rightarrow \lim_{n \to
      \infty} \sqrt[n]{a} =1 $.

    \item $ a< 1 \Rightarrow \frac{1}{a} > 1 $. Τότε από την 
      προηγούμενη πρόταση έχουμε 
      $ \lim_{n \to \infty} \sqrt[n]{\frac{1}{a} } = 1 \Rightarrow 
      \lim_{n \to \infty} \frac{1}{\sqrt[n]{\frac{1}{a}}} = 
      \frac{1}{1} = 1 \Rightarrow \lim_{n \to \infty} \sqrt[n]{a} 
      = 1$.
  \end{myitemize}
\end{proof}

\begin{mybox3}
  \begin{prop}
    $ \lim_{n \to \infty} \sqrt[n]{n!} = +\infty $ 
  \end{prop}
\end{mybox3}
\begin{proof}
  Χωρίς Απόδειξη
\end{proof}

\begin{mybox3}
  \begin{prop}
    \label{prop:monoton}
    Κάθε ακολουθία έχει \textbf{μονότονη} (αύξουσα ή φθίνουσα) υπακολουθία.
  \end{prop}
\end{mybox3}
\begin{proof}
  Χωρίς απόδειξη.
\end{proof}

\begin{mybox2}
  \begin{thm}[Bolzano-Weierstrass]{Κάθε φραγμένη ακολουθία πραγματικών αριθμών 
    έχει \textbf{συγκλίνουσα} υπακολουθία}
  \end{thm}
\end{mybox2}
\begin{proof}
  Έστω $ {(a_{n})}_{n \in \mathbb{N}} $ φραγμένη ακολουθία. Τότε
  \[
    \exists M>0 \; : \; \abs{a_{n}} \leq Μ, \; \forall n \in \mathbb{N} \Leftrightarrow 
    -M \leq a_{n} \leq M, \; \forall n \in \mathbb{N}
  \] 
  Από την πρόταση~\ref{prop:monoton} $ \exists (a_{k_{n}})_{n \in \mathbb{N}} $ 
  η οποία είναι γνησίως μονότονη, αύξουσα ή φθίνουσα. Έστω ότι 
  $ {(a_{k_{n}})}_{n \in \mathbb{N}} $ είναι γνησίως αύξουσα. Τότε για την 
  $ {(a_{k_{n}})}_{n \in \mathbb{N}} $ ισχύουν:
  \begin{myitemize}[labelindent=1em]
    \item $ \abs{a_{k_{n}}} < M, \; \forall n \in \mathbb{N} $, δηλαδή η 
      $ {(a_{k_{n}})}_{n \in \mathbb{N}} $ είναι φραγμένη.
    \item $ \abs{a_{k_{n}}} $, είναι γνησίως αύξουσα
      $ {(a_{k_{n}})}_{n \in \mathbb{N}} $ είναι φραγμένη.
  \end{myitemize}
  Επομένως, σύμωνα με την πρόταση~\ref{thm:aukssygk} 
  η $ {(a_{k_{n}})}_{n \in \mathbb{N}} $ συγκλίνει σε πραγματικό αριθμό.
\end{proof}

\section{Ακολουθίες Cauchy}

\begin{mybox1}
\begin{dfn}
  Μια ακολουϑία $ {(a_{n})}_{n \in \mathbb{N}} $ πραγματικών αριθμών καλείται 
  \textcolor{Col1}{ακολουθία Cauchy}, αν 
  $ \forall \varepsilon >0, \; \exists n_{0} \in \mathbb{N} \; : \; \forall n,m \geq
  n_{0} \quad \abs{a_{n} - a_{m}} < \varepsilon $ 
\end{dfn}
\end{mybox1}

\begin{mybox3}
  \begin{prop}\label{prop:cauchyfragm}
  Κάθε ακολουθία Cauchy είναι φραγμένη.
\end{prop}
\end{mybox3}
\begin{proof}
  Έστω η $ {(a_{n})}_{n \in \mathbb{N}} $ είναι μια ακολουθία Cauchy. Τότε για 
  $ \varepsilon = 1>0$ 
  \[
    \; \exists n_{0} \in \mathbb{N} \; : \; \forall n,m \geq n_{0}
    \quad \abs{a_{n}-a_{m}} < 1 
  \] 
  Συνεπώς, για $ m = n_{0} $ και $ \forall n \geq n_{0} $, έχουμε 
  $ \inlineequation[eq:cauchy1]{\abs{a_{n}- a_{n_{0}}} < 1} $. 
  Άρα, $ \forall n \geq n_{0} $, έχουμε
  \[
    \abs{a_{n}} = \abs{a_{n} + a_{n_{0}}- a_{n_{0}}} \leq \abs{a_{n}- a_{n_{0}}} + 
    \abs{a_{n_{0}}} < 1 + \abs{a_{n_{0}}}
   \] 
   Θέτουμε $ M = \max \{ 1+ \abs{a_{n_{0}}} , \abs{a_{1}} , \abs{a_{2}} , \ldots,
   \abs{a_{n_{0}-1}} \} $. Τότε, $ \forall n \geq n_{0} $, έχουμε ότι 
   $ \abs{a_{n}} \leq M $, δηλαδή, η ακολουθία $ {(a_{n})}_{n \in \mathbb{N}} $ 
   είναι φραγμένη.
\end{proof}

\begin{mybox3}
\begin{prop}
  Μια ακολουθία $ {(a_{n})}_{n \in \mathbb{N}} $ είναι Cauchy $ \Leftrightarrow $ 
  $ {(a_{n})}_{n \in \mathbb{N}} $ συγκλίνει σε πραγματικό αριθμό.
\end{prop}
\end{mybox3}
\begin{proof}
\item {}
  \begin{description}
    \item [($ \Leftarrow $)] Έστω $ \lim_{n \to \infty} a_{n} = a \in \mathbb{R} $. 
      Θα δέιξουμε ότι η ακολουθία $ {(a_{n})}_{n \in \mathbb{N}} $ είναι Cauchy. 

      Έστω $ \varepsilon >0 $. Επειδή $ \lim_{n \to \infty} a_{n} = a $, για 
      $ \frac{\varepsilon}{2} > 0 $,
      \[ 
        \exists n_{0} \in \mathbb{N} \; : \; \forall n,m \geq n_{0} \quad 
        \abs{a_{n} - a_{m}} = 
        \abs{a_{n} - a + a - a_{m}} \leq \abs{a_{n}-a} + \abs{a_{m}-a} < 
        \frac{\varepsilon}{2} + \frac{\varepsilon}{2} = \varepsilon 
      \]
    \item [($ \Rightarrow $)] Έστω $ {(a_{n})}_{n \in \mathbb{N}} $ ακολουθία Cauchy. 
      Θα δείξουμε ότι η $ {(a_{n})}_{n \in \mathbb{N}} $ συγκλίνει σε πραγματικό
      αριθμό.Πράγματι, 
      Επειδή η $ {(a_{n})}_{n \in \mathbb{N}} $ είναι Cauchy, από την
      πρόταση~\ref{prop:cauchyfragm}, είναι φραγμένη. Επομένως, από το θεώρημα 
      Bolzano-Weierstrass υπάρχει ακολουθία $ (a_{k_{n}})_{n \in \mathbb{N}} $ 
      που συγκλίνει σε πραγματικό αριθμό, έστω $ a \in \mathbb{R} $. Θα δέιξουμε ότι 
      $ \lim_{n \to \infty} a_{n} = a $. Έστω $ \varepsilon >0 $. Επειδή $ \lim_{n \to
      \infty} a_{k_{n}} = a $, για $ \frac{\varepsilon}{2} > 0 $, 
      \[
        \exists n_{1} \in \mathbb{N} \; : \; \forall n \geq n_{1} \quad 
        \abs{a_{k_{n}}-a} < \frac{\varepsilon}{2}
      \] 
      Επειδή η $ {(a_{n})}_{n \in \mathbb{N}} $ είναι Cauchy, για $
      \frac{\varepsilon}{2} >0 $, 
      \[
        \exists n_{2} \in \mathbb{N} \; : \; \forall n,m \geq n_{2} \quad 
        \abs{a_{n}-a_{m}} < \frac{\varepsilon}{2}
      \] 
      Αν θέσουμε $ n_{0} = \max \{ n_{1}, n_{2} \} $, έχουμε ότι $ \forall n \geq n_{0}
      $, 
      \[
        \abs{a_{n}-a} = \abs{a_{n}-a_{k_{n}} + a_{k_{n}} - a} \leq \abs{a_{n}-
        a_{k_{n}}} + \abs{a_{k_{n}} - a} < \frac{\varepsilon}{2} + \frac{\varepsilon}{2}
        = \varepsilon 
      \] 
  \end{description}
\end{proof}

\end{document}
