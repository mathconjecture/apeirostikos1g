\input{preamble_ask.tex}
\input{definitions_ask.tex}

\thispagestyle{empty}

\begin{document}

\begin{center}
  \minibox{\large \bfseries \textcolor{Col1}{Θέματα Απειροστικός Ι}}
\end{center}

\vspace{\baselineskip}

\section*{2019}

\begin{enumerate}
    \item 
        \begin{enumerate}[i)]
            \item Έστω $A$, $B$ δύο μη κενά και άνω φραγμένα υποσύνολα των 
                πραγματικών αριθμών.  Να αποδείξετε ότι 
                \[
                    \sup (A \cup B) = \max \{ \sup A, \sup B \} 
                \] 
            \item Να αποδείξετε ότι 
                \[
                    \sum_{n=2}^{N-2} \frac{1}{(n+1)(n+2)} = \frac{1}{3} - \frac{1}{N}, 
                    \quad N \in \mathbb{N}, \; N \geq 4
                \] 
        \end{enumerate}

    \item 
        \begin{enumerate}[i)]
            \item Έστω $ {(a_{n})}_{n \in \mathbb{N}} $ και 
                $ {(b_{n})}_{n \in \mathbb{N}} $ δυο συγκλίνουσες ακολουθίες 
                πραγματικών αριθμών, τέτοιες ώστε $ a_{n} < b_{n}, \; \forall n 
                \in \mathbb{N} $. Να δείξετε με τη χρήση του 
                ορισμού της σύγκλισης ακολουθίας ότι $ \lim_{n \to \infty} a_{n} 
                \leq \lim_{n \to \infty} b_{n}$.

            \item Να δώσετε παράδειγμα συγκλινουσών ακολουθιών 
                $ {(a_{n})}_{n \in \mathbb{N}} $ και
                $ {(b_{n})}_{n \in \mathbb{N}} $ για τις οποίες ισχύουν οι 
                ανισώσεις 
                \[
                    a_{n} < b_{n}, \; \forall n \in \mathbb{N} \quad \text{και} \quad 
                    \lim_{n \to \infty} a_{n} = \lim_{n \to \infty} b_{n}
                \] 
        \end{enumerate}

    \item 
        \begin{enumerate}[i)]
            \item Να υπολογιστούν τα όρια των ακολουθιών:
                \begin{enumerate*}[i), ]
                    \item $ \lim_{n \to \infty} \sqrt[n]{\frac{1}{2^{n}} + 
                        \frac{1}{3^{n}}} $
                    \item $ \lim_{n \to \infty} \frac{n^{3}}{3^{n}} $
                \end{enumerate*}

            \item Να εξεταστούν οι παρακάτω σειρές ως προς τη σύγκλιση.
                \begin{enumerate*}[i), ]
                    \item $ \sum_{n=1}^{\infty} \frac{5^{n}-n}{6^{n} +n} $
                    \item $ \sum_{n=1}^{\infty} \frac{\cos{(n^{2})}}{n^{2}}$
                \end{enumerate*}
        \end{enumerate}

    \item Να υπολογιστούν τα όρια, αν υπάρχουν.
        \begin{enumerate*}[i), ]
            \item $ \lim_{x \to 0} \frac{e^{2018x} - 1}{x^{2018}} $
            \item $ \lim_{x \to \infty} (\cos{x} + \sin{x} ) $
        \end{enumerate*}


    \item Δίνεται η συνάρτηση $ f \colon \mathbb{R} \to \mathbb{R} $ με τύπο:
        \[
            f(x) = 
            \begin{cases} 
                e^{x-1}+1, & x \leq 1 \\ 
                x^{2}+1, & x >1 
            \end{cases}  
        \] 
        Να εξετάσετε αν η $f$ είναι μονότονη και να βρείτε την $ f^{-1} $, αν 
        υπάρχει.
\end{enumerate}

\section*{2019}


\begin{enumerate}
    \item 
    \begin{enumerate}[i)]
        \item Έστω $ A = \mathbb{Q} \cup (0,3) \subseteq \mathbb{R} $. Να 
            αποδειχθεί ότι $ \sup A = 3 $ και $ \inf A = 0 $
        \item Να βρεθεί φυσικός αριθμός $N \in \mathbb{N} $ τέτοιος ώστε 
            \[
                \sum_{n=1}^{N} \frac{1}{2^{n}} > \frac{2018}{2019} 
            \] 
    \end{enumerate}
   
 
\item Να υπολογιστούν τα όρια των ακολουθιών 
    \begin{enumerate}[i)]
        \begin{enumerate*}[i)]
            \item $ \lim_{n \to \infty} \sqrt[n]{\frac{1}{3^{n}}+ \frac{1}{4^{n}}} + 
                \frac{1}{5^{n}} $
            \item $ \lim_{n \to \infty} \frac{n^{4}+5n-6}{2^{n}} $
        \end{enumerate*}
    \end{enumerate}
    
    \item Να εξετάσετε τις σειρές ως προς τη σύγκλιση    
        \begin{enumerate*}[i)]
            \item $ \sum_{n=1}^{\infty} \frac{2^{2n}+1}{4^{n}} $
            \item $ \sum_{n=2}^{\infty} \frac{n^{3}+8}{n^{5}-6} $
        \end{enumerate*}

    \item
        \begin{enumerate}[i)]
            \item Αν μια ακολουθία έχει δυο υπακολουθίες που συγκλίνουν σε διαφορετικά 
                όρια μπορεί η ίδια να συγκλίνει;
            \item Αν μια ακολουθία έχει δυο υπακολουθίες που συγκλίνουν 
                στο ίδιο όριο είναι αλήθεια ότι η ίδια συγκλίνει;
            \item Αν για μια ακολουθία ισχύει $ a_{n} = a_{n+3} $ και η 
                $ {(a_{n})}_{n \in \mathbb{N}} $ συγκλίνει, να αποδείξετε ότι 
                είναι σταθερή.
        \end{enumerate}

    \item 
        \begin{enumerate}[i)]
            \item Έστω $ f \colon \mathbb{R} \to \mathbb{R} $ συνεχής με 
                $ \lim_{x \to \infty} f(x) = \infty$ και $ \lim_{x \to - \infty} f(x) 
                = - \infty $. Να αποδείξετε ότι η $f$ είναι επί.

            \item Έστω ότι $ f \colon \mathbb{R} \to \mathbb{R} $ με $ f(\frac{1}{n}) 
                = (-1)^{n} $. Να δείξετε ότι η $f$ δεν είναι συνεχής στο 0.
        \end{enumerate}

    \item Να εξετάσετε ως προς την παραγωγισιμότητα τη συνάρτηση
        \[
            f(x) = 
            \begin{cases} 
                x^{2}, & x \in \mathbb{Q} \\ 
                x^{4}, & x \not \in \mathbb{Q} 
            \end{cases} 
         \] 
\end{enumerate}
\end{document}
