\documentclass{book}

\usepackage{etex}
\usepackage{etoolbox}

%%%%%%%%%%%%%%%%%%%%%%%%%%%%%%%%%%%%%%
% Babel language package
%\usepackage[english,greek]{babel}
% Inputenc font encoding
%\usepackage[utf8]{inputenc}


% \usepackage{xltxtra} 
% \usepackage{xgreek} 
% \setmainfont[Mapping=tex-text]{GFS Didot} 

%\usepackage{kmath,kerkis} % The order of the packages matters; kmath changes the default text font
%\usepackage[T1]{fontenc}
\usepackage{ifxetex}
\ifxetex
    % IF XELATEX

% MINION new font
\usepackage{fontspec}
%\usepackage{mathspec}

\setmainfont[Extension=.ttf,UprightFont=*-Regular,BoldFont=*-Bold,ItalicFont=*-Italic,BoldItalicFont=*-Bold-Italic]{Minion-Pro}
\setsansfont[Extension=.ttf,UprightFont=*H,BoldFont=*HB,ItalicFont=*HI,BoldItalicFont=*HBI]{Vera}
\usepackage{unicode-math}
\setmathfont{latinmodern-math.otf}
%\setmathfont[range=\mathup]{Minion-Pro-Regular.ttf}
%\setmathfont[range=\mathit]{Minion-Pro-Italic.ttf}
%\setmathfont[range=\mathbf]{Minion-Pro-Bold.ttf}
%\setmathfont[range={0222B}]{Minion-Pro-Italic.ttf}
%\setmathfontface\mathfoo{Minion-Pro-Regular.ttf}
%\setoperatorfont\mathfoo
    \usepackage{polyglossia}
    \setdefaultlanguage{greek}
    \setotherlanguage{english}
    %\RequirePackage{unicode-math}
    %\setmathfont{Latin Modern Math}
    %\newcommand{\smblkcircle}{•}
\else
    % IF PDFLATEX
    %\usepackage{tgheros}
    %\renewcommand*\familydefault{\sfdefault}
    %\usepackage[eulergreek]{sansmath}
    %\sansmath
    \usepackage[T1]{fontenc}
    \usepackage[utf8]{inputenc}
    \usepackage[english,greek]{babel}
\fi


\usepackage{anyfontsize}
\newlength{\FONTmain}\setlength{\FONTmain}{9pt}
\newlength{\FONTmainbl}\setlength{\FONTmainbl}{1.2\FONTmain}
\renewcommand{\tiny}        {\fontsize{0.6\FONTmain}{0.6\FONTmainbl}\selectfont}
\renewcommand{\scriptsize}  {\fontsize{0.7\FONTmain}{0.7\FONTmainbl}\selectfont}
\renewcommand{\footnotesize}{\fontsize{0.8\FONTmain}{0.8\FONTmainbl}\selectfont}
\renewcommand{\small}       {\fontsize{0.9\FONTmain}{0.9\FONTmainbl}\selectfont}
\renewcommand{\normalsize}  {\fontsize{1.0\FONTmain}{1.0\FONTmainbl}\selectfont}
\renewcommand{\large}       {\fontsize{1.2\FONTmain}{1.2\FONTmainbl}\selectfont}
\renewcommand{\Large}       {\fontsize{1.4\FONTmain}{1.4\FONTmainbl}\selectfont}
\renewcommand{\LARGE}       {\fontsize{1.6\FONTmain}{1.6\FONTmainbl}\selectfont}
\renewcommand{\huge}        {\fontsize{1.8\FONTmain}{1.8\FONTmainbl}\selectfont}

%%%%%%%%%%%%%%%%%%%%%%%%%%%%%%%%%%%%%%
\usepackage[table,RGB]{xcolor}

\usepackage{geometry}
\geometry{a5paper,top=15mm,bottom=15mm,left=15mm,right=15mm}
\setlength{\parindent}{0pt}


%\usepackage{extsizes}
\usepackage{multicol}

%%%%% math packages %%%%%%%%%%%%%%%%%%
\usepackage[intlimits]{amsmath}
\usepackage{amssymb}
\usepackage{amsfonts}
\usepackage{amsthm}
\usepackage{proof}
\usepackage{mathtools}
\usepackage{extarrows}

\usepackage[italicdiff]{physics}
\usepackage{siunitx}
\usepackage{xfrac}

%%%%%%% symbols packages %%%%%%%%%%%%%%
\usepackage{bm} %for use \bm instead \boldsymbol in math mode
\usepackage{dsfont}
%\usepackage{stmaryrd}
%%%%%%%%%%%%%%%%%%%%%%%%%%%%%%%%%%%%%%%


%%%%%% graphics %%%%%%%%%%%%%%%%%%%%%%%
\usepackage{graphicx}
%\usepackage{color}
%\usepackage{xypic}
%\usepackage[all]{xy}
%\usepackage{calc}

%%%%%% tables %%%%%%%%%%%%%%%%%%%%%%%%%
\usepackage{array}
\usepackage{booktabs}
\usepackage{multirow}
\usepackage{makecell}
\usepackage{minibox}
\usepackage{systeme}
%%%%%%%%%%%%%%%%%%%%%%%%%%%%%%%%%%%%%%%

\usepackage{enumitem}
\usepackage{tikz}
\usetikzlibrary{shapes,angles,calc,arrows,arrows.meta,quotes,intersections}
\usetikzlibrary{decorations.pathmorphing}
\usetikzlibrary{decorations.pathreplacing} 
\usetikzlibrary{decorations.markings,patterns} 
\usepackage{pgfplots}
\pgfplotsset{compat=1.15}

\tikzset{dot/.style={ draw, fill, circle, inner sep=1pt, minimum size=3pt }}
\usepackage{fancyhdr}
%%%%% header and footer rule %%%%%%%%%
%\setlength{\headheight}{14pt}
\renewcommand{\headrulewidth}{0pt}
\renewcommand{\footrulewidth}{0pt}
\fancypagestyle{plain}{\fancyhf{}\rfoot{\thepage}}
\fancypagestyle{vangelis}{\fancyhf{}
    \fancyfootoffset[LE,RO]{10mm}
    \rfoot[]{\thepage}
    \lfoot[\thepage]{}
    \rhead[]{\tikz[remember picture,overlay]{\node[rotate=90,anchor=east] (text) at ([shift={(-5mm,-8mm)}]current page.north east) {\textcolor{Col\thechapter}{\small\strut\leftmark}};
    \fill[Col\thechapter] ([xshift={-2.5mm}]text.east) rectangle++(5mm,5mm);
    }}
    %\lhead[\textcolor{Col\thechapter}{\leftmark}]{}
}
%%%%%%%%%%%%%%%%%%%%%%%%%%%%%%%%%%%%%%%

\usepackage[space]{grffile}


% \definecolor{Col1}{HTML}{eb3b79}
% \definecolor{Col2}{HTML}{9a529f}
% \definecolor{Col3}{HTML}{775ba6}
% \definecolor{Col4}{HTML}{5a68b0}
% \definecolor{Col5}{HTML}{55a0d8}
% \definecolor{Col6}{HTML}{34b0e5}
% \definecolor{Col7}{HTML}{34c1d7}
% \definecolor{Col8}{HTML}{65bc6a}
% \definecolor{Col9}{HTML}{9acb62}
% \definecolor{Col10}{HTML}{d1dd5b}
% \definecolor{Col11}{HTML}{f9ec5d}
% \definecolor{Col12}{HTML}{fbc82a}
% \definecolor{Col13}{HTML}{faa725}
% \definecolor{Col14}{HTML}{f26f47}
% \definecolor{Col15}{HTML}{8e6d65}
% \definecolor{Col16}{HTML}{bdbcbc}
% \definecolor{Col17}{HTML}{79919d}

\definecolor{Col1}{rgp}{0.74, 0.2, 0.64}
\definecolor{Col2}{rgp}{0.0, 0.55, 0.55}
\definecolor{Col3}{rgp}{0.74, 0.2, 0.64}
\definecolor{Col4}{rgp}{0.0, 0.55, 0.55}
\definecolor{Col5}{rgp}{0.74, 0.2, 0.64}
\definecolor{Col6}{rgp}{0.0, 0.55, 0.55}
\definecolor{Col7}{rgp}{0.74, 0.2, 0.64}
\definecolor{Col8}{rgp}{0.0, 0.55, 0.55}
\definecolor{Col9}{rgp}{0.74, 0.2, 0.64}
\definecolor{Col10}{rgp}{0.0, 0.55, 0.55}

\everymath{\displaystyle}

\usepackage[most]{tcolorbox}

\usepackage[explicit]{titlesec}
%%%%%% titlesec settings %%%%%%%%%%%%%
% \titleformat{ command }[ shape ]{ format }{ label }{ sep }{ before-code }[ after-code 
% \titlespacing*{ command }{ left }{ before-sep }{ after-sep }[ right-sep ]
% Chapter


% \titleformat{\chapter}[block]{\huge\bfseries}{\begin{tcolorbox}[colback=Col\thechapter,left=3pt,right=3pt,top=18pt,bottom=18pt,sharp
% corners,boxrule=0pt]\centering\huge\bfseries\textcolor{white}{#1}\end{tcolorbox}}{0pt}{\markboth{#1}}[\clearpage]
% \titlespacing*{\chapter}{0cm}{6\baselineskip}{0\baselineskip}[0ex]
% % Section
% \titleformat{\section}[hang]{\pagestyle{plain}\Large\bfseries\centering}{\begin{tcolorbox}[colback=Col\thechapter!75!white,left=1pt,right=1pt,top=2pt,bottom=2pt,sharp
% corners,boxrule=0pt]\centering\strut\textcolor{white}{#1}\end{tcolorbox}}{0ex}{}
% \titlespacing*{\section}{0cm}{2\baselineskip}{\baselineskip}[0ex]
% % subsection
% \titleformat{\subsection}[hang]{\pagestyle{plain}\large\bfseries\centering}{\begin{tcolorbox}[colback=Col\thechapter!55!white,left=1pt,right=1pt,top=2pt,bottom=2pt,sharp
% corners,boxrule=0pt]\centering\strut\textcolor{white}{#1}\end{tcolorbox}}{0ex}{}
% \titlespacing*{\section}{0cm}{2\baselineskip}{\baselineskip}[0ex]
% % Subsubsection
% \titleformat{\subsubsection}[hang]{\normalsize\bfseries\centering}{}{0ex}{\color{Col\thechapter!45}{#1}}{}
% \titlespacing*{\subsubsection}{0cm}{\baselineskip}{\baselineskip}[0ex]

%% Subsection
%\titleformat{\subsection}[hang]{\large\bfseries\centering}{\tcbox[colback=Col\thechapter!50!white,left=1pt,right=1pt,top=1pt,bottom=1pt,sharp corners]{#1}}{0ex}{}
%\titlespacing*{\subsection}{0cm}{2\baselineskip}{\baselineskip}[0ex]
%% Subsubsection
%\titleformat{\subsubsection}[hang]{\normalsize\bfseries\centering}{}{0ex}{\color{Col\thechapter}{#1}}{}
%\titlespacing*{\subsubsection}{0cm}{\baselineskip}{\baselineskip}[0ex]
%%%%%%%%%%%%%%%%%%%%%%%%%%%%%%%%%%%%%%%




\AtBeginDocument{\pagestyle{vangelis}\normalsize\raggedright}


\newcommand{\twocolumnside}[2]{\begin{minipage}[t]{0.45\linewidth}\raggedright
#1
\end{minipage}\hfill{\color{Col\thechapter}{\vrule width 1pt}}\hfill\begin{minipage}[t]{0.45\linewidth}\raggedright
#2
\end{minipage}
}

\newcommand{\twocolumnsides}[2]{\begin{minipage}[t]{0.45\linewidth}\raggedright
#1
\end{minipage}\hfill\begin{minipage}[t]{0.45\linewidth}\raggedright
#2
\end{minipage}
}

\newcommand{\twocolumnsidesc}[2]{\begin{minipage}{0.45\linewidth}\raggedright
#1
\end{minipage}\hfill\begin{minipage}[c]{0.45\linewidth}\raggedright
#2
\end{minipage}
}

\newcommand{\twocolumnsidesp}[2]{\begin{minipage}[t]{0.35\linewidth}\raggedright
#1
\end{minipage}\hfill\begin{minipage}[t]{0.55\linewidth}\raggedright
#2
\end{minipage}
}

\newcommand{\twocolumnsidesl}[2]{\begin{minipage}[t]{0.55\linewidth}\raggedright
#1
\end{minipage}\hfill\begin{minipage}[t]{0.35\linewidth}\raggedright
#2
\end{minipage}
}


\usepackage{calc}
\usepackage{array}
\definecolor{TabLine}{RGB}{254,254,254}
\newcommand{\TabRowHead}{\rowcolor{TabHeadRow}}
\newcommand{\TabRowHeadCor}{\cellcolor{white}}
\newcommand{\TabRowHCol}{\color{white}\bfseries\boldmath}
\newcommand{\TabCellHead}{\cellcolor{TabHeadRow}\TabRowHCol}
\newenvironment{Mytable}%
    {\begingroup\setlength{\arrayrulewidth}{2pt}\arrayrulecolor{TabLine}
    \colorlet{TabHeadRow}{Col\thechapter}
    \colorlet{TabRowOdd}{Col\thechapter!50!white}
    \colorlet{TabRowEven}{Col\thechapter!25!white}
    \rowcolors{1}{TabRowOdd}{TabRowEven}
    }%
    {\endgroup

}

\usepackage{fancyhdr}
%%%%% header and footer rule %%%%%%%%%
\setlength{\headheight}{14pt}
\renewcommand{\headrulewidth}{0pt}
\renewcommand{\footrulewidth}{0pt}
\fancypagestyle{plain}{\fancyhf{}
\fancyhead{}
\lfoot{\small \hrule \vspace{5pt}\color{Col1} Βαγγέλης Σαπουνάκης}
\cfoot{\small \hrule \vspace{5pt}\color{Col2!75} Φοιτητικό Πρόσημο}
\rfoot{\small \hrule \vspace{5pt} \thepage}}
\fancypagestyle{vangelis}{\fancyhf{}
\lfoot{\small \hrule \vspace{5pt}\color{Col1} Βαγγέλης Σαπουνάκης}
\cfoot{\small \hrule \vspace{5pt}\color{Col2!75} Φοιτητικό Πρόσημο}
\rfoot{\small \hrule \vspace{5pt} \thepage}}

%%%%%%%%%%%%Watermark%%%%%%%%%%%%%%%%%%
 \usepackage[printwatermark]{xwatermark} 
 \newwatermark[allpages,color=blue!8,angle=45,scale=3,xpos=0,ypos=0]{ΠΡΟΣΗΜΟ}
%%%%%%%%%%%%%%%%%%%%%%%%%%%%%%%%%%%%%%

\input{definitions_ask.tex}

\pagestyle{askhseis}
\everymath{\displaystyle}

\begin{document}

\begin{center}
  \minibox[c]{\large \bfseries \textcolor{Col1}{Πραγματικοί Αριθμοί} \\ \large
  \textcolor{Col1}{Ασκήσεις}}
\end{center}

\vspace{\baselineskip}


\begin{enumerate}
  \item Να υπολογιστούν τα $ \sup $, $ \inf $, $ \max $ και $ \min $ των 
    παρακάτω συνόλων.
    \begin{enumerate}[i)]
      \renewcommand{\arraystretch}{1.3}
    \item $ A = \left\{ \frac{1}{n} \; : \; n \in \mathbb{N} \right\} $ 
      \hfill Απ: $\left\{\begin{tabular}{l} $ \sup A = \max A = 1 $ \\
      $ \inf A = 0 $   \end{tabular}\right.$
    \item $ A = \left\{ \frac{1}{n} \; : \; n \in \mathbb{Z}, \; n \neq 0
      \right\} $ 
      \hfill Απ: $\left\{\begin{tabular}{l} $ \sup A = \max A = 1 $ \\
      $ \inf A = \min A = -1 $ \end{tabular}\right.$
    \item $ A = \left\{ x \; : \; x=0 \; \text{ή} \;
      x = \frac{1}{n}, \; n \in \mathbb{N} \right\} $
      \hfill Απ: $\left\{\begin{tabular}{l} $ \sup A = \max A = 1 $ \\
      $ \inf A = \min A = 0 $ \end{tabular}\right.$
    \item $ A = \left\{ x \in \mathbb{R} \; : \; 0 \leq x \leq \sqrt{2}, \; 
      x \in \mathbb{Q}  \right\}  $ 
      \hfill Απ: $\left\{\begin{tabular}{l} $ \sup A = \sqrt{2} $ \\
      $ \inf A = \min A = 0 $  \end{tabular}\right.$
    \item $ A = \left\{ x \in \mathbb{R} \; : \; x^{2} + x + 1 \geq 0 
      \right\} $
      \hfill Απ: δεν υπάρχουν 
    \item $ A = \{ x \in \mathbb{R} \; : \; x^{2} + x - 1 < 0 \} $
      \hfill Απ: $\left\{\begin{tabular}{l} $ \sup A = 
          \frac{-1 + \sqrt{5}}{2} $ \\ 
      $ \inf A = \frac{-1 - \sqrt{5}}{2} $ \end{tabular}\right.$
    \item $ A = \{ x \in \mathbb{R} \; : \; x<0, \; x^{2} + x - 1 < 0 \} $
      \hfill Απ: $\left\{\begin{tabular}{l} $ \sup A = 0 $ \\
      $ \inf A = \frac{-1 - \sqrt{5}}{2} $ \end{tabular}\right.$
    \item $A = \left\{ 1 - \frac{(-1)^{n}}{n} \; : \; n \in \mathbb{N}
      \right\}  $
      \hfill Απ: $\left\{\begin{tabular}{l} $ \sup A = \max A = 2 $ \\
      $ \inf A = \min A = \frac{1}{2}  $\end{tabular}\right.$
  \end{enumerate}

\item Για τα παρακάτω υποσύνολα να υπολογιστούν και να αποδειχθούν τα supremum και 
  infimum.
  \begin{enumerate}[i)]
    \renewcommand{\arraystretch}{1.3}
  \item $ A = (0,2) $ \hfill Απ: $ \left\{ \begin{tabular}{l}
        $\sup A = 2$ \\
        $\inf A = 0$
    \end{tabular}\right. $
  \item $ A = \{ x \in \mathbb{R} \; : \; x<0 \} $ 
    \hfill Απ: $ \left\{ \begin{tabular}{l}
        $ \sup A = 0 $ \\
        $ \inf A = - \infty  $
    \end{tabular} \right.$ 
  % \item $ A = \left\{ \frac{1}{n} \; : \; n \in \mathbb{N} \right\} $ 
  %   \hfill Απ: $\left\{\begin{tabular}{l} $ \sup A = \max A = 1 $ \\
  %   $ \inf A = 0 $   \end{tabular}\right.$
  % \item $ A = \left\{ \frac{n}{n+1} \; : \; n \in \mathbb{N} \right\} $
  %   \hfill Απ: $\left\{\begin{tabular}{l} $ \sup A = 1 $ \\
  %   $ \inf A = \min A = \frac{1}{2} $\end{tabular}\right.$
  % \item $ A = \left\{ \frac{1}{n} + (-1)^{n} \; : \; n \in \mathbb{N} 
  %   \right\} $
  %   \hfill Απ: $\left\{\begin{tabular}{l} $ \sup A = \max 
  %       A = \frac{3}{2}  $ \\
  %   $ \inf A = -1 $ \end{tabular}\right.$
  % \item $ A = \left\{ \frac{1}{n} - \frac{1}{m} \; : \; n,m \in \mathbb{N} 
  %   \right\} $
  %   \hfill Απ: $\left\{\begin{tabular}{l} $ \sup A = 1 $ \\ $ \inf A = -1 $ 
  %   \end{tabular}\right.$
\end{enumerate}

    \item \label{ask:monosynolo} Έστω $A$ φραγμένο υποσύνολο του $ \mathbb{R} $ 
      τέτοιο ώστε $ \sup A = \inf A $. Να δείξετε ότι το $ A $ είναι μονοσύνολο.

    \item Έστω $ A, B $ μη-κενά υποσύνολα του $ \mathbb{R} $ τέτοια ώστε 
      $ a \leq b, \; \forall a \in A $ και $ \forall b \in B $.
      Να δείξετε ότι:
      \begin{enumerate*}[label=\roman*)]
        \item $ \sup A \leq b, \;  \forall b \in B $
        \item $ \sup A \leq \inf B $
      \end{enumerate*}

    \item Έστω $ A \subseteq \mathbb{R} $ μη-κενό, και κάτω φραγμένο και έστω 
      $ B $ το σύνολο των κάτω φραγμάτων του $A$. Να δείξετε ότι:
      \begin{enumerate*}[i)]
        \item $ B \neq \emptyset $
        \item $B$ άνω φραγμένο.
        \item $ \sup B = \inf A $
      \end{enumerate*}
      
    \item Να αποδείξετε ότι το σύνολο $ \mathbb{Z} $ δεν είναι φραγμένο.

    \item \label{ask:3z} Να αποδείξετε ότι το σύνολο 
      $ A = \{ 3k \; : \; k \in \mathbb{Z} \} $ δεν είναι άνω φραγμένο.

    \item Να αποδείξετε με χρήση της Μαθηματικής Επαγωγής τους παρακάτω τύπους.
      \label{eq:epagsums}
      \begin{enumerate}[i)]
        \item $ \sum_{n=1}^{N} n = \frac{N(N+1)}{2},\; \forall N \in
          \mathbb{N} $
        \item $ \sum_{n=1}^{N} n^{2} = \frac{N(N+1)(2N+1)}{6},\; \forall 
          N \in \mathbb{N} $
        \item $ \sum_{n=1}^{N} n^{3} = (1+2+\cdots + N)^{2}, \; 
          \forall N \in \mathbb{N} $
        \item $ \sum_{n=0}^{N} a^{n} = \frac{a^{N+1} - 1}{a-1},\; 
          \forall N \in \mathbb{N}$
      \end{enumerate}

    \item \label{ask:sums} Βρείτε ένα κλειστό τύπο για τα παρακάτω 
      αθροίσματα: 
      \begin{enumerate}[i)]
        \item $ \sum_{k=1}^{n} (2k-1) = 1 + 3 + 5 + \cdots + (2n-1)  
          $ \hfill Απ: $ n^{2} $ 

        \item $ \sum_{k=1}^{n} (2k-1)^{2} = 1^{2} + 3^{2} + 5^{2} + \cdots 
          + (2n-1)^{2}  $ \hfill Απ: $ \frac{n(2n+1)(2n-1)}{3} $ 
      \end{enumerate}

    \item \label{ask:thema18sum} ({\bfseries Θέμα: 2018}) Να αποδείξετε ότι 
      $ \sum_{n=2}^{N-2} \frac{1}{(n+1)(n+2)} = \frac{1}{3} - \frac{1}{N} $

    \item Να αποδείξετε ότι $ n^{5} - n $ είναι πολλαπλάσιο του 5,
      $ \forall n \in \mathbb{N} $.

    \item Να αποδείξετε ότι $ n! > 2^{n}, \forall n \geq 4 $

    \item Έστω $ a \in \mathbb{R} $ και $ n \in \mathbb{N} $. Να δείξετε
      με τη βοήθεια της μαθηματικής επαγωγής της παρακάτω ανισότητες.
      \begin{enumerate}[i)]
        \item Αν $ 0<a< \frac{1}{n} $ τότε $ (1+a)^{n} < \frac{1}{1-na} $
        \item Αν $ 0 \leq a \leq 1$  τότε $ 1-na \leq (1-a)^{n} \leq
          \frac{1}{1+na} $
      \end{enumerate}

    \item Αν $a > 0$ τότε να αποδείξετε ότι $ (1+a)^{n} \geq 1 + na + 
      \frac{n(n-1)a^{2}}{2},\; \forall n \in \mathbb{N}   $ 

    \item Να δείξετε ότι οι παρακάτω αριθμοί είναι άρρητοι.
      \begin{enumerate*}[i)]
        \item $ \sqrt{3} $ και  $ \sqrt{5} $
        \item $ \sqrt[3]{2} $ και $ \sqrt[3]{3} $
        \item $ \sqrt{2} + \sqrt{3} $ και $ \sqrt{2} + \sqrt{6} $ 
      \end{enumerate*}
  \end{enumerate}

  \vspace{\baselineskip}

  \section*{Υποδείξεις} 

  \subsection*{Άθροισμα $n$ πρώτων όρων αριθμητικής προόδου}
  Έστω αριθμητική πρόοδος $a_{n} = a_{1} + (n-1)\omega, \quad \forall n \in \mathbb{N}$.
  Τότε το άθροισμα των $n$ πρώτων όρων της ακολουθίας είναι
  \[
    S_{n} = a_{1} + a_{2} + \cdots + a_{n} = \frac{(a_{1} + a_{n})n}{2} =
    \frac{2a_{1}+(n-1)\omega}{2}
  \]
  \subsection*{Άθροισμα $n$ πρώτων όρων γεωμετρικής προόδου}
  Έστω γεωμετρική πρόοδος $ a_{n} = a r^{n-1}, \quad \forall n \in \mathbb{N} $.
  Τότε το άθροισμα των $n$ πρώτων όρων της ακολουθίας είναι
  \[
    S_{n} = \sum_{k=1}^{n} ar^{k-1} = a + ar + ar^{2} + \cdots + ar^{n-1} = 
    a_{1}\frac{1 - r^{n}}{1-r}, \quad r \neq 1 
  \] 

  \begin{description}
    \item [Άσκηση \ref{ask:monosynolo}] Υποθέστε ότι $ \sup A = \inf A = c
      $. Ποιές ανισότητες ισχύουν για το $c$ και τα στοιχεία του $A$; 
      Τι συμπέρασμα βγάζουμε;

    \item [Άσκηση \ref{ask:sums}] Προσθαφαιρέστε τους κατάλληλους κάθε φορά  
      άρτιους όρους που λείπουν από το άθροισμα και ομαδοποιήστε έτσι ώστε να 
      μπορέσετε να χρησιμοποιήσετε τους τύπους αθροίσματος αριθμητικής προόδου 
      καθώς και τα αθροίσματα της άσκησης~\ref{eq:epagsums}.

    \item [Άσκηση \ref{ask:thema18sum}] Κάντε ανάλυση σε απλά κλάσματα 
      του όρου που βρίσκεται μέσα στο άθροισμα κ στη συνέχεια, γράψτε
      αναλυτικά τους όρους του αθροίσματος του 1ου μέλους. Τι παρατηρείτε?
  \end{description}
  \end{document}
