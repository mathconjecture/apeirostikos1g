\input{preamble_ask.tex}
\input{definitions_ask.tex}


\everymath{\displaystyle}
\pagestyle{askhseis}

\begin{document}

\begin{center}
  \minibox{\large \bfseries  \textcolor{Col1}{Σειρές (Ασκήσεις)}}
\end{center}

\vspace{\baselineskip}



\begin{enumerate}
    \item Να εξεταστούν ως προς τη σύγκλιση οι παρακάτω σειρές 

      \twocolumnside{
      \begin{enumerate}[i)]
        \item $ \sum_{n=1}^{\infty} (n^{2}+1) $ \hfill Απ: αποκλίνει 
        \item $ \sum_{n=1}^{\infty} \frac{n^{2}+1}{3n^{2}-1} $ \hfill Απ: αποκλίνει
      \end{enumerate}
      }{
      \begin{enumerate}[i),start=3]
        \item $ \sum_{n=1}^{\infty} \frac{2n^{3}+n-1}{n^{3}+4}  $ 
          \hfill Απ: αποκλίνει 
      \end{enumerate}
    }

    \item Να εξεταστούν ως προς τη σύγκλιση οι παρακάτω σειρές (Κριτήριο Σύγκρισης ή
      Ορίου) 
      \label{ask:sygk}

      \twocolumnside{
        \begin{enumerate}[i)]
          \item $ \sum_{n=1}^{\infty} \frac{2n}{n^{2}+1} $ \hfill Απ: αποκλίνει 
          \item $ \sum_{n=1}^{\infty} \frac{10n+2020}{n^{2}+1} $ 
            \hfill Απ: αποκλίνει
          \item $ \sum_{n=1}^{\infty} \frac{3n^{3}+1}{4n^{4}-1} $ \hfill Απ: 
            αποκλίνει
          \item $ \sum_{n=1}^{\infty} \frac{n^{2}+1}{n} $ \hfill Απ: αποκλίνει 
          \item $ \sum_{n=1}^{\infty} \frac{\sqrt[3]{n^{2}+1}}{n+1} $ 
            \hfill Απ: αποκλίνει 
          \item $ \sum_{n=1}^{\infty} \frac{1}{\sqrt{n(n+1)}} $ \qquad 
            \hfill Απ: αποκλίνει 
          \item $ \sum_{n=1}^{\infty} \frac{n}{3n^{2}-4}  $ \hfill Απ: αποκλίνει 
          \item $ \sum_{n=1}^{\infty} \frac{n^{3}+2n+1}{5n^{5}-7}  $ \hfill Απ: 
            συγκλίνει
        \end{enumerate}
      }{
      \begin{enumerate}[i),start=8]
        \item $ \sum_{n=1}^{\infty} \frac{\sqrt{n}}{n^{3}+1} $ \hfill Απ: συγκλίνει 
        \item $ \sum_{n=1}^{\infty} \frac{\sqrt{n}}{n+5 \sqrt{n}} $ \qquad 
          \hfill Απ: αποκλίνει 
        \item $ \sum_{n=1}^{\infty} \frac{3^{n}}{5^{n}+1} $ \hfill Απ: συγκλίνει 
        \item $ \sum_{n=1}^{\infty} \frac{1}{n} \left(\frac{2}{5} \right)^{n} $ 
          \hfill Απ: συγκλίνει
        % \item $ \sum_{n=1}^{\infty} \frac{1}{2^{n}} 
        %   \abs{\sin{\left(\frac{n^{2}+1}{n+2}\right)}} $ \hfill Απ: συγκλίνει 
        % \item $ \sum_{n=1}^{\infty} \frac{\abs{\sin{(n^{2}+1)} \cdot 
        %   \cos{(n+5)}}}{n^{4}+5} $ \hfill Απ: συγκλίνει 
        \item $ \sum_{n=1}^{\infty} \frac{\sin^{4}n}{1+ \sqrt{n^{5}}} $
          \hfill Απ: συγκλίνει 
        \item $ \sum_{n=1}^{\infty} \frac{3^{n}+5}{4^{n}+n^{2}} $ \hfill Απ: 
          συγκλίνει
        \item $ \sum_{n=1}^{\infty} \frac{1}{n!} $ 
          % \qquad (\textcolor{Col1}{υπόδειξη:} $ \frac{1}{n!} \leq \frac{1}{2^{n-1}}) $ 
          \hfill Απ: συγκλίνει 
        \item $ \sum_{n=1}^{\infty} \frac{n!}{2^{n}+1} $ \qquad
          % (\textcolor{Col1}{υπόδειξη:} 
          % $ \frac{n!}{2^{n}+1} \geq \cdots \geq \frac{1}{4} $) 
          \hfill Απ: αποκλίνει 
      \end{enumerate}
    }

    \item Να εξεταστούν ως προς τη σύγκλιση οι παρακάτω σειρές (Κριτήριο Λόγου)

      \twocolumnside{
        \begin{enumerate}[i)]
          \item $ \sum_{n=1}^{\infty} \frac{n!}{3^{n}} $ \hfill Απ: αποκλίνει 
            \item $ \sum_{n=1}^{\infty} \frac{2^{n}}{n^{2}} $ \hfill Απ: αποκλίνει
            \item $ \sum_{n=1}^{\infty} \frac{3^{n}\cdot n!}{n^{n}} $ \hfill Απ: 
                αποκλίνει
            \item $ \sum_{n=1}^{\infty} \frac{6^{n}}{2n+7} $ \hfill Απ: αποκλίνει 
          \end{enumerate}
      }{
      \begin{enumerate}[i),start=5]
        \item $ \sum_{n=1}^{\infty} \frac{3^{n}+4^{n}}{5^{n}+4^{n}} $ \hfill Απ: 
          συγκλίνει
        \item $ \sum_{n=1}^{\infty} \frac{(n!)^{2}}{(2n)!} $ \hfill Απ: συγκλίνει
        \item $ \sum_{n=1}^{\infty} \frac{n!}{1\cdot 3 \cdots (2n-1)} $ \hfill Απ: 
          συγκλίνει
      \end{enumerate}
    }

    \pagebreak

    \item Να εξεταστούν ως προς τη σύγκλιση οι παρακάτω σειρές (Κριτήριο Ρίζας)

      \twocolumnside{
        \begin{enumerate}[i)]
            \item $ \sum_{n=1}^{\infty} \left(\frac{3n}{n+1} \right)^{n} $ 
                \hfill Απ: αποκλίνει 
            \item $ \sum_{n=1}^{\infty} \frac{2^{n}}{n^{n}} $ \hfill Απ: συγκλίνει 
            \item $ \sum_{n=1}^{\infty} \frac{e^{n}}{5n} $ \hfill Απ: αποκλίνει 
            \item $ \sum_{n=1}^{\infty} \frac{n^{3}}{e^{n^{2}}} $ \hfill Απ: συγκλίνει 
          \end{enumerate}
      }{
      \begin{enumerate}[i),start=5]
            \item $ \sum_{n=1}^{\infty} \frac{2^{n}}{n\cdot e^{n+1}} $ \hfill Απ: 
                συγκλίνει
            \item $ \sum_{n=1}^{\infty} \left(\sqrt[n]{n} -1\right)^{n} $ 
                \hfill Απ: συγκλίνει 
            \item $ \sum_{n=1}^{\infty} \frac{1}{3^{n}} \cdot 
                \left(\frac{n+1}{n} \right)^{n^{2}} $ \hfill Απ: συγκλίνει 
            \item $ \sum_{n=1}^{\infty} \left(1+ \frac{1}{4n} \right)^{-n^{2}} $
                \hfill Απ: συγκλίνει 
        \end{enumerate}
    }

    \item Να εξεταστούν ως προς τη σύγκλιση οι παρακάτω σειρές (Κριτήριο Ορίου)

      \twocolumnside{
        \begin{enumerate}[i)]
          \item $ \sum_{n=1}^{\infty} \frac{1}{n^{2}+n-1} $ \hfill Απ: συγκλίνει 
          \item $ \sum_{n=1}^{\infty} \frac{2n+1}{n^{2}+3} $ \hfill Απ: αποκλίνει  
      \end{enumerate}}
      {
        \begin{enumerate}[i),start=3]
          \item $ \sum_{n=1}^{\infty} \frac{1}{\sqrt{n^{3}+n^{2}}}  $ \hfill Απ: 
            συγκλίνει
          \item $ \sum_{n=1}^{\infty} \frac{n}{\sqrt{n^{3}+n}} $ \hfill Απ: 
            αποκλίνει
        \end{enumerate}
      }

    \item Να εξεταστούν ως προς τη σύγκλιση οι παρακάτω σειρές (Κριτήριο Leibnitz)

      \twocolumnside{
        \begin{enumerate}[i)]
          \item $ \sum_{n=0}^{\infty} (-1)^{n} \frac{1}{2n+1} $ 
            \hfill Απ: συγκλίνει 
          % \item $ \sum_{n=1}^{\infty} (-1)^{n-1} \frac{1}{\sqrt{n}}  $ \hfill Απ: 
          %   συγκλίνει
          \item $ \sum_{n=1}^{\infty} (-1)^{n+1} \frac{1}{\sqrt{n^{2}+3}} $ 
            \hfill Απ: συγκλίνει 
        \end{enumerate}
        }{
        \begin{enumerate}[i),start=4]
          \item $ \sum_{n=1}^{\infty} \left(- \frac{1}{2}\right)^{n} $ \hfill Απ: 
            συγκλίνει
          \item $ \sum_{n=1}^{\infty} \frac{\cos{n \pi}}{\sqrt{n}} $ 
            \hfill Απ: \begin{tabular}{c} 
              ($ \cos{(n \pi)} = (-1)^{n} $) \\ συγκλίνει 
            \end{tabular}
        \end{enumerate}
      }

    \item Να εξεταστούν ως προς τη σύγκλιση οι παρακάτω σειρές (Κριτήριο Απόλυτης 
      Σύγκλισης)

      \twocolumnside{
        \begin{enumerate}[i)]
          \item $ \sum_{n=1}^{\infty} (-1)^{n-1} \frac{ne}{n!} $ \hfill Απ: 
            συγκλίνει
          \item $ \sum_{n=1}^{\infty} (-1)^{n-1} \frac{n^{2}}{n^{4}+2}  $ \hfill Απ: 
            συγκλίνει
          \item $ \sum_{n=1}^{\infty} \frac{\sin{(3n+1)}}{10^{n}} $ \hfill Απ:
            συγκλίνει 
          \item $ \sum_{n=1}^{\infty} (-1)^{n+1} \frac{\cos{n} + \sin{n}}{5^{n}} $ 
            \hfill Απ: συγκλίνει 
        \end{enumerate}
        }{
        \begin{enumerate}[i),start=5]
          \item $ \sum_{n=1}^{\infty} \frac{(-1)^{n}}{3^{n}} $ \hfill Απ: συγκλίνει 
          \item $ \sum_{n=1}^{\infty} \frac{1}{2^{n}} \cdot 
            \sin{\left(\frac{n^{2}+1}{n+2}\right)}$ \hfill Απ: συγκλίνει 
          \item $ \sum_{n=1}^{\infty} 
            \left[\sin{(n^{2}+5)} \cdot \frac{n+1}{n^{3}+7}\right] $ \hfill Απ:
            συγκλίνει 
          \item $ \sum_{n=1}^{\infty} \frac{1}{n} \cdot \sin{\frac{\pi}{n}} $ 
            \hfill Απ: \begin{tabular}{c} 
              $\left(\abs{\sin{\frac{\pi}{n}}} \leq \abs{\frac{\pi}{n}} \right)$ \\ 
              συγκλίνει 
            \end{tabular}
        \end{enumerate}
      }
  \end{enumerate}

  \section*{Υποδείξεις}

  \begin{myitemize}
    \item Στην άσκηση~\ref{ask:sygk} \textbf{όλα} τα παραδείγματα, λύνονται με 
      κριτήριο σύγκρισης. Εννοείται, πως σε κάποια με ρητές, και αυτά με τις ρίζες, 
      βολεύει επίσης και το κριτήριο του Ορίου.
      \textbf{Προσοχή}, όμως, γιατί στο κριτήριο του Ορίου, πρέπει οι ακολουϑίες να είναι
      \textbf{θετικών} όρων, αλλιώς εφαρμόζω (κατάλληλα), μόνο το κριτήριο σύγκρισης.
      Σε αυτές με τα παραγοντικά, βολεύει και το κριτήριο Λόγου. 
      Για να εφαρμόσετε, κρ. σύγκρισης, θυμηθείτε ότι
      $ n! \geq 2^{n-1}, \; \forall n \in \mathbb{N} $.
  \end{myitemize}

  \end{document}

