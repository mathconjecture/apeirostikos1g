\begin{enumerate}
      \item {\boldmath\textcolor{Col2}{$A = (0,2) $}}
        \begin{proof}(Με Ορισμό)
        \item {}
          Έχουμε ότι $ \sup A = 2 $ και $ \inf A = 0 $.
          \begin{myitemize}
            \item Θα δείξουμε ότι $ \sup A = 2 $. Πράγματι:
              $2$ α.φ. του $A$, γιατί $ a < 2, \; \forall a \in A $.
              Έστω $M$ α.φ. του $A$, δηλ. $a \leq M, \; \forall a \in A $.
              Θ.δ.ο. $ 2 \leq M $ (Με άτοπο). Πράγματι:
              Έστω $ M < 2 $. Τότε $ (M,2) \neq \emptyset \Rightarrow (M,2) \cap A \neq
              \emptyset $, άρα 
              $ \exists a \in (M,2) $.
              Δηλαδή $ \exists a \in A $ με $ a > M $. 
              άτοπο, γιατί $ M $ α.φ. του $A$.
            \item Θα δείξουμε ότι $ \inf A = 0 $. Πράγματι:
              $0$ κ.φ. του $A$, γιατί $ a > 0, \; \forall a \in A $.
              Έστω $m$ κ.φ. του $A$, δηλ. $a \geq m, \; \forall a \in A $.
              Θ.δ.ο. $ 0 \geq m $ (Με άτοπο). Πράγματι:
              Έστω $ m > 0 $. Τότε $ (0,m) \neq \emptyset \Rightarrow (0,M) \cap A \neq
              \emptyset $, άρα 
              $ \exists a \in (0,m) $. 
              Δηλαδή $ \exists a \in A $ με $ a < m $, 
              άτοπο, γιατί $ m $ κ.φ. του $A$.
          \end{myitemize}
        \end{proof}
      \item {\boldmath\textcolor{Col2}{$ A = \{ x \in \mathbb{R} \; 
        : \; x < 0 \} $}}. 
        \begin{proof}(Με Ορισμό)
        \item {}
          Έχουμε ότι $ \sup A = 0 $ και $ \inf A = - \infty $.
          \begin{myitemize}
            \item Θα δείξουμε ότι $ \sup A = 0 $. Πράγματι:
              $ 0 $ α.φ. του $A$, γιατί $ a < 0, \; \forall a \in A $.
              Έστω $M$ α.φ. του $A$, δηλ. $a \leq M, \; \forall a \in A $.
              Θ.δ.ο. $ 0 \leq M $ (Με άτοπο). Πράγματι:
              Έστω $ M < 0 $. Τότε $ (M,0) \neq \emptyset $, άρα 
              $ \exists a \in (M,0) $.
              Δηλαδή $ \exists a \in A $ με $ a > M $, 
              άτοπο, γιατί $ M $ α.φ. του $A$.
            \item Θα δείξουμε ότι $ \inf A = -\infty $. Πράγματι:
              $ A \neq \emptyset $ και $A$ όχι κάτω φραγμένο. Άρα γράφουμε
              $ \inf A = - \infty $.
          \end{myitemize}
        \end{proof}

      \item {\boldmath \textcolor{Col2}{$ A = \left\{ \frac{1}{n} \; 
        : \; n \in \mathbb{N} \right\} $}}.
        \begin{proof}
        \item {}
          Έχουμε ότι $ \sup A = 1 $ και $ \inf A = 0 $.
          \begin{myitemize}
            \item Θα δείξουμε ότι $ \sup A = 1 $. Πράγματι:
              $ \frac{1}{n} \leq 1, \; \forall n \in \mathbb{N} $, άρα το 1 είναι 
              α.φ. του $A$ και επίσης $ 1 \in A $ (για $ n=1 $). 
              Άρα $ 1 = \max A $. Άρα $ \sup A = \max A = 1 $.
            \item Θα δείξουμε ότι $ \inf A = 0 $. Πράγματι:
              Προφανώς $ 0 \leq \frac{1}{n}, \; \forall n \in \mathbb{N} $, 
              άρα το $ 0 $ κ.φ. του $A$.
              Έστω $ \varepsilon > 0 $. Τότε $ \exists n_{0} \in \mathbb{N} $
              με $ \frac{1}{n_{0}} < \varepsilon $ τ.ω. $ \frac{1}{n_{0}} 
              < 0 + \varepsilon$, με $ \frac{1}{n_{0}} \in A $.
          \end{myitemize}
        \end{proof}
      \item {\boldmath \textcolor{Col2}{$ A = \left\{ \frac{n}{n+1} \; 
        : \; n \in \mathbb{N} \right\} $ }}
        \begin{proof}
        \item {}
          Έχουμε ότι $ \sup A = 1 $ και $ \inf A = \frac{1}{2} $
          \begin{myitemize}
            \item Θα δείξουμε οτι $ \sup A = 1 $. Πράγματι:
              $ 1 $ α.φ. του $A$, γιατί $ \frac{n}{n+1} < 1, \; \forall n \in 
              \mathbb{N}$.
              (Δοκιμή: $ 1- \varepsilon < \frac{n}{n+1} \Leftrightarrow 
              \varepsilon > 1 - \frac{n}{n+1} \Leftrightarrow \frac{1}{n+1} 
              < \varepsilon \Leftrightarrow n+1 > \frac{1}{\varepsilon} 
              \Leftrightarrow n > \frac{1}{\varepsilon} -1$ )
              Έστω $ \varepsilon > 0 $. Τότε $ \exists n_{0} \in \mathbb{N}  $ με 
              $ n_{0} > \frac{1}{\varepsilon} - 1 $ τ.ω. $ 1 - \varepsilon < 
              \frac{n_{0}}{n_{0}+1}$, με $ \frac{n_{0}}{n_{0} +1} \in A $.
            \item Θα δείξουμε ότι $ \inf A = \frac{1}{2} $. Πράγματι: $ \frac{1}{2}
              $ κ.φ. του $A$, γιατί 
              $ \frac{n}{n+1} \geq \frac{1}{2} \Leftrightarrow n+1 \leq 2n \Leftrightarrow
              n \geq 1, \; \forall n \in \mathbb{N} $ (ισχύει)
              και $ \frac{1}{2} \in A$ (για $ n = 1 $).
              Άρα $ \frac{1}{2} = \min A $. Άρα $ \inf A = \min A = \frac{1}{2} $.
          \end{myitemize}
        \end{proof}
      \item {\boldmath \textcolor{Col2}{$ A = \left\{ \frac{1}{n} + (-1)^{n} \; 
        : \; n \in \mathbb{N} \right\} $}}.
        \begin{proof}
        \item {}
          Παρατηρούμε ότι \begin{align*} 
            A &= \left\{ 0, \frac{1}{2} + 1, \frac{1}{3} -1, 
            \frac{1}{4} + 1, \frac{1}{5} -1, \ldots  \right\} \\ 
              &= \left\{ 0, \frac{1}{3} -1, \frac{1}{5} -1, \ldots \right\} \cup 
              \left\{ \frac{1}{2} + 1, \frac{1}{4} + 1, \ldots \right\} \\ 
              &= \left\{ \frac{1}{2n-1} - 1 \; : \; n \in \mathbb{N}\right\}
              \cup \left\{ \frac{1}{2n} + 1 \; : \; n \in \mathbb{N}\right\} \\
              &= A_{1} \cup A_{2} 
            \end{align*}
            Αν παραστήσουμε τα στοιχεία του 
            $A=\{ 0, \frac{3}{2}, -\frac{2}{3}, \frac{5}{4}, -\frac{4}{5}, \ldots \}$, 
            πάνω στην ευθεία των πραγματικών αριθμών, παρατηρούμε ότι 
            $ \inf A = \inf A_{1} = -1 $ και $ \sup A = \sup A_{2} = \max A_{2} = 
            1+ \frac{1}{2} = \frac{3}{2} $. Πράγματι, από γνωστή πρόταση έχουμε ότι
            $ \sup A = \max \{ \sup A_{1}, \sup A_{2}\} = 
            \max \{0,\frac{3}{2}\} = \frac{3}{2}  $ και 
            $ \inf A = \min \{ \inf A_{1}, \inf A_{2} \} = \min \{ -1, 1 \} = -1 $ 
            Οπότε αρκεί να αποδείξουμε τα sup και inf των $A_{1}, A_{2} $. 
            \begin{myitemize}
              \item Θα δείξουμε ότι $ \sup A_{2} = \frac{3}{2} $. Πράγματι:
                $ \frac{3}{2} $ α.φ. του $A_{2} $, γιατί $ \frac{1}{2n}+1 \leq
                \frac{3}{2} \Leftrightarrow \frac{1+2n}{2n} \leq \frac{3}{2}
                \Leftrightarrow 2+4n \leq 6n \Leftrightarrow n \geq 1, 
                \; \forall n \in \mathbb{N} $ (ισχύει) 
                και $ \frac{3}{2} \in A_{2} $ (για $ n=1 $).
                Άρα $ \frac{3}{2} = \max A_{2} $. Άρα $ \sup A_{2} = \max A_{2} 
                = \frac{3}{2} $.
              \item Θα δείξουμε ότι $ \inf A_{1} = -1 $. Πράγματι:
                $ -1 $ κ.φ. του $A_{1}$, γιατί προφανώς $ \frac{1}{2n-1} - 1 \geq -1, 
                \; \forall n \in \mathbb{N} $.
                (Δοκιμή: $ \frac{1}{2 n -1} -1 < -1 + \varepsilon 
                \Leftrightarrow \frac{1}{2 n -1} < \varepsilon \Leftrightarrow 
                2n-1 > \frac{1}{\varepsilon} \Leftrightarrow n > \frac{1/ \varepsilon +
                1}{2}). $
                Έστω $ \varepsilon > 0 $. Τότε $ \exists n_{0} \in \mathbb{N}  $ 
                με $ n_{0} > \frac{1/ \varepsilon +1}{2} $ τ.ω. $ \frac{1}{2 n_{0} -1} 
                - 1 < \varepsilon -1 $, με $ \frac{1}{2 n_{0}-1} - 1 \in A $.
            \end{myitemize}
          \end{proof}
        \item {\boldmath \textcolor{Col2} {$ A = \left\{ \frac{1}{n} - 
          \frac{1}{m} \; : \; n,m \in \mathbb{N} \right\}$ }}
          Έχουμε ότι $ \sup A = 1 $  και $\inf A = -1 $.
          \begin{myitemize}
            \item Θα δείξουμε ότι $ \inf A = -1 $. Πράγματι:
              $ -1 $ κ.φ. του $A$, γιατί $ \frac{1}{n} - \frac{1}{m} \geq \frac{1}{n} 
              - 1 > -1, \; \forall n \in \mathbb{N} $.
              'Εστω $ \varepsilon > 0 $. Τότε, $ \exists n_{0} \in \mathbb{N} $ με 
              $ \frac{1}{n_{0}} < \varepsilon $ ώστε
              $ \frac{1}{n_{0}} - 1 < \varepsilon -1 $ με $ \frac{1}{n_{0}} -1 \in A $.
            \item Θα δείξουμε ότι $ \sup A = 1 $. Πράγματι:
              Παρατηρούμε ότι $ A = -A $, άρα από γνωστή πρόταση έχουμε ότι 
              \[ \inf A = - \sup (-A) = - \sup A \]
              Άρα \[ \sup A = - \inf A = -(-1) = 1 \]
          \end{myitemize}
      \end{enumerate}

    \item \textcolor{Col1}{Έστω $ A,B $ μη-κενά, φραγμένα υποσύνολα του $ \mathbb{R} $.
        Να δείξετε ότι:
        \begin{enumerate}
          \item $ A \cup B  $ είναι φραγμένο
          \item $ \sup {(A\cup B)} = \max \{ \sup A, \sup B \} $
          \item $ \inf {(A\cup B)} = \min \{ \inf A, \inf B \} $
          \item Ισχύει κάτι ανάλογο για το $ \sup {(A\cap B)} $ και 
            $ \inf {(A\cap B)} $;
      \end{enumerate}}
      \begin{proof}
      \item {}
        \begin{enumerate}
          \item {}
            $A$ φραγμένο $ \Leftrightarrow \exists M \in \mathbb{R} 
            > 0, \; -M < a < M, \; \forall a \in A $ 

            $B$ φραγμένο $ \Leftrightarrow \exists N \in \mathbb{R} 
            > 0, \; -N < b < N, \; \forall b \in B $ 

            Θέτουμε $ K = \max \{ M,N \} $. 

            Θα αποδείξουμε (με άτοπο) ότι $ -K < c < K, \; 
            \forall c \in A \cup B $.Πράγματι:

            Έστω $ c \in A \cup B $ με $ \abs{c} \geq K = \max \{ 
            M,N\} $. Τότε 

            $ 
            \left.
              \begin{tabular}{l}
                Αν $c \in A$ τότε $M > \abs{c} \geq \max \{ M,N \}$, 
                άτοπο \\
                Αν $c \in B$ τότε $N > \abs{c} \geq \max \{ M,N \}$,
                άτοπο
              \end{tabular} 
            \right\}  \Rightarrow \\
            -K < c < K, \; \forall c \in A \cup B $.
          \item 
            $
            \left.
              \begin{tabular}{l}
                $ A \neq \emptyset $, και άνω φραγμένο $
                \overset{\text{Α.Π.}}{\Rightarrow} \exists \sup A  $ \\

                $ B \neq \emptyset $, και άνω φραγμένο $
                \overset{\text{Α.Π.}}{\Rightarrow} \exists \sup B  $ \\
              \end{tabular}
            \right\}  \Rightarrow $ \\ 
            χ.β.γ. έστω $ \max \{ \sup A, \sup B \} = \sup A$. 

            Θα δείξουμε ότι $ \sup (A \cup B) = \sup A $. Πράγματι:

            Έστω $ c \in A \cup B $. Τότε, αν $ c \in A \Rightarrow c \leq 
            \sup A$, ενώ αν $ c \in B \Rightarrow c \leq \sup B \leq \sup A $. 
            Άρα σε κάθε περίπτωση $ c \leq \sup A, \; \forall c \in A \cup 
            B$. Άρα $ \sup A $ α.φ. του $ A \cup B $.

            Ισχύει ότι \inlineequation[eq:first]{ \sup (A \cup B) 
            \leq \sup A }, γιατί $ \sup A $ α.φ. 
            του $ A \cup B $. Θα δείξουμε ότι \inlineequation[eq:two]{ 
            \sup A \leq \sup (A \cup B) }.
            Πράγματι: 

            Αρκεί να δείξουμε ότι $ \sup (A \cup B) $ α.φ. του $A$. Πράγματι:

            έστω $ a \in A \Rightarrow a \in A \cup B \Rightarrow a \leq \sup 
            (A \cup B), \; \forall a \in A$, άρα το $ \sup A $ α.φ. του $A$.

            Οπότε από τις $ \eqref{eq:first} $ και $ \eqref{eq:two} $, 
            προκύπτει το ζητούμενο.
        \end{enumerate}
      \end{proof}

    \item \textcolor{Col1}{Έστω $ A, B $ μη-κενά, άνω φραγμένα υποσύνολα του
      $ \mathbb{R} $.}

      Αν $ A+B = \{ a+b \; : \; a \in A, \; b\in B \} $ και $A \cdot B = 
      \{ a\cdot b \; : \; a \in A, \; b \in B\}$ . Να δείξετε ότι 
      \begin{enumerate}
        \item $ \sup {(A+B)} = \sup A + \sup B $.
        \item $ \sup {(A\cdot B)} = \sup A \cdot \sup B $
      \end{enumerate}
      \begin{proof}
      \item {}
        $
        \left.
          \begin{tabular}{l}
            $ a \leq \sup A, \; \forall a \in A$ \\
            $ b \leq \sup B, \; \forall b \in B $
          \end{tabular}
        \right\} \Rightarrow a+b \leq \sup A + \sup B, \; \forall a \in A 
        $ και $ b \in B $.

        Οπότε $ \sup A + \sup B $ α.φ. του $ A + B \Rightarrow \sup (A+B) \leq 
        \sup A + \sup B$.
        Θ.δ.ο. $ \sup A + \sup B \leq \sup (A+B) $. Πράγματι:

        Έστω $ a \in A $ και $ b \in B \Rightarrow a+b \leq \sup (A+B) 
        \Leftrightarrow a \leq \sup (A+B) - b, \; \forall a \in A $ 

        Άρα $ \sup (A+B) - b $ α.φ. του $A, \; \forall b \in B $, 

        άρα $ \sup A \leq \sup (A+B) - b, \; \forall b \in B \Leftrightarrow 
        b \leq \sup (A+B) - \sup A , \; \forall b \in B$. 

        Άρα $ \sup (A+B) - \sup A $ α.φ. του $B$, και άρα 

        $ \sup B \leq \sup (A+B) - \sup A \Leftrightarrow \sup A + \sup B \leq 
        \sup (A+B)$.
      \end{proof}

