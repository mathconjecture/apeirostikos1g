\input{preamble_ask.tex}
\input{definitions_ask.tex}

\everymath{\displaystyle}
\pagestyle{askhseis}

\begin{document}

\begin{center}
  \minibox[c]{\large \bfseries \textcolor{Col1}{Ακολουθίες}\\ \large 
  \textcolor{Col1}{Ασκήσεις}}
\end{center}

\vspace{\baselineskip}


\setcounter{chapter}{1}
\section{Φραγμένες Ακολουθίες}

\begin{enumerate}
  \item Να δείξετε ότι η ακολουθία $ a_{n} = (-1)^{n}\frac{1}{2n} $ είναι 
    φραγμένη.
    \hfill Απ: $ \abs{a_{n}} \leq \frac{1}{2} $ 
  \item Να δείξετε ότι η ακολουθία $ a_{n} = \frac{5 \cos^{3}{n}}{n+2} $ 
    είναι φραγμένη.
    \hfill Απ: $ \abs{a_{n}} < \frac{5}{2}  $ 
  \item Να δείξετε ότι η ακολουθία $ a_{n} = \frac{\cos{n} + n \sin{n}}{n^{2}} $ 
    είναι φραγμένη. 
    \hfill Απ: $ \abs{a_{n}} \leq 2 $ 
  \item Να δείξετε ότι η ακολουθία $ a_{n} = \frac{n}{3^{n}} $ είναι 
    φραγμένη. 
    \hfill Απ: $ 0<a_{n}< \frac{1}{2} $
  \item Να δείξετε ότι η ακολουθία $ a_{n} = \frac{n!}{n^{n}} $ είναι 
    φραγμένη. 
    \hfill Απ: $ 0 \leq a_{n} \leq 1 $ 
  \item Να δείξετε ότι η ακολουθία $ a_{n} = 1 + \frac{1}{1!} +
    \frac{1}{2!} + \cdots + \frac{1}{n!} $ είναι άνω φραγμένη.
    \hfill Απ: $ a_{n} < 3 $ 
    % \item Να δείξετε ότι η ακολουθία $ a_{n} = \frac{3 \sin{3n}}{n^{2}} $ 
    % είναι φραγμένη.
    % \hfill Απ: $ \abs{a_{n}} \leq 3 $ 
  \item Να δείξετε ότι η ακολουθία $ a_{1} = 3, \; a_{n+1} =
    \frac{a_{n}+4}{2}, \; \forall n \in \mathbb{N} $ είναι άνω φραγμένη.
    \hfill Απ: $ a_{n} < 4 $ 
  \item Να δείξετε ότι η ακολουθία $ a_{1} = \sqrt{2}, \; a_{n+1} =
    \sqrt{2+ a_{n}}, \; \forall n \in \mathbb{N} $ είναι φραγμένη.
    \hfill Απ: $ 0 < a_{n} < 2$ 
  \item Να δείξετε ότι η ακολουθία $ a_{n} = 3-2n $ δεν είναι κάτω φραγμένη. 
  \item Να δείξετε ότι η ακολουθία $ a_{n} = 2^{n} $ δεν είναι άνω 
    φραγμένη.
    % \item Να δείξετε ότι η ακολουθία $ a_{n} = \frac{n^{3} + \sin{5n}}{n} $ δεν είναι 
    %     φραγμένη.
  % \item Να δείξετε ότι η ακολουθία $ a_{n} = \frac{n^{2}}{3n+ \sin^{2}{n}} $ δεν 
    % είναι άνω φραγμένη.
\end{enumerate}

\section{Μονότονες Ακολουθίες}

\begin{enumerate}
  \item Να δείξετε ότι ακολουθία $ a_{n} = \frac{n}{5n-1} $ είναι 
    γνησίως φθίνουσα.
  \item Να δείξετε ότι ακολουθία $ a_{n} = \frac{2n^{2}-1}{n} $ είναι γνησίως 
    αύξουσα.
  \item Να δείξετε ότι ακολουθία $ a_{n} = \frac{n}{3^{n}} $ είναι 
    γνησίως φθίνουσα.
  \item Να δείξετε ότι ακολουθία $ a_{n} = \frac{2^{n}}{n!} $ είναι 
    γνησίως φθίνουσα.
  \item Να δείξετε ότι η ακολουθία $ a_{1}=0, \; a_{n+1}= 
    \frac{2 a_{n}+4}{3}, \; \forall n \in \mathbb{N} $ είναι γνησίως αύξουσα.
  \item Να δείξετε ότι ακολουθία $ a_{1}=1, \; a_{n} = \sqrt{a_{n}+1}, \; 
    \forall n \in \mathbb{N}$ είναι γνησίως αύξουσα.
    % \item Να δείξετε ότι ακολουθία $ a_{n} = \frac{1}{1\cdot 2} + \frac{1}{2\cdot 3} 
    % + \cdots + \frac{1}{n(n+1)} $ είναι γνησίως αύξουσα.
  \item Να δείξετε ότι ακολουθία $ a_{n} =  \frac{(-1)^{n}}{n^{2}+2} $ 
    δεν είναι μονότονη. 
\end{enumerate}

\section{Ορισμός του Ορίου}

\begin{enumerate}
  \item Να δείξετε με τη βοήθεια του ορισμού τα παρακάτω όρια.
    \begin{enumerate}[i)]
      \item $ \lim_{n \to \infty} \frac{n}{3n-1} = \frac{1}{3} $
      \item $ \lim_{n \to \infty} \frac{n+2}{n^{2}} = 0 $
      \item $ \lim_{n \to \infty} \frac{3n -2}{2n+1} = \frac{3}{2} $ 
      \item $ \lim_{n \to \infty} \frac{5n-4}{2-3n} = - \frac{5}{3} $ 
      \item $ \lim_{n \to \infty} \frac{n^{2}+n}{n^{2}+3} = 1 $ 
      \item $ \lim_{n \to \infty} \frac{\sin{\frac{n^{3}}{3}}}{n^{3}}=0 $
      \item $ \lim_{n \to \infty} \frac{\sin{n} + \cos{3n}}{n^{2}} = 0 $
      \item $ \lim_{n \to \infty} (\sqrt{n+2} - \sqrt{n}) = 0 $
    \end{enumerate}
\end{enumerate}


\end{document}

